%-------------------------------------------------------------------------------
% menu
%-------------------------------------------------------------------------------
%
% \file        menu.tex
% \library     Documents
% \author      Chris Ahlstrom
% \date        2015-08-31
% \update      2025-05-30
% \version     $Revision$
% \license     $XPC_GPL_LICENSE$
%
%     Provides the Menu section of seq66-user-manual.tex.
%
%-------------------------------------------------------------------------------

\section{Menu}
\label{sec:menu}

   The \textsl{Seq66} menu structure is more complex than
   that of \textsl{Seq24}.  In particular, the \textsl{File} menu has two
   variants:  a normal file menu, and a file menu when \textsl{Seq66} is
   running under the \textsl{New/Non Session Manager}.
   (See \sectionref{subsec:sessions_nsm}.)

\subsection{Menu / File}
\label{subsec:menu_file}

   The \textbf{File} menu is used to save and load files in
   Standard MIDI Format 0 or 1, \textsl{Cakewalk} "WRK",
   and \textsl{Seq66} MIDI files.
   It also supports a list of recent files, and
   sub-menus for import and export functions,
   which have expanded quite a bit.
   The \textsl{Seq66} \textbf{File} menu contains the sub-items shown below.
   The next few sub-sections discuss
   the sub-items in the \textbf{File} menu.
   Please note that these entries are different
   if \textsl{Seq66} is started under the control of the
   \textsl{New/Non Session Manager}.  
   See \sectionref{subsubsec:sessions_file_menu}.
   However, the import and export menus remain the same, although there are
   slight differences in how they work.

\begin{figure}[H]
   \centering 
   \includegraphics[scale=0.75]{main-menu/file/file-import-export-menus.png}
   \caption{Seq66 File Menu Plus Import/Export, Composite View}
   \label{fig:menu_file_items}
\end{figure}

   \begin{enumber}
      \item \textbf{New}
      \item \textbf{Open}
      \item \textbf{Open Playlist}
      \item \textbf{Recent MIDI files}
      \item \textbf{Save}
      \item \textbf{Save As}
      \item \textbf{Import}
      \begin{enumber}
         \item \textbf{Project Configuration...}
         \item \textbf{MIDI to Current Set...}
         \item \textbf{Playlist...}
         \index{restart!automatic}
         Once the playlist is imported,
         \textsl{Seq66} is automatically \textsl{\textbf{restarted}}
         in order to load the playlist.
         Be careful!
      \end{enumber}
      \item \textbf{Export}
      \begin{enumber}
         \item \textbf{Project Configuration...}
         \item \textbf{MIDI Only...}
         \item \textbf{Song...}
         \item \textbf{SMF 0...}
      \end{enumber}
      \item \textbf{Quit} (\textbf{Exit} in \textsl{Windows})
   \end{enumber}

   For information on the \textbf{File} menu when \textsl{Seq66} is
   running under the \textsl{Non Session Manager}, see
   \sectionref{subsubsec:sessions_file_menu}.

\subsubsection{Menu / File / New}
\label{subsec:menu_file_new}

   The \textbf{New} menu entry clears the current song.
   (A play-list or mute-groups setup, if loaded, are not affected.)
   If unsaved changes are pending, the user is prompted to save the changes.
   Prompting for changes is more comprehensive than \textsl{Seq24}.
   However, when in doubt, save!
   Keep backups of your tunes and configuration files!

\subsubsection{Menu / File / Open}
\label{subsubsec:menu_file_open}

   The \textbf{Open} menu entry opens a song (MIDI file or \textsl{Cakewalk}
   WRK file), replacing the current song (after a prompt if the song was
   modified).
   It opens up a standard file dialog:

\begin{figure}[H]
   \centering 
   \includegraphics[scale=0.65]{main-menu/file/light-menu-file-open.png}
   \caption{File / Open}
   \label{fig:menu_file_open}
\end{figure}

   This dialog lets one type a file-name, highlighting the first file
   that matches the characters typed.
   \textsl{Seq66} can open \textsl{Seq66}, MIDI SMF 0, and SMF 1 files, and
   \textsl{Cakewalk} WRK files.
   If the file is an SMF 0 file, where all channels appear on one track, the
   track can be split so that each channel (0 to 15)
   is stored in the corresponding
   pattern, and pattern 16 contains the original track.
   This feature is enabled via
   \textbf{Edit / Preferences / Pattern / Automatic conversion of SMF 0 to SMF
   1} or the 'usr' setting \texttt{convert-to-smf-1 = true}.

   Note that a MIDI file can be drag-and-dropped from a file manager onto
   the grid to open a file.

\subsubsection{Menu / File / Open Playlist}
\label{subsubsec:menu_file_open_playlist}

   The \textbf{Open Playlist...} menu entry opens a \textsl{Seq66}
   play-list file.

\begin{figure}[H]
   \centering 
   \includegraphics[scale=0.65]{main-menu/file/dark-menu-file-open-playlist.png}
   \caption{File / Open Playlist}
   \label{fig:menu_file_open_playlist}
\end{figure}

   The playlist file contains a list of "playlist sections",
   each listing a number of MIDI songs.
   These playlists and songs can be
   selected by the arrow keys or by MIDI control,
   and are displayed and editiable in the \textsl{Playlist} tab
   in the main window.
   See \sectionref{sec:playlist}.

\subsubsection{Menu / File / Recent MIDI files}
\label{subsubsec:menu_file_recent}

   This menu entry provides a list of the last few MIDI files created or opened;
   play-list selections are \textsl{not} included in this list.

\begin{figure}[H]
   \centering 
   \includegraphics[scale=0.65]{main-menu/file/menu-recent-files.png}
   \caption{Seq66 Menu File Recent Files}
   \label{fig:menu_file_recent_files}
\end{figure}

   Here is the long form when the 'rc' file's
   \texttt{[recent-files] full-paths} value is set to true:

\begin{figure}[H]
   \centering 
   \includegraphics[scale=0.65]{main-menu/file/menu-recent-files-long.png}
   \caption{Seq66 Menu File Recent Files, Full Paths}
   \label{fig:menu_file_recent_files_full_paths}
\end{figure}

   This list is saved in the \texttt{[recent-files]} section of the
   'rc' configuration file.
   In the 'rc' file, the full path to the file-name is stored.
   This path is in "UNIX" format, using the forward slash (solidus),
   as the path separator, even in \textsl{Windows}.
   The \texttt{full-paths} option can be set to show the full path in the
   recent-files drop-down menu.
   Only unique entries are included in the recent-files list.
   The limit is 12 recent-file entries.
   This is a feature from \textsl{Kepler34} \cite{kepler34}.
   One can also set \textsl{Seq66} to load the most-recent file at startup.
   Here is an example from an 'rc' file:

\begin{verbatim}
   [recent-files]
   full-paths = false
   load-most-recent = true
   count = 3
   /home/user/git/seq66/data/b4uacuse-gm-patchless.midi
   /home/user/git/seq66/data/midi/colours.midi
   /home/user/git/Julian-data/TestBeeps.midi
\end{verbatim}

\subsubsection{Menu / File / Save and Save As}
\label{subsubsec:menu_file_open_save_as}

   The \textbf{Save} menu entry saves the song under its current file-name.
   If there is no current file-name, it opens up a standard file
   dialog to name and save the file.
   The \textbf{Save As} menu entry saves a song under a different name.
   It opens up the following standard file dialog, very similar to the 
   \textbf{File Open} dialog, with an additional \textbf{Name} text-edit field.
   The exact look of the dialog depends on system Qt settings or the current
   Qt theme.

\begin{figure}[H]
   \centering 
   \includegraphics[scale=0.65]{main-menu/file/dark-menu-file-save-as.png}
   \caption{File / Save As}
   \label{fig:menu_file_save_as}
\end{figure}

   To save a new file or save the current file to a new name,
   enter the name in the name field, without an extension.
   \textsl{Seq66} will append a \texttt{.midi} extension to the filename.
   The file will be saved in a format that the Linux \textsl{file} command
   will tag as something like:

   \begin{verbatim}
      colours.midi: Standard MIDI data (format 1) using 16 tracks at 1/192
   \end{verbatim}

   It looks like a simple MIDI file, and yet, if one re-opens it in
   \textsl{Seq66}, one sees that the mute-groups, labeling, pattern
   information, and song layout have been preserved in this file.
   This information is saved in a way that MIDI-compliant software
   should be able to use or ignore without failure.
   After the last track in the file, a number of
   \index{SeqSpec}
   sequencer-specific (SeqSpec) items are saved, to preserve
   the extra information that \textsl{Seq66} adds to the song.
   There is no way to save a \textsl{Cakewalk} "WRK" file.
   \textsl{Seq66} can only read them, and then save them as
   \textsl{Seq66} files.

   \index{Meta events}
   Meta events are now handled by \textsl{Seq66}.
   Meta events \textbf{Set Tempo}
   and \textbf{Time Signature}
   are now fully supported.
   Other meta events,
   such as \textbf{Meta MIDI Channel}
   and \textbf{Meta MIDI Port}
   are now read as events, and are saved back when the file is saved.
   They cannot be edited in \textsl{Seq66}, but they are not lost.
   (Channel and port meta events are
   considered \textsl{obsolete} in the MIDI standard.)
   Various meta text events, such as \textbf{Lyric},
   can be edited and saved.

\subsubsection{Menu / File / Import / Project Configuration}
\label{subsubsec:menu_file_import_project_configuration}

   This command is useful to grab an existing project configuration
   (i.e. the set of \texttt{qseq66.*} files) and copy it
   to the current "home" configuration directory.
   This command is most useful in importing a project into a new
   NSM session.
   Previously, the "home" project would be imported automatically
   into a new NSM session, but this was deemed confusing by some users, and
   properly so!

   This command brings up a file dialog box. Navigate to the desired
   source directory and then select the desired 'rc' file.
   Any configuration files in the current "home" directory are removed
   and replaced with all files and subdirectories from the selected
   directory.
   If NSM is not running, \textsl{Seq66} restarts and loads the
   imported configuration.
   Be aware!
   Otherwise, a prompt to restart using the NSM client is shown.
   For more information, see
   \sectionref{subsubsec:midi_export_file_import_project}.

\subsubsection{Menu / File / Import / MIDI to Current Set}
\label{subsubsec:menu_file_import}

   The \textbf{Import} menu entry imports an SMF 0
   or SMF 1 MIDI file as one or more patterns, one pattern per track,
   into the specified screen-set.
   This functionality is explained in detail in
   \sectionref{subsubsec:midi_export_file_import}.

\subsubsection{Menu / File / Import / Playlist}
\label{subsubsec:menu_file_import_playlisMIDI to Current Sett}

   A user can create a playlist that accesses MIDI files anywhere in the file
   system.
   However, in a session manager, it is preferable to have the configuration
   self-contained.
   Even without a session manager, it can be useful to copy a playlist to a
   subdirectory in order to separate it and its MIDI files from other
   playlists.
   Once a project has been imported or saved, then a playlist can also be
   imported, along with all of the MIDI files it references.

   This command brings up a file dialog box. Navigate to the desired
   source directory and then select the desired 'playlist' file.
   This menu entry copies the playlist file and its associated
   MIDI files; see
   \sectionref{subsubsec:midi_export_file_import_playlist}.

\subsubsection{Menu / File / Export / Project Configuration}
\label{subsubsec:menu_file_export_project}

   This menu entry lets the user select a destination directory.
   Then the project files from the current "home" directory are copied
   to that destination directory. Useful for backup or for making
   an experimental configuration official.
   See \sectionref{subsubsec:midi_export_configuration_export}.

\subsubsection{Menu / File / Export / Song}
\label{subsubsec:menu_file_export_song_as_midi}

   Thanks to the \textsl{Seq32} project, the ability to export songs to MIDI
   format has been added.  In this export, a complete song performance is
   recoded so that other MIDI sequencers can play the performance properly.
   This functionality is explained in detail in
   \sectionref{subsubsec:midi_export_song_export}.

\subsubsection{Menu / File / Export / MIDI Only}
\label{subsubsec:menu_file_export_midi_only}

   Sometimes it might be useful to export only the non-vendor-specific
   (non-SeqSpec) data from a \textsl{Seq66} song, in order to reduce the
   size of the file or to accomodate non-compliant sequencers.
   This functionality is explained in detail in
   \sectionref{subsubsec:midi_export_file_export_midi_only}.

\subsubsection{Menu / File / Export / SMF 0}
\label{subsubsec:menu_file_export_smf_0}

   This feature allows all tracks in the song to
   be consolidated and exported in MIDI's SMF 0 format.
   It follows rules similar to song export.
   See \sectionref{subsubsec:midi_export_file_export_smf_0}.

\subsection{Menu / Edit}
\label{subsec:menu_edit}

   The \textbf{Edit} menu has undergone some expansion in \textsl{Seq66}.

   \begin{enumber}
      \item \textbf{Preferences...}
      \item \textbf{Song Editor}
      \item \textbf{Apply Song Transpose}
      \item \textbf{Clear Mute Groups}
      \item \textbf{Reload Mute Groups}
      \item \textbf{Mute All Tracks}
      \item \textbf{Unute All Tracks}
      \item \textbf{Toggle All Tracks}
      \item \textbf{Copy Current Set}
      \item \textbf{Paste To Current Set}
   \end{enumber}

   \setcounter{ItemCounter}{0}      % Reset the ItemCounter for this list.

   \itempar{Preferences}{edit!preferences}
   This entry brings up a \textbf{Preferences} menu entry,
   to allow viewing and tweaking MIDI I/O ports, displays options, JACK
   options, and more.
   It can also be brought up by \texttt{Ctrl-P}.
   It has a lot of configuration items, and is
   discussed in \sectionref{sec:edit_preferences}.

   \itempar{Song Editor}{edit!song editor}
   \index{performance editor}
   \index{song editor}
   This item toggles the presence of the main song/performance editor.
   Note that the song editor is also available in the
   \textbf{Song} tab in the main window.
   The song/performance editor allows specifying exact numbers of loop replays;
   this provides a canned rendition of all the patterns in theMIDI tune.

   \itempar{Apply Song Transpose}{edit!song transpose}
   \index{song transpose}
   Selecting this item applies the global song transposition value to
   all sequences / patterns marked as transposable.
   This actively changes the note / pitch value of all note and aftertouch
   events in every pattern.
   The change can range from -12 semitones to 12 semitones.
   For the setting of the global song transpose value, see
   \sectionref{sec:song_editor}.
   Normally, drum tracks are \textsl{not} transposable.
   This can be set by clicking the "transposable" button in the
   pattern editor.
%  Note that transpose can be enabled in the
%  in the sequence editor
%  (see \sectionref{sec:pattern_editor}).

   \itempar{Clear Mute Groups}{edit!clear mute groups}
   \index{mute groups}
   A feature of \textsl{Seq66} is that the mute groups
   can be saved in both the 'rc' file \textsl{and} in the "MIDI" file.
   This menu entry clears them. If this resulted in any mute-group sequences
   status being set to false, then the user is prompted to save the MIDI
   file, so that it will no longer have any
   mute-group information.  And then, if the
   application exits, the cleared mute-group information is also saved to
   the 'rc' file.

   \itempar{Reload Mute Groups}{edit!load mute groups}
   \index{rc!mute groups}
   This menu entry reloads the mute-groups from the 'rc' file.
   So, if one loads a MIDI file that has its own mute groups that one does not
   like, this command will restore one's favorite mute-grouping from the 'rc'
   file.

   \itempar{Mute All Tracks}{edit!mute all tracks}
   \index{mute all}
   This menu entry, useful mostly in \textbf{Live} mode,
   immediately mutes \textsl{all} patterns in the entire song.
   The hard-wired menu short-cut for this action is \texttt{Ctrl-M}.

   \itempar{Unmute All Tracks}{edit!unmute all tracks}
   \index{unmute all}
   This menu entry, useful mostly in \textbf{Live} mode,
   immediately unmutes \textsl{all} patterns in the entire song.
   The hard-wired menu short-cut for this action is \texttt{Ctrl-U}.

   \itempar{Toggle All Tracks}{edit!toggle all}
   \index{toggle all}
   This option toggles the mute/armed status of \textbf{all} tracks.
   It is useful mostly \textbf{Live} mode, which overrides \textbf{Song}
   mode even if the Song Editor is focused.
   The hard-wired menu short-cut for this action is \texttt{Ctrl-T}.
   The default keystroke in the 'ctrl' file is \texttt{F8}.

   \itempar{Copy Current Set}{edit!copy set}
   \index{copy set}
   This item marks the current set (i.e. the play-set)
   for the copying of all its patterns to another set.
   After clicking this menu entry, one can move to another set to paste it,
   using the following menu entry.

   \itempar{Paste To Current Set}{edit!paste set}
   \index{paste set}
   Once a set has been copied into the internal set clipboard,
   then this menu item is enabled.
   Move to the desired set (whether empty or note), and then
   click this menu item.
   All of the patterns in the original set are pasted into the current set,
   \textsl{overwriting all patterns} already in the set.
   Also note that the set clipboard can be pasted after a
   \textbf{File / New} or \textbf{File / Open},
   to copy it into another file.

\subsection{Menu / Help}
\label{subsec:menu_help}

   The usual \textbf{Help} dialog is provided.
   As of version 0.98.8, it has been beefed up with a way to access a
   tutorial and the user manual.

   These new help items are a work in progress, so please apprise
   us of any issues; include information on the operating system and,
   if \textsl{Linux}, the desktop/window manager in use.

\subsubsection{Menu / Help / About...}
\label{subsubsec:menu_help_about}

   \index{Help!about}
   This menu entry shows the "About" dialog.
   That dialog provides access to some credits for the program as well.
   authors and the project documentors, and active link to them.
   It also shows Git version-control information as well.

\subsubsection{Menu / Help / Build Info...}
\label{subsubsec:menu_help_build_info}

   \index{Help!build info}
   This menu entry shows the "Build Info" dialog.  This list of
   build options enabled in the current application is the same list
   that it generated via this command line:

   \begin{verbatim}
      $ seq66 --version
   \end{verbatim}

\subsubsection{Menu / Help / Song Summary File...}
\label{subsubsec:menu_help_song_summary_file}

   \index{Help!song summary}
   This menu entry allows one to write a summary of the song data into a text
   file. It brings up a file dialog which defaults to the name of the
   currently-loaded MIDI file, with the extenstion \texttt{.text} and
   the directory from where the MIDI file was loaded.
   It shows the filename, the information about the sets and tracks,
   MIDI format (0 or 1), and the PPQN.

   It also shows each sequence: name, channel (128 mean there is no output
   channel), the time signature, buss number (and any mapping), the length in
   pulses, the event and trigger count, transposability, key and scale, and
   color number (if any).
   For each trigger in the pattern, its start, stop, offset, and transposition
   values are shown.
   This file can be helpful for trouble-shooting or solving puzzling effects in
   the tune.

\subsubsection{Menu / Help / App Keys}
\label{subsubsec:menu_help_app_keys}

   \index{Help!app keys}
   This entry brings up a dialog that shows brief descriptions of the
   non-automation keys available in various contexts.
   These keys are almost exclusively hardwired and currently cannot be
   changed via a configuration file.  By pressing a button, the desired
   keystrokes can be quickly viewed. Note that the descriptions come from small
   HTML files that are part of the installation.

\subsubsection{Menu / Help / Tutorial}
\label{subsubsec:menu_help_tutorial}

   \index{Help!tutorial}
   This entry brings up a short tutorial of \textsl{Seq66} in the default
   browser. This tutorial is meant only to jump-start a new user of
   \textsl{Seq66}, and is a work in progress.
   It does not cover nearly as much as the user manual, so check that out in
   the next section.

   Normally, the tutorial will open a web page.  If it does not, one might need
   to set up a default browser.  On Linux, make sure that there is a "desktop"
   file for the browser, as in
   \texttt{/usr/share/applications/firefox.desktop}.
   If so, then run the following command, and then test it:

   \begin{verbatim}
      $ xdg-settings set default-web-browser firefox.desktop
      $ xdg-open https://ahlstromcj.github.io/docs/seq66/tutorial/index.html 
   \end{verbatim}

   On Windows, this procedure is still \textsl{to be determined}.

   In both systems, one can override the default applications by opening
   the specified 'usr' file (usually \texttt{qseq66.usr} or
   \texttt{qpseq66.usr} and specifying the full path to the desired
   applications (Linux paths shown here):

   \begin{verbatim}
      [user-options]
      log = "/home/user/.config/seq66/seq66.log"
      pdf-viewer = "/usr/bin/zathura"
      browser = "/usr/bin/google-chrome"
   \end{verbatim}

   Also see \sectionref{subsubsec:usr_file_user_options}.

\subsubsection{Menu / Help / User Manual}
\label{subsubsec:menu_help_user_manual}

   \index{Help!user manual}
   This menu entry first tries to locate the user manual on the internet and
   open it in the default browser. If not found, or the network is down,
   then this entry brings up the full \textsl{Seq66} user manual in the default
   PDF viewer.  It currently looks in the possible installation areas and in
   the \textsl{Seq66} source tree to find the PDF.

   On Linux, one can follow the setup procedure in the previous section and
   test it via the following command, which will show the manual in the default
   browser.:

   \begin{verbatim}
      $ xdg-open https://ahlstromcj.github.io/docs/seq66/seq66-user-manual.pdf
   \end{verbatim}

%-------------------------------------------------------------------------------
% vim: ts=3 sw=3 et ft=tex
%-------------------------------------------------------------------------------
