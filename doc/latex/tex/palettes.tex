%-------------------------------------------------------------------------------
% palettes
%-------------------------------------------------------------------------------
%
% \file        palettes.tex
% \library     Documents
% \author      Chris Ahlstrom
% \date        2020-12-29
% \update      2021-02-19
% \version     $Revision$
% \license     $XPC_GPL_LICENSE$
%
%     Provides a discussion of the MIDI GUI palettes that Seq66
%     supports.
%
%-------------------------------------------------------------------------------

\section{Palettes for Coloring}
\label{sec:palettes}

   Many user-interface elements in \textsl{Seq66} are drawn independently of
   the Qt theme in force, and they have their own coloring.  Also, patterns can
   be colored, and the color is stored (as a color number) in the pattern when
   the tune is saved.

   There are four palettes:

   \begin{itemize}
      \item \textbf{Pattern}.  This palette contains 32 color entries, and each
         can be used to add color to a pattern in the \textsl{Live} grid or in
         the \textsl{Song} editor.  The color of a pattern, if used, is saved
         with the pattern in the MIDI file.
      \item \textbf{Ui}.  This palette contains 16 color entries.  These
         color entries are used in drawing text, backgrounds, grid lines,
         background patterns, drum notes, and more.  These colors each have a
         counterpart that is used with the \texttt{-{}-inverse} option is applied
         to a run of \textsl{Seq66}.
      \item \textbf{Inverse Ui}.  This palette contains 16 color entries.
         These colors are used when the \texttt{-{}-inverse} option is applied
         to a run of \textsl{Seq66}.
      \item \textbf{Brushes}.  This "palette" provides a way to specify the
         fill type for the drawing of notes, the scale (if shown) in the
         pattern editor, and the background sequence (if shown).  It allows the
         user to select solid file, hatching, and some other fill patterns.
   \end{itemize}

   All palettes have default values built into the application.  However, the
   user can also include 'palette' files to change the colors used.  For
   example, the normal colored palette can be changed to a gray-scale palette.
   The name of the palette file is specified in the 'rc' file by lines like the
   following:

   \begin{verbatim}
      [palette-file]
      1     # palette_active
      qseq66-alt-gray.palette
   \end{verbatim}

   If this palette file is active, it is loaded, changing all of the palettes,
   and thus the coloring of \textsl{Seq64}.

\subsection{Palettes Setup}
\label{subsec:palettes_setup}

   The palette file is a standard \textsl{Seq66} configuration file with a name
   something like \texttt{qseq66.palette}, plus two sections:

   \begin{verbatim}
      [palette]
      [ui-palette]
   \end{verbatim}

   The first section is the "Pattern" palette, and the second section is the
   "Ui" palette, which includes the inverse palette as well.

\subsubsection{Palettes Setup / Pattern}
\label{subsubsec:palettes_setup_pattern}

   The following shows the pattern palette, with some entries elided for
   brevity:

   \begin{verbatim}
      [palette]
       0            "Black" [ 0xFF000000 ]      "White" [ 0xFFFFFFFF ]
       1              "Red" [ 0xFFFF0000 ]      "White" [ 0xFFFFFFFF ]
       2            "Green" [ 0xFF008000 ]      "White" [ 0xFFFFFFFF ]
       3           "Yellow" [ 0xFFFFFF00 ]      "Black" [ 0xFF000000 ]
       4             "Blue" [ 0xFF0000FF ]      "White" [ 0xFFFFFFFF ]
       ...            ...       ...             ...       ...
      29      "Dark Violet" [ 0xFF9400D3 ]      "Black" [ 0xFF000000 ]
      30       "Light Grey" [ 0xFF778899 ]      "Black" [ 0xFF000000 ]
      31        "Dark Grey" [ 0xFF2F4F4F ]      "Black" [ 0xFF000000 ]
project.
   \end{verbatim}

   The names are color names, and these names are what show up in the popup
   color menus for the pattern buttons in the \textsl{Live} grid.
   The colors on the left are the background colors, and the colors on the
   right are the foreground colors, which are chosen for contrast with the
   background.  The colors are in \texttt{\#AARRGGB} format, with the "\#"
   replaced by "0x" because "\#" starts a comment in \textsl{Seq66}
   configuration files.  Note that all the alpha values are "FF" (opqque); we
   have not yet experimented with changing them.
   Lastly, only 32 entries are accepted.

\subsubsection{Palettes Setup / Ui and Inverse Ui}
\label{subsubsec:palettes_setup_ui}

   The following shows the pattern palette, with some entries elided for
   brevity:

   \begin{verbatim}
      [ui-palette]
       0       "Foreground" [ 0xFF000000 ] "Foreground" [ 0xFFFFFFFF ]
       1       "Background" [ 0xFFFFFFFF ] "Background" [ 0xFF000000 ]
       2            "Label" [ 0xFF000000 ]      "Label" [ 0xFFFFFFFF ]
       3        "Selection" [ 0xFFFFA500 ]  "Selection" [ 0xFFFF00FF ]
       4             "Drum" [ 0xFFFF0000 ]       "Drum" [ 0xFF000080 ]
             ...            ...       ...       ...       ...
      13        "Beat Line" [ 0xFF2F4F4F ]  "Beat Line" [ 0xFF2F4F4F ]
      14        "Step Line" [ 0xFF778899 ]  "Step Line" [ 0xFF808080 ]
      15            "Extra" [ 0xFF778899 ]      "Extra" [ 0xFFBD6BB7 ]
   \end{verbatim}

   Here, the names are feature names, not color names.  The first color is the
   normal color, and the second color is the inverse color.  Only 16 entries
   are accepted.

\subsubsection{Palettes Setup / Brushes}
\label{subsubsec:palettes_setup_brushes}

   The last palette is small, allowing the fill-pattern of a few pattern-editor
   items to be changed.

   \begin{verbatim}
      [brushes]
      empty = nobrush
      note = solid
      scale = dense3
      backseq = dense2
   \end{verbatim}

   On the left of the equals sign is the item than can be filled, and on the
   right side is the \textsl{Qt} brush to be used.  The defaults for most are
   solid fill.

   The entry \texttt{empty} isn't used yet.
   The entry \texttt{note} affects the fill of normal/selected notes.
   The entry \texttt{note} affects the fill for the piano roll scale.  The
   hatching used here makes it easier to recognize that the scale is just there
   for orientation.
   The entry \texttt{note} affects the fill of the background sequence.  The
   hatching used here helps further distinguish the real notes from the
   background notes.

\subsection{Palettes Summary}
\label{subsec:palettes_summary}

   There are some obvious enhancements to this scheme, including increasing the
   number of palette items, synchronizing the palette with the current desktop
   them semi-automatically, and providing a user interface to drag-and-drop
   colors.

%-------------------------------------------------------------------------------
% vim: ts=3 sw=3 et ft=tex
%-------------------------------------------------------------------------------
