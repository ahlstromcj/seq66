%-------------------------------------------------------------------------------
% pattern_editor
%-------------------------------------------------------------------------------
%
% \file        pattern_editor.tex
% \library     Documents
% \author      Chris Ahlstrom
% \date        2015-08-31
% \update      2021-01-07
% \version     $Revision$
% \license     $XPC_GPL_LICENSE$
%
%-------------------------------------------------------------------------------

\section{Pattern Editor}
\label{sec:pattern_editor}

   The \textsl{Seq66} \textbf{Pattern Editor} can edit and preview a
   pattern, configure its buss, channel, transpose, musical
   scale, and many other settings.
   A rudimentary version of the \textbf{Pattern Editor} appears in the
   \textbf{Edit} tab in the main window, and a more powerful version can be
   brought up in an external window.

\begin{figure}[H]
   \centering 
   \includegraphics[scale=0.75]{pattern-editor/pattern-edit-window.png}
   \caption{External Pattern Editor Window}
   \label{fig:pattern_editor_window}
\end{figure}
  
   The \textbf{Pattern Editor} is complex, and we will discuss the external
   window only since its features are a superset of the \textbf{Edit} tab.
   For exposition, we break the window into the following sections:

   \begin{enumber}
      \item \textbf{First Row}
      \item \textbf{Second Row}
      \item \textbf{Time}
      \item \textbf{Piano Roll}
      \item \textbf{Events}
      \item \textbf{Data View}
      \item \textbf{Bottom Row}
      \item \textbf{Common Actions}
   \end{enumber}

   Before we describe this window, there are some things to recognize.
   First, if the pattern is empty when play is started, the progress bar will
   still move, so that the user can play a MIDI instrument and record new notes.
   Second, to add a note with the mouse, one must press the \textsl{right}
   mouse button (the pointer changes to a pencil) and,
   \textsl{while holding it}, press the left mouse button.
   Or click in the pattern editor, press the
   \index{keys!p}
   \texttt{p} key to select the "pencil" or "paint" mode, then
   \index{mouse!left-click}
   left-click to add a note or
   \index{mouse!left-click-drag}
   left-click-drag to add multiple notes as the mouse moves.
   \index{keys!x}
   Press or release the right mouse button, or press
   \texttt{x} to "eXit" or "eXscape" from paint mode.
   Another option is to press the "hand" button to toggle between note-entry
   and note-selection.
   Third, notes are drawn only with the length selected by the "notes" button
   near the top of the pattern window.  There are tricks to
   modifying the new notes that are described later.

   \textsl{Seq66} automatically scrolls
   horizontally through the sequence/pattern editor window when
   playback moves the progress bar outside of the current frame of data.  This
   feature makes it easier to follow patterns that are longer than a measure or
   two.

   One might want to print out the following figure to follow along.  There is
   a lot of functionality in this window.

\begin{figure}[H]
   \centering 
   \includegraphics[scale=0.95]{pattern-editor/pattern-edit-window-annotated.png}
   \caption{Pattern Editor Window (Annotated)}
   \label{fig:pattern_editor_window_annotated}
\end{figure}

\subsection{Pattern Editor / First Row}
\label{subsec:pattern_editor_first_row}

   The top bar (horizontal panel) of the Pattern (sequence) Editor
   lets one change the name of
   the pattern/loop/sequence/track, the time signature of the piece, how long
   the track is, and some other configuration items.

   \begin{enumber}
      \item \textbf{Track Number}
      \item \textbf{Track Title}
      \item \textbf{Beats Per Bar Reset} and \textbf{Beats Per Bar}
      \item \textbf{Beat Width Reset} and \textbf{Beat Width}
      \item \textbf{Pattern Length Reset} and \textbf{Pattern Length}
      \item \textbf{Chord Types}
      \item \textbf{Buss Reset} and \textbf{Buss Selection}
      \item \textbf{Channel Reset} and \textbf{Channel Selection}
   \end{enumber}

   \setcounter{ItemCounter}{0}      % Reset the ItemCounter for this list.

   \itempar{Track Number}{pattern editor!number}
   This item shows the sequence/track/pattern/loop
   number, to make it easier to pick it out when a lot of patterns are being
   edited at once.

   \itempar{Track Name}{pattern editor!name}
   Provides the name of the pattern.
   This name should be short and memorable.
   It is displayed in the \textbf{Live Grid} (the \textbf{Patterns Panel}),
   on the top line of its pattern slot.

   \itempar{Beats Per Bar}{pattern editor!beats/bar}
   \index{beats per bar}
   Specifies the number of beat units per bar in the time signature.
   The possible values range from 1 to 16, if the drop-down menu is used.
   Arbitrary values up to 32 can be entered by typing the number.
   The "Reset" button resets the value to 4.

   \itempar{Beat Width}{pattern editor!beat width}
   \index{beat width}
   Specifies the size of the bottom beat unit of the time signature:
   1 for whole notes; 2 for half notes; 4 for quarter notes; 8 for eight notes;
   16 for sixteenth notes; and 32 for thirty-second notes.
   The whole time signature is display at the bottom center of the
   corresponding pattern slot in the \textbf{Live Grid}.
   Arbitrary values up to 32 can be entered by typing the number.
   The "Reset" button resets the value to 4.

   \itempar{Pattern Length}{pattern editor!length}
   Sets the length of the current pattern, in measures.
   The possible values range from 1 to 64.
   Arbitrary values up to 1024 can be entered by typing the number.
   \textsl{However}, when opening or importing a non-\textsl{Seq66}
   MIDI tune, the length of each track will be used, and so other values
   are possible.

   Bringing up a pattern less than one measure or bar in
   length in the pattern editor will adjust the pattern to pad it to the
   length of one measure.
   \index{pattern editor!progress bar}
   \textsl{Seq66} will, when it reads such a short pattern
   from a MIDI file,

   A feature from user \textsl{stazed} allows the pattern to expand
   indefinitely while the user inputs MIDI from a controller, via the
   \textbf{Expand} option of the \textbf{Loop Record Type}.

%  See \sectionref{subsec:pattern_editor_bottom} below.

%  WHAT IS THIS? A feature where a short loop plays only once.
%
%  (Also nice would be a "one-shot"
%  pattern, useful for live intro patterns, for example.)

   \itempar{Chord Types}{pattern editor!chord types}
   \index{chord generation}
   This setting allows one to select a chord type (e.g. "major" or "minor").
   When active, a note is treated like the base note of the selected chord
   type, and extra notes are generated to create that chord.
   The \textbf{Chord Generation Reset} button is at the left of the
   \textbf{Data View}.

   \itempar{Chord Generation}{pattern editor!chord generation}
   \index{chord generation}

   \itempar{MIDI Out Device (Buss)}{pattern editor!midi out device}
   This setting specifies a virtual MIDI output buss or a
   MIDI output device set up by the computer and
   attached MIDI equipment.
   The button resets it to buss 0.
   Note that, if the pattern's selected buss is not found, this entry will be
   blank.  The user must select a valid buss from this dropdown.

%  The settings look a lot like
%  \figureref{fig:pattern_window_right_click_midi_bus}.

   \itempar{MIDI Out Channel}{pattern editor!midi out channel}
   This settings select the MIDI output channel.
   The possible values range from 1 to 16.
   If instruments are assigned in the 'usr' configuration file
   to that device and channel, their names will be shown in the dropdown.

\subsection{Pattern Editor / Second Row}
\label{subsec:pattern_editor_second_row}

   The second horizontal panel of the Pattern Editor provides a number
   of additional settings and functions:

   \begin{enumber}
      \item \textbf{Undo}
      \item \textbf{Redo}
      \item \textbf{Quantize Selection}
      \item \textbf{Tools Popup}
      \item \textbf{Follow Progress}
      \item \textbf{Reset Snap} and \textbf{Grid Snap}
      \item \textbf{Note Length Reset} and \textbf{Note Length}
      \item \textbf{Zoom Reset} and \textbf{Zoom}
      \item \textbf{Key Reset} and \textbf{Key of Sequence}
      \item \textbf{Scale Reset} and \textbf{Musical Scale}
      \item \textbf{Background Sequence}
   \end{enumber}

   \setcounter{ItemCounter}{0}      % Reset the ItemCounter for this list.

   \itempar{Undo}{pattern editor!undo}
   The \textbf{Undo} button rolls back any changes to the pattern from this
   session.  It will roll back one change each time pressed.
   \index{keys!ctrl-z}
   Pressing \texttt{Ctrl-Z} is the same as using the \textbf{Undo} button.

   \itempar{Redo}{pattern editor!redo}
   The \textbf{Redo} button will restore any undone changes to the pattern from
   this session.
   It will restore one change each time it is pressed.
   There is currently no redo key.

   \itempar{Quantize Selection}{pattern editor!quantize}
   This button quantizes the selected events as per
   the \textbf{Grid Snap} setting.

   \itempar{Tools}{pattern editor!tools}
   This button brings up a nested menu of tools for modifying selected
   events and notes:

   \begin{enumber}
      \item \textbf{Select}.  This menu
      provides two sets of selections for notes:
      \begin{itemize}
         \item \textbf{All Notes}, which selects all notes in the pattern;
            The \index{keys!ctrl-a} \texttt{Ctrl-A} will also select
            all of the events in the pattern editor.
         \item \textbf{Inverse Notes}, which inverts the selection of notes.
      \end{itemize}
      \item \textbf{Modify Time}. This menu
      offers two ways to tweak the timing of the selected note:
      \begin{itemize}
         \item \textbf{Quantize Selected Notes}
            \index{quantize}
            quantizes the selected notes, the same way as the
            \textbf{Quantize} ("\textbf{Q}") button.
         \item \textbf{Tighten Selected Notes},
            \index{tighten}
            which is merely a less strict form of quantization.
         \item \textbf{Modify Pitch} has only one entry by default,
            \index{modify pitch}
            \textbf{Transpose Selected}.
            Selecting the \textbf{Transpose Selected} entry brings up a sub-menu.
            If the user has selected a
            \textbf{Musical Scale} setting other than \textbf{Off},
            then \textbf{Modify Pitch} has two entries:
            \begin{itemize}
               \item \textbf{Transpose Selected}, discussed above, plus
                  another sub-menu,
               \item \textbf{Harmonic Transpose Selected}, which makes sure
                  that all transpositions stay on the selected scale.
                  Again, the harmonic-transpose option will not be
                  available unless a scale has been selected.
            \end{itemize}
      \end{itemize}
   \end{enumber}

   \itempar{Follow Progress}{pattern editor!tools}
   This button toggles whether or not the progress bar follows
   progress in long patterns.  Turning off this feature is useful when
   one wants to concentrate on the current measure without the paging to
   subsequent measures that occurs with the "follow progess" feature.

   \itempar{Grid Snap}{pattern editor!grid snap}
   Grid snap selects where the notes will snap when drawn.
   That is, it selects the snap-spacing for the notes
   The following values are supported:
   \textbf{1}, \textbf{1/2}, \textbf{1/4}, \textbf{1/8},
   \textbf{1/16} (\textsl{the default value}),
   \textbf{1/32}, \textbf{1/64}, and \textbf{1/128}.
   Additional values are also supported:
   \textbf{1/3}, \textbf{1/6}, \textbf{1/12}, \textbf{1/24},
   \textbf{1/48}, \textbf{1/96}, and \textbf{1/192}.
   The button to the left of this control resets it to the default value.

   \itempar{Note Length}{pattern editor!note length}
   Note length determines the duration of inserted notes.
   Like the \textbf{Grid Snap} values,
   the following values are supported:
   \textbf{1}, \textbf{1/2}, \textbf{1/4}, \textbf{1/8},
   \textbf{1/16} (\textsl{the default value}),
   \textbf{1/32}, \textbf{1/64}, and \textbf{1/128}.
   Additional values are also supported:
   \textbf{1/3}, \textbf{1/6}, \textbf{1/12}, \textbf{1/24},
   \textbf{1/48}, \textbf{1/96}, and \textbf{1/192}.
   The button to the left of this control resets it to the default value.

   \itempar{Zoom}{pattern editor!zoom}
   Horizontal zoom is the ratio between MIDI pixels and ticks, written as
   "pixels:ticks", where "ticks" is the "pulses" in "PPQN".
   For example, 1:4 = 4 ticks per pixel.
   Supported values are
   \textbf{1:1}, \textbf{1:2} (\textsl{the default value}),
   \textbf{1:4}, \textbf{1:8}, \textbf{1:16},
   and \textbf{1:32}, along with
   more values to support higher PPQN tunes:
   \textbf{1:64}, \textbf{1:128}, \textbf{1:256}, and \textbf{1:512}.
   The default zoom is 2 for the standard PPQN value, 192, but it
   increases for higher PPQN values, so that the default zoom looks sensible.
   As the right number (ticks) goes higher,
   the effect is to zoom out, and show more of the pattern.
%  Also see \sectionref{subsubsec:pattern_editor_zoom_keys}.

   \itempar{Key of Sequence}{pattern editor!key}
   Selects the desired musical key for the pattern.  The following keys are
   supported:
   \textbf{C}, \textbf{C\#},
   \textbf{D}, \textbf{D\#},
   \textbf{E}, \textbf{F}, \textbf{F\#},
   \textbf{G}, \textbf{G\#},
   \textbf{A}, \textbf{A\#},
   and \textbf{B}.
   Changing the key shifts the marked note-rows
   for the \textbf{Musical Scale} setting and indicates the base notes
   of the key in a \textbf{bold} font.
   The small key button resets the key to \textbf{C}.

   \index{save musical key}
   The musical key that a sequence/pattern is set to is
   saved in the MIDI file along with the rest of the data for the sequence.
   \textbf{However},
   a change made to the key, scale, or background sequence in
   the pattern editor can be saved in the whole song,
   so that opening another sequence
   will apply the same settings to that sequence.  This is an optional feature,
   supported as noted below.

   \index{global-sequence}
   If the global-sequence feature is enabled, and the user selects
   a different key, scale, or background sequence in the pattern editor, 
   then \textsl{all} patterns share the selected key, scale, or background
   sequence.  Furthermore, these settings are saved in the "proprietary"
   section of the MIDI file, where they are available for all patterns.

   If the global-sequence feature is \textsl{not} enabled, and the user selects
   a different key, scale, or background sequence in the pattern editor, 
   then only that pattern will use the selected key, scale, or background.
   The key, scale, or background sequence change will be saved in the MIDI file
   only for that pattern, as a SeqSpec meta event.
   The global-sequence feature setting can be made in the 'usr' configuration
   file.

   \itempar{Musical Scale}{pattern editor!scale}
   Selects the desired background scale for the pattern; it provides a way for
   someone to key in notes that are only in that scale.
   When a scale is selected, the following features are supported:

   \begin{itemize}
      \item The notes that are \textsl{not}
         in the scale are shown as grey in the piano roll.
      \item For harmonic transposition, the notes are shifted
         so that they remain in the selected scale.
      \item The exact notes that are considered "in-scale" shift according
         to the value of the selected \textbf{Key of Sequence}.
   \end{itemize}

   \index{musical scales}
   The following musical scales are supported:

   \begin{itemize}
      \item \textbf{Off (Chromatic)}
      \item \textbf{Major (Ionian)}
      \item \textbf{Minor (Aeolian)}
      \item \textbf{Harmonic Minor}
      \item \textbf{Melodic Minor}
      \item \textbf{Whole Tone}
      \item \textbf{Blues}
      \item \textbf{Major Pentatonic}
      \item \textbf{Minor Pentatonic}
   \end{itemize}

   Please let us know of any mistakes in the new scales.
   Note that the \textbf{Melodic Minor} scale is supposed to
   descend in the same way as the natural \textbf{Minor} scale, but
   there is no way to support that trick in \textsl{Seq66}.

   One can select which \textbf{Musical Scale} and
   \textbf{Key} the piece is in nominally,
   and \textsl{Seq66} will grey those keys on the piano-roll that
   are \textsl{not} in the selected scale for the selected key.
   This is purely visual; a user can still add off-key notes.
   This feature makes it easier to stay in key while playing and recording.
   The scale will shift when a different \textbf{Key} is selected.

   \index{save musical scale}
   The scale that a pattern is set to is
   saved in the MIDI file along with the rest of the data for the pattern.
   A change made to the key, scale, or background pattern in
   the pattern editor can be saved globally, so that opening another pattern
   apply the same settings to that pattern.  This is a configurable feature in
   the 'usr' file; see "global\_seq\_feature".
   This option allows applying the key/scale/background-sequence
   either globally (all patterns) or locally (per-pattern), with each pattern
   holding its key, scale, and background-sequence settings in
   SeqSpec meta events.

   \itempar{Background Sequence}{pattern editor!background sequence}
   One can select another pattern to draw on the background to help with
   writing corresponding parts.
   The button brings up a small menu with values of \textbf{Off} and
   \textbf{Set 0} (at a minimum).
   The 0 is a set number; sets are numbered from 0 to 31.
   Additional set numbers appear in the menu for each set that has data in it.
   Under the \textbf{Set 0} entry, a menu appears.
   Once the desired pattern is selected from that list, it appears as dark cyan
   note bars, along with the normal notes that are part of the pattern.

   \index{save background sequence}
   The background sequence that shows is saved in the MIDI file
   along with the rest of the data for the sequence/pattern.
   A change made to the key, scale, or background sequence in
   the pattern editor is saved in the editor, so that opening another sequence
   will apply the same settings to that sequence.
   This is an optional feature, as noted earlier.

   \itempar{Chord Generation}{pattern editor!chord generation}
   One can insert chords with one click.
   (This feature comes from user "stazed"
   and his \textsl{Seq32} project \cite{seq32}.)
   Select the desired chord type first.
   Once a value other than \textbf{Off} is selected,
   drawing mode will add multiple notes representing the chord
   created, with the clicked note value as the base of the chord.

\subsection{Pattern Editor / Piano Roll}
\label{subsec:pattern_editor_piano_roll}

   The piano roll is the center of the pattern/loop/track/sequence editor.
   It is accompanied by a thin "event bar" ("event area", "event strip")
   just below it,
   and a taller "data bar" or "data area" just below that.
   While the pattern
   editor is very similar to note editors in other sequencers, it is a bit
   different in feel.  A good mouse with at least 3 buttons is very helpful
   for editing.  Buttons and keystrokes support enhanced editing.

   One can page vertically in the piano roll using the
   \index{keys!page-up} \texttt{Page Up} and 
   \index{keys!page-down} \texttt{Page Down} keys.
%  One can go to the top using the 
%  \index{keys!home} Home key, and
%  to the bottom using the
%  \index{keys!end} End key.
   One can go to the leftmost position using the 
   \index{keys!ctrl-home} \texttt{Ctrl-Home} key,
   and to the rightmost position using the
   \index{keys!ctrl-end} \texttt{Ctrl-End} key,

   \index{step}
   \index{note step}
   With the note-step feature, if one paints notes with the mouse,
   the note position advances with each click.
   If one paints notes via an external MIDI keyboard, the notes are painted and
   advanced.
   To preview notes entered via a MIDI device, click the
   \textbf{pass MIDI in to output} button to activate so that they will be
   passed to the sound generator or software synthesizer.

\subsubsection{Pattern Editor / Piano Roll Items}
\label{subsubsec:pattern_editor_piano_roll_items}

   The center of the pattern editor consists of a time panel at the top,
   a virtual keyboard at the left, a note grid, a vertical scrollbar, an event
   panel, and a data panel at the bottom.

   \begin{enumber}
      \item \textbf{Beat}
      \item \textbf{Measure}
      \item \textbf{Virtual Piano Keyboard}
      \item \textbf{Notes}
   \end{enumber}

   \setcounter{ItemCounter}{0}      % Reset the ItemCounter for this list.

   \itempar{Beat}{piano roll!beat}
   The light vertical lines represent the beats defined by the configuration
   for the pattern.  The even lighter dotted lines between the beats are useful
   for snapping notes.

   \itempar{Measure}{piano roll!measure}
   The heavy vertical lines represent the measures (bars) defined by the
   configuration for the pattern.
   \index{pattern!end marker}
   Also note that the end of the pattern
   occurs at the end of a measure, and is marked by a blocky \textbf{END}
   marker.

   \itempar{Virtual Piano Keyboard}{piano roll!virtual piano keyboard}
   The virtual keyboard is a fairly powerful interface.  It shows,
   by shadowing, which note on the keyboard will be drawn. It can be
   played with a mouse, using left-clicks, to preview a short motif.
%  It can show marks to indicate off-scale notes, to make them easy to
%  avoid.
   Every octave, a note letter and octave number are shown, as in
   "C4".  If there is a difference scale in force, then the letter changes to
   match, as in "F\#5".

   \index{virtual keyboard!right-click}
   A right-click in the virtual keyboard area toggles the display
   between octave-note letters, MIDI note-numbers, and other views.
   The following figure shows all views, superimposed for comparison.

\begin{figure}[H]
   \centering 
   \includegraphics[scale=1.00]{pattern-editor/pattern-edit-window-key-numbers.png}
   \caption{Virtual Keyboard Number and Note Views}
   \label{fig:pattern_editor_key_numbers}
\end{figure}

   \itempar{Notes}{piano roll!notes}
   Musical notes are indicated in the piano roll
   by thick horizontal bars with white
   centers.  Each bar provides
   a visual representation of the pitch of a note and the length of a note.
   The current scale and background pattern can also be shown in the piano
   roll.

\subsubsection{Pattern Editor / Note Painting}
\label{subsubsec:pattern_editor_note_painting}

   When we say "editing" in the context of the piano roll, in part we mean that
   we will "draw"
   \index{draw mode}
   \index{mode!draw}
   \index{paint mode}
   \index{mode!paint}
   or "paint" notes.
   Drawing, modifying, copying, and deleting notes is easy in \textsl{Seq66}.

   The \textsl{Seq24} note-editing style is as expected for basic
   actions such as selecting and moving notes using the left mouse button.
   Drawing a note or event is a bit different, in that one must first
   enter the drawing mode ("paint mode").
   One way is to \textsl{click and hold} the right mouse button, and then
   \textsl{click and drag} the mouse to insert notes.
   Note that some \textsl{Seq66} windows
   can use the \textsl{Ctrl-left-click} as a middle click. 
   \index{keys!p}
   Another way is to use the \texttt{p} key to enter the "paint" mode.
   To get out of the "paint" mode, press the
   \index{keys!x}
   \texttt{x} key while in the sequence editor.
   Also available is a "finger" button
   (\textbf{Note Select/Note Entry})
   to click to toggle the mode.

   \index{notes!inserting}
   Notes are inserted to be at the current length and grid-snap values for
   the sequence editor for as long as the buttons are pressed while the mouse
   is dragged.
   The length of the note will
   be that specified in the note-length setting (e.g. "1/16").
   \index{auto-note}
   This is the "auto-note" feature.
   The auto-note feature also works with chord-generation.
   Notes are inserted only up to the specified sequence length.
   Once notes are inserted, moving the mouse with the left button still
   held down moves the notes to the new note value of the mouse.
   If one releases the left button, then presses and holds it again,
   more notes will be added in the same way.
   This is a good way to layer notes in a short sequence.
   The draw mode has the following features:

   \begin{itemize}
      \item Notes are continually added as the mouse is dragged ("auto-notes").
      \item Notes cannot be added past the "END" marker of the pattern, which
         marks the \textbf{Sequence Length in bars} setting.
      \item As the mouse is dragged while the left button is held in draw mode,
         notes are either added, or, if already present at that note-on time,
         are moved up and down.
      \item If the draw mode is exited, and entered again, then the original
         notes will not be altered.  Instead, new ones will be added.
      \item Notes can be added while the pattern is playing, and will be heard
         the next time the progress bar passes over them.
   \end{itemize}

   Drawing/painting can also be done while the sequence is playing,
   and notes will be added to be played the next time the progress bar crosses
   them.

%  \index{keys!Mod4}
%  \index{mouse!Mod4}
%  \label{new_mod4_mode}
%  \textbf{Mod4 key preserves add (paint) mode in song and pattern editors}.
%  \index{mouse!split mode}
%  \label{new_split_mode}
%  Middle click splits song triggers at nearest snap (instead of
%  the halfway point).
%  Move this section to the right place and simply create a section-reference to
%  it here.

%  The \texttt{p} and \texttt{x} keys also works in the small event strip just
%  above the white data area.

\subsubsection{Pattern Editor / Note Editing}
\label{subsubsec:pattern_editor_note_editing}

   Once notes are in place, whether by recording or using "paint" mode,
   the piano roll provides a sophisticated set of note-editing
   actions.

   \setcounter{ItemCounter}{0}      % Reset the ItemCounter for this list.

   \itempar{Event Selection}{event!select}
   There are various ways to select events using
   the mouse or the keyboard in the piano roll:

   \begin{itemize}
      \item
         \index{keys!ctrl-a}
         \index{selection!all}
         \textbf{\texttt{Ctrl-A}}.
         Pressing the \texttt{Ctrl-A} key will select all of the events in the
         pattern editor.
      \item
         \index{mouse!left-click}
         \index{pattern editor!left click}
         \index{pattern editor!select note}
         \textbf{Left Click}.
         Pressing the left button on a note or a event deselects all other
         notes or events, and selects the item clicked on.
         The selected note will turn orange (or the configured palette color).
      \item
         \index{mouse!left-click-drag}
         \index{pattern editor!select multiple notes}
         \textbf{Left Click Drag}.
         Pressing the left mouse button and dragging also lets one
         select ("lasso") multiple events and notes.
         The selected notes will turn orange.
         Adjustments can be made to one or more notes by selecting one or more
         notes, and then applying one or more special
         \index{selection action}
         "selection actions" to the selection.
         Be careful!  If you \texttt{Ctrl-left-click-drag}
         on an already-selected note,
         the drag will change the length of
         \textsl{all of the notes in the selection}.
      \item \index{mouse!ctrl-left-click}
         \textbf{Ctrl Left Click}.
         Pressing the \texttt{Ctrl} key and the left button on a note or an
         unselected event \textsl{adds} that event to the selection.
      \item
         \index{pattern editor!transpose notes}
         \textbf{Left Click Drag Selection Up/Down}.
         To move notes in pitch, once selected, grab one of the notes in the
         selection and drag it upward or downward.
         \index{down arrow}
         \index{up arrow}
         Also, when a selection is in force, the
         \texttt{Up} and \texttt{Down} arrow keys will
         change the pitch of every note in the selection.
         The smallest unit of pitch change is one MIDI note value.
      \item
         \index{pattern editor!move notes in time}
         \textbf{Left Click Drag Selection Left/Right}.
         To move notes in time, once selected, grab one of the notes in the
         selection and drag it leftward or rightward.
         \index{left arrow}
         \index{right arrow}
         Also, since a selection is in force, the
         \texttt{Left} and \texttt{Right} arrow keys can also
         be used to change the time of every note in the selection.
         The smallest unit of time change is the \textbf{Grid snap} value,
         which might be a 16th note, for example.
      \item
         \index{mouse!ctrl-left-click-drag}
         \textbf{Ctrl Left Click Drag}.
         \begin{itemize}
            \item Pressing the \texttt{Ctrl} while left-click-dragging
               \textsl{on unselected events} lets one make additional
               selections of multiple events and notes.
            \item Pressing the \texttt{Ctrl} while left-click-dragging
               \textsl{on an already-selected event} lets one stretch or
               compress the lengths of all notes in the selection.
%              Also achievable via a \textbf{Middle Click Drag}.
               \index{pattern editor!event stretch}
               \index{event!stretch}
               \index{pattern editor!event compression}
               \index{event!compression}
               This feature is called \textsl{event stretch}
               or \textsl{event compression}.
               Notes can be shortened below the default note length by event
               compression.  There is currently no way to change the length of
               the note using a keystroke.
         \end{itemize}
      \item \index{pattern editor!deselect notes} \index{selection!deselect}
         \textbf{Deselect Notes}.
         To deselect the notes, click somewhere else in the piano roll, and the
         notes should change back to white.  There is no way to deselect a
         single note, with, say, a \texttt{Shift-click} or
         \texttt{Ctrl-click} action.
   \end{itemize}

   \index{warning!wrap-around notes}
   \textbf{Warning}:  Reducing or increasing the length of a note selection
   by too much causes the note or notes to "wrap-around" to the end
   of the pattern boundary and grow more from the beginning of the sequence. 
   If it happens, one probably ought to undo it.

   The \textbf{Tools} button described in
   \sectionref{subsec:pattern_editor_second_row} can also be used to
   modify selections.
   Once one or more notes are selected, they can be modified in time,
   pitch, or length, as described above.

   \textbf{Warning:}
   \index{warning!down arrow}
   \index{warning!up arrow}
   \index{warning!note loss}
   If one moves the selection too low or too high in pitch, whether with the
   mouse or the arrow keys, any notes that go below the lowest MIDI pitch or
   above the highest MIDI pitch \textbf{will be lost}!
   If done using the mouse, the undo feature (\texttt{Ctrl-Z}) will work.
   If done using the arrow keys, the undo feature does not work!
   Be careful, especially if you have a fast keyboard repeat rate!

   Note that there is no possibility of note loss with a change in time.  When
   a note disappears at one end of the pattern boundary, it wraps around to the
   other end.  Cool.

   \itempar{Copy/Paste}{pattern editor!copy/paste}
   Copying, cutting, and pasting is supported by selecting a number of events
   or notes, and using the
   \index{pattern editor!cut}
   \index{keys!ctrl-x} Cut (\texttt{Ctrl-X}), 
   \index{pattern editor!copy}
   \index{keys!ctrl-c} Copy (\texttt{Ctrl-C}), and
   \index{pattern editor!paste}
   \index{keys!ctrl-v} Paste (\texttt{Ctrl-V})
   keys.
   When the notes are selected,
   \index{pattern editor!delete}
   \index{keys!del}
   \index{keys!backspace}
   one can delete them with the \texttt{Delete} or \texttt{Backspace} key.
   If the events are \textsl{cut}, using the \texttt{Ctrl-X} key, then
   they can be pasted, using the \texttt{Ctrl-V} key, then
   moving the cursor to the desired place, and clicking.

   One can move the selection box using the arrow keys, to the
   desired location, and then click to
   drop the notes at that location.
   Selected notes that are cut or copied can also be
   pasted into \textsl{other} pattern editor dialogs; that is, they can be
   pasted into other sequences.

\subsubsection{Pattern Editor / Zoom Keys}
\label{subsubsec:pattern_editor_zoom_keys}

   \index{zoom keys}
   \index{keys!0}
   \index{keys!z}
   \index{keys!shift-z}
   After a left-click in the piano roll, the
   \textbf{z}, \textbf{Z}, and \textbf{0}
   can be used to zoom the piano-roll view \textsl{horizontally}.
   The \textbf{z} key zooms out (smaller),
   the \textbf{Z} key zooms in (larger),
   and the \textbf{0} key resets the zoom to the default value.
   The horizontal zoom feature also affects the time-line
   (measures indicator) and the data area.

   \index{keys!v}
   \index{keys!shift-v}
   The note display can also be zoomed vertically.
   The \textbf{v} key zooms out vertically to make the notes thinner,
   the \textbf{V} key zooms in vertically to make the notes fatter,
   and the \textbf{0} key resets the zoom to the value of the "key height"
   setting in the 'usr' configuration file.
   
\subsection{Event Editor}
\label{subsec:pattern_editor_events}

   Also known as the "events pane" or "events panel".
   The narrow (a few pixels high) events strip shows discrete events,
   such as \texttt{Note On} and \texttt{Note Off}.
   We currently recommend not editing or selecting events
   in that pane (feel free to disobey), because \textbf{more work is needed}.

   \index{event strip}
   \textsl{Note On}, \textsl{Note Off}, and other events appear
   as small squares in the event strip, along with a black vertical bar
   in the \textbf{Data View} with a
   height proportional to the velocity of the note event, plus a numeric
   representation of that value.
   Note events should not be inserted in the event strip; it is too easy to
   screw up.

%  \index{warning!unterminated notes}
%  \textsl{They can be inserted there, but they end up as short
%  events of the lowest possible note, 0 or C1, and they don't have a Note
%  Off event.  Don't do that!})

\begin{comment}
   \index{events!insert}
   Other event types can be inserted via the event strip.
   To do that, first
   select the kind of event to insert using the \textbf{Event} button in the
   bottom panel.  The place the mouse cursor in the event strip.
   Right-click to make the drawing pencil appear at the exact spot where the
   event must go.  While holding the right button, click the left button.
   A small square for the event should appear.

   One can also left-click in that section,
   \index{keys!p}
   then hit the \texttt{p} key to go into "paint" mode,
   \index{keys!x}
   and hit the \texttt{x} key to escape that mode.

   Should one want more of the same event, continue to hold both buttons and
   drag the mouse.  One event should appear at each beat position (e.g. at
   each 16th note position) that is crossed.

   To move the event(s) to a different spot, select it or them via the left
   button.  Then drag it or them to where one wants them.
   \index{todo!high precision events}
   it is currently not possible to move them to positions smaller than the
   beat size.  The work-around is to temporarily reduce the beat size,
   but this requires caution.
\end{comment}

\subsection{Data View}
\label{subsec:pattern_editor_data_view}

   Once the event positions are set, the next step is to modify the
   data values of the events.  But first, note the buttons at the left.

   \begin{enumber}
      \item \textbf{Transpose}
      \item \textbf{Drum Note Mode}
      \item \textbf{Chord Generation Reset}
   \end{enumber}

   \setcounter{ItemCounter}{0}      % Reset the ItemCounter for this list.

   \itempar{Transpose}{pattern editor!transpose toggle}
   This button allows the sequence to be transposed by
   the global transpose selection made in the song editor.  If transpose is
   enabled for that pattern, the button will be highlighted as per the current
   desktop theme.  Patterns for drums should, in general, not be transposable.

   \itempar{Drum Note Mode}{pattern editor!drum mode}
   This button changes from normal note mode to drum note mode. In the drum
   mode, the notes are drawn as small red diamonds without any duration.
   They are also entered the same way.
   This is a feature adopted from \textsl{Kepler34}.

   \itempar{Chord Generation}{pattern editor!chord generation}
   \index{chord generation}
   This button resets the chord-generation feature to \textbf{Off}.
   It's located by the data pane in order to save space in the first row.

   Now on to the \textbf{Data View} itself.
   Also known as the "data pane"
   \index{data pane}
   or "data panel".
   \index{data panel}

   The events values for the currently selected category of events are shown
   in this window as vertical lines of a height proportional to the value.
   These values can be easily modified by
   \index{mouse!left-click-drag}
   left-click-dragging the
   mouse past each line, to chop it off at the given value.  Easier to try
   it than explain it.
   \index{mouse!right-click-drag}
   Right-click-drag also works the same.

   There are many things that can be done with selected data and events:

   \index{modify event-data}
   $\bullet$ \textbf{Modify Event Data} offers a way to
   alter the event data values in 
   the lower pane of the pattern editor, the "data pane".
   By left-dragging the mouse in the data pane across the value lines that are
   shown, the values are chopped or set to the height of the mouse pointer at
   each event.
   When notes are selected, and the
   mouse is used to change the values (heights) of the lines in the event-data
   area,
   \textsl{only the events that are selected} are changed.  The data-values of
   \textsl{unselected} events are left unchanged.
   A cool feature from \textsl{Seq24}.

   Once the event positions are set, the next step is to modify the
   data values of the events.
   \index{event data}
	The event value (data) editor (directly under the event strip) is used 
	to change note velocities, channel pressure, control codes,
	patch select, etc.
   \index{event data editor!draw}
   \index{event data editor!left click}
   \index{event data editor!right click}
   \index{event data editor!middle click}
   Just left-click+drag the mouse across the window to draw a line.  The
   values will match that line.  
   middle-click+drag and right-click+drag also
   draw the value line.

   \textbf{Bug:}
   \index{bugs!event editing can fail}
   Sometimes the editing of event values in the event data section will not work.
   The workaround is to do a \texttt{Ctrl-A}, and the click in the roll
   to deselect the selection; that makes the event value editing work again.
   
   \index{event data editor!mouse wheel}
   Any events that are selected in the piano roll or event strip can have
   their values modified with the mouse wheel.

   Data values can also be modified using the \textbf{LFO} pane.

\subsection{Pattern Editor / Bottom Panel}
\label{subsec:pattern_editor_bottom}

   The bottom horizontal panel of the Pattern Editor provides for
   selecting events for viewing and editing, MIDI playback,
   pass-through, and recording.

%\begin{figure}[H]
%   \centering 
%   \includegraphics[scale=0.50]{new/pattern-edit-bottom-panel.png}
%   \caption{Pattern Editor, Bottom Panel Items}
%   \label{fig:pattern_editor_bottom_panel_items}
%\end{figure}

   Missing from this diagram is the "Existing Event Selector" which zeroes
   in only on events already present in the pattern.

%  Until we can get a completely labelled screenshot, here is the
%  latest bottom panel, with the new \textbf{LFO}
%  and \textbf{Recording Type} buttons, and the popup menu for the latter.
%
%\begin{figure}[H]
%   \centering 
%   \includegraphics[scale=0.75]{new/lfo_and_rectype_buttons.png}
%   \caption{Pattern Editor, Additional Bottom Panel Items}
%   \label{fig:pattern_editor_added_bottom_panel_items}
%\end{figure}

   \begin{enumber}
      \item \textbf{Event Selector}
      \item \textbf{Existing Event Menu}
      \item \textbf{Event Selection}
      \item \textbf{LFO}
%     \item \textbf{Time Scroll}
      \item \textbf{Data To MIDI Buss}
      \item \textbf{MIDI Data Pass-Through}
      \item \textbf{Record MIDI Data}
      \item \textbf{Quantized Record}
      \item \textbf{Recording Type} (Merge, Replace, Expand)
      \item \textbf{Select Recording Volume}
   \end{enumber}

   \setcounter{ItemCounter}{0}      % Reset the ItemCounter for this list.

   \itempar{Event Selector}{pattern editor!event selector}
   This button brings up the following context menu, so that the user can
   select the category of events to view and edit.

%\begin{figure}[H]
%   \centering 
%  \includegraphics[scale=0.75]{pattern/event-context-menu.png}
%   \includegraphics[scale=0.65]{roll.png}
%   \caption{Pattern Editor, Event Button Context Menu}
%   \label{fig:pattern_editor_bottom_event_context_menu}
%\end{figure}

   Note the squares.  Some of filled (black), others are empty.  The filled
   squares indicate that the sequence does indeed have some events of that
   type.  Otherwise, there are no such events in the sequence.
   Useful in deciding if it is worth selecting the event.

   The sub-menus of this context menu show 128 controller messages,
   so we won't try to show all of them here.  They also use the squares to
   indicate if there are any events of the type shown in the menu.
   These sub-menus can be modified by editing the file
   
   \begin{verbatim}
      $HOME/.config/seq66/seq66.usr
   \end{verbatim}

   to make it match one's instrument.  See \sectionref{sec:usr_file}.

   \itempar{Existing Event Menu}{pattern editor!existing events}
   The existing-event selector is a small button (with a black-square icon)
   that brings up a menu with only existing events shown.
   Unlike the event-selector described above, this menu
   shows only the actual events existing in the track, for quicker selection.

   \itempar{Event Selection}{pattern editor!event selection}
   Shows the selection event, with its number shown in hexadecimal notation,
   and the name of the event shown.

   \itempar{LFO}{pattern editor!LFO}
   A low-frequency oscillator allows data events
   can be modulated by some rudimentary wave functions.
   By clicking on the \textbf{LFO} button or using the \texttt{Ctrl-L} key,
   the following window appears, shown as the set of 5 vertical sliders:

%\begin{figure}[H]
%   \centering 
%  \includegraphics[scale=0.65]{new/pattern_editor_with_LFO.png}
%   \includegraphics[scale=0.65]{roll.png}
%   \caption{Pattern Editor, LFO Support}
%   \label{fig:pattern_editor_bottom_lfo_support}
%\end{figure}

%  Note the controls in this window:

   \begin{enumber}
      \item \textbf{Value}:
         Provides a kind of DC offset for the data value. Starts at 64, and
         ranges from 1 to 127.
      \item \textbf{Range}:
         Controls the depth of modulation. Starts at 64, and ranges from 1 to
         127.
      \item \textbf{Speed}:
         Indicates the number of periods per pattern (divided by beat width,
         normally 4).  For long patterns, this parameter needs to be set high,
         to even show an effect.  It is also subject to an 'anti-aliasing'
         effect in some parts of the range, especially for short patterns.
         Try it!  For short patterns, try a value of 1 at first.  For a pattern
         of one measure in length, this will create four periods of the wave.
      \item \textbf{Phase}:
         Provides the phase shift within a period of the LFO wave.
         A value of 1 is a phase shift of 360 degrees (or maybe it is one
         radian?).
      \item \textbf{Wave Type}:
         Selects the kind of wave to use for the LFO:
         \begin{enumber}
            \item \textbf{Sine wave}.
            \item \textbf{Ramp (up) sawtooth}.
            \item \textbf{Decay (down) sawtooth}.
            \item \textbf{Triangle wave}.
         \end{enumber}
   \end{enumber}

   We may have more to explain about this dialog at some point.  For now,
   try it out on the file \texttt{one-measure.midi}, and be sure to hover over
   each control to see the tooltips.
   Note that it works best with short patterns.

   \itempar{Time Scroll}{pattern editor!time scroll}
   Allows one to pan through the whole pattern, if it is too long to fit in
   the window horizontally.

   \itempar{Data To MIDI Buss}{pattern editor!data to midi buss}
   This button causes the pattern to be output to the
   selected MIDI output buss,
   which will normally be connected to a software or hardware
   synthesizer, to be heard.
   Generally, this control should always be activated.

   \itempar{MIDI Data Pass-Through}{pattern editor!midi data pass-through}
   This button routes incoming MIDI data through
   \textsl{Seq66}, which then writes it to the MIDI output buss.

   \itempar{Record MIDI Data}{pattern editor!record midi data}
   This button routes incoming MIDI data into
   \textsl{Seq66}, which then saves the data to its buffer, and also
   displays the new information (notes) in the piano roll view.

   \itempar{Quantized Record}{pattern editor!quantized record}
   This button will causes MIDI data to be recorded, but be
   quantized on the fly before recording it.
   The quantization is to the current snap value.

   \itempar{Recording Type}{pattern editor!recording type}
   In \textsl{Seq24}, the pattern recording worked by merging new notes played
   as the pattern to be recorded was looped.  This method allows a loop to be
   built up bit-by-bit.  \textsl{Seq66} adds two more methods from
   Stazed's \textsl{Seq32} project.  The three methods are:

   \begin{enumber}
      \item \textbf{Merge}. \index{merge}
         \index{recording type!merge}
         This is the "legacy" style of recording loops, where notes can
         accumulate.
      \item \textbf{Replace.}
      \index{replace}
      \index{recording type!replace}
         When the loop starts over, and a note is pressed,
         then the existing notes in that loop are erased,
         and the new note is added.
         This provides a good way of correcting major mistakes, live.
         It will not work if adding notes while not recording.
         This mode can cause incomplete notes if one
         holds the note and releases it in the next iteration, leaving a
         partially-drawn note behind.  The workaround is to try again.
      \item \textbf{Expand}.
         \index{expand}
         \index{recording type!expand}
         Once the end of the loop is near, whether or
         not any notes are being input, another measure is added to the length of
         the loop.  This continues indefinitely, whether or not any notes are
         being played/recorded.
   \end{enumber}

   \itempar{Vol}{pattern editor!vol}
   This button allows setting the volume of the recording.
   The velocity of the notes will be set to the selected value upon recording.
   If the \textbf{Free} item is selected, then the incoming note velocity is
   preserved.

%\begin{figure}[H]
%   \centering 
%  \includegraphics[scale=0.75]{pattern/vol-context-menu-new.png}
%   \includegraphics[scale=0.65]{roll.png}
%   \caption{Pattern Recording Volume Menu}
%   \label{fig:pattern_edit_recording_volume_menu}
%\end{figure}

   The velocity values are shown at the right side of each menu entry.
   These values correspond to MIDI volume levels from 127 down to 16, as
   shown in the figure.

%  One thing fixed in the 0.90.x version is the ability to store MIDI note-on
%  events with the actual velocity provided by the MIDI keyboard used to
%  generate the notes.  Previously, even in \textsl{seq24},
%  the \textbf{Free} option in the \textbf{Vol} menu option did not work.
%  This is fixed.

\subsection{Pattern Editor / Common Actions}
\label{subsec:pattern_editor_common}

   This section is a catch-all for actions not described above.

\subsubsection{Pattern Editor / Common Actions / Scrolling}
\label{subsec:pattern_editor_scrolling}

   Let us describe the actions that can be performed with a
   scroll wheel, or with the scrolling features of multi-touch touchpads.
   There are three major scrolling actions available when using mouse
   scrolling, with the mouse hovering in the piano-roll area:

   \begin{itemize}
      \item \textbf{Vertical Panning (Notes Panning)}
         \index{scroll!normal scroll}
         \index{scroll!vertical pan}
         \index{scroll!notes pan}
         \index{pan!seqroll notes}
         Using the vertical scroll action of a mouse or touchpad moves the
         view of the sequence/pattern notes up and down.
         One can also click in the piano roll, and then use the
         \texttt{Page-Up} \index{keys!page-up}
         and \texttt{Page-Down} \index{keys!page-down}
         keys to move the view up and down in pitch.
      \item \textbf{Horizontal Panning (Timeline Panning)}
         \index{scroll!shift scroll}
         \index{scroll!horizontal pan}
         \index{scroll!timeline pan}
         \index{pan!seqroll time}
         Holding the Shift key, and then using the vertical scroll action of a
         mouse or touchpad moves the view of the sequence/pattern time forward
         and backward.
         One can also click in the piano roll, and then use the
         \texttt{Shift Page-Up} \index{keys!shift page-up}
         and \texttt{Shift Page-Down} \index{keys!shift page-down}
         keys to move the view left and right in time.
      \item \textbf{Horizontal Zoom (Timeline Zoom)}
         \index{scroll!ctrl scroll}
         \index{scroll!horizontal zoom}
         \index{scroll!timeline zoom}
         \index{zoom!seqroll time}
         Holding the Ctrl key, and then using the vertical scroll action of a
         mouse or touchpad zooms the view of the sequence/pattern time to
         compress it or expand it.
         One can also click in the piano roll, and then use the
         \texttt{z} \index{keys!z},
         \texttt{Z} \index{keys!Z}, and
         \texttt{0} \index{keys!0} keys to change the timeline zoom.
   \end{itemize}

   The actions of this scrolling are smooth and fast.
   If an event is selected in the piano-roll area or the (thin) event area,
   then the scrolling increases or decreases the value of the event.
   In the case of a note, this increases or decreases the velocity of the note.
   For all events, this increases or decreases the length of the vertical line
   that represents the value of the event.

\subsubsection{Pattern Editor / Common Actions / Close}
\label{subsec:pattern_editor_close}

   There is no \textbf{Close} button in the pattern editor.  One can use
   window-manager actions, such as clicking on the X button of the window
   frame, or pressing the exit key defined in the window manager.
   \index{window!close}
   \textsl{Seq66} also provides the Ctrl-W key to close the pattern
   editor window.
   However, be aware that this convention does not apply to the other
   application windows of \textsl{Seq66}.

%-------------------------------------------------------------------------------
% vim: ts=3 sw=3 et ft=tex
%-------------------------------------------------------------------------------
