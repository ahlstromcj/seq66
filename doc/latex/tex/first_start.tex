%-------------------------------------------------------------------------------
% first_start
%-------------------------------------------------------------------------------
%
% \file        first_start.tex
% \library     Documents
% \author      Chris Ahlstrom
% \date        2015-11-01
% \update      2023-05-17
% \version     $Revision$
% \license     $XPC_GPL_LICENSE$
%
%     This document provides LaTeX documentation for Seq66.
%
%-------------------------------------------------------------------------------

\section{Let's Go!}
\label{sec:introduction_lets_go}

   \textsl{Seq66} requires ALSA or JACK on Linux;
   it does not support
   \index{pipewire}
   Pipewire.
   It might work with a Pipewire-JACK setup, but that usage has not been
   tested.
   On Windows, \textsl{Seq66} uses the MultiMedia API.
   This manual assumes that one is running \textsl{Seq66} on Linux, for the
   most part.

   For the first run, make sure all desired MIDI devices are plugged in and
   that all desired software synthesizers are running.
   Only ports present at startup will appear; this version of
   \textsl{Seq66} does not
   support automatic detection of port changes at run-time.
   In addition, port-mapping (see \sectionref{sec:port_mapping}) is
   enabled by default.

   Start \textsl{Seq66}, verify that it comes up, and then exit it
   immediately.
   This first start creates the configuration files in
   \texttt{/home/user/.config/seq66/} (Linux) or
   \texttt{C:/Users/your\_user\_name/AppData/Local/seq66} (Windows).
   The main configuration file is
   \texttt{qseq66.rc} (Linux) or
   \texttt{qpseq66.rc} (Windows).
   The '.rc' file contains ports and port-mapping for the devices
   initially on the system

   Now start \textsl{Seq66} to use JACK or ALSA for MIDI.
   For USB MIDI devices on JACK, the \texttt{a2jmidid} program must be running
   (see \sectionref{sec:jack}, for details.)
   On \textsl{Windows}, just run it \texttt{qpseq66.exe};
   see \sectionref{sec:windows}.

   For better trouble-shooting, run it from the command-line at first and
   provide a MIDI file.
   The port settings will depend on your system.
   On our system, the synthesizer (\textsl{Yoshimi}) comes up on MIDI buss 5.
   (On \textsl{Windows}, buss 0 is the "MIDI Mapper", while buss 1 is the
   built-in wavetable synthesizer, which is normally under control of buss 0.)
   The \texttt{-{}-buss} option remaps all events to the desired buss:

   \begin{verbatim}
      $ qseq66 --alsa --buss 5 /usr/share/seq66/midi/b4uacuse-gm-patchless.midi
      $ qseq66 --jack --buss 5 /usr/share/seq66/midi/b4uacuse-gm-patchless.midi
   \end{verbatim}

   The buss-override setting can also be made via the port drop-down control
   in the main window. It will modify the MIDI file.
   The "data" directory is a directory created upon installation of the
   application:

   \begin{verbatim}
      /usr/share/seq66-0.90/                          (Linux)
      C:/Program Files (x86)/Seq66/data/              (Windows)
   \end{verbatim}

   It contains sample MIDI files and configuration files.
   Some of the files in those directories apply to both operating systems, so
   be sure to look at them.
   The user configration files are stored in the user's "home" area:

   \begin{verbatim}
      /home/user/.config/seq66/qseq66.*               (Linux)
      C:/Users/username/AppData/Local/seq66           (Windows)
   \end{verbatim}

   The configuration files in the "home" directory
   are created after the first run of \textsl{Seq66}.
   If all is well, the "play" button will start playback and the tune sounds.
   If not, look into \textbf{Edit / Preferences} or the configuration files.
   Don't change the configuration files while \textsl{Seq66} is running, as
   they are generally re-saved at application exit.

   \textbf{Bug}:
      The irst-start change of input port setting does not get saved
      to the 'rc' file. The workaround is to exit and rerun Seq66, and
      make the setting again. Mysterious.

\subsection{Device Changes}
\label{subsec:introduction_device_changes}

   After running \textsl{Seq66} and getting it to work, one might
   either add new MIDI devices and software synthesizers, or remove
   them.
   If port-mapping is enabled (the default, and recommended), then
   one can go to
   \textbf{Edit / Preferences / MIDI Clock} and
   click the \textbf{Make Maps} button.
   This action stores the current set of MIDI devices.
   Restart \textsl{Seq66} and the new set of devices should appear.

   In version 2 of \textsl{SAeq66}, a year or two from now, we hope to make
   the adjustments automatic.

\subsection{Windows}}
\label{subsec:introduction_windows}

   Details about running in \textsl{Windows} can be found in the installed file
   \texttt{C:\Program Files (x86)\Seq66\data\midi\readme.windows}.
   But basically, the first-start depends on if it is a bare \textsl{Windows}
   install, or if supplement MIDI support such as 
   the CoolSoft MIDIMapper and VirtualMIDISynth have been installed.
   If they haven't, the default MIDI setup will contain no MIDI inputs,
   and two MIDI outputs, the \textsl{MIDI Mapper} and the
   \textsl{Microsoft GS Wavetable} synthesizer.
   The former grabs control of the latter, making it unavailable!
   \textsl{Seq66} detects this situation and enables only the MIDI Mapper.

   Therefore, run \textsl{Seq66} for the first time, then exit it and
   restart it. One should then be able to play MIDI files to the MIDI Mapper,
   if nothing else.

   We hope to get some input from Windows users, as our main usage is Linux
   and we run Windows only in an old virtual machine.

%-------------------------------------------------------------------------------
% vim: ts=3 sw=3 et ft=tex
%-------------------------------------------------------------------------------
