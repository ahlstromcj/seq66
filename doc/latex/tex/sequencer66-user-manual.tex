%-------------------------------------------------------------------------------
% sequencer66-user-manual
%-------------------------------------------------------------------------------
%
% \file        sequencer66-user-manual.tex
% \library     Documents
% \author      Chris Ahlstrom
% \date        2015-11-01
% \update      2019-08-13
% \version     $Revision$
% \license     $XPC_GPL_LICENSE$
%
%     This document provides LaTeX documentation for Sequencer66.
%
%-------------------------------------------------------------------------------

\documentclass[
 11pt,
 twoside,
 a4paper,
 headinclude,
 footinclude,
 final                                 % versus draft
]{article}

%-------------------------------------------------------------------------------
% docs-structure
%-------------------------------------------------------------------------------
%
% \file        docs-structure.tex
% \library     Documents
% \author      Chris Ahlstrom
% \date        2015-04-20
% \update      2021-01-28
% \version     $Revision$
% \license     $XPC_GPL_LICENSE$
%
%     This "include file" provides LaTeX options for a document.
%
%     Note that enumitem is an extension of enumerate, and comes from
%     Debian's texlive-latex-recommended package.
%
%-------------------------------------------------------------------------------

\usepackage{enumitem}         % setting the whitespace between and within lists
\setlistdepth{9}
\setlist{noitemsep}           % spacing within the list

\usepackage{color}            % provide colors?
\usepackage{nameref}          % Provide references by name instead of number
\usepackage[colorlinks=true,linkcolor=webgreen,filecolor=webbrown,citecolor=webgreen]{hyperref}
\definecolor{webgreen}{rgb}{0,.5,0}
\definecolor{webbrown}{rgb}{.6,0,0}

\usepackage{ragged2e}         % For underfull boxes in the bibliography
\usepackage{verbatim}         % For the comment macro
\usepackage{url}              % Required for including URLs
\usepackage{hyperref}         % Required for including hyperlinks
\usepackage{amsthm}           % Helps avoid "destination with same
\usepackage[hypcap]{caption}  % make labels point to figure, not the caption
\usepackage[pdftex]{graphicx} % Required for including images
\graphicspath{{../images}}    % Set the default folder for images
\usepackage{float}            % For more control of location of Figures
\usepackage[T1]{fontenc}      % Remove font warnings for textleftbrace, etc.
\usepackage{geometry}         % Page & text layout
\geometry{
  letterpaper,
  top=2.5cm,
  bottom=2.5cm,
  left=2.5cm,
  right=2.5cm
}

\usepackage{longtable}        % For making multi-page tables
\usepackage{makeidx}          % For making an index

% Try to reduce the space before or after verbatim sections.
% Doesn't affect the spacing after the verbatim, though.

\usepackage{etoolbox}
\makeatletter
\preto{\@verbatim}{\topsep=10pt \partopsep=0pt}
\makeatother

% Let's try to reduce the size of quotations.

\usepackage{relsize,etoolbox}          % http://ctan.org/pkg/{relsize,etoolbox}
\AtBeginEnvironment{quotation}{\smaller}   % Step font down one size relatively

% For the MIDI Implementation Chart

\usepackage{makecell}

% This package isn't available easily on CentOS:
%
% \usepackage[subtle]{savetrees} % For tightening document vertical spacing

\hypersetup{                  % HYPERLINKS
% draft,                      % Uncomment removes links (e.g. for B&W printing)
 colorlinks=true,
 breaklinks=true,
 bookmarksnumbered,
 urlcolor=webbrown,
 linkcolor=blue,              % RoyalBlue
 citecolor=webgreen,
 pdftitle={},
 pdfauthor={\textcopyright},
 pdfsubject={},
 pdfkeywords={},
 pdfcreator={pdfLaTeX},
 pdfproducer={LaTeX with hyperref and ClassicThesis}
}

% Make an "enumber" style that makes all levels of enumerated lists show
% arabic numerals.

\newlist{enumber}{enumerate}{10}
\setlist[enumber]{nolistsep,label=\arabic*.}

% Make "paragraph" a fourth level, and make it shown in the table of
% contents.

\makeatletter
\renewcommand\paragraph{\@startsection{paragraph}{4}{\z@}%
   {-2.5ex\@plus -1ex \@minus -.25ex}%
   {1.25ex \@plus .25ex}%
   {\normalfont\normalsize\bfseries}}
\makeatother
\setcounter{secnumdepth}{4} % how many sectioning levels to assign numbers to
\setcounter{tocdepth}{4}    % how many sectioning levels to show in ToC

% Provide a way of counting user-interface items without putting them in an
% enumberation.

\newcounter{ItemCounter}

% Makes a numbered paragraph out of an item, and allows two index entries
% for it.

\newcommand{\itempar}[2] {
   \noindent
   \stepcounter{ItemCounter}
   \textbf{\arabic{ItemCounter}. #1.}
   \index{#1}
   \index{#2}
}

% Provides for two forms of an option, as might be shown in a man page.

\newcommand{\optionpar}[2] {
   \textbf{\texttt{#1}} \textbf{\texttt{#2}} \\
   \index{#1}
   \index{#2}
}

% Similar, but with no line break.

\newcommand{\optionline}[2] {
   \textbf{\texttt{#1}} \textbf{\texttt{#2}}
   \index{#1}
   \index{#2}
}

% Now deprecated in preference to \itempar

\newcommand{\settingdesc}[2] {
   \textbf{#1}
   \index{#1}
   \index{#2}
}

% Reference to a configuration file setting
%
%     \configref{xxx}{xxxxx}{xxxx}.

\newcommand{\configref}[3] {
   \index{#1!#2}
   \texttt{qseq66.#1: [#2] #3}.
}

% Make a full reference to a figure using its number, its name, and its page
% number.  Very useful if you have a hard-copy of the document to deal with.

\newcommand{\figureref}[1] {
   Figure~\ref{#1}
   "\nameref{#1}"
   on page~\pageref{#1}\ignorespaces
}

% Make a full reference to a section using its number, its name, and its page
% number.  Very useful if you have a hard-copy of the document to deal with.

\newcommand{\sectionref}[1] {%
   section~\ref{#1}
   "\nameref{#1}"
   on page~\pageref{#1}\ignorespaces
}

% Make a full reference to a "paragraph"  using its number, its name, and
% its page number.  Very useful if you have a hard-copy of the document to
% deal with.

\newcommand{\paragraphref}[1] {%
   paragraph~\ref{#1}
   "\nameref{#1}"
   on page~\pageref{#1}\ignorespaces
}

% Make a full reference to a table using its number, its name, and its page
% number.  Very useful if you have a hard-copy of the document to deal with.

\newcommand{\tableref}[1] {%
   table~\ref{#1}
   "\nameref{#1}"
   on page~\pageref{#1}\ignorespaces
}

% For lining up enumerated items.  Doesn't really work well, better
% to create a table.

\newcommand{\itab}[1]{\hspace{0em}\rlap{#1}}
\newcommand{\tab}[1]{\hspace{.1\textwidth}\rlap{#1}}

% Change the fragction of the page that can be filled with graphics from 0.7
% to 0.9.

\renewcommand\floatpagefraction{.9}
\renewcommand\dblfloatpagefraction{.9}
\renewcommand\topfraction{.9}
\renewcommand\dbltopfraction{.9}
\renewcommand\bottomfraction{.9}

\raggedbottom                          % avoid excessive vertical justification

%-------------------------------------------------------------------------------
% vim: ts=3 sw=3 et ft=tex
%-------------------------------------------------------------------------------
                 % specifies document structure and layout

% Replacing normal header/footer with a fancier version.  These two symbols of
% document class were showing up as "unused" in the log file.
%
% headinclude,
% footinclude,
%
% So we add the fancyhdr package, clear the default layout, and set it up for
% our wider pages.

\usepackage{fancyhdr}
\pagestyle{fancy}
\fancyhead{}
\fancyfoot{}
\fancyheadoffset{0.005\textwidth}
\lhead{Sequencer66 Live MIDI Sequencer}
\chead{}
\rhead{User Manual}
\lfoot{}
\cfoot{\thepage}
\rfoot{}

\makeindex

\begin{document}

\title{Sequencer66 User Manual 0.90.2}
\author{Chris Ahlstrom \\
   (\texttt{ahlstromcj@gmail.com})}
\date{\today}
\maketitle

\begin{figure}[H]
   \centering 
%  \includegraphics[scale=0.40]{Sequencer66-0_94.png}
%  \includegraphics[scale=0.40]{Sequencer66-0_90.png}
   \includegraphics[scale=0.65]{roll.png}
   \caption*{"New Version"}
\end{figure}

\clearpage                             % moves Contents to next page

\tableofcontents
\listoffigures                         % print the list of figures
\listoftables                          % print the list of tables

% Changes the paragraph style to remove indenting and put a line between each
% paragraph.  This could be moved up into the preamble, but then would
% affect the spacing of the TOC and LOF, LOT noted above.

\setlength{\parindent}{2em}
\setlength{\parskip}{1ex plus 0.5ex minus 0.2ex}

\section{Introduction}
\label{sec:introduction}

   This document describes \textsl{Sequencer66}
   \cite{sequencer66}, through version 0.90.0.
   The following project supports \textsl{Sequencer66}:

   \begin{itemize}
      \item \url{https://github.com/ahlstromcj/sequencer66.git}.
      \item \url{https://github.com/ahlstromcj/sequencer64-doc.git}.
   \end{itemize}

   We include the old \textsl{Sequencer64} documentation project because we would
   like a more lean-and-mean manual for \textsl}Sequencer66].

   \textsl{Sequencer66} is \textsl{Sequencer64} refactored for newer versions of
   \textsl{C++} and with a lot of kruft removed.  It drops the Gtkmm
   user-interface in favor of \textsl{Qt 5} and has better handling of sets and
   configuration files.

   We have many contributors to acknowledge.  Please see
   \sectionref{sec:kudos}.

\subsection{Sequencer66: What?}
\label{subsec:what_is_sequencer66}

   \textsl{Sequencer66} is an ongoing reboot of \textsl{Seq24},
   a live-looping sequencer with an interface more like a hardware sequencer
   than track-based MIDI sequencers.
   \textsl{Sequencer66} is not a synthesizer.  It requires a hardware
   synthesizer, or a software synthesizer such as Timidity \cite{timidity},
   FluidSynth \cite{fluidsynth}, etc.

\subsection{Sequencer66: Why?}
\label{subsec:introduction_seq66_vs_others}

   The first reason to refactor \textsl{Sequencer64} is ...

\subsection{Improvements}
\label{subsec:improvements}

   The following improvements are some that have been made in
   \textsl{Sequencer66} versus \textsl{Sequencer64}.

   \begin{itemize}
      \item A mutes editor.
      \item A sets editor.
      \item A better live frame (main window).
      \item More to write!!!
   \end{itemize}

   For developers, \textsl{Sequencer66} is customizable via C macros,
   by enabling/disabling options at build-configuration time, and by many
   command-line arguments.  We cannot show all permutations of settings in this
   document, so don't be surprised if some screenshots don't quite match
   one's setup.  Distro maintainers might pick their favorite build
   configurations.

\subsection{Document Structure}
\label{subsec:introduction_document_structure}

   The structure of this document follows the user-interface of
   \textsl{Sequencer66}.  The sections are provided in the order
   their contents appear in the user-interface of \textsl{Sequencer66}.  To
   help the reader jump around this document, it provides
   multiple links, references, and index entries.

\subsection{Let's Go!}
\label{subsec:introduction_lets_get_started}

   Make sure no other sound application is running, for the first run.
   Start \textsl{Sequencer66} to use JACK for MIDI, or
   on \textsl{Windows}, just run it (\texttt{qpseq66.exe};
   on \textsl{Windows}, PortMidi is used). The port
   settings will be different.  Provide a MIDI file.
   On our system, the synthesizer
   (\textsl{Yoshimi}) comes up on MIDI buss 5; an option remaps
   all events to that buss:

\begin{verbatim}
   $ seq66 --jack-midi --buss 5 contrib/midi/b4uacuse-seq24.midi
   C:\> qpseq66 --buss 1  contrib/midi/b4uacuse-seq24.midi
\end{verbatim}

   If the \texttt{--alsa} option is used instead of
   \texttt{--jack-midi}, then the "JACK" button shows "ALSA" instead
   (Linux only).  The following figure is for the Linux
   version.

\begin{figure}[H]
   \centering 
%  \includegraphics[scale=0.5]{new/seq66-first-screen-0-94.png}
   \includegraphics[scale=0.65]{roll.png}
   \caption{Sequencer66 Main Screen, Linux ALSA MIDI}
   \label{fig:seq66_main_screen}
\end{figure}

   The following figure shows the user-interface used for
   \textsl{Windows}.  It uses the Qt 5 framework.

\begin{figure}[H]
   \centering 
%  \includegraphics[scale=0.65]{new/seq66-first-screen-0-96-qt.png}
   \includegraphics[scale=0.65]{roll.png}
   \caption{Sequencer66 Main Screen, Qt 5 Interface, With Colors}
   \label{fig:seq66_main_screen_qt}
\end{figure}

   The \textsl{Sequencer66} main window appears, as shown above.
%  \figureref{fig:seq66_main_screen}, and
%  \figureref{fig:seq66_main_screen_qt}.
   These figures have some differences from the \textsl{Seq24} main window
   and from each other, but the functionality is about the same.
   Most features, including the "look" of the application,
   can be configured via the "rc" and "user"
   configuration files or command-line options.
   There are many new front-panel items in \textsl{Sequencer66}, but
   the Qt and Gtkmm user-interfaces differ in details.

   \begin{itemize}
      \item Control buttons:
      \begin{itemize}
         \item Start, Stop, and Pause.
         \item Toggle and show the status of "Live" mode versus "Song" mode.
         \item Mute/show the mute status of all tracks.
         \item Enable/disable the menu bar and show its status.
         \item Set JACK Slave/Master transport, and
            ALSA/JACK (native) mode.
         \item Set the kind of time display, between "bars:beat:ticks"
            and "hours:minute:seconds".
         \item Panic button, to stop all tracks and turn off all notes.
         \item Song-recording snap, the \textbf{S} button.
         \item Tap Tempo, the \textbf{0} (zero) button.
         \item Keep-queue toggling and status.
      \end{itemize}
      \item Current time in bars, beats, and ticks.
      \item Song recording records all muting changes to the Song Editor.
      \item Log Tempo, which inserts the current tempo into the tempo track
         as an event.
      \item Tempo recording, which inserts all tempo changes as tempo events.
   \end{itemize}

   Many of these buttons have configurable keystrokes as well.
   See \sectionref{subsec:seq66_patterns_panel_bottom}.
   Some of these items are not present in the Qt interface.

\rhead{\rightmark}         % shows section number and section name

% Menu

%-------------------------------------------------------------------------------
% seq66_menu
%-------------------------------------------------------------------------------
%
% \file        seq66_menu.tex
% \library     Documents
% \author      Chris Ahlstrom
% \date        2015-08-31
% \update      2019-08-20
% \version     $Revision$
% \license     $XPC_GPL_LICENSE$
%
%     Provides the Menu section of sequencer66-user-manual.tex.
%
%-------------------------------------------------------------------------------

\section{Menu}
\label{sec:seq66_menu}

   The \textsl{Sequencer66} menu
   (\figureref{fig:seq66_main_screen}) is simple, but important.

\subsection{Menu / File}
\label{subsec:seq66_menu_file}

   The \textbf{File} menu is used to save and load MIDI files
   (Standard MIDI Format 0 or 1) and \textsl{Sequencer66} MIDI
   files.
   The \textsl{Sequencer66} menu entry contains the sub-items shown below.
%  \figureref{fig:seq66_menu_file_items}.
   The next few sub-sections discuss
   the sub-items in the \textsl{File} sub-menu.

\begin{figure}[H]
   \centering 
   \includegraphics[scale=0.65]{roll.png}
   \caption{Sequencer66 File Menu Items}
   \label{fig:seq66_menu_file_items}
\end{figure}

   \begin{enumber}
      \item \textbf{New}
      \item \textbf{Open}
      \item \textbf{Open Playlist}
      \item \textbf{Recent MIDI files}
      \item \textbf{Save}
      \item \textbf{Save As}
      \item \textbf{Export Song as MIDI}
      \item \textbf{Export MIDI Only}
      \item \textbf{Import MIDI to Current Set}
%     \item \textbf{Options}
      \item \textbf{Exit or Quit}
   \end{enumber}

%  The Qt version of this menu is similar, except that the
%  \textbf{Options...} menu is placed in \textbf{Edit / Preferences}.

\subsection{Menu / File / New}
\label{subsec:menu_file_new}

   The \textbf{New} menu entry clears out the current song or play-list.
   If unsaved changes are pending, the user is prompted to save the changes.
   Prompting for changes is more comprehensive than \textsl{Seq24}.
   However, when in doubt, save!  Keep backups of your tunes!

\subsubsection{Menu / File / Open}
\label{subsubsec:seq66_menu_file_open}

   The \textbf{Open} menu entry opens a song (MIDI file or \textsl{Cakewalk}
   WRK file), replacing the current song.
   It opens up a standard Qt 5 file dialog:

\begin{figure}[H]
   \centering 
%  \includegraphics[scale=0.5]{menu/menu_file_open.png}
   \includegraphics[scale=0.65]{roll.png}
   \caption{File / Open}
   \label{fig:seq66_menu_file_open}
\end{figure}

   This dialog lets one type a file-name, highlighting the first file (if any)
   that matches the characters typed so far.
   \textsl{Sequencer66} can open regular MIDI files and
   Cakewalk WRK files.
   It will read and store some MIDI Meta events 
   (e.g. Tempo and Time Signature).

\subsubsection{Menu / File / Open Playlist}
\label{subsubsec:seq66_menu_file_open}

   The \textbf{Open Playlist...} menu entry opens a \textsl{Sequencer66}
   play-list file.
   This file contains a list of "playlist sections",
   each listing a number of MIDI songs.
   These playlists and songs can be
   selected by the arrow keys or by MIDI control.
   See \sectionref{sec:playlist}.

   Once activated, the current play-list file-name is saved in the
   \texttt{[playlist]} section of the "rc" configuration file when
   \textsl{Sequencer66} exits.
   The file extension is \texttt{.playlist}.
   The Qt user-interface will eventually support editing of the play-list.
   Currently, the user must understand the play-list format, and use a
   text editor to edit the play-list file.

\subsubsection{Menu / File / Recent MIDI files}
\label{subsubsec:seq66_menu_file_recent}

   This menu entry provides a list of the last few MIDI files created or opened;
   play-list selections are \textsl{not} included.
   This list is saved in the \texttt{[recent-files]} section of the
   "rc" configuration file.
   In the menu, only the last part of the file-name is
   shown, but in the "rc" configuration file,
   the full path to the file-name is stored.
   This path is in "UNIX" format, using the forward slash, or solidus,
   as the path separator, even in \textsl{Windows}.
   Only unique entries are included in the recent-files list.
   The limit is 10 recent-file entries.
   This is a feature from \textsl{Kepler34} (\cite{kepler34}).
   Here is an example from an "rc" file:

\begin{verbatim}
   [recent-files]
   # Holds a list of the last few recently-loaded MIDI files.
   3
   /home/chris/git/seq66/data/b4uacuse-gm-patchless.midi
   /home/chris/git/seq66/contrib/midi/colours.midi
   /home/chris/git/seq66/Julian-data/TestBeeps.midi
\end{verbatim}

\subsubsection{Menu / File / Save and Save As}
\label{subsubsec:menu_file_open_save_as}

   The \textbf{Save} menu entry saves the song under its current file-name.
   If there is no current file-name, it opens up a standard file
   dialog to name and save the file.
   The \textbf{Save As} menu entry saves a song under a different name.
   It opens up the following standard file dialog, very similar to the 
   \textbf{File Open} dialog, with an additional \textbf{Name} text-edit field.

\begin{figure}[H]
   \centering 
%  \includegraphics[scale=0.5]{menu/menu_file_save_as.png}
   \includegraphics[scale=0.65]{roll.png}
   \caption{File / Save As}
   \label{fig:seq66_menu_file_save_as}
\end{figure}

   To save a new file, or to save the current existing file to a new name,
   enter the name in the name field, without an extension.
   \textsl{Sequencer66} will append a \texttt{.midi} extension to the filename.
   The file will be saved in a format that the Linux \textsl{file} command
   will tag as something like:

   \begin{verbatim}
      myfile.midi: Standard MIDI data (format 1) using 16 tracks at 1/192
   \end{verbatim}

   It looks like a simple MIDI file, and yet, if one re-opens it in
   \textsl{Sequencer66}, one sees that all of the labeling, pattern
   information, and song layout has been preserved in this file.
%  Even the pattern layout (arrangement), as discussed in
%  \sectionref{subsubsec:seq66_song_editor_arrangement_panel_roll},
%  have been saved.
%  (But the L and R marker positions are not saved.)
   This information is saved in a way that MIDI-compliant software
   should be able to use, or ignore without failure.

   The output MIDI file after \textsl{Sequencer66} saves the original MIDI file
   is larger.
   After the last track in the file, a number of
   \index{SeqSpec}
   MIDI-compliant sequencer-specific (SeqSpec) items are saved, to preserve
   the extra information that \textsl{Sequencer66} adds.
   In legacy mode, \textsl{Sequencer66} saves this information
   in the same format as \textsl{Seq24}.
   Otherwise, it saves it in a more MIDI-compatible format.
%  Some of this extra information (mute-groups) will be stripped from the MIDI
%  file.
   Normally, \textsl{Sequencer66} saves this information by marking
   each SeqSpec section as vendor-specific information, and marking this
   section as a regular MIDI track.
   The legacy and new formats of the final "track" are explained in
   \sectionref{subsec:legacy_midi_format}.

   \index{Meta events}
   Meta events are now partially handled by \textsl{Sequencer66}.
   Meta events Set Tempo
%  (\texttt{FF 51 03 tt tt tt}),
   and Time Signature
%  (\texttt{FF 58 04 nn dd cc bb}),
   are now fully supported.
   Other Meta events,
   such as Meta MIDI Channel
%  (\texttt{FF 20 01 cc}),
   and Meta MIDI Port
%  (\texttt{FF 21 01 pp}),
   are now read as events, and are saved back when the file is saved.
   They cannot be edited in \textsl{Sequencer66}, but they are
   not lost.
   Please note that the channel and port meta events are
   considered \textsl{obsolete} in the MIDI standard.

\subsubsection{Menu / File / Import MIDI}
\label{subsubsec:seq66_menu_file_import}

   The \textbf{Import} menu entry imports an SMF 0
   or SMF 1 MIDI file as one or more patterns, one pattern per track,
   into the specified screen-set.
   This functionality is explained in detail in
   \sectionref{subsec:seq66_midi_export_file_import}.

\subsubsection{Menu / File / Export Song as MIDI}
\label{subsubsec:seq66_menu_file_export}

   Thanks to the \textsl{Seq32} project, the ability to export songs to MIDI
   format has been added.  In this export, a complete song performance is
   recoded so that other MIDI sequencers can play the performance properly.
   This functionality is explained in detail in
   \sectionref{subsec:seq66_midi_export_file_export}.

\subsubsection{Menu / File / Export MIDI Only}
\label{subsubsec:seq66_menu_file_export_midi_only}

   Sometimes it might be useful to export only the non-sequencer-specific
   (non-SeqSpec) data from a \textsl{Sequencer66} song, in order to reduce the
   size of the file or to accomodate non-compliant sequencers.
   This functionality is explained in detail in
   \sectionref{subsec:seq66_midi_export_file_export_midi_only}.

\subsubsection{Menu / File / Options}
\label{subsubsec:seq66_menu_file_options}

   In the \textsl{Portmidi / Windows / Qt 5} version of
   \textsl{Sequencer66}, the \textbf{Options} dialog has been moved to 
   \textbf{Edit / Preferences}.
   See \sectionref{subsubsec:qt_portmidi_qt5_edit_prefs}.
   There are still a few configuration items yet to be represented in that
   new user-interface.

   \textbf{Options} provides a number of settings in one
   tabbed dialog, shown in the figures that follow.
   It allows one to set MIDI clocking,
   what incoming MIDI events control the sequencer, what keys are
   mapped to functions, how the mouse works, and some JACK parameters.
   Note that there is a new tab-page for \textbf{Ext Keys}, to support
   more keystroke controls.

\begin{figure}[H]
   \centering 
%  \includegraphics[scale=0.65]{menu/options-tab-0-9-18.png}
   \includegraphics[scale=0.65]{roll.png}
   \caption{Edit / Preference}
   \label{fig:seq66_options_tab_0_9_18}
\end{figure}

\paragraph{Menu / File / Options / MIDI Clock}
\label{paragraph:seq66_menu_file_options_midi_clock}

   The \textbf{MIDI Clock} tab provides a way to set MIDI clock for
   the available MIDI output busses.
   It configures to what busses the MIDI clock and data gets dumped.
   It also shows the devices that can play music.
   The items that appear in this tab depend the setup.

   \begin{itemize}
      \item What MIDI devices are connected to the computer.
         MIDI controllers, USB MIDI cables, applications with virtual
         ports, and other connected devices will add MIDI
         output devices (ports) to the system.
         In \textsl{Windows}, the available devices are shown as well.
%     \item What MIDI software applications are running on the computer.
%        For example, running MIDI software synthesizers such as
%        \textsl{Timidity} and \textsl{Yoshimi} will add extra output devices
%        (playback ports) to a system.
      \item The setting of the "manual-ports" option, which tells
         \textsl{Sequencer66} to set up virtual MIDI ports.
         It is enabled by the
         \texttt{--manual-ports} command-line option or the
         \texttt{[manual-ports]} section of the
         \texttt{sequencer66.rc} configuration file, described in
         \sectionref{subsec:seq66_rc_file_manual_ports}.
      \item The setting of the \textsl{Sequencer66}-specific
         "reveal ALSA ports" option,
         \texttt{--reveal-ports} command-line option or the
         \texttt{[reveal-ports]} section of the
         \texttt{sequencer66.rc} configuration file, described in
         \sectionref{subsec:seq66_rc_file_reveal_ports}.
   \end{itemize}

   For the current discussion, a USB MIDI cable was plugged into the system,
   and the \textsl{Timidity} and \textsl{Yoshimi} (in ALSA mode) software
   synthesizers were running.  \textsl{Sequencer66} was also running,
   without virtual ports enabled, and \texttt{--alsa} turned on.
%  with the option of "manual-ports" (\texttt{-m} or
%  \texttt{--manual-ports}) and ALSA (\texttt{-A} or
%  \texttt{--alsa} turned on.
   Here are the devices shown when running \textsl{aplaymidi}
   from the command-line:

   \begin{verbatim}
      $ aplaymidi -l
       Port    Client name                      Port name
       14:0    Midi Through                     Midi Through Port-0
       24:0    E-MU XMidi1X1 Tab                E-MU XMidi1X1 Tab MIDI 1
      128:0    TiMidity                         TiMidity port 0
      128:1    TiMidity                         TiMidity port 1
      128:2    TiMidity                         TiMidity port 2
      128:3    TiMidity                         TiMidity port 3
      129:0    seq66                            seq66 in
   \end{verbatim}

%     129:16   sequencer66                      sequencer66 in

   Note that \textsl{Yoshimi} does not appear.  Perhaps
   \texttt{aplaymidi} does not subscribe properly to all ALSA notifications.
   A better command might be this one (a different setup than above):

   \begin{verbatim}
      $ aconnect -lio
      client 0: 'System' [type=kernel]
          0 'Timer           '
          1 'Announce        '
         Connecting To: 15:0, 128:0
      client 14: 'Midi Through' [type=kernel]
          0 'Midi Through Port-0'
      client 129: 'yoshimi' [type=user,pid=26335]
          0 'input           '
   \end{verbatim}

   \textsl{Sequencer66} will detect \textsl{Yoshimi}.
   One can also run the following command instead:

   \begin{verbatim}
      $ aconnect -io
   \end{verbatim}

   One other note... of late, we're seeing cases where running the
   \textsl{Timidity} daemon hides \textsl{Yoshimi}.  Just be aware.

%  (For some reason, the \textsl{Yoshimi} input port is not showing up
%  in the output of \texttt{aplaymidi}, though, as shown in
%  \figureref{fig:seq66_midi_clock_4_devices_manual_0},
%  \textsl{Sequencer66} sees it on port 7.  Perhaps that application is not
%  providing a good ALSA device name.)
%
%  Also, currently we do NOT see sequencer66/seq66 in the output shown
%  above!  What's up with that!?  Need the -m option!

   Turning to \figureref{fig:seq66_midi_clock_4_devices_manual_1},
   with the option of "manual-ports" (\texttt{-m} or
   \texttt{--manual-ports}) and ALSA in force,
   note the 16 devices provided by \textsl{Sequencer66}.
%  Also note that its first value is 1, not 0, due to
%  the MIDI Thru port occupying slot 0.
   This figure shows the result with "manual-ports" turned on.
   Remember that this option also applies to the native JACK MIDI
   mode of \textsl{Sequencer66}.

%  Also note that the new \textbf{Port Disabled} radio button is not shown in
%  here.
%  See \sectionref{fig:qt5_prefs_clock_windows}, it shows this button.
%  It was added because Windows can have issues with its built-in MIDI mapper,
%  causing some ports to be uninitialized, and hence not available.
%  We need to ignore those ports.

\begin{figure}[H]
   \centering 
%  \includegraphics[scale=0.75]{menu/midi-clock-4-devices-manual-1.png}
%  \includegraphics[scale=0.75]{new/midi-clock-4-devices-manual-1.png}
%  \includegraphics[scale=0.65]{jack/jack-nano-yosh-manual-clock-seq66.png}
   \includegraphics[scale=0.65]{roll.png}
   \caption{MIDI Clock, Manual Option On}
   \label{fig:seq66_midi_clock_4_devices_manual_1}
\end{figure}

   It shows the 16 virtual MIDI output busses that \textsl{Sequencer66} can
   drive.  One needs to use a JACK or ALSA MIDI
   connection application to connect a device on each of those outputs.  The
   fact that the the buss names can
   start with different numbers, depending on the system setup, can complicate
   the playing of MIDI in this manner.  Also, the "user" configuration file can
   change the visible names of the ports, causing further confusion.
   The following elements are present in this dialog:

   \begin{enumber}
      \item \textbf{Index Number}
      \item \textbf{Client Number}
      \item \textbf{Port Number}
      \item \textbf{Buss Name}
      \item \textbf{Port Disabled}
      \item \textbf{Off}
      \item \textbf{On (Pos)}
      \item \textbf{On (Mod)}
      \item \textbf{Clock Start Modulo}
   \end{enumber}

   The format of the left side of the entry listing is like the following:

   \begin{verbatim}
      [4] 4:4 seq66 midi out 4:5
       ^  ^ ^ ^
       |  | | |
       |  | |  -------- Buss name
       |  |  ---------- Port number
       |   ------------ Client number
        --------------- Index number
   \end{verbatim}

   \setcounter{ItemCounter}{0}      % Reset the ItemCounter for this list.

   \itempar{Index Number}{midi clock!index number}
   \index{index number}
   The number in square brackets is an ordinal indicating the position
   of the output buss in the list.
   \index{buss override}
   It can be used with the \texttt{-b --buss --bus} option to redirect all
   output to that one buss, which is useful if only one buss is active, and the
   \textsl{Sequencer66} MIDI song patterns route to non-existent busses.

   \itempar{Client Number}{midi clock!client number}
   \index{client number}
   The number that precedes the colon is the "client number".
   It is useful mainly in ALSA, where clients can have numbers like "14",
   "128", "129", etc.  For native JACK mode, it matches the index number.

   \itempar{Port Number}{midi clock!port number}
   \index{port number}
   The number that follows the colon is the "port number".
   It is useful mainly in ALSA.
   For native JACK mode, it matches the index number.

   \itempar{Buss Name}{midi clock!buss name}
   \index{port name}
   \index{midi clock!port name}
   These labels indicate the output busses (ports) of \textsl{Sequencer66}.
%  They range from \textbf{[1] sequencer66 1} to \textbf{[16] sequencer66 16}
%  in the legacy application, \texttt{sequencer66}.
   They range from \textbf{[1] seq66 1} to \textbf{[16] seq66 16}.
   in native JACK, when manual/virtual ports are active.

   \itempar{Port Disabled}{midi clock!port disabled}
   The \textbf{Port Disabled} clock choice marks a port
   that one does not want to use or that the operating system
   (\textsl{Windows}, I'm looking at \textsl{you}!)
   is locking or disabling that output port.
   Normally, this inaccessible port would cause \textsl{Sequencer66} to exit.
   With the port disabled, the inaccessible port is ignored.

   When the \textsl{Windows} version of \textsl{Sequencer66}
   (\texttt{qpseq66} is first started, it may error out.
   It will then write "erroneous.rc" and "erroneous.usr" configuration
   files, which can be examined to find the offending buss.
%  See \sectionref{fig:qt5_prefs_clock_windows}, it shows this
%  new feature.

   \itempar{Off}{midi clock!off}
   Disables the MIDI \textsl{clock} for the given output buss.
   MIDI output is still sent to those ports, and
   each port that has a device connected to it will play music.
   Some synthesizers may require this setting.
%  For feeding \textsl{Yoshimi} (running in ALSA mode)
%  with MIDI data, we found that this
%  setting is the one that must be made in order for \textsl{Yoshimi} to
%  produce a sound.

   \itempar{On (Pos)}{midi clock!on (pos)}
   MIDI clock will be sent to this buss.
   MIDI Song Position and MIDI Continue will be sent if playback starts
   at greater than tick 0 in Song mode.  Otherwise, MIDI Start will be sent.

   \itempar{On (Mod)}{midi clock!on (mod)}
   MIDI clock will be sent to this buss.
   MIDI Start will be sent, and clocking will begin
   once the Song Position has reached the start modulo of the specified size
   (see the next item's description).
   This setting is used for gear that does not respond to Song Position.

   \itempar{Clock Start Modulo}{midi clock!clock start modulo}
   Clock Start Modulo (1/16 Notes).
   This value starts at 1 and ranges up to 16384, and defaults to 64.
   It is used by the \textbf{On (Mod)} setting discussed above.
   It is the \texttt{[midi-clock-mod-ticks]} option in the \textsl{Sequencer66}
   "rc" file as described in
   \sectionref{subsec:seq66_rc_file_midi_cmt}.

   \itempar{Meta Events}{midi clock!meta events}
   \index{tempo-track-number}
   This section consists of one item, the Tempo Track number.
   It allows the user to move the tempo track from pattern 0 to
   another pattern.  Changing this option is not recommended, since track 1 (0)
   is the official track for tempo events, but \textsl{Sequencer66} allows the
   user to record tempo events to another track.  \textsl{Sequencer66} will
   process tempo events in any pattern.
   \textsl{Not supported yet in the Qt user-interface, but
   it can be set manually in the "rc" configuration file.}

   In addition, the \textbf{Set as Song Tempo Track} button sets this
   track as part of the currently-load MIDI song file, and will be saved when
   exited.  This value will override the global tempo-track value, which is
   stored in the 'rc' configuration file.  However, if 0, it will be ignored,
   so that the global value will take hold.
   See \sectionref{subsec:seq66_rc_file_midi_meta_events}, which discusses this
   setting in the "rc" configuration file.

   One thing to remember about tempo...
   there is a "global" tempo which is saved as part of the song in a SeqSpec
   section.  However, that tempo is not saved in the tempo track.
   In order to do so, one can click the \textbf{Log Tempo} button to write it
   to the tempo track as a tempo event.

\begin{figure}[H]
   \centering 
%  \includegraphics[scale=0.75]{menu/midi-clock-4-devices-manual-0.png}
%  Updated for version 0.93.1:
%  \includegraphics[scale=0.65]{new/midi-clock-4-devices-manual-0.png}
   \includegraphics[scale=0.65]{roll.png}
   \caption{MIDI Clock, Manual Option Off (ALSA View, Old Screenshot)}
   \label{fig:seq66_midi_clock_4_devices_manual_0}
\end{figure}

   In the figure above, with "manual-ports" turned off, and
   all of the real (non-virtual) devices that can be driven by MIDI output are
   shown, including the MIDI Thru port, and
%  the MIDI port on the \textsl{E-MU XMidi1x1} USB cable,
   the four ports provided by \textsl{Timidity} on our setup.
%  , and the unlabelled
%  port provided by the \textsl{Yoshimi} synthesizer running in ALSA mode.
%  (However, \texttt{seq66} does show the name "yoshimi" as the client name.)
%  One could theoretically play music through 6 or 7 devices using
%  \textsl{Sequencer66} with this setup.
   (The \textbf{Port Disabled} column is missing from this old screenshot.  We
   need to fix that someday.)

   See \sectionref{subsec:seq66_jack_native_midi},
   for a lot more information about native JACK support, and examples of JACK
   MIDI ports and connections.

   \index{todo!manual alsa gui option}
   There is currently no user-interface item corresponding to the "manual-ports"
   command-line and "rc" configuration file option.
   We should rename this option to "virtual"
   eventually, since it can also apply to JACK MIDI.

\paragraph{Menu / File / Options / MIDI Input}
\label{paragraph:seq66_menu_file_options_midi_input}

   To allow \textsl{Sequencer66} to record MIDI from MIDI devices such as
   controllers and keyboards, the output of the ALSA MIDI recording
   command-line application is relevant:

   \begin{verbatim}
      $ arecordmidi -l
       Port    Client name                      Port name
       14:0    Midi Through                     Midi Through Port-0
       24:0    USB2.0-MIDI                      USB2.0-MIDI MIDI 1
      129:1    seq66                            seq66 midi out 0
      129:2    seq66                            seq66 midi out 1
       . . .   . . .                               . . .
      129:16   seq66                            seq66 midi out 15
   \end{verbatim}

% Should the above be offset re 0, not 1?  Check it out!!!

   We see that we can record MIDI from the MIDI Thru port, from the USB MIDI
   cable, and MIDI from any of the 16 output ports provided by the manual
   port mode of \textsl{Sequencer66}.

   If the "manual-ports" option is turned \textsl{off} (e.g. by using the
   \texttt{-a} option),
   then the only item in the \textbf{MIDI Input} tab is the single MIDI input
   buss provided by \textsl{Sequencer66}:  \textbf{[0] seq66 0}.
   (The figure shown is currently out-of-date.)

% Should the above be offset re 0, not 1?  Check it out!!!

\begin{figure}[H]
   \centering 
%  \includegraphics[scale=0.75]{menu/midi-input-4-devices-manual-1.png}
%  \includegraphics[scale=0.65]{new/midi-input-4-devices-manual-1.png}
   \includegraphics[scale=0.65]{roll.png}
   \caption{MIDI Input, Manual Ports Off (Condensed View)}
   \label{fig:seq66_midi_input_4_devices_manual_1}
\end{figure}

% WE NEED a composite view of the -m, -A, and -a options for ALSA for
% both the Clock and the Input tabs!!!!!

% Check the following out!  I am not 100% certain it is correct!!!
% The above is shown with the -A option.

   Any item checked allows \textsl{Sequencer66} to record MIDI
   from another source,
   which must be connected to this port via
   another application).
%  , or pass it through to the output busses
%  that are configured to allow pass-through
%  (in the Pattern Editor, as discussed in 
%  \sectionref{subsec:seq66_pattern_editor_bottom}.)

   \textbf{Warning:}
   \index{warnings!usr config}
   \index{usr config}
   If the 
   \texttt{[user-midi-bus-definitions]} value in the "user" configuration file
   is non-zero, and the
   corresponding number of
   \texttt{[user-midi-bus-N]} settings are provided, then
   the list of existing hardware will be ignored, and those values will be
   shown instead.
   (This feature can be overridden with the
   \texttt{--reveal-ports} (\texttt{-r}) option.)

%  New in \textsl{Sequencer66} is the option to record MIDI input into
%  more than one pattern based on the MIDI channel, as discussed below.

   If the "auto ALSA ports" option is turned on, via the \texttt{-a} or
   \texttt{--auto-ports} option, then
   the input ports from the system are shown:

\begin{figure}[H]
   \centering 
%  \includegraphics[scale=0.75]{menu/midi-input-4-devices-manual-0.png}
%  \includegraphics[scale=0.65]{new/midi-input-4-devices-manual-0.png}
   \includegraphics[scale=0.65]{roll.png}
   \caption{MIDI Input, \texttt{-a} Option (Condensed View)}
   \label{fig:seq66_midi_input_4_devices_manual_0}
\end{figure}

   For example, one could check input \#1 to have \textsl{Sequencer66} record
   MIDI from an old-fashioned MIDI keyboard that is connected to another
   USB MIDI cable (the \textsl{E-MU Xmidi}).  If the keyboard didn't have a
   sound generator, one would also want \textsl{Sequencer66} to pass this MIDI
   on to a sound generator, such as a software or hardware synthesizer attached
   to one of the ports shown in
   \figureref{fig:seq66_midi_clock_4_devices_manual_0}.

   \textbf{Warning:}
   \index{warnings!usr config}
   \index{usr config}
   The "user" configuration file can override what is actually
   displayed as hardware.  If you define these sections, they should match your
   hardware exactly, and your hardware should not change from session to
   session.

   Note the two sections of this configuration page:

   \index{input buses}
   \textbf{Input Buses} delineates the MIDI input devices as noted above.
   \index{input options}
   \textbf{Input Options} adds further refinements to MIDI input.

   \index{input by channel}
   \textbf{Record input into sequence according to channel}
   causes MIDI input with multiple channels to be distributed to
   each sequence according to MIDI channel number.
   When disabled, the legacy recording behavior dumps all data into the current
   sequence, regardless of channel.

%  Needs to be clarified!!!

\paragraph{Menu / File / Options / Keyboard }
\label{paragraph:seq66_menu_file_options_keyboard}

   \textsl{Seq24} allows extensive use of
   keyboard shortcuts to make operations go faster than with a mouse,
   and \textsl{Sequencer66} extends that tradition.
   The \textbf{Keyboard} tab (currently in Gtkmm only)
   allows for the configuration of these keyboard shortcuts.

   These settings can also be modified by editing the appropriate "rc"
   configuration file, stored in one of the following directories, depending on
   the operating system:
   
   \begin{verbatim}
         /home/username/.config/sequencer66
         C:/Users/username/AppData/Local/sequencer66
   \end{verbatim}

   \textbf{Warning:}
   \index{keys!gotchas}
   There are a number of "gotchas" to be aware of when assigning keys to the
   fields in the \textbf{Keyboard} tab:

   \begin{itemize}
      \item This configuration dialog is not yet present in the
         \textbf{Qt 5} version of \textsl{Sequencer66}.  For now,
         one has to edit the "rc" file to configure the keystrokes.
         It is easiest to just use the sample \texttt{qpseq66.rc}
         or \texttt{qseq66.rc} from the
         \texttt{data} directory in the source-code package.
         Another option is to just run \textsl{Sequencer66} the first time and
         tweak the "rc" and "usr" files that are created in the directories
         noted above.
         Internally, the Qt key-codes are remapped to Gtk key-codes.
      \item Whenever one of the text fields in this dialog has the focus (and
         that is usually the case), then
         \textsl{any} keystroke, including keys like
         \texttt{Ctrl},
         \texttt{Alt}, and
         \texttt{Super} (also known as Mod4 or the Windows key),
         can alter the value of a
         field to that of the keystroke.  This change is very easy to do
         accidentally!  \textsf{Use the mouse} to move this window and to click
         its \textbf{OK} button!
      \item Some of the keys traditionally used (or used by default) for
         control have been adapted for other uses, and are not configurable.
         One example is \texttt{Ctrl-L}, which brings up the learn mode
         that can be started using the "L" button or the "glearn"
         (group-learn) MIDI control.
         Some other hard-wired keystrokes are the "arrow" keys or "page
         up/down" keys.
      \item \textsl{Sequencer66} has appropriated the
         \index{keys!shift} Shift key so that it
         modifies a click on a pattern so that all of the other patterns are
         \textsl{toggled}.  Therefore, using characters that require the Shift
         key while clicking, such as \texttt{\{} and \texttt{\}}, when used
         to set the \textbf{Replace} function, becomes surprising.
         Instead, look to the remaining keys: \texttt{F11}, \texttt{F12},
         and the "keypad" keys if more options are wanted.  Be sure to
         look at the \textbf{Ext Keys} tab to see what other keys are in use.
         \index{auto-shift}
         Also, for the group-learn feature, the \texttt{Shift} key is 
         automatically enabled, using an "auto-shift" feature.
   \end{itemize}

\begin{figure}[H]
   \centering 
%  \includegraphics[scale=0.65]{new/menu_file_options_keyboard.png}
   \includegraphics[scale=0.65]{roll.png}
   \caption{File / Options / Keyboard}
   \label{fig:seq66_menu_file_options_keyboard}
\end{figure}

   \texttt{[keyboard-control]}.
   We won't attempt to cover every user-interface item in this busy
   dialog, just the categories.  Some items might be discussed in other parts
   of this manual.

   There are some things to note in the \texttt{[keyboard-control]} section.
   First, it is laid out like the main patterns panel, in an 8 x 4 grid.
   The keys in there correspond directly to the patterns panel slot, and the
   conventional keyboard layout is shown.
   Second, if one is looking for more keys to map to other functions, the
   default layout keeps open the following keys:
   \texttt{9 o l 0 p}.
%  Third, these key-mappings are stored in the
%  \texttt{[keyboard-control]} section of the "rc" configuration file.

   The \texttt{[mute-group]} section is laid out in a similar manner, except
   that the \textbf{upper-case} versions of the keys are used.  Additional keys
   are available for other functions:
   \texttt{( O L ) P}.
   Currently, one must be very carefully about assigning the same key to
   different functions.  Confusion will ensue!  We've done it!
%  Note that these key-mappings are stored in the
%  \texttt{[mute-group]} section of the "rc" configuration file.

%  \index{new!pause}
   \index{pause}
%  If the application has been built with the "pause" option, an
   An additional key definition is shown for the Pause key.
   By default, the Pause key is the period (".").  An old version of
   the "rc" file is automatically fixed to include this new option.
%  (The pause feature can be removed by rebuilding the application
%  after configuring with the \texttt{--disable-pause} option, but
%  why would you want to do that?)

% \begin{figure}[H]
%    \centering 
%    \includegraphics[scale=0.75]{new/keyboard-options-0_9_10_1.png}
%    \caption{File / Options / Keyboard, with Pause}
%    \label{fig:seq66_menu_file_options_keyboard_pause}
% \end{figure}

   \index{pattern edit}
   New features try to achieve being able to edit a pattern using only the
   keyboard.  \textsl{Sequencer66} now supports two modifier keys.
   The first modifier key causes the usual pattern-toggle key (hot-key) for a
   given slot to instead bring up the pattern editor.  By default, this key is
   the equals ("=") key.
   \index{event edit}
   The second modifier key causes the usual
   pattern-toggle key (hot-key) for a given slot to instead bring up the event
   editor.  By default, this key is the minus ("-") key.
   These keys are configurable in the
   \textbf{File / Options / Ext Keys} page.
%  As with the other
%  keys, these keys can be reconfigured to a different set of keys in the
%  \textbf{File / Options / Keyboard} page.

   To continue with a listing of the keyboard options:

   \begin{enumber}
      \item \textbf{Show sequence hot-key labels on sequences}
      \item \textbf{Show sequence numbers on sequences}
      \item \textbf{Control keys [keyboard-group]}
      \item \textbf{Sequence toggle keys [keyboard-control]}
      \item \textbf{Mute-group slots [mute-group]}
      \item \textbf{Learn}
      \item \textbf{Disable}
      \item \textbf{Enable}
   \end{enumber}

   These categories are described below.

   \setcounter{ItemCounter}{0}      % Reset the ItemCounter for this list.

   \itempar{Show key labels on sequence}{keyboard!show labels}
   This option shows the key labels in the lower-right corner of
   each loop/pattern slot in the Patterns window (the main window).
   It is useful for live playback and control of a song.
   It is configurable in the "rc" configuration file.
   It also enables the display of the pattern length, in
   measures, at the top right of the pattern slot.

   \itempar{Show sequence numbers on sequence}{keyboard!sequence numbers}
   \index{new!sequence numbers}
   If this option is on, the
   empty slots in the pattern window show the prospective sequence number.
   See the following figure for one look of this feature.

\begin{figure}[H]
   \centering 
%  \includegraphics[scale=0.75]{pattern-window-with-numbering.png}
%  \includegraphics[scale=0.65]{new/pattern-window-with-numbering.png}
   \includegraphics[scale=0.65]{roll.png}
   \caption{Pattern Window with One Kind of Numbering}
   \label{fig:seq66_build_with_numbering}
\end{figure}

   The option also changes the visibility of sequence numbers
   in active sequences and in the Song Editor's names column.
   If one doesn't like it, turn off the option in the "rc" configuration file,
   or try other grid options in the "user" configuration file.

   \itempar{Control keys}{keyboard!control keys}
   \texttt{[keyboard-group]}.
   This block of fields in the \textbf{Options / Keyboard} tab
   provides shortcut keys for many operations of
   \textsl{Sequencer66}.  There is a default mapping built into
   \textsl{Sequencer66}, but open the keyboard options tab to see
   the actual values.
   
%  Some of the old \textsl{Seq24} defaults were ill-advised.

   \begin{enumber}
      \item \textbf{Start}.
      \item \textbf{Stop}.
      \item \textbf{Pause}.
      \item \textbf{Slot Shift}.
      \item \textbf{Snapshot 1}.
      \item \textbf{Snapshot 2}.
      \item \textbf{bpm up}.
      \item \textbf{bpm down}.
      \item \textbf{Replace}.
      \item \textbf{Queue}.
      \item \textbf{Keep queue}.
      \item \textbf{Screenset down}.
      \item \textbf{Screenset up}.
      \item \textbf{Set playing screenset}.
   \end{enumber}

   Some of the keys have positional mnemonic value.  For example,
   for BPM control, the semicolon is at the left (down), and the apostrophe
   is at the right (up).
   Note that the keys definable in this tab are only a subset of the
   various keys that can be used, especially keys used with the
   \texttt{Ctrl} key or other modifier keys.

   \index{slot-shift}
   \index{keys!slot-shift}
   The \textbf{slot shift} key is useful when using pattern grids larger
   than 8 x 4 patterns.  Pressing the slot-shift key basically adds 32 to the
   pattern number of the slot-key that is pressed.
   Not all builds of \textsl{Sequencer66} support this option.

   \index{snapshot}
   \index{keys!snapshot}
   A \textbf{snapshot} is a briefly-preserved state of the patterns.
   One can press a snapshot key, change the state of the patterns for live
   playback, and then release the snapshot key to revert to the state when
   the snapshot key was first pressed.

%   Holding 'Alt' will save the state of playing sequences
%   and restore them when 'Alt' is lifted.
%
%   Holding 'Left Ctrl' and 'Alt' at the same time will enable
%   you to flip over to new sequences briefly and then
%   flip right back upon lifting 'Alt'.
%
%	Is this Snapshot 1 versus Snapshot 2?  In Seq24's code, either key
%  does exactly the same thing!

   \index{queue}
   \index{keys!queue}
   To \textbf{queue}
   a pattern means to ready it for playback upon the next repeat
   of a pattern.  A pattern can be armed immediately with a hot-key,
   or it can be queued to play back the next time the pattern repeats.
   A pattern can be queued by holding the queue key (defined in
   \textbf{File / Options / Keyboard / queue}) and pressing a pattern-slot
   hot-key.  Instead of the pattern turning on
   immediately, it turns on at the next repeat of the pattern.

   \index{keep queue}
   \index{keys!keep queue}
   \index{queue!keep}
   \textbf{Keep queue}
   allows the queue to be held without holding
   down the queue button the whole time.  First, press the keep-queue key
   (defined in \textbf{File / Options / Keyboard / Keep queue}).  Now, hitting
   any of the slot hot-keys, no matter how many, sets up the corresponding
   pattern slot to be queued.  Also, in keep-queue mode, clicking on the
   pattern slot will queue the pattern.  The keep-queue mode is disabled by
   hitting the "queue" key again (any currently active queues remain active
   until finished).  There is also a "Q" button to toggle the keep-queue
   status.

   Be sure to note the new option, \textbf{one-shot queue}, in the
   extended-keys section (\sectionref{subsubsec:seq66_patterns_pattern_slot}).

   \itempar{Sequence toggle keys}{keyboard!sequence toggle keys}
   Each of these keys toggles the playing/muting of one of the 32
   loop/pattern boxes.  These keys are layed out logically on the keyboard,
   and can also be shown in each loop/pattern box.  No need to list them all
   here!  Please note that we often call them "shortcut keys" or
   "hot-keys" where the context
   makes it clear that they apply to the armed/unarmed state of a pattern.

   \itempar{Mute-group slots}{keyboard!mute-group slots}
   There can be up to 32 mute-groups.
   \index{playing set}
   When activated, a mute-group
   sets the muted/unmuted status of the current "playing set"
   to the pattern-muting statuses of the selected mute-group.
   Each of these keys operates on the mute-grouping of one of the 32
   stored mute groups.
   These keys are layed out logically on the keyboard.
   No need to list them all here!
   Generally, they are the shifted versions of the
   keyboard keys used as hot-keys for the patterns.
   Note that a mute-group key will be memorized only when
   \textsl{Sequencer66} is in
   \index{group-learn} \textsl{group-learn} mode.

%  \index{mute-group}
%  One thing to explain is just what mute-grouping means.
%  \textsl{Mute groups} are shortcuts to play a defined group of patterns
%  on the current set, while stopping other patterns from the current set, and
%  all patterns from other sets.

   \itempar{Learn}{keyboard!learn}
   \index{group!learn}

   To define the group of patterns for one mute group, press and hold the
   configured Learn key (the \texttt{Insert} key by default,
   the \texttt{Ctrl-L} key, or the "L" button in the user-interface.
   Simultaneously (not needed with the "L" button),
   press one of the mute group keys: \textsl{Sequencer66}
   will save the currently-playing pattern slots into the corresponding mute
   group.
   \index{auto-shift}
   The default mute group keys must be the shifted version of the key,
   but one does not need the \texttt{Shift} key while pressing
   \texttt{Insert} to learn the group, only to trigger it.
   \textsl{Sequencer66} will automatically assign the corresponding key with
   \texttt{Shift} activated.  Try pressing the \texttt{Shift} key in Learn mode
   and see what happens!

%  Now, on some keyboards (like the author's), 
%  the \texttt{Insert} key is a clumsy two-key button.  So, an alternative
%  is to click the \index{L button} \textbf{L button} and release it,
%  then hit the desired mute-group (no need for the \texttt{Shift} key).
%  One can also press \texttt{Ctrl-L} to enter group-learn.

   Group-mute can be globally enabled or disabled (with default keys apostrophe
   \texttt{'} \index{grave} \index{igrave} and igrave or grave \texttt{`}).
   So make sure it is enabled before trying to use it.

%  \itempar{Learn}{keyboard!learn}
%  \index{group!learn}
%  Learn (while pressing a mute-group key).
%  This items sets the key used to initiate a learn mode.
%  It is the \textbf{Insert} key by default.
%  \index{auto-shift}
%  \index{group-learn!auto-shift}
%  When in group-learn mode, the \texttt{Shift} key cannot be hit, so the
%  group-learn mode automatically converts the keys to their shifted versions.
%  \index{shift-lock}
%  \index{group-learn!shift-lock}
%  This feature known as \textsl{shift-lock} or \textsl{auto-shift}.
%  It is new to \textsl{Sequencer66}.
%
%  Also, currently necessary because pressing \texttt{Shift} can clear the
%  arming of the patterns in the current set.

   \itempar{Disable}{keyboard!disable}
   \index{keys!apostrophe}
   It is the \textbf{apostrophe} key by default.
   \index{group!off}
   \index{keyboard!group off}
   This key is the \textsl{group off} key.

   \itempar{Enable}{keyboard!enable}
   \index{keyboard!igrave}
   It is the \textbf{igrave} (back-tick) key by default.
   \index{group!on}
   \index{keyboard!group on}
   This key is the \textsl{group on} key.

\paragraph{Menu / File / Options / Ext Keys }
\label{paragraph:seq66_menu_file_options_ext_keys}

   \texttt{[extended-keys]}.
   A number of additional functions have been added to \textsl{Sequencer66},
   and keystrokes have been provided for those new functions, in the
   \textbf{Ext Keys} page.  This page is needed because the original page is
   completely filled.  The new tab has its own section in the "rc" file.

\begin{figure}[H]
   \centering 
%  \includegraphics[scale=0.75]{menu/menu_file_options_ext_keys_condensed.png}
%  \includegraphics[scale=0.65]{new/menu_file_options_ext_keys_condensed.png}
   \includegraphics[scale=0.65]{roll.png}
   \caption{File / Options / Ext Keys (Condensed View)}
   \label{fig:seq66_menu_file_options_ext_keys}
\end{figure}

% Currently there is only one block of fields.  New blocks should also be
% marked by "itempar".

   \setcounter{ItemCounter}{0}      % Reset the ItemCounter for this list.

   \itempar{Ext Keys}{keyboard!extended keys}
   This block of fields in the \textbf{Options / Ext Keys} tab
   provides shortcut keys for more operations of \textsl{Sequencer66}, many of
   them ported from \textsl{Seq32} or \textsl{Kepler34}.

   \begin{enumber}
      \item \textbf{Song/Live toggle}.
      \item \textbf{Toggle JACK}.
      \item \textbf{Menu mode}.
      \item \textbf{Follow transport}.
      \item \textbf{Fast forward}.
      \item \textbf{Rewind}.
      \item \textbf{Pointer Position}.
      \item \textbf{Toggle mutes}.
      \item \textbf{Tap BPM}.
      \item \textbf{Song Record}.
      \item \textbf{One-shot Queue}.
         This new feature allows one-shot queuing of a pattern.
   \end{enumber}

   Most of these extended keys implement operations performed with button
   presses.  Some of the new keystrokes may not have a corresponding
   button.
   These values are saved as the \texttt{[extended-keys]} section of the "rc"
   configuration file.
%  However, not all builds of \textsl{Sequencer66} will
%  support the new keys.  Some of the new features can be enabled or disabled
%  during the build-configure step, and one's favorite Linux distro may decide to
%  disable some features.
%  An option might be disabled during the build process.
%  In this case, although the values will still be
%  stored in the "rc" file, they will be disabled or missing in this tab:

\begin{figure}[H]
   \centering 
%  \includegraphics[scale=0.65]{menu/menu_file_options_ext_keys_disabled.png}
   \includegraphics[scale=0.65]{roll.png}
   \caption{File / Options / Ext Keys (Disabled)}
   \label{fig:seq66_menu_file_options_ext_keys_disabled}
\end{figure}

   \index{song mode}
   Note the \textbf{Song/Live toggle} key.
   The \textsl{song mode} normally is in effect only when playback is started
   from the \textbf{Song Editor}.  Now this mode can be used from any
   window, if enabled by pressing this key.  There is also
   a button in the main window for this function, which shows the current state
   of this flag.  Note that this flag is also stored in the "rc" configuration
   file, as well as this hot-key value, which defaults to \texttt{F1}.

   \index{toggle JACK}
   \index{JACK toggle}
   The \textsl{JACK mode} is set via the
   \textbf{File / Options / JACK / JACK Connect} or 
   \textbf{JACK Disconnect} buttons.
%  But, if \textsl{Sequencer66} is built for
%  using the \textsl{Seq32} JACK support, then
   This keystroke will toggle between JACK connect and JACK disconnect.
%  Note that, with this kind of build,
   The \textbf{Song Editor} will also have a \textbf{JACK} button.
   The hot-key for this function defaults to \texttt{F2}.

   \index{menu mode}
   The \textsl{menu mode} indicates if the main menu of the
   main window is accessible or not.  It is disabled during playback
   so that more hot-keys can be used without triggering menu functions.
   It can also be disabled by the user; the default hot-key is \texttt{F3}.
   This feature is needed because the original \textsl{Seq24} had numerous
   conflicts between the menu key bindings and the default key bindings for the
   main window.

%  Here is Stazed's explanation of the feature, mildly edited:
%
%  \begin{quotation}
%     \textsl{"why disabling is needed when playing"}
%     The original seq24 had numerous conflicts between the menu key binding
%     and the default seq24 key binding for the mainwind sequence triggers.
%     For example: Ctrl-q (quits the program without prompt). If you place a
%     sequence in the default 'q' slot, you cannot use it with Ctrl-l or Ctrl-r
%     (default replace or queue) because the menu grabs the keys. Same goes for
%     the Alt-l or Alt-r (default snapshot 1 or 2). Try same as above with
%     Alt-f, Alt-v, Alt-h, Ctrl-n, Ctrl-o...  etc. So I just shut off all the
%     menus by default when playing because it seems that they should not be
%     needed then... especially in a live performance.
%
%     \textsl{"why a button?"}
%     On occasion I wanted to use the mainwnd key binding when stopped to set
%     the sequences to be ready before starting. It's also a sort of safety
%     feature as well, just toggle the menus off before going live so that you
%     don't hit Ctrl-q, Ctrl-n etc. forgetting things are not playing....
%  \end{quotation}

   \index{follow transport}
   \textsl{Follow transport} is a feature ported from \textsl{Seq32}.
   The default key is \texttt{F4}.
   It determines if \textsl{Sequencer66} follows JACK transport.

   \index{fast forward}
   \textsl{Fast forward} is a feature ported from \textsl{Seq32}.
   The default key is \texttt{F5}.
   While this key is held, the song pointer will fast-forward
   through the song.
   This feature does not have a corresponding button.
   This feature requires that the \textsl{Seq32} transport option be
   enabled at build time.

   \index{rewind}
   \textsl{Rewind} is a feature ported from \textsl{Seq32}.
   The default key is \texttt{F6}.
   While this key is held, the song pointer will rewind.
   This feature does not have a corresponding button.
   This feature requires that the \textsl{Seq32} transport option be
   enabled at build time (and now that is the default).
   Be sure not to use the following keys, which are already
   hardwired for other functions in the Pattern Editor and Song Editor:

   \begin{itemize}
      \item \texttt{p}.  Paint mode.
      \item \texttt{x}.  Escape paint mode.
   \end{itemize}

   \index{pointer position}
   \textsl{Pointer position} is a feature ported from \textsl{Seq32}.
   The default key is \texttt{F7}.
   When this key is pressed, the song pointer will move to the
   current position of the mouse, snapped.
   This feature does not have a corresponding button.

   \index{toggle mutes}
   \textsl{Toggle mutes} toggles the mute status of every
   pattern on every screen-set.  It corresponds to the
   \textbf{Edit / Toggle mute all tracks} or the 
   \textbf{Song / Toggle All Tracks}
   menu entries.  There is also a button in the main window for this function,
   which shows the current state of this flag.  Note that this
   hot-key value is stored in the "rc" configuration file, and
   defaults to \texttt{F8}.

   \index{tap bpm}
   \textsl{Tap bpm} allows the user to "tap" in time with some
   other music, and see the tap sequence translated into beats/minute (BPM).
   There is also a "0" button for this function.
   After 5 seconds, this feature resets automatically, so the user can try
   again if not satisfied.  At least two taps are needed for the
   BPM to be registered.

% VERIFY and the UNCOMMENT
%
%  Tap BPM causes events to be logged to the tempo track which is the first
%  track (track 0) by default.

\paragraph{Menu / File / Options / Mouse }
\label{paragraph:seq66_menu_file_options_mouse}

   This item selects the mouse-interaction method.

\begin{figure}[H]
   \centering 
%  \includegraphics[scale=0.65]{menu/menu_file_options_mouse_condensed.png}
   \includegraphics[scale=0.65]{roll.png}
   \caption{File / Options / Mouse (Condensed View)}
   \label{fig:seq66_menu_file_options_mouse}
\end{figure}

   \index{interaction method}
   \index{mouse interaction}
   \textbf{Interaction Method}

   The default mouse interaction method is \textbf{Seq24 (original style)}.
   The alternate mouse interaction method is \textbf{Fruity (similar to a
   certain well known sequencer)}.
   The "Fruity" interaction method
   is currently only available in the Gtkmm-2.4 user-interface, and it
   is not comprehensive.
   For Qt 5, only the original style is available, at this time.

   \index{mouse!fruity}
   The alternate method is presumably that of the \textsl{Fruity Loops}
   (now \textsl{FL Studio}) sequencer.  The fruity mode seems to involve the
   following rules:

   \begin{itemize}
      \item \textbf{Left-click left side}.
         Begin a grow/shrink operation for the left side.
         However, even in \textsl{Seq24}, this action is \textsl{broken}.
         It does allow one to move the note, however.
      \item \textbf{Left-click right side}.
         Begin a grow/shrink operation for the right side.
      \item \textbf{Left-click middle}.
         Move the object.  To clarify, each note image
         has an invisible "handle" on the left or the right side, with the
         middle providing a third area of interaction in the "fruity" mode.
      \item \textbf{Left-click}.
         Add an event if nothing selected.
      \item \textbf{Middle-click}.
         Split the note?
   \end{itemize}

   There may be a few more rules to add, when time allows.

% This paragraph needs to be moved to the pattern editor.

   The \textsl{Seq24} note-editing style is as expected for basic
   actions such as selecting and moving notes using the left mouse button.
   Drawing a note or event is a bit different, in that one must first
   \textsl{click and hold} the right mouse button, and then
   \textsl{click and drag} the right mouse button to insert notes,
   Notes are inserted to be at the current length and grid-snap values for
   the sequence editor for as long as the buttons are pressed.
   Notes are inserted only up to the specified sequence length.
   Once notes are inserted, moving the mouse with the left button still
   held down moves the notes to the new note value of the mouse.
   If one releases the left button, then presses and holds it again,
   more notes will be added in the same way.
   This is unconventional, but a powerful way to layer notes in a short
   sequence.
   We call it the
   \index{draw mode}
   \index{mode!draw}
   "draw mode" or
   \index{paint mode}
   \index{mode!paint}
   "paint mode".
   Drawing/painting can also be done while the sequence is playing,
   and notes will be added to be played the next time the progress bar crosses
   them.

%  \index{sequencer66 options}
%  The \textbf{Sequencer66 Options} section contains a couple of new options.

   \index{keys!Mod4}
   \index{mouse!Mod4}
   \label{new_mod4_mode}
   \textbf{Mod4 key preserves add (paint) mode in song and pattern editors}.
   In order to work with trackpads that don't permit simultaneous left
   and right clicks, the
   "Seq24" mode of mouse interaction can be modified in the
   Pattern or Song editors so that the Mod4 key (Super or Windows key)
   can be pressed when releasing the right mouse button.
   This keeps the mouse in note-add mode.
   Another right-click, without pressing Mod4, will exit this mode.

%  The reason for this feature is the crummy FocalTech touchpad on one of
%  the author's laptops.  This trackpad seems to have only a single button,
%  which the driver interprets as left or right depending where the finger
%  is when it is clicked.  There's no way to click the right and left
%  buttons at the same time.  There's no way to make a middle-click action.
%  What a crock!

   This option will not interfere with the Mod4 key being set
   in the \textbf{Keyboard} option tab, since the keys there mainly apply to
   the Patterns Panel (main window), not the pattern-editor window.

   \index{mouse!split mode}
   \label{new_split_mode}
   Middle click splits song triggers at nearest snap (instead of
   the halfway point).
% Move this section to the right place and simply create a section-reference to
% it here.

   \index{paint mode}
   Another way to turn on the paint mode has been added.
%  , based on a feature
%  found in a patch that someone posted about in some mailing list somewhere on
%  the internet.
   To turn on the paint mode, press the
   \index{keys!p}
   \texttt{p} key while in the sequence editor.
   This is just like pressing the right mouse button, but the draw/paint mode
   stays on.
   To get out of the paint mode, press the
   \index{keys!x}
   \texttt{x} key while in the sequence editor.
   These keys, however, do not work while the sequence is playing.

%  \index{todo:extend mouse support}
%  These convenience options are limited to the
%  pattern/sequence editor window and the performance editor window, and may
%  need some heavier testing.
   Note that some \textsl{Sequencer66} windows
   can use the ctrl-left-click as a middle click. 
 
\paragraph{Menu / File / Options / Jack Sync}
\label{paragraph:seq66_menu_file_options_jack_sync}

   This tab sets up JACK transport, if \textsl{Sequencer66}
   was built with JACK support.
%  It now also supports native JACK MIDI.
   This tab also sets up options for using LASH session management, \textsl{if}
   \textsl{Sequencer66} was built with LASH support, which is no longer the
   default, even though it is shown in the figure below.

\begin{figure}[H]
   \centering 
%  \includegraphics[scale=0.75]{menu/menu_file_options_jack_sync.png}
%  \includegraphics[scale=0.75]{new/menu_file_options_jack_sync.png}
%  \includegraphics[scale=0.65]{jack/menu_file_options_jack_sync.png}
   \includegraphics[scale=0.65]{roll.png}
   \caption{File / Options / JACK}
   \label{fig:seq66_menu_file_options_jack_sync}
\end{figure}

   The main sections in this dialog are:

   \begin{enumber}
      \item \textbf{JACK Transport/MIDI}
      \item \textbf{JACK Start Mode}
      \item \textbf{JACK Transport Connect and Disconnect}
      \item \textbf{LASH Options}
   \end{enumber}

   \setcounter{ItemCounter}{0}      % Reset the ItemCounter for this list.

   \itempar{Transport/MIDI}{jack sync!transport/midi}
   These settings are stored in the "rc" file settings group
   \texttt{[jack-transport]}.
   See \sectionref{subsec:seq66_rc_file_jack_transport},
   which describes this configuration option.
   This items collects the following settings:

   \begin{itemize}
      \item \textbf{Jack Transport}.
         \index{JACK!transport}
         Enables slave synchronization with JACK Transport.
         The command-line option is \texttt{--jack-transport}.
         The behavior of this mode of operation is perhaps not quite
         correct.  Even as a slave, \textsl{Sequencer66} can start and
         stop playback.
         Note that this option cannot be disabled via the mouse if the
         \textbf{Transport Master} option is enabled.  Disable that one first.
      \item \textbf{Transport Master}.
         \index{JACK!transport master}
         \textsl{Sequencer66} will attempt to serve as the JACK Master.
         The command-line option is \texttt{--jack-master}.
         \textbf{Tip}:
         \textsl{Sequencer66} generally works better as JACK Master.
         If this option is enabled the \textbf{JACK Transport} option is
         automatically enabled as well.
      \item \textbf{Master Conditional}.
         \index{JACK!master conditional}
         \textsl{Sequencer66} will fail to serve as the JACK Master if there is
         already a Master.
         The command-line option is \texttt{--jack-master-cond}.
         If this option is enabled the \textbf{JACK Transport} option is
         automatically enabled as well.
      \item \textbf{Native JACK MIDI}.
         \index{JACK!native midi}
         This option is for the \texttt{seq66} version of
         \textsl{Sequencer66}.
         If set, MIDI input and output use native JACK MIDI,
         rather than ALSA.  However, if JACK is not running on the
         system, then \texttt{seq66} will fall back to ALSA mode.
         The command-line option is \texttt{--jack-midi}.
   \end{itemize}

%  Note that there are long-standing issues with the JACK support of
%  \textsl{Seq24}, and \textsl{Sequencer66} currently inherits some of them,
%  in spite of some bug fixes.  Generally, if one experiences issues in
%  transport control, try making one of the other sequencer applications the
%  JACK Master.
%  If one starts \textsl{Sequencer66} in JACK mode without JACK running,
%  it will take a little while for \textsl{Sequencer66} to start up, and it
%  will fall back to ALSA usage.

   If one makes a change in the JACK transport settings, it is best to
   then press the \textbf{JACK Transport Disconnect} button, then the
   \textbf{JACK Transport Connect} button.  Another option is to restart
   \textsl{Sequencer66}... the settings are automatically saved when
   \textsl{Sequencer66} exits.

   \itempar{JACK Start mode}{jack sync!start mode}
   This item collects the following settings, also stored in the "rc" file
   settings group \texttt{[jack-transport]}.

   \begin{itemize}
      \item \textbf{Live Mode}.
         \index{JACK!live mode}
         \index{live mode}
         \index{non-playback mode}
         Playback will be in live mode.  Use this option to allow muting and
         unmuting of patterns.  This option might also be called "non-song
         mode".
         The command-line option is \texttt{--jack-start-mode 0}.
      \item \textbf{Song Mode}.
         \index{JACK!song mode}
         \index{song mode}
         \index{playback mode}
         \index{performance mode}
         Playback will use only the Song Editor's data.
         The command-line option is \texttt{--jack-start-mode 1}.
   \end{itemize}

   \textsl{Sequencer66} also selects the playback modes
   according to which window started the playback,
   reverting back to legacy \textsl{Seq24} behavior.
   \textsl{The main window}, or pattern
   window, causes playback to be in live mode.  The user can arm and mute
   patterns in the main window by clicking on sequences, using their hot-keys,
   and by using the group-mode and learn-mode features.
   The song editor causes playback to be in performance mode, also known as
   "playback mode", or \textbf{Song} mode.

   \itempar{Connect}{jack sync!connect}
   Connect to JACK Sync.
   This button is useful to restart JACK sync when making changes to it,
   or when \textsl{Sequencer66} was started in ALSA mode.

   \itempar{Disconnect}{jack sync!disconnect}
   Disconnect from JACK Sync.
   This button is useful to stop JACK sync when making changes to it.

   \itempar{LASH Options}{lash!option}
   Currently contains only one item, which enables the usage of LASH session
   management.  Currently, \textsl{Sequencer66} needs to be restarted to
   complete the enabling or disabling of LASH support.  Like the rest of the
   options, this one is written to the "rc" configuration file.
   However, LASH is no longer supported in the default build.

   Finally, there is a new button (labelled \textsl{Master} in the following
   figure) in the main window to bring up directly the
   \textbf{JACK} (or \textbf{JACK/LASH}) page.

\begin{figure}[H]
   \centering 
%  \includegraphics[scale=0.75]{menu/menu_file_options_jack_sync.png}
%  \includegraphics[scale=0.65]{new/main_master_button.png}
   \includegraphics[scale=0.65]{roll.png}
   \caption{JACK Connection Button}
   \label{fig:seq66_main_master_button}
\end{figure}

   This button not only brings up the JACK page, but also shows the current
   status of the MIDI connection:
   \textbf{Master} (JACK Transport Master),
   \textbf{Slave} (JACK Transport Slave),
   \textbf{JACK} (native JACK MIDI, overrides any transport label),
   and \textbf{ALSA} (overridden by any transport label).
   \index{jack page!ctrl-p}
   \index{keys!ctrl-p}
   The \texttt{Ctrl-P} key will also bring up this page.

   One thing to note is that, while playing, the JACK/ALSA button is disabled.
   However, one can still get to the JACK options via the main File menu.
   JACK connection and disconnection are disabled during playback, but the
   buttons don't yet reflect that status.

\subsection{Menu / Edit}
\label{subsec:seq66_menu_edit}

   The \textbf{Edit} menu has undergone some expansion lately.

\begin{figure}[H]
   \centering 
%  \includegraphics[scale=0.65]{new/menu_edit_0_90.png}
   \includegraphics[scale=0.65]{roll.png}
   \caption{Edit Menu}
   \label{fig:seq66_menu_edit_0_90}
\end{figure}

   \begin{enumber}
      \item \textbf{Preferences...} (only in the Qt user-interface)
      \item \textbf{Song Editor...}
      \item \textbf{Apply song transpose}
      \item \textbf{Clear mute groups}
      \item \textbf{Reload mute groups}
      \item \textbf{Mute all tracks}
      \item \textbf{Unute all tracks}
      \item \textbf{Toggle mute all tracks}
   \end{enumber}

   \setcounter{ItemCounter}{0}      % Reset the ItemCounter for this list.

   \itempar{Preferences}{edit!preferences}
   In the Qt user-interface, there is a \textbf{Preferences} menu entry,
   corresponding to the Gtkmm user-interface's \textbf{File / Options}
   menu entry.

   \itempar{Song Editor}{edit!song editor}
   \index{song editor}
   This item is the same as the 
   \textbf{View / Song Editor toggle} menu entry.  It toggles the presence of
   the main song editor.

   \itempar{Apply song transpose}{edit!song transpose}
   \index{song transpose}
   Selecting this item applies the song transposition value to
   \textbf{all} sequences/patterns that are marked as transposable.
   (Normally, drum tracks are \textsl{not} transposable).
   This actively changes the note/pitch value of all note and aftertouch events
   in the pattern.
%  Once the transpositions are done, the transposition value is set to 0.
   For the setting of song transpose, see
   \sectionref{sec:seq66_song_editor}, for more information.
   Also note that transpose can be enabled in the patterns panel for each
   pattern (see \sectionref{subsubsec:seq66_patterns_pattern_filled}) and
   in the sequence editor
   (see \sectionref{sec:seq66_pattern_editor}).

   \itempar{Clear mute groups}{edit!clear mute groups}
   \index{mute groups}
   A feature of \textsl{Seq24} and \textsl{Sequencer66} is that the mute groups
   are saved in both the "rc" file
   (see \sectionref{subsec:seq66_rc_file_mute_group})
   \textsl{and} in the "MIDI" file
   (see \sectionref{subsec:legacy_midi_format}).

   This menu entry clears them. If this resulted in any mute-group sequences
   status being set to false, then the user is prompted to save the MIDI
   file, so that it will no longer have any
   mute-group information.  And then, if the
   application exits, the cleared mute-group information is also saved to
   the "rc" file.
%  We'd like to be able to handle the "rc" and "MIDI"
%  mute-groups separately in the future.

   \itempar{Reload mute groups}{edit!load mute groups}
   \index{rc!mute groups}
   This menu entry reloads the mute-groups from the "rc" file.
   So, if one loads a MIDI file that has its own mute groups that one does not
   like, this command will restore one's favorite mute-grouping from the "rc"
   file.

   \itempar{Mute all tracks}{edit!mute all tracks}
   \index{mute all}
   This menu entry, available only in \textbf{Live} mode,
   immediately mutes \textsl{all} patterns in the entire song.

   \itempar{Unmute all tracks}{edit!unmute all tracks}
   \index{unmute all}
   This menu entry, available only in \textbf{Live} mode,
   immediately unmutes \textsl{all} patterns in the entire song.

   \itempar{Toggle mute all tracks}{edit!toggle all tracks}
   \index{toggle mute all}
   This option toggles the mute/armed status of \textbf{all} tracks.
   It is only available in \textbf{Live} mode, which overrides \textbf{Song}
   mode even if the Song Editor is focussed.
   \textsl{Do not confuse it with the main \textbf{Mute} button, which toggles the
   status only of the tracks that are armed and remembers them.}

\subsection{Menu / View}
\label{subsec:seq66_menu_view}

   If the "allow two perfedits" option is turned off in the "user"
   configuration file, this menu item has only one entry, \textbf{Song Editor}, 
   which is already covered by a button at the bottom of the Patterns
   window.  Selecting this item bring up the Song Editor window.
   See \figureref{fig:song_editor_window}.
   The Song Editor window can also be brought up via the
   \index{song editor!ctrl-e}
   \index{keys!ctrl-e}
   Ctrl-E key.

   If the \textbf{allow two perfedits} option is turned on in the "user"
   configuration file, this menu item has two entries,
   as shown in the following figure:

\begin{figure}[H]
   \centering 
%  \includegraphics[scale=0.65]{menu/menu_view-dual-song-editors.png}
   \includegraphics[scale=0.65]{roll.png}
   \caption{Dual Song Editor Entries in View Menu}
   \label{fig:seq66_menu_view_song_editors}
\end{figure}

   Note that only the first Song Editor has a user-interface button and
   a hot-key.  Also note that there can be issues bringing up the second
   song-editor with the hot-key.  The menu entry will always work.
   If two song editors are up, they each track any changes made in the other
   song editor.  But the main purpose of two song editors is to arrange two
   different parts of the performance at the same time when not all the
   patterns will fit in one window.

   In the Qt user-interface, there is a \textbf{Song} tab in the main window.
   But there is also song-editor button in the main window
   and a menu entry in \textbf{Edit / Song Editor},
   which bring up a separate window for the song-editor.
   Do not try to use both windows for song-editing at the same time; they
   are not synchronized.

\subsection{Menu / Help / About...}
\label{subsec:seq66_menu_about}

   This menu entry shows the "About" dialog.

\begin{figure}[H]
   \centering 
%  \includegraphics[scale=0.75]{menu/menu_help_about.png}
%  \includegraphics[scale=0.65]{new/menu_help_about.png}
   \includegraphics[scale=0.65]{roll.png}
   \caption{Help / About}
   \label{fig:seq66_menu_help_about}
\end{figure}

   The Qt version is slightly different.
%  That dialog provides access to the credits for the program, including the
%  authors and the project documentors.  It has recently been updated
%  to show Git version-control information as well.

\begin{figure}[H]
   \centering 
%  \includegraphics[scale=0.65]{menu/menu_help_credits.png}
   \includegraphics[scale=0.65]{roll.png}
   \caption{Help Credits}
   \label{fig:seq66_menu_help_credits}
\end{figure}

   Shows who has worked on the program, with the original author at the top
   of the list.

\begin{figure}[H]
   \centering 
%  \includegraphics[scale=0.65]{menu/menu_help_doc.png}
   \includegraphics[scale=0.65]{roll.png}
   \caption{Help Documentation}
   \label{fig:seq66_menu_help_doc}
\end{figure}

   Shows who has documented this project.

\subsection{Menu / Help / Build Info...}
\label{subsec:seq66_menu_build_info}

   This menu entry shows the "Build Info" dialog.  This list of
   build options enabled in the current application is the same list
   that it generated via this command line:

   \begin{verbatim}
      $ seq66 --version
   \end{verbatim}

\begin{figure}[H]
   \centering 
%  \includegraphics[scale=0.50]{new/menu_help_build_info.png}
   \includegraphics[scale=0.65]{roll.png}
   \caption{Help / Build Info}
   \label{fig:seq66_menu_help_build_info}
\end{figure}

%-------------------------------------------------------------------------------
% vim: ts=3 sw=3 et ft=tex
%-------------------------------------------------------------------------------


% Patterns Panel

\input{seq66_patterns_panel}

% Pattern Editor

\input{seq66_pattern_editor}

% Song Editor

\input{seq66_song_editor}

% Event Editor

\input{seq66_event_editor}

% Import/Export

\input{seq66_midi_export}

% Tables of keyboard and mouse actions

%-------------------------------------------------------------------------------
% seq66_kbd_mouse
%-------------------------------------------------------------------------------
%
% \file        seq66_kbd_mouse.tex
% \library     Documents
% \author      Chris Ahlstrom
% \date        2016-04-07
% \update      2018-09-21
% \version     $Revision$
% \license     $XPC_GPL_LICENSE$
%
%     Provides tables for keyboard and mouse support in Sequencer66.
%
%-------------------------------------------------------------------------------

\section{Sequencer66 Keyboard and Mouse Actions}
\label{sec:kbd_mouse_actions}

   This section presents some tables summarizing keyboard and mouse actions
   available in \textsl{Sequencer66}.
   It does not cover mute keys and group keys, which are well
   described in the keyboard options for the main window.
   See \sectionref{paragraph:seq66_menu_file_options_keyboard}).
   It does not cover the "fruity" mouse actions, though they are touched
   on in \sectionref{paragraph:seq66_menu_file_options_mouse}.

%  Any volunteers to fill in the table?

   This section describes the keystrokes that are currently hardwired
   in \textsl{Sequencer66}.
   This description only includes
   items not defined in the \textbf{File / Options}
   dialog.  That is, hardwired values.
   "KP" stands for "keypad".
   \index{keys!focus}
   The effect that keystrokes have depends upon
   which window has the keyboard/mouse focus.
   \index{keys!qt}
   It must be noted that the Qt 5 user-interface does not (yet) support the
   full set of keystrokes supported by the legacy Gtkmm-2.4 user-interface.

\subsection{Main Window}
\label{subsec:kbd_mouse_main_window}

   The main window keystrokes are all defined via the options dialog
   and "rc" configuration file, or are stock Gtk window-management keystrokes.
   The main window has a very complete setup for live control of the MIDI tune
   via keystrokes.  These actions are not included in
   \tableref{table:main_window_support}.
%  There may be some other keystrokes to be documented at some point.

   \begin{table}[H]
      \centering
      \caption{Main Window Support}
      \label{table:main_window_support}
      \begin{tabular}{l l l l l l}
         \textbf{Action} & \textbf{Normal} & \textbf{Double} &
            \textbf{Shift} & \textbf{Ctrl} & \textbf{Mod4} \\
         \textbf{e} & --- & --- & --- & Open song editor & --- \\
         \textbf{l} (el) & --- & --- & --- & Enter Learn mode & --- \\
         Left-click slot & Mute/Unmute & New/Edit & Toggle other slots &
            --- & --- \\
         Right-click slot & Edit menu & --- & Edit menu & Edit Menu &
            --- \\
      \end{tabular}
   \end{table}

   The new mouse features of this window for \textsl{Sequencer66},
   as noted in \sectionref{sec:seq66_patterns_panel}, are:

   \begin{itemize}
      \item \textsl{Shift-left-click}:
         Over one pattern slot, this action toggles the mute/unmute
         (armed/unarmed) status of all other patterns
         (even the patterns in other, unseen sets).
      \item \textsl{Left-double-click}:
         Over a pattern slot, this action quickly toggles the mute/unmute status,
         which is confusing.  But it ultimately brings up the pattern editor
         (sequence editor) for that pattern.
%        It acts like Ctrl-left-click.
   \end{itemize}

\subsection{Performance Editor Window}
\label{subsec:kbd_mouse_performance_editor_window}

   The "performance editor" window is also known as the "song editor" window.
   It's main sections are the "piano roll" (perfroll) and the "performance
   time" (perftime) sections, discussed in the following sections.
   Also, some keystrokes are handled by the frame of the window.

   \begin{itemize}
      \item \texttt{Ctrl-z}. Undo.
      \item \texttt{Ctrl-r}. Redo.
   \end{itemize}

\subsubsection{Performance Editor Piano Roll}
\label{subsubsec:kbd_mouse_performance_editor_piano_roll}

%  \begin{itemize}
%     \item \texttt{Ctrl-x}. Cut.
%     \item \texttt{Ctrl-c}. Copy.
%     \item \texttt{Ctrl-v}. Paste.
%     \item \texttt{Ctrl-z}. Undo.
%     \item \texttt{Ctrl-r}. Redo.
%     \item \texttt{Shift-Up}.   Move backward one small unit (which is...?)
%     \item \texttt{Shift-Down}.   Move forward one small unit (which is...?)
%     \item \texttt{Shift-Page Up}.   Move backward one frame.
%     \item \texttt{Shift-Page Down}.   Move forward one frame.
%     \item \texttt{Shift-Home, Shift-KP Home}.  Move to beginning of piano roll.
%     \item \texttt{Shift-End, Shift-KP End}.  Move to end of piano roll.
%     \item \texttt{Shift-z (Z)}.  Zoom in.
%     \item \texttt{0}.  Set default zoom.
%     \item \texttt{z}.  Zoom out.
%     \item \texttt{Left}.  Move item left one snap unit.
%     \item \texttt{Right}.  Move item right one snap unit.
%     \item \texttt{Up}.  Move frame up one small scroll unit.
%     \item \texttt{Down}.  Move frame down one small scroll unit.
%     \item \texttt{Home}.  Move to top of piano roll.
%     \item \texttt{End}.  Move to bottom of piano roll.
%     \item \texttt{Page Up}.  Move up one frame (page-increment).
%     \item \texttt{Page Down}.  Move down one frame (page-increment).
%  \end{itemize}

   Note that the keystrokes in this table
   (see \tableref{table:perf_window_piano_roll})
   require that the focus first be
   assigned to the piano roll by left-clicking in an empty area within it.
   Otherwise, another section of the performance editor might receive the
   keystroke.

   \begin{table}[H]
      \centering
      \caption{Performance Window Piano Roll}
      \label{table:perf_window_piano_roll}
      \begin{tabular}{l l l l l l}
         \textbf{Action}   & \textbf{Normal} & \textbf{Double}    & \textbf{Shift}     & \textbf{Ctrl}   & \textbf{Mod4}      \\
         Space             & Start playback  & ---                & ---                & ---             & ---                \\
         Esc               & Stop playback   & ---                & ---                & ---             & ---                \\
         Period (.)        & Pause playback  & ---                & ---                & ---             & ---                \\
         Del               & Cut section     & ---                & ---                & ---             & ---                \\
         c key             & ---             & ---                & ---                & Copy            & ---                \\
         p key             & Paint mode      & ---                & ---                & ---             & ---                \\
         v key             & ---             & ---                & ---                & Paste           & ---                \\
         x key             & Escape paint    & ---                & ---                & Cut             & ---                \\
         z key             & Zoom out        & ---                & ---                & Undo            & ---                \\
         0 key             & Reset zoom      & ---                & ---                & ---             & ---                \\
         Z key             & Zoom in         & ---                & ---                & Undo            & ---                \\
         Left-arrow        & Move earlier    & ---                & ---                & ---             & ---                \\
         Right-arrow       & Move later      & ---                & ---                & ---             & ---                \\
         Left-click        & Select section  & ---                & ---                & ---             & ---                \\
         Right-click       & Paint mode      & ---                & Paint mode         & Paint mode      & Lock Paint mode    \\
         Scroll-up         & Scroll up       & ---                & Scroll Left        & Scroll Up       & ---                \\
         Scroll-down       & Scroll down     & ---                & Scroll Right       & Scroll Down     & ---                \\
      \end{tabular}
   \end{table}

   This section of the performance editor also handles the start, stop, and
   pause keys.  These can be modified in the \textbf{Options / Keyboard} page.
   A "section" in the performance editor is actually a box that
   specifies a trigger for the pattern in that sequence/pattern slot.
   Note that the "toggle other slots" action occurs only if shift-left-clicked
   in the "names" area of the performance editor.
   Left-click is used to select performance blocks if clicked within
   a block, or to deselect them if clicked in an empty area of the piano roll.
   Also note that all scrolling is done by the internal horizontal and vertical
   step increments.
   Some features of this window for \textsl{Sequencer66},
   as noted in \sectionref{sec:seq66_song_editor}, are explained here:

   \begin{itemize}
      \item \textsl{p}:  Enters the paint mode, until right-click is pressed or
         until the "x" key is pressed.
      \item \textsl{x}:  Exits the paint mode.  Think of the made-up term
         "x-scape".
      \item \textsl{z}:  Zooms out the performance view.  It makes the view
         look smaller, so that more of the performance can be seen.
         Opening a second performance view is another way to see more
         of the performance.
      \item \textsl{0}:  Resets the zoom to its normal value.
      \item \textsl{Z}:  Zooms in the performance view, making the view
         larger, so that more details of the performance can be seen.
%     \item \textsl{.}:  The period (configurable) is a new key devoted to the
%        new pause functionality.
      \item \textsl{Left Arrow}:  Moves the selected item to the left (earlier
         in time) in the performance layout.
      \item \textsl{Right Arrow}:  Moves the selected item to the right (later
         in time) in the performance layout.
      \item \textsl{Mod4-right-click, release}:  Locks the paint mode,
         until right-click is pressed.
      \item Once selected (rendered in grey), a pattern section (trigger)
         can be moved by the mouse.
         To move it using the left or right
         arrow keys, the paint mode must be entered, but only via the "p"
         key.
%        -- the right mouse button deselects the greyed pattern.
%        Too tricky, we might try fixing it later.
   \end{itemize}

\subsubsection{Performance Editor Time Section}
\label{subsubsec:kbd_mouse_performance_editor_time_section}

   \begin{itemize}
      \item \texttt{l}.  Set to move L marker.
      \item \texttt{r}.  Set to move R marker.
      \item \texttt{x}.  Escape ("x-scape") the movement mode.
      \item \texttt{Left}.  Move the selected marker left.
      \item \texttt{Right}.  Move the selected marker right.
   \end{itemize}

   This section of the performance editor is also known as the "measure ruler"
   or the "bar indicator", and is discussed in
   \sectionref{subsubsec:seq66_song_editor_arrangement_panel_measures_ruler}.
   See \tableref{table:performance_editor_time_section}.

   \begin{table}[H]
      \centering
      \caption{Performance Editor Time Section}
      \label{table:performance_editor_time_section}
      \begin{tabular}{l l l l l l}
         \textbf{Action}   & \textbf{Normal} & \textbf{Double}    & \textbf{Shift} & \textbf{Ctrl}   & \textbf{Mod4}      \\
         l                 & Move L [1]      & ---                & ---            & ---             & ---                \\
         r                 & Move R [1]      & ---                & ---            & ---             & ---                \\
         x                 & Escape Move     & ---                & ---            & ---             & ---                \\
         Left-Click        & Set L [2]       & ---                & ---            & ---             & ---                \\
         Middle-Click      & ---             & ---                & ---            & ---             & ---                \\
         Right-Click       & Set R [2]       & ---                & ---            & ---             & ---                \\
      \end{tabular}
   \end{table}

   \begin{enumerate}
      \item Activates movement of this marker using the left and right arrow
         keys.  Movement is in increments of the snap value.  This mode is
         exited by pressing the 'x' key.  Also see note [2].
      \item Controlled in the pertime section.
   \end{enumerate}

   The new features of this window for \textsl{Sequencer66},
   as noted in
   \sectionref{subsubsec:seq66_song_editor_arrangement_panel_measures_ruler},
   are:

   \begin{itemize}
      \item \textsl{l}:  Enters a mode where the left and right arrow keys move
         the L marker, until the "x" key is pressed.
      \item \textsl{r}:  Enters a mode where the left and right arrow keys move
         the R marker, until the "x" key is pressed.
      \item \textsl{x}:  Exits the marker-movement  mode.
   \end{itemize}

\subsubsection{Performance Editor Names Section}
\label{subsubsec:kbd_mouse_performance_editor_names_section}

   \begin{table}[H]
      \centering
      \caption{Performance Editor Names Section}
      \label{table:performance_editor_names}
      \begin{tabular}{l l l l l l}
         \textbf{Action}   & \textbf{Normal}    & \textbf{Double}    & \textbf{Shift}        & \textbf{Ctrl}   & \textbf{Mod4}      \\
         Left-Click        & Toggle track       & ---                & Toggle other tracks   & ---             & ---                \\
         Middle-Click      & ---                & ---                & ---                   & ---             & ---                \\
         Right-Click       & New/Edit menu      & ---                & ---                   & ---             & ---                \\
      \end{tabular}
   \end{table}

\subsection{Pattern Editor}
\label{subsec:kbd_mouse_pattern_editor}

   The pattern/sequencer editor piano roll is a complex and powerful event
   editor;
   \tableref{table:pattern_editor_piano_roll},
   doesn't begin to cover its functionality.
   Here are some keystrokes handled by the main frame of the piano roll:

   \begin{itemize}
      \item \texttt{Ctrl-L}.  Bring up the LFO event modulation editor.
      \item \texttt{Ctrl-W}.  Exit the sequence (pattern) editor.
      \item \texttt{Ctrl-Page Up}.  Zoom in.
      \item \texttt{Ctrl-Page Down}.  Zoom out.
      \item \texttt{Shift-Page Up}.  Scroll leftward.
      \item \texttt{Shift-Page Down}.  Scroll rightward.
      \item \texttt{Shift-Home}.  Scroll leftward to the beginning.
      \item \texttt{Shift-End}.  Scroll rightward to the end.
%     \item \texttt{Shift-z (Z)}.  Zoom in.
%     \item \texttt{0}.  Set default zoom.
%     \item \texttt{z}.  Zoom out.
      \item \texttt{Page Down}.  Scroll downward.
      \item \texttt{Page Up}.  Scroll upward.
      \item \texttt{Home}.  Scroll upward to the beginning.
      \item \texttt{End}.  Scroll downward to the end.
      \item \texttt{Delete}.  Deletes (not cuts) the currently-selected notes
         in the piano roll; can be undone with the \textbf{Undo} button.
   \end{itemize}

\subsubsection{Pattern Editor Piano Roll}
\label{subsubsec:kbd_mouse_pattern_editor_piano_roll}

   Here are the keystrokes handled by the piano roll:
   These keystrokes require that the focus be set to the piano roll by clicking
   in it with the mouse.

   \begin{itemize}
%     \item \texttt{Ctrl-x}. Cut.
%     \item \texttt{Ctrl-c}. Copy.
%     \item \texttt{Ctrl-v}. Paste.
%     \item \texttt{Ctrl-z}. Undo.
      \item \texttt{Ctrl-r}. Redo.
      \item \texttt{Ctrl-a}. Select all.
      \item \texttt{Ctrl-Left}.  Shrink selected notes.
      \item \texttt{Ctrl-Right}.  Grow selected notes.
      \item \texttt{Delete}.  Remove selected notes.
      \item \texttt{Backspade}.  Remove selected notes.
      \item \texttt{Home.  Set sequence to beginnging of sequence}.  (Verify!)
%     \item \texttt{Left}.  Move selected notes one snap left.
%     \item \texttt{Down}.  Move selected notes one pitch downward.
%     \item \texttt{Up}.  Move selected notes one pitch upward.
      \item \texttt{Enter, Return}.
         Paste the selected notes at the current position.
%     \item \texttt{p}.  Enter "paint" (also known as "adding") mode.
%     \item \texttt{x}.  Escape ("x-scape") the paint mode.
   \end{itemize}

   And here is the table, which includes items not described above:

   \begin{table}[H]
      \centering
      \caption{Pattern Editor Piano Roll}
      \label{table:pattern_editor_piano_roll}
      \begin{tabular}{l l l l l l}
         \textbf{Action}   & \textbf{Normal} & \textbf{Double}    & \textbf{Shift} & \textbf{Ctrl}   & \textbf{Mod4}      \\
         Del               & Delete Selected & ---                & ---            & ---             & ---                \\
         c                 & ---             & ---                & ---            & Copy            & ---                \\
         p                 & Paint mode      & ---                & ---            & ---             & ---                \\
         v                 & ---             & ---                & ---            & Paste           & ---                \\
         x                 & Escape Paint    & ---                & ---            & Cut             & ---                \\
         z                 & Zoom Out        & ---                & Zoom In        & Undo            & ---                \\
         0                 & Reset Zoom      & ---                & ---            & ---             & ---                \\
         Left-Arrow        & Move Earlier [1] & ---               & ---            & ---             & ---                \\
         Right-Arrow       & Move Later [1]  & ---                & ---            & ---             & ---                \\
         Up-Arrow          & Increase Pitch  & ---                & ---            & ---             & ---                \\
         Down-Arrow        & Decrease Pitch  & ---                & ---            & ---             & ---                \\
         Left-Click        & Deselect        & ---                & ---            & ---             & ---                \\
         Right-Click       & Paint mode      & ---                & Edit Menu      & Edit/Edit Menu  & Lock Paint mode    \\
         Left-Middle-Click & Grow Selected   & ---                & Stretch Sel.   & ---             & ---                \\
         Scroll-Up         & Zoom Time In    & ---                & Scroll Left    & Zoom Time In    & ---                \\
         Scroll-Down       & Zoom Time Out   & ---                & Scroll Right   & Zoom Time Out   & ---                \\
      \end{tabular}
   \end{table}

   \begin{enumerate}
      \item Once selected (and thus rendered in grey), a pattern segment
         can be moved by the mouse.  To move it using the left or right
         arrow keys, the paint mode must be entered, but only via the
         \texttt{p} key -- the right mouse button deselects the greyed pattern.
         Too tricky, we might try fixing it later.
   \end{enumerate}

   Features of this window section for \textsl{Sequencer66}, as noted in
   \sectionref{subsubsec:seq66_pattern_editor_piano_roll_items}, are:

   \begin{itemize}
      \item \textsl{p}:  Enters the paint mode, until right-click is pressed or
         until the \texttt{x} key is pressed.  Notes are added
         by clicking or click-dragging.
      \item \textsl{x}:  Exits ("x-scapes") the paint mode.
      \item \textsl{z}:  Zooms out.
      \item \textsl{0}:  Resets zoom to its normal value.
      \item \textsl{Z}:  Zooms in.
      \item \textsl{.}:  The period (configurable) does the pause function.
      \item \textsl{Left Arrow}:  Moves selected events to the left.
      \item \textsl{Right Arrow}:  Moves selected events to the right.
      \item \textsl{Up Arrow}:  Moves selected notes upward in pitch.
      \item \textsl{Down Arrow}:  Moves selected notes downward in pitch.
      \item \textsl{Mod4-Right-Click}:  Locks the paint mode, until right-click
         is pressed again.
   \end{itemize}

\subsubsection{Pattern Editor Event Panel}
\label{subsubsec:kbd_mouse_pattern_editor_event_panel}

   \begin{itemize}
      \item \texttt{Ctrl-x}. Cut.
      \item \texttt{Ctrl-c}. Copy.
      \item \texttt{Ctrl-v}. Paste.
      \item \texttt{Ctrl-z}. Undo.
      \item \texttt{Delete}.  Delete (not cut!) the selected events.
      \item \texttt{p}.  Enter "paint" (also known as "adding") mode.
      \item \texttt{x}.  Escape ("x-scape") the paint mode.
   \end{itemize}

\subsubsection{Pattern Editor Data Panel}
\label{subsubsec:kbd_mouse_pattern_editor_data_panel}

   Currently, no keystroke support is provided in the data panel.
   One potential upgrade would be the ability to change the value of the event
   with the Up and Down arrow keys.

\subsubsection{Pattern Editor Virtual Keyboard}
\label{subsubsec:kbd_mouse_pattern_editor_virtual_keyboard}

   \begin{table}[H]
      \centering
      \caption{Pattern Editor Virtual Keyboard}
      \label{table:pattern_editor_virtual_keyboard}
      \begin{tabular}{l l l l l l}
         \textbf{Action}   & \textbf{Normal} & \textbf{Double}    & \textbf{Shift} & \textbf{Ctrl}   & \textbf{Mod4}      \\
         Left-Click        & Play note       & ---                & ---            & ---             & ---                \\
         Right-Click       & Toggle labels   & ---                & ---            & ---             & ---                \\
      \end{tabular}
   \end{table}

\subsection{Event Editor}
\label{subsec:kbd_mouse_event_editor}

   \begin{itemize}
      \item \texttt{Down}.  Move one slot down.
      \item \texttt{Up}.  Move one slot up.
      \item \texttt{Page Down}.  Move one frame down.
      \item \texttt{Page Up}.  Move one frame up.
      \item \texttt{Home}.  Move to top frame.
      \item \texttt{End}.  Move to bottom frame.
      \item \texttt{Asterisk, KP Multiply}.  Delete the currently-selected event.
   \end{itemize}

%-------------------------------------------------------------------------------
% vim: ts=3 sw=3 et ft=tex
%-------------------------------------------------------------------------------


% Meta-event support

\input{seq66_meta_events}

% Configuration file

\input{seq66_rc_file}

% User file

%-------------------------------------------------------------------------------
% seq66_usr_file
%-------------------------------------------------------------------------------
%
% \file        seq66_usr_file.tex
% \library     Documents
% \author      Chris Ahlstrom
% \date        2015-08-31
% \update      2018-10-28
% \version     $Revision$
% \license     $XPC_GPL_LICENSE$
%
%     Provides the usr_file.
%
%-------------------------------------------------------------------------------

\section{Sequencer66 "usr" Configuration File}
\label{sec:seq66_usr_file}

   There are two \textsl{Sequencer66} configuration files:
   \texttt{sequencer66.rc} and \texttt{sequencer66.usr}.
   See \sectionref{sec:seq66_rc_file}; it describes the "rc" file.
   It is a bit different in how it is handled.

   The \textsl{Sequencer66} "usr" (or "user")
   configuration file provides a way to give more
   informative names to the MIDI busses, MIDI channels, and MIDI controllers of
   a given system setup.  This configuration will override the default values
   of some drop-down lists and menu items, and make them reflect your names for
   them.  In \textsl{Sequencer66} it, also includes some items that affect the
   user-interface's look, and other new configuration items.

   \index{usr!-u}
   \index{usr!--user-save}
   Unlike the "rc" file, the "user" file is \textsl{not} written every time
   \textsl{Sequencer66} exits.  If the "user" files does not exist, one is
   created, but it is normally not overwritten thereafter.  To
   cause it to be overwritten at exit, run \textsl{Sequencer66} with the
   \texttt{-u} or \texttt{--user-save} option:

   \begin{verbatim}
      $ seq66 --user-save
   \end{verbatim}

   This option is recommended when one installs a new version of
   \textsl{Sequencer66}, which might add new options to the "user" file.
   See \sectionref{sec:seq66_man_page}; it discusses more options involving the
   "user" file.

   Another difference between the "rc" file and the "user" file is that
   the "user" file currently has no graphical user-interface dialog to
   configure the "user" settings.  One has to edit the file manually.

   The original purpose for the "user" file was to create familiar names for the
   system MIDI devices.
   By default, the list of MIDI devices that \textsl{Sequencer66} shows depends
   on one's system setup and whether the manual-alsa-port option is specified
   or not.  Here's our system, which has Timidity installed and running as a
   service, and the \texttt{[manual-alsa-port]} option turned off, shown in a
   composite view with all menus one can look at for MIDI settings:

\begin{figure}[H]
   \centering 
%  \includegraphics[scale=0.65]{buss/manual-0-buss-dropdown.png}
   \includegraphics[scale=0.65]{roll.png}
   \caption{Sequencer66 Composite View of Native Devices}
   \label{fig:seq66_manual_0_buss_dropdown}
\end{figure}

   At the top center, the dropdown menu contains the 5 MIDI busses/ports
   supported by this computer.  At right, the MIDI channel shows
   the channels numbers that can be picked for buss 0.  At bottom left, we see
   the default controller values that \textsl{Sequencer66} includes.  We have
   no idea if these correspond to any controllers that the selected MIDI buss
   supports.  We \textsl{can} use this dropdown to see if any such controller
   events are in the loaded MIDI file, of course; a solid black square
   indicates that such an event was found in the pattern.

   Assume we have 3 MIDI "buss" devices hooked to our system:
   two Model "2x2" MIDI port devices, and an old PCR-30 MIDI controller
   keyboard.  Let's number them:

   \begin{enumerate}
      \item Model 2x2 A
      \item Model 2x2 B
      \item PCR-30
   \end{enumerate}

   Then assume that we have nine different MIDI instruments in our kit.
   Let's number them, too:

   \begin{enumerate}
      \item Waldorf Micro Q
      \item SuperNova
      \item DrumStation
      \item TX81Z
      \item WaveStation
      \item ESI-2000
      \item ES-1
      \item ER-1
      \item TB-303
   \end{enumerate}

   The Waldorf Micro Q, the SuperNova, and the DrumStation all have a large
   number of special MIDI controller values for affecting the sound they
   produce.  The DrumStation accepts MIDI controllers that change various
   features of the sound of each type of drum it supports.

   The buss devices can be configured to route certain
   MIDI channels to certain MIDI devices.  Assume we have them
   set up this way:

   \begin{enumerate}
      \item Model 2x2 A
      \begin{itemize}
         \item SuperNova: channels 1 to 8
         \item TX81Z: channels 9 to 11
         \item Waldorf Micro Q: channels 12 to 15
         \item DrumStation: channel 16
      \end{itemize}
      \item Model 2x2 B
      \begin{itemize}
         \item WaveStation: channels 1 to 4
         \item ESI-2000: channels 5 to 14
         \item ES-1: channel 15
         \item ER-1: channel 16
      \end{itemize}
      \item PCR-30
      \begin{itemize}
         \item TB-303: channel 1
      \end{itemize}
   \end{enumerate}

   How can we get \textsl{Sequencer66} to show these items with the proper
   names associated with each device, channel, and controller value?
   We use the oddly-named \textbf{"user" configuration file}.

   \index{sequencer66.usr}
   \index{[sequencer66.usr]}   % for convenience
   The \textsl{Seq24} configuration file was called
   \texttt{.seq24usr}, and it was stored in the user's \texttt{\$HOME}
   directory.
   \textsl{Sequencer66} uses a new file-name
   to take its place, with a fall-back to the original file-name if the new
   file does not exist, or if \textsl{Sequencer66} is running in
   \index{legacy mode}
   legacy mode.
   After one runs \textsl{Sequencer66} for the first time (or after deleting
   the configuration files), it will generate a
   \texttt{sequencer66.usr} file in your home directory:

   \begin{verbatim}
      /home/ahlstrom/.config/sequencer66/sequencer66.usr
   \end{verbatim}

   It allows you to give an alias to 
   each MIDI bus, MIDI channel, and MIDI control 
   codes, per channel.
   The file-name is a bit misleading... do not confuse this file with the
   \texttt{sequencer66.rc} file.

   The process for setting up the user file is to:

   \begin{enumber}
      \item Define one or more MIDI busses, the name of each, and what
         instruments are on which channels.  Each buss is configured in a
         section of the form "\texttt{[user-midi-bus-X]}", where "X" ranges
         from 0 on up.  Each buss then defines up to 16 channel entries.
         Each entry includes the channel number and the number of a
         section in the user-instrument section described next.
      \item Define all of the instruments and their controller
         names, if they have them.  Each instrument is configured in a
         section of the form "\texttt{[user-instrument-X]}", where "X"
         ranges from 0 on up.
   \end{enumber}

   Let's walk through the structure of this setup, since it is a little bit
   tricky.  Here is a diagram of the relationships between the buss definitions
   and instrument definitions:

\begin{figure}[H]
   \centering 
%  \includegraphics[scale=0.50]{user-busses-and-instruments.png}
   \includegraphics[scale=0.65]{roll.png}
   \caption{Busses and Instruments in the "usr" File}
   \label{fig:seq66_manual_user_busses_and_instruments}
\end{figure}

   The first section in the "usr" file (after \texttt{[comments]})
   is \texttt{[user-midi-bus-definitions]}.  The solid diamond link, with the
   "*" marker, indicates that this section contains an arbitrary number ("*")
   of \texttt{[user-midi-bus-N]} sections, where "N" ranges from 0 on upward.
   These correspond to the MIDI busses expected to be in the system, ignoring
   the "announce" buss.

   Each of the busses contains 16 (0 to 15) channel entries.
   These channels are referred to as "instrument numbers", and are
   represented as and linked to "instruments" in the
   \texttt{[user-instrument-definitions]} section.  Each instrument contains up
   to 128 controller values; these controller values are available in the
   \textbf{Event} button in the Pattern Editor, and their names are shown.

   So, each instrument is setup as a "channel" in a particular "buss".
   In the Pattern Editor, when a particular buss and channel is selected,
   the \textbf{Event} menu entries should match the controller entries set up
   in the "usr" file.

   Taking our list of devices and channels we created above, which
   can be seen in the \textsl{Sequencer66} sample file
   \texttt{contrib/configs/sequencer66.usr.example}, and 
   deducting 1 from each device number and channel number (so that numbering
   starts from 0), and consulting the device manuals to determine the
   controller values it supports, we can assemble a "user" configuration file
   that makes the setup visible in \textsl{Sequencer66}.

   Peruse the next couple of sections to understand a bit about the format of
   this file.  Look at the example files in the \texttt{contrib/configs}
   directory as well, to see the whole thing put together.
   Once satisfied, go to
   \sectionref{subsec:seq66_usr_file_midi_bus_results}, and 
   see what it all looks like.

\subsection{"usr" File / MIDI Bus Definitions}
\label{subsec:seq66_usr_file_midi_bus_definitions}

   \index{usr!user-midi-bus-definitions}
   \index{[user-midi-bus-definitions]}
   This section begins with an
   "INI" group marker \texttt{[user-midi-bus-definitions]}.
   It defines the number of user busses that will be configured in this file.

   \begin{verbatim}
      [user-midi-bus-definitions]
      3     # number of user-defined MIDI busses
   \end{verbatim}

   \index{usr!user-midi-bus-n}
   \index{[user-midi-bus-n]}
   This means that the \texttt{sequencer66.usr} file will have three MIDI buss
   sections: [user-midi-bus-0], [user-midi-bus-1], and [user-midi-bus-2].
   Here's is an annoted example of one such section:

   \begin{verbatim}
      [user-midi-bus-0]
      2x2 A (SuperNova,Q,TX81Z,DrumStation)     # name of the device
      16                                        # number of channels

      # NOTE: Channels are 0-15, not 1-16.  Instruments set to -1 = GM

      0 1                                       # channel and instrument
      1 1      # Instrument #1 of the [user-instrument-definitions] section
      2 1
      . . .
      7 1
      8 3      # Instrument #3 of the [user-instrument-definitions] section
      9 3
      10 3
      11 0     # Instrument #0 of the [user-instrument-definitions] section
      12 0     # This is the Waldorf Micro Q device defined below
      13 0
      14 0
      15 2     # Instrument #2 of the [user-instrument-definitions] section
   \end{verbatim}

   Here's an example of one that needs only one override:

   \begin{verbatim}
      [user-midi-bus-2]                         # Instrument 2, see ch. 15 above
      PCR-30 (303)
      1                                         # number of channels
      0 8                                       # channel and instrument
      # The rest default to -1... General MIDI
   \end{verbatim}

   Note that these user-instrument entries can be quickly disabled by changing
   the count values to 0.
   Also, these instrument-definition
   sections are read from the "user" configuration file only if
   the \texttt{--reveal-alsa-ports} option is \textsl{off} ("0");
   this command-line option can also be specified in the
   \texttt{[reveal-alsa-ports]} section of the "rc" file,
   see \sectionref{subsec:seq66_rc_file_reveal_ports}, where this section is
   discussed.
   Otherwise, the actual port names reported by ALSA are shown.
   The \texttt{user-midi-bus-definitions} and \texttt{user-midi-bus-N} sections
   can be misleading if one wants to have access to the
   actual ALSA ports that exist on the system.
   Therefore, if the \texttt{--reveal-alsa-ports} option is turned on, then the
   definitions in the "user" configuration file are \textsl{not} read from that
   file.  The following figures show the results of various settings with an
   active "user" file.  They have been clipped to save space.

\begin{figure}[H]
   \centering 
%  \includegraphics[scale=0.50]{user/seq66-clock-m.png}
   \includegraphics[scale=0.65]{roll.png}
   \caption{Clocks View, -m (--manual-alsa-ports)}
   \label{fig:seq66_clock_m}
\end{figure}

   Above, the virtual (manual) output ports are shown just as created by
   \textsl{Sequencer66}.
   The \texttt{--reveal-alsa-ports} option is \textsl{off} here.

\begin{figure}[H]
   \centering 
%  \includegraphics[scale=0.50]{user/seq66-input-m.png}
   \includegraphics[scale=0.65]{roll.png}
   \caption{Inputs View with -m (--manual-alsa-ports) Option}
   \label{fig:seq66_input_m}
\end{figure}

   Above, the single virtual (manual) input port is shown just as created by
   \textsl{Sequencer66}.
   Again, the \texttt{--reveal-alsa-ports} option is \textsl{off} here.

\begin{figure}[H]
   \centering 
%  \includegraphics[scale=0.50]{user/seq66-clock-m-R.png}
   \includegraphics[scale=0.65]{roll.png}
   \caption{Clocks View, -m (--manual-alsa-ports) and -R (--hide-alsa-ports)}
   \label{fig:seq66_clock_m_R}
\end{figure}

   Above, by adding the "hide" ports option, the system port labels are
   replaced by the labels from the "usr" file.

\begin{figure}[H]
   \centering 
%  \includegraphics[scale=0.50]{user/seq66-clock-r.png}
   \includegraphics[scale=0.65]{roll.png}
   \caption{Clocks View, -r (--reveal-alsa-ports)}
   \label{fig:seq66_clock_r}
\end{figure}

   Above, the "reveal" ports option overrides the device names given in the
   "usr" file, so that the native system names of the output ports are shown.

\begin{figure}[H]
   \centering 
%  \includegraphics[scale=0.50]{user/seq66-input-r.png}
   \includegraphics[scale=0.65]{roll.png}
   \caption{Inputs View with -r (--reveal-alsa-ports) Option}
   \label{fig:seq66_input_r}
\end{figure}

   Above, the "reveal" ports option overrides the device names given in the
   "usr" file, so that the native system names of the input ports are shown.
   However, \textsl{Sequencer66} no longer overrides the
   names of the input ports via the "usr" file.  This is done to
   save some trouble in displaying the input port names, which are shown
   only in this dialog.  We may consider offering a separate override section
   for the input ports in the future.

\begin{figure}[H]
   \centering 
%  \includegraphics[scale=0.50]{user/seq66-clock-R.png}
   \includegraphics[scale=0.65]{roll.png}
   \caption{Clocks View with -R (--hide-alsa-ports) Option}
   \label{fig:seq66_clock_R}
\end{figure}

   The figure above shows how hiding the system port names shows the names
   defined in the "usr" file.  But notice that the actual port names are shown
   in square brackets, for reference.

\begin{figure}[H]
   \centering 
%  \includegraphics[scale=0.50]{user/seq66-input-R.png}
   \includegraphics[scale=0.65]{roll.png}
   \caption{Inputs View with -R (--hide-alsa-ports) Option}
   \label{fig:seq66_input_R}
\end{figure}

   Although the "hide" ports option is specified above, this view is
   currently also the normal view of the input ports, even with device names
   defined in the "usr" file.
   At this time, there's no real need to show the user-instrument names
   on the input port.  If there turns out to be such a need, the definitions
   would likely need to be different from the definitions for the output ports.
   Another complexity is the possible existence, under ALSA, of the
   \texttt{system:announce} port.

\subsection{"usr" File / MIDI Instrument Definitions}
\label{subsec:seq66_usr_file_midi_instrument_definitions}

   \index{usr!user-instrument-definitions}
   \index{[user-instrument-definitions]}
   This section begins with an
   "INI" group marker \texttt{[user-instrument-definitions]}.
   It defines the number of user instruments that will be configured in this
   file.  This section defines characteristics, such as
   the meanings of MIDI controller values, of the instruments themselves,
   not the MIDI busses to which they attached.

   \begin{verbatim}
      [user-instrument-definitions]
      9     # number of user instrument
   \end{verbatim}

   \index{usr!user-instrument-n}
   \index{[user-instrument-n]}
   So this "usr" file will define 9 instruments.  We provide only one section
   as an example.  Note that items without text default to the values
   prescribed by the General MIDI (GM) specification.

   \begin{verbatim}
      [user-instrument-0]
      Waldorf Micro Q                     # name of instrument
      128                                 # number of MIDI controllers
      0                                   # first controller value, unnamed
      1 Modulation Wheel
      2 Breath Control
      3 
      4 Foot Control
         . . .
      119
      120 All Sound Off (0)
      121 Reset All Controllers (0)
      122 Local Control (0-127) (Off,On)
      123 All Notes Off (0)
      124                                 # defaults to GM
      125 Unsupported
      126 Unsupported
      127                                 # defaults to GM
   \end{verbatim}

   Note the unnamed control numbers above.
   An unnamed control number might be an unsupported control number.
   It is termed to be "inactive".  In this case, the event menu of
   the pattern editor will show the default name of this controller.
   Again, though, the function denoted by this name might not be supported by
   the device.  In that case, it might be better to call it "Unsupported".
   See the examples above.
   Here is an instrument that has synthesis parameters that can be controlled:

   \begin{verbatim}
      [user-instrument-1]
      SuperNova
      128
      0 Bank Select MSB
      1 Modulation Wheel
      2 Breath Controller
      3 Arp Pattern Select
         . . .
      121 Reset Controllers
      122 Local Control [*]
      123 All Notes Off
      124 All Notes Off
      125 All Notes Off
      126 All Notes Off
      127 All Notes Off
   \end{verbatim}

   Here is an instrument that perhaps has no controllers, or perhaps is simply
   not configured by the musician yet:

   \begin{verbatim}
      [user-instrument-4]
      WaveStation
      0
   \end{verbatim}

   The sample file
   \texttt{contrib/configs/sequencer66-timidity-yoshimi.usr.example}
   contains examples of some other kinds of instruments.
   It is a minimal resource, useful to study when creating one's own settings.

\subsection{"usr" File / User Interface Settings}
\label{subsec:seq66_usr_file_user_interface_settings}

   \index{usr!user-interface-settings}
   \index{[user-interface-settings]}
   This section, new to \textsl{Sequencer66}, begins with an
   "INI" group marker \texttt{[user-interface-settings]}.

   It provides for a feature we will hopefully be able to complete some day:
   the complete specificition of the appearance of the user-interface.
   There is plenty of room to change the appearance of
   \textsl{Sequencer66} already!  Please try the settings and see what you
   like.

   \index{usr!grid-style}
   \begin{verbatim}
      #   ======== Sequencer66-Specific Variables Section ========

      [user-interface-settings]

      # These settings specify the soon-to-be-modifiable sizes of
      # the Sequencer66 user-interface elements.

      # Specifies the style of the main-window grid of patterns.
      #
      # GTK:
      #   0 = Normal style, matches the GTK theme, has brackets.
      #   1 = White grid boxes that have brackets.
      #   2 = Black grid boxes (no brackets, our favorite).
      #
      # Qt:
      #   0 = Slot coloring matches Kepler34.
      #   1 = Slot coloring more like GTK.
      1       # grid_style
   \end{verbatim}

   \index{usr!grid-brackets}
   \begin{verbatim}
      # Specifies box style box around a main-window grid of patterns.
      # 0  = Draw a whole box around the pattern slot.
      # 1  = Draw brackets on the sides of the pattern slot.
      # 2 and up = make the brackets thicker and thicker.
      # -1 = same as 0, draw a box one-pixel thick.
      # -2 and lower = draw a box, thicker and thicker.
      2       # grid_brackets
   \end{verbatim}

   \index{usr!mainwnd-rows}
   \index{variset}
   \begin{verbatim}
      # Specifies the number of rows in the main window.
      # Values of 4 (the default) through 8 (the best alternative value)
      # are allowed.
      4       # mainwnd_rows
   \end{verbatim}

   \index{usr!mainwnd-cols}
   \index{variset}
   \begin{verbatim}
      # Specifies the number of columns in the main window.
      # At present, only values from 8 (the default) to 12 are supported.
      8       # mainwnd_cols
   \end{verbatim}

   \index{usr!max-sets}
   \begin{verbatim}
      # Specifies the maximum number of sets, which defaults to 1024.
      # It is currently never necessary to change this value.
      32      # max_sets
   \end{verbatim}

   \index{usr!mainwid-border}
   \begin{verbatim}
      # Specifies the border width in the main window.
      0      # mainwid_border
   \end{verbatim}

   \index{usr!mainwid-spacing}
   \begin{verbatim}
      # Specifies the border spacing in the main window.
      2      # mainwid_spacing
   \end{verbatim}

   \index{usr!control-height}
   \begin{verbatim}
      # Specifies some quantity, it is not known what it means.
      0      # control_height
   \end{verbatim}

   \index{usr!zoom}
   \begin{verbatim}
      # Specifies the initial zoom for the piano rolls.  Ranges from 1.
      # to 32, and defaults to 2 unless changed here.
      2      # zoom
   \end{verbatim}

   \index{usr!global-seq-feature}
   \begin{verbatim}
      # Specifies if the key, scale, and background sequence are to be
      # applied to all sequences, or to individual sequences.  The
      # behavior of Seq24 was to apply them to all sequences.  But
      # Sequencer66 takes it further by applying it immediately, and
      # by saving to the end of the MIDI file.  Note that these three
      # values are stored in the MIDI file, not this configuration file.
      # Also note that reading MIDI files not created with this feature
      # will pick up this feature if active, and the file gets saved.
      # It is contagious.
      #
      # 0 = Allow each sequence to have its own key/scale/background.
      #     Settings are saved with each sequence.
      # 1 = Apply these settings globally (similar to seq24).
      #     Settings are saved in the global final section of the file.
      1      # global_seq_feature
   \end{verbatim}

   \index{usr!use-new-font}
   \begin{verbatim}
      # Specifies if the old, console-style font, or the new anti-
      # aliased font, is to be used as the font throughout the GUI.
      # In legacy mode, the old font is the default.
      #
      # 0 = Use the old-style font.
      # 1 = Use the new-style font.
      1      # use_new_font
   \end{verbatim}

   \index{usr!allow-two-perfedits}
   \begin{verbatim}
      # Specifies if the user-interface will support two song editor
      # windows being shown at the same time.  This makes it easier to
      # edit songs with a large number of sequences.
      #
      # 0 = Allow only one song editor (performance editor).
      # 1 = Allow two song editors.
      1      # allow_two_perfedits
   \end{verbatim}

   \index{usr!perf-h-page-increment}
   \begin{verbatim}
      # Specifies the number of 4-measure blocks for horizontal page
      # scrolling in the song editor.  The old default, 1, is a bit
      # small.  The new default is 4.  The legal range is 1 to 6, where
      # 6 is the width of the whole performance piano roll view.
      4      # perf_h_page_increment
   \end{verbatim}

   \index{usr!perf-v-page-increment}
   \begin{verbatim}
      # Specifies the number of 1-track blocks for vertical page
      # scrolling in the song editor.  The old default, 1, is a bit
      # small.  The new default is 8.  The legal range is 1 to 18, where
      # 18 is about the height of the whole performance piano roll view.
      8      # perf_v_page_increment
   \end{verbatim}

   \index{usr!progress-bar-colored}
   \begin{verbatim}
      # Specifies if the progress bar is colored black, or a different
      # color.  The following integer color values are supported:
      # 
      # 0 = black
      # 1 = dark red
      # 2 = dark green
      # 3 = dark orange
      # 4 = dark blue
      # 5 = dark magenta
      # 6 = dark cyan
      6      # progress_bar_colored
   \end{verbatim}

   \index{usr!progress-bar-thick}
   \begin{verbatim}
      # Specifies if the progress bar is thicker.  The default is 1
      # pixel.  The 'thick' value is 2 pixels.  (More than that is not
      # useful.  Set this value to 1 to enable the feature, 0 to disable
      # it.
      1      # progress_bar_thick
   \end{verbatim}

   \index{usr!inverse-colors}
   \begin{verbatim}
      # Specifies using an alternate (darker) color palette.  The
      # default is the normal palette.  Not all items in the user
      # interface are altered by this setting, and it's not perfect.
      # Set this value to 1 to enable the feature, 0 to disable it.
      0      # inverse_colors
   \end{verbatim}

   \index{usr!window-redraw-rate}
   \begin{verbatim}
      # Specifies the window redraw rate for all windows that support
      # that concept.  The default is 40 ms.  Some windows used 25 ms.
      40     # window_redraw_rate
   \end{verbatim}

   \index{usr!use-more-icons}
   \begin{verbatim}
      # Specifies using icons for some of the user-interface buttons
      # instead of text buttons.  This is purely a preference setting.
      # If 0, text is used in some buttons (the main window buttons).
      # Otherwise, icons are used.  One will have to experiment :-).
      0      # use_more_icons (currently affects only main window)
   \end{verbatim}

   \index{multi-wid}
   \index{usr!block-rows}
   \begin{verbatim}
      # Specifies the number of set window ('wid') rows to show.
      # The long-standing default is 1, but 2 or 3 may also be set.
      # Corresponds to 'r' in the '-o wid=rxc,f' option.
      2      # block_rows (number of rows of set blocks/wids)
   \end{verbatim}

      This option is \textsl{not supported} in the Qt version.
      Instead, the user can create an arbitrary number of
      live-frame windows.

   \index{multi-wid}
   \index{usr!block-columns}
   \begin{verbatim}
      # Specifies the number of set window ('wid') columns to show.
      # The long-standing default is 1, but 2 may also be set.
      # Corresponds to 'c' in the '-o wid=rxc,f' option.
      2      # block_columns (number of columns of set blocks/wids)
   \end{verbatim}

   \index{usr!block-independent}
   \begin{verbatim}
      # Specifies if the multiple set windows are 'in sync' or can
      # be set to arbitrary set numbers independently.
      # The default is false (0), means that there is a single set
      # spinner, which controls the set number of the upper-left 'wid',
      # and the rest of the set numbers follow sequentially.  If true
      # (1), then each 'wid' can be set to any set-number.
      # Corresponds to the 'f' (true, false, or 'indep') in the
      # '-o wid=rxc,f' option.  Here, 1 is the same as 'indep' or false,
      # and 0 is the same as f = true.  Backwards, so be careful.
      1      # block_independent (separate set spinner for blocks/wids)
   \end{verbatim}

   Note that the window-redraw rate option is meant more for experimentation
   than anything else.  It probably doesn't affect CPU usage much, but might
   provide a smoother-running cursor on some systems.

\subsection{"usr" File / User MIDI Settings}
\label{subsec:seq66_usr_file_user_midi_settings}

   \index{[user-midi-settings]}
   This section begins with an
   "INI" group marker \texttt{[user-midi-settings]}.
   It supports files with different PPQN, and and allows one to specify the
   global defaults for tempo, beats per measure, and so on.

   \index{usr!midi-ppqn}
   \begin{verbatim}
      [user-midi-settings]

      # These settings specify MIDI-specific value that might be
      # better off as variables, rather than constants.
      # Specifies parts-per-quarter note to use, if the MIDI file.
      # does not override it.  Default is 192, but we'd like to go
      # higher than that.  BEWARE:  STILL GETTING IT TO WORK!
      192     # midi_ppqn
   \end{verbatim}

   \index{usr!midi-beats-per-measure}
   \begin{verbatim}
      # Specifies the default beats per measure, or beats per bar.
      # The default value is 4.
      4       # midi_beats_per_measure/bar
   \end{verbatim}

   \index{usr!midi-beats-per-minute}
   \begin{verbatim}
      # Specifies the default beats per minute.  The default value
      # is 120, and the legal range is 1 to 600.
      120     # midi_beats_per_minute
   \end{verbatim}

   \index{usr!midi-beat-width}
   \begin{verbatim}
      # Specifies the default beat width. The default value is 4.
      4       # midi_beat_width
   \end{verbatim}

   \index{usr!midi-buss-override}
   \begin{verbatim}
      # Specifies the buss-number override. The default value is -1,
      # which means that there is no buss override.  If a value
      # from 0 to 31 is given, then that buss value overrides all
      # buss values specified in all sequences/patterns.
      # Change this value from -1 only if you want to use a single
      # output buss, either for testing or convenience.  And don't
      # save the MIDI afterwards, unless you really want to change
      # all of its buss values.
      -1     # midi_buss_override
   \end{verbatim}

   For the new 0.90 series, additional values for the
   \texttt{[user-midi-settings]} section have been added:

   \index{usr!velocity-override}
   \begin{verbatim}
      # Specifies the default velocity override when adding notes in the
      # sequence/pattern editor.  This value is obtained via the 'Vol'
      # button, and ranges from 0 (not recommended :-) to 127.  If the
      # value is -1, then the incoming note velocity is preserved.
      80     # velocity_override (-1 = 'Free')
   \end{verbatim}

   \index{usr!bpm-precision}
   \begin{verbatim}
      # Specifies the precision of the beats-per-minutes spinner and
      # MIDI control over the BPM value.  The default is 0, which means
      # the BPM is an integer.  Other values are 1 and 2 decimal digits
      # of precision.
      1      # bpm_precision
   \end{verbatim}

   \index{usr!bpm-step-increment}
   \begin{verbatim}
      # Specifies the step increment of the beats/minute spinner and
      # MIDI control over the BPM value.  The default is 1. For a
      # precision of 1 decimal point, 0.1 is a good value.  For a
      # precision of 2 decimal points, 0.01 is a good value, but one
      # might want somethings a little faster, like 0.05.
      0.1    # bpm_step_increment
   \end{verbatim}

   \index{usr!bpm-page-increment}
   \begin{verbatim}
      # Specifies the page increment of the beats/minute field. It is
      # used when the Page-Up/Page-Down keys are pressed while the BPM
      # field has the keyboard focus.  The default value is 10.
      5.0    # bpm_page_increment
   \end{verbatim}

   \index{usr!midi-bpm-minimum}
   \begin{verbatim}
		# Specifies the minimum value of beats/minute in tempo graphing.
		# By default, the tempo graph ranges from 0.0 to 127.0.
		# This value can be increased to give a magnified view of tempo.
		0       # midi_bpm_minimum
   \end{verbatim}

   \index{usr!midi-bpm-maximum}
   \begin{verbatim}
		# Specifies the maximum value of beats/minute in tempo graphing.
		# By default, the tempo graph ranges from 0.0 to 127.0.
		# This value can be increased to give a magnified view of tempo.
		360       # midi_bpm_maximum
   \end{verbatim}

   \index{usr!velocity-override}
      The \texttt{velocity-override} option fixes a long standing (from
      \textsl{Seq24}) bug where the actual incoming note velocity was always
      replaced by a hard-wired value.

   \index{usr!bpm-step-increment}
   \index{usr!bpm-page-increment}
      The \texttt{bpm-precision}, \texttt{bpm-step-increment}, and
      \texttt{bpm-page-increment} values allow more precise control over tempo,
      which makes it easier to match the tempo of external music sources.  Note
      that the step-increment is used by the up/down arrow buttons, the up/down
      arrow keys, and the MIDI BPM control values.  The page-increment is used
      if the BPM field has focus and the Page-Up/Page-Down keys are pressed,
      and new MIDI control values have been added to support coarse MIDI
      control of tempo.

   \index{usr!midi-bpm-minimum}
   \index{usr!midi-bpm-maximum}
		The \texttt{midi-bpm-minimum} and \texttt{midi-bpm-maximum} settings
		are used in scaling the display of Tempo events.
      By adjusting these values, one can more easily see the variations in
      tempo.  In a main window pattern slot, or in the song editor tempo track,
      this range is scaled to the full range of note values, 0 to 127.
      Generally, one wants to select a range that keeps the main tempo line at
      the middle height of the pattern display.

   To obtain these new settings, remember to backup the existing
   \textsl{sequencer66.usr}, then run \textsl{Sequencer66} with the
   \texttt{--user-save} option, and then do a "diff" on the new file and the
   original to merge any old values that need to be preserved.  Then make any
   further tweaks to the new values.

\subsection{"usr" File / User Options}
\label{subsec:seq66_usr_file_user_options}

   \index{[user-options]}
   This section begins with an
   "INI" group marker \texttt{[user-options]}.
   It provides for additional options keyed by the
   \texttt{-o}/\texttt{--option} options.
   This group of options serves to expand the options that are available, since
   \textsl{Sequencer66} is  running out of single-character options.
   This group of options are shown below.

   \index{usr!option-daemonize}
   \begin{verbatim}
		# The daemonize option is used in seq66cli to indicate that the
		# application should be gracefully run as a service.
		0       # option_daemonize
   \end{verbatim}

   If this option is not used when running \texttt{seq66cli}, then the
   application stays in the console window and dumps informational output to
   it.  If this option is in force, then the only way to affect
   \texttt{seq66cli} is to send a signal (e.g. SIGKILL) to it, or use
   MIDI control.

   \index{usr!option-logfile}
   \begin{verbatim}
      # This value specifies an optional log-file that replaces output
      # to standard output and standard error.  To indicate no log-file,
      # the string "" is used.
      "seq66.log"
   \end{verbatim}

   This log-file is written to the same directory as the "rc" and "usr" files.
   If this file-name is empty, then a valid file-name must be specified
   in the "--option log=filename.log" option.  Note that this file
   is always written to the \textsl{Sequencer66} configuration directory.

\subsection{"usr" File / Device and Control Names}
\label{subsec:seq66_usr_file_midi_bus_results}

   Okay, now we have this file copied to our home directory:

   \begin{verbatim}
      /home/ahlstrom/.config/sequencer66/sequencer66.usr
   \end{verbatim}

   If we'd already run \textsl{Sequencer66} at least once, we'd have
   overwritten the skeleton sample file that \textsl{Sequencer66}
   writes by default.  We now have a full-fledged "user" file.

   However, because we don't actually have all that equipment (we got the
   example from the Web, for cryin' out loud), let's see what we end up with
   when we run \textsl{Sequencer66} this time and show the pattern editor
   settings:

\begin{figure}[H]
   \centering 
%  \includegraphics[scale=0.65]{buss/manual-0-userfile-buss-dropdown.png}
   \includegraphics[scale=0.65]{roll.png}
   \caption{Sequencer66 Composite View of Non-Native Devices}
   \label{fig:seq66_manual_0_userfile_buss_dropdown}
\end{figure}

   Compare that diagram to \figureref{fig:seq66_manual_0_buss_dropdown}.
   If the original figure, we saw the 5 native busses (ports) on our system,
   their bare-bones channel numbers, and the default controller values.  In
   this new figure, we see the three buss devices (ports), plus the two
   Timidity ports.  If we stopped the Timidity service, these would go away.

   Look at the selected buss, "[0]".  It's 16 channels are now associated with
   the devices to which the channels have been assigned.
   Thus, when we have a new pattern we've created in \textsl{Sequencer66},
   we can assign it to exactly the buss and device we want.

   If we don't have port-mappers installed, and thus have only one playback
   device plugged into the buss, we can still create a setup that
   shows the device and a specific program setup.  Doing so would be tedious,
   but perhaps there's some automated way to do it?
   Lastly, note the following figure.

\begin{figure}[H]
   \centering 
%  \includegraphics[scale=0.65]{buss/manual-0-userfile-seq-buss-dropdowns.png}
   \includegraphics[scale=0.65]{roll.png}
   \caption{The MIDI Bus Menu for a Specific Pattern}
   \label{fig:seq66_manual_0_userfile_seq_buss_dropdown}
\end{figure}

   This figure shows that we can also select the desired port and channel
   directly from the main window.
   There's more to the "user" configuration file than we've exposed here.
%  but finding more information about this file has proven a bit tricky.

%  Sometime we would like to create a "user" that sets up the
%  \textsl{Yoshimi} 1.3.5+ software synthesizer as a device and instrument.

%-------------------------------------------------------------------------------
% vim: ts=3 sw=3 et ft=tex
%-------------------------------------------------------------------------------


% Playlists

\input{seq66_playlist}

% Qt / PortMidi / Windows version

\input{seq66_qt_portmidi}

% Man page

%-------------------------------------------------------------------------------
% seq66_manpage
%-------------------------------------------------------------------------------
%
% \file        seq66_manpage.tex
% \library     Documents
% \author      Chris Ahlstrom
% \date        2015-08-31
% \update      2018-10-02
% \version     $Revision$
% \license     $XPC_GPL_LICENSE$
%
%     Provides the man page section of seq24-user-manual.tex.
%
%-------------------------------------------------------------------------------

\section{Sequencer66 Man Page}
\label{sec:seq66_man_page}

   This section presents the contents of the \textsl{Sequencer66} man page, but
   not exactly in \textsl{man} format.  Also, an item or two are shown that
   somehow didn't make it into the man page, and minor corrections and
   formatting tweaks were made.
   For example, we replaced the underscore with the hyphen in the names of some
   options.  The legacy Seq24 options, which use underscores or are missing the
   option hyphen, are still unofficially supported.

   \texttt{\$HOME/.config/sequencer66/sequencer66.rc} holds the "rc" settings
   for \textsl{Sequencer66}.

   \texttt{\$HOME/.config/sequencer66/sequencer66.usr} holds the "user" settings
   for \textsl{Sequencer66}.

   But the old style names are used for the "legacy" mode.  See the
   \texttt{--legacy} option below. Here is the basic command line:

%  \textsl{Sequencer66} is a real-time MIDI sequencer. It was created to
%  provide a very simple interface for editing and playing MIDI 'loops'.

   \begin{verbatim}
       sequencer66 [OPTIONS] [FILENAME]
       seq66 [OPTIONS] [FILENAME]
   \end{verbatim}

   \textsl{Sequencer66} accepts the following options, plus an optional name of
   a MIDI file.  Please note that many of the options are 'sticky'.  If they
   are used on the command-line, their settings are saved to the configuration
   files when \textsl{Sequencer66} exits.

   \setcounter{ItemCounter}{0}      % Reset the ItemCounter for this list.

   \optionpar{-h}{--help}
      Display a list of all command-line options, then exit.

   \optionpar{-v}{--version}
      Display the program version, then exit.

   \optionpar{-H}{--home [directory]}
      \textbf{New:}
      \index{new!home directory}
      Change the "home" directory from \texttt{.contrib/sequencer66}
      (always relative to \texttt{\$HOME}.
      This option causes the \texttt{sequencer66.rc}
      and \texttt{sequencer66.usr} files to be loaded from or
      saved to a different directory.
      Format: \texttt{--home dirname}.

   \optionpar{-l}{--legacy}
      \textbf{New:}
      \index{new!legacy mode}
      Save the MIDI file in the old Seq24 format, as unspecified
      binary data, instead of as a legal MIDI track with meta events.
      Also read the configuration, if provided, from the "legacy"
      \texttt{\textasciitilde/.seq24rc} and
      \texttt{\textasciitilde/.seq24usr} files.

%     instead of the new
%     \texttt{\textasciitilde/.config/sequencer66/sequencer66.rc} and
%     \texttt{\textasciitilde/.config/sequencer66/sequencer66.usr} files.

      The user-interface will indicate this mode with a small text
      note.
      This mode is also used if \textsl{Sequencer66} is invoked as the
      \texttt{seq24} command (one can create a soft link to the sequencer66
      executable to make that happen).

   \optionpar{-b}{--bus [buss]}
      \textbf{New:}
      \index{new!MIDI buss override}
      Supports modifying the buss number on all tracks when a MIDI file is
      read.  All tracks are loaded with this buss-number override.  This
      feature is useful for testing, making it easy to map the MIDI file onto
      the system's current hardware/software synthesizer setup.
      Also note that this option applies the MIDI buss override to any new
      sequences, as well.

      Format: \texttt{--bus bussnumber} or
      \texttt{--buss bussnumber}.

      Most of the time, one will want to set this value to -1.

   \optionpar{-q}{--ppqn [ppqn]}
      \textbf{New:}
      \index{new!ppqn override}
      Supports modifying the PPQN value of Sequencer66, which is
      defaults to a value of 192.  This setting should allow MIDI files to
      play back at the proper speed, and be written with the new PPQN value.
      This feature is basically done.
      One can load MIDI files of arbitrary PPQN, and they
      play normally and look normal in the editor windows.  They can also be
      saved, and reloaded with the new PPQN value. 
      Format: \texttt{--ppqn ppqnnumber}.
      The \textsl{ppqnnumber} can range from 32 to 19200.
      \index{new!ppqn from MIDI file}
      It can also be set to 0... in this case, \textsl{Sequencer66}
      uses the PPQN from the loaded file as its internal PPQN value.

   \optionpar{-L}{--lash}
      \textbf{New:}
      \index{new!LASH runtime enabling}
      If LASH support is compiled into the program, this option
      enables it.
      If LASH support is not compiled into the program, this option will not
      be shown in the output of the --help option.

   \optionpar{-n}{--no-lash}
      \textbf{New:}
      \index{new!LASH runtime disabling}
      If LASH support is compiled into the program, this option
      disables it, even if the default or configuration file set it.
      If LASH support is not compiled into the program, this option will not
      be shown in the output of the --help option.

   \optionpar{N/A}{--file [filename]}
      Load a MIDI file on startup.
      \textbf{Bug:}
      \index{bugs!--file option doesn't exist}
      This option does not exist.
      Instead, specify the file itself as the last command-line argument.

   \optionpar{-m}{--manual-alsa-ports}
      \textsl{Sequencer66} won't attach the system's existing ALSA ports.
      Instead, it will create is own set of input and output busses/ports.

   \optionpar{-a}{--auto-alsa-ports}
      \textsl{Sequencer66} will attach the system's existing ALSA ports.
      This variant is useful for overriding the rc configuration file.

   \optionpar{-r}{--reveal-alsa-ports}
      \textbf{New:}
      \index{new!reveal ALSA ports}
      \textsl{Sequencer66} will show the names of the ALSA port that the system
      defines, rather than the names defined in the 'user' configuration file.

   \optionpar{-R}{--hide-alsa-ports}
      \index{new!hide ALSA ports}
      \textsl{Sequencer66} will show the names of the ALSA port that the 'user'
      configuration file define, rather than the names defined by ALSA.

   \optionpar{-A}{--alsa}
      \textsl{Sequencer66} will not run the JACK support, even if specified
      in the configuration file.  The configuration options are sticky (they
      are saved), and sometimes they aren't what you want to run.

   \optionpar{-s}{--show-midi}
      Dumps incoming MIDI to the screen.

   \optionpar{-p}{--priority}
      Runs at higher priority with a FIFO scheduler.

   \optionpar{N/A}{--pass-sysex}
      Passes any incoming SYSEX messages to all outputs.
		Not yet supported.

   \optionpar{-i}{--ignore [number]}
      Ignore ALSA device [number].

   \optionpar{-k}{--show-keys}
      Prints pressed key value.

   \optionpar{-K}{--inverse}
      \index{inverse colors}
      \index{night mode}
      Changes the color scheme for the sequence editor and performance editor
      piano rolls.  It basicially inverts the colors.  It can be considered
      kind of a "night mode".

   \optionpar{-X}{--playlist [filename]}
      \index{playlist}
      This option loads the given file-name as a play-list file.
      See \sectionref{sec:playlist}.
      This file provides one or more play-list
      entries, each providing a list of one or more songs.  Once loaded, the
      user can use the four arrow keys to move between play-lists and the songs
      in each play-list.  The play-list entries are also controllable via MIDI
      control values set in the "rc" file.  See the Sequencer66 manual for
      more information.  Note that, once set, this option is, by default, saved
      in the "rc" file.

   \optionpar{-x}{--interaction-method [number]}
      Select the mouse interaction method.
      0 = seq24 (the default); and 1 = fruity loops method.
      The latter does not completely support all actions supported by the Seq24
      interaction method, at this time.

      The following options will not be shown by --help if the application is
      not compiled for JACK support.

   \optionpar{-j}{--jack-transport}
      \textsl{Sequencer66} will sync to JACK transport.

   \optionpar{-J}{--jack-master}
      \textsl{Sequencer66} will try to be JACK master.

   \optionpar{-C}{--jack-master-cond}
      JACK master will fail if there is already a master.

   \optionpar{-M}{--jack-start-mode [x]}
      When \textsl{Sequencer66} is synced to JACK, the following play modes
      are available: 0 = live mode; and 1 = song mode, the default.

   \optionpar{-S}{--stats}
      Print statistics on the command-line while running.
      Not available unless this option has been compiled in at build time,
      which can be determined by using the \texttt{--version} option.

   \optionpar{-U}{--jack-session-uuid [uuid]}
      Set the UUID for the JACK session.

   \optionpar{-u}{--user-save}
      Save the "user" configuration file when exiting Sequencer66.
      Normally, it is saved only if not present in the configuration directory,
      so as not to get stuck with temporary settings such as the --bus option.
      Note that the "rc" configuration option are generally also saved.
      But see the "auto-option-save" directive in the "rc" file.
      It is new with version 0.9.9.15.

   \optionpar{-f}{--rc filename}
      Use a different "rc" configuration file.  It must be a file in the user's
      \$HOME/.config/sequencer66 directory or the directory specified by the
      --home option.  Not supported by the --legacy mode.  The '.rc' extension
      is added if no extension is present in the filename.

   \optionpar{-F}{--usr filename}
      Use a different "usr" configuration file.  It must be a file in the
      user's \$HOME/.config/sequencer66 directory or the directory specified by
      the --home option.  Not supported by the --legacy mode.  The '.usr'
      extension is added if no extension is present in the filename.

   \optionpar{-c}{--config basename}
      Use a different configuration file base name for the 'rc' and 'usr'
      files.  For example, one can specify a full configuration for "testing",
      for "jack", or for "alsa", to set up
      \texttt{testing.rc} and \texttt{testing.usr},
      \texttt{jack.rc} and \texttt{jack.usr},
      \texttt{alsa.rc} and \texttt{alsa.usr}.

   \optionpar{-o}{--option opvalue}
      Provides additional options, since the application is running out of
      single-character options.  The \texttt{opvalue} set supported is:
      \begin{itemize}
         \item \texttt{daemonize}.
            \index{daemonize}
            Makes the \texttt{seq66cli} application
            fork to the background.
         \item \texttt{no-daemonize}.
            \index{no-daemonize}
            Makes the \texttt{seq66cli} application
            run in the foreground, where it is easy to see the informational
            output written to the console.
         \item \texttt{log=filename.log}.
            \index{log}
            Reroutes standard error and standard
            output messages to the given log-file.  This file is located in the
            directory for the "rc" and "usr" files (which can be altered via
            the \texttt{-H}/\texttt{--home} directory option).  If this file is
            already present, additional log information is appended to it.  If
            the "=filename.log" portion is missing, the log-file name in the
            \texttt{[user-options]} section of the "usr" file is used as the
            default log-file, if this file-name is not empty.
         \item \texttt{wid=3x2,i}.
            \index{multi-wid}
            Not supported in the Qt version.  Instead, an arbitrary number of
            external live-frame windows can be created.
            Provides for multiple main windows ("mainwids") to be shown in a
            big grid, so that multiple sets can be viewed at the same time.
            The default is to show the usual single mainwid.
            The first number specified is the row count, which can range from 1
            to 3.  The second number specified is the column count, which can
            only be 1 or 2.  The third parameter starts with 't' ("true")
            to indicate that the multiple sets will be controlled by a single
            set spinner, which is the default.  Specifying 'f' ("false") or 'i'
            ("indep") will show one set spinner for each mainwid.  Note that it
            is possible, in this mode, to show the same set in two different
            mainwids, but this is not recommended, as there are minor
            unavoidable issues with that.  Also note that the format for this
            command line option is very strict, no deviations or added spaces.
            Finally, to save these options to the "usr" file, add the
            \texttt{--user-save} option to the command line.
            In that file, the options modified are \texttt{block\_rows} and
            \texttt{block\_columns}.
         \item \texttt{sets=8x8}.
            \index{variset}
            This option, informally known as "variset", allow some changes in
            the set size and layout from the default 4x8 = 32 sets layout.
            \textbf{Warning:}
            \textsl{seq24} was hardwired for supporting 32 patterns per
            set, and there are still places where that is true.  Thus,
            consider this option to be experimental.
            To save these options to the "usr" file, add the
            \texttt{--user-save} option to the command line.
            In that file, the options modified are \texttt{mainwnd\_rows} and
            \texttt{mainwnd\_cols}.
         \item \texttt{scale=x.y}.
            \index{scaling}
            This option scales the main window by the factor x.y, which can
            range from 0.75 to 3.0.  This makes the main window larger, and
            most of its contents larger.  It does not currently scale the
            fonts used in the display.  A scale factor of 2.5 is about the
            maximum useful value.  If multi-wid is active, then you probably
            need a larger monitor to use this factor.
      \end{itemize}

   \texttt{\$HOME/.config/sequencer66.rc} holds the main configuration settings
   for all versions of Sequencer66.  If it does not exist, it will be generated
   when Sequencer66 exits.  If it does exist, it will be rewritten with the
   current configuration of Sequencer66.  Many, or most, of the command-line
   options are "sticky", in that they will be written to the configuration
   file.

   \texttt{\$HOME/.config/sequencer66.usr} stores the MIDI-configuration
   settings and some of the user-interface settings for Sequencer66.  If it
   does not exist, it will be generated with a minimal configuration when
   Sequencer66 exits.  If it does exist, it will be rewritten with the current
   configuration of Sequencer66, but \textsl{only} if the
   \texttt{--user-save} option was provided on the command-line.
   Note that the
   \texttt{--legacy} option causes the old
   configuration-file names (\texttt{.seq66rc} and \texttt{.seq66usr}
   in the \texttt{\$HOME} directory)
   to be used.

   The current Sequencer66 project homepage is a git repository at

   \url{https://github.com/ahlstromcj/sequencer66.git}.

   Up-to-date instructions can be found in the project at

   \url{https://github.com/ahlstromcj/sequencer66-doc.git}.

   The old Seq24 project homepage is at
   \url{http://www.filter24.org/seq24/} the new
   one is at \url{https://edge.launchpad.net/seq24/}.
   It is released under the GNU GPL license.
   Sequencer66 is also released under the GNU GPL license.

   \textsl{Sequencer66} was written by Chris Ahlstrom
   \href{mailto:ahlstromcj@gmail.com}{ahlstromcj@gmail.com}
   (with a fair amount of help).
   \textsl{Seq24} was written by Rob C. Buse
   \href{mailto:seq24@filter24.org}{seq24@filter24.org}
   and the \textsl{Sequencer66} team.

   This manual page was written by Dana Olson
   \url{mailto:seq24@ubuntustudio.com} with additions from Guido Scholz
   \url{mailto:guido.scholz@bayernline.de} and Chris Ahlstrom
   \url{mailto:ahlstromcj@gmail.com}.

%   \begin{verbatim}
% Version 0.94.0                      June 6 2017                  sequencer66(1)
%   \end{verbatim}

%-------------------------------------------------------------------------------
% vim: ts=3 sw=3 et ft=tex
%-------------------------------------------------------------------------------


% Headless version

\input{seq66_headless}

% MIDI implementation chart

% \pagebreak
% %-------------------------------------------------------------------------------
% seq66_midi_impl_chart
%-------------------------------------------------------------------------------
%
% \file        seq66_midi_impl_chart.tex
% \library     Documents
% \author      Chris Ahlstrom
% \date        2018-02-04
% \update      2018-02-04
% \version     $Revision$
% \license     $XPC_GPL_LICENSE$
%
%     Provides a reasonable facsimile of a MIDI Implementation Chart for
%     Sequencer66.
%
%-------------------------------------------------------------------------------

\section{MIDI Implementation Chart}
\label{sec:midi_impl_chart}

   This section provides a basic and still-in-progress MIDI implementation
   chart for \textsl{Sequencer66}.

   \begin{table}[htb]
      \label{table:midi_impl_chart_top}
      \begin{tabular}{l c r}
         Sequencer66 Live MIDI &   & Date: \date{\today} \\
         Seq66 rtmidi & \textbf{MIDI Implementation Chart} & Version 0.94 \\
      \end{tabular}
   \end{table}

%  \begin{table}[htb]
   \begin{table}[H]
      \label{table:midi_impl_chart_main}
      \begin{tabular}{| l | l | l | l |}

% Each section is a row.

         \cline{1-4}
         \textbf{Function} &
            \textbf{Transmitted} &
            \textbf{Recognized} &
            \textbf{Remarks} \\

         \cline{1-4}
         \begin{tabular}{l l}
            Basic & Default \\
            Channel & Changed \\
         \end{tabular} &
            \makecell[l]{1-16 \\ 1-16} &
            \makecell[l]{1-16 \\ 1-16} &
            \makecell[l]{Songs sets out channel(s) \\ Config sets in port(s)} \\

         \cline{1-4}
         \begin{tabular}{l l}
             & Default \\
             Mode & Messages \\
             & Altered \\
         \end{tabular} &
            \makecell[l]{ TODO \\ TODO } &
            \makecell[l]{ TODO \\ TODO } &
            \makecell[l]{ TODO \\ TODO } \\

         \cline{1-4}
         \begin{tabular}{l l}
             Note & \\
             Number & True Voice \\
         \end{tabular} &
            \makecell[l]{ TODO \\ TODO } &
            \makecell[l]{ TODO \\ TODO } &
            \makecell[l]{ TODO \\ TODO } \\

         \cline{1-4}
         \begin{tabular}{l l}
             Velocity & Note ON \\
             & Note OFF \\
         \end{tabular} &
            \makecell[l]{ TODO \\ TODO } &
            \makecell[l]{ TODO \\ TODO } &
            \makecell[l]{ TODO \\ TODO } \\

         \cline{1-4}
         \begin{tabular}{l l}
             After & Key's \\
             Touch & Ch's \\
         \end{tabular} &
            \makecell[l]{ TODO \\ TODO } &
            \makecell[l]{ TODO \\ TODO } &
            \makecell[l]{ TODO \\ TODO } \\

         \cline{1-4}
         \begin{tabular}{l l}
             Pitch Bender & \\
         \end{tabular} &
            \makecell[l]{ TODO \\ TODO } &
            \makecell[l]{ TODO \\ TODO } &
            \makecell[l]{ TODO \\ TODO } \\

         \cline{1-4}
         \begin{tabular}{l l}
             Control & 1-127 \\
         \end{tabular} &
            \makecell[l]{ TODO \\ TODO } &
            \makecell[l]{ TODO \\ TODO } &
            \makecell[l]{ TODO \\ TODO } \\

         \cline{1-4}
         \begin{tabular}{l l}
             Prog & \\
             Change & True No. \\
         \end{tabular} &
            \makecell[l]{ TODO \\ TODO } &
            \makecell[l]{ TODO \\ TODO } &
            \makecell[l]{ TODO \\ TODO } \\

         \cline{1-4}
         \begin{tabular}{l l}
             System Exclusive & \\
         \end{tabular} &
            \makecell[l]{ TODO \\ TODO } &
            \makecell[l]{ TODO \\ TODO } &
            \makecell[l]{ TODO \\ TODO } \\

         \cline{1-4}
         \begin{tabular}{l l}
             System & Song Pos \\
             Common & Song Sel \\
             & Tune \\
         \end{tabular} &
            \makecell[l]{ TODO \\ TODO } &
            \makecell[l]{ TODO \\ TODO } &
            \makecell[l]{ TODO \\ TODO } \\

         \cline{1-4}
         \begin{tabular}{l l}
             System & Clock \\
             Real Time & Commands \\
         \end{tabular} &
            \makecell[l]{ TODO \\ TODO } &
            \makecell[l]{ TODO \\ TODO } &
            \makecell[l]{ TODO \\ TODO } \\

         \cline{1-4}
         \begin{tabular}{l l}
             & Local ON/OFF \\
             Aux & All Notes OFF \\
             Message & Active Sense \\
             & Reset \\
         \end{tabular} &
            \makecell[l]{ TODO \\ TODO } &
            \makecell[l]{ TODO \\ TODO } &
            \makecell[l]{ TODO \\ TODO } \\

         \cline{1-4}
            Notes &
            \makecell[l]{ TODO \\ TODO } &
            \makecell[l]{ TODO \\ TODO } &
            \makecell[l]{ TODO \\ TODO } \\

% Bottom of chart.

         \cline{1-4}

      \end{tabular}
   \end{table}

   \begin{table}[htb]
      \label{table:midi_impl_chart_bottom}
      \begin{tabular}{l l l}
         Mode 1: OMNI ON, POLY  & Mode 2: OMNI ON, MONO  & -O- : Yes \\
         Mode 3: OMNI OFF, POLY & Mode 4: OMNI OFF, MONO & -X- : No  \\
      \end{tabular}
   \end{table}

%-------------------------------------------------------------------------------
% vim: ts=3 sw=3 et ft=tex
%-------------------------------------------------------------------------------

% \pagebreak

% Important Concepts

%-------------------------------------------------------------------------------
% seq66_concepts
%-------------------------------------------------------------------------------
%
% \file        seq66_concepts.tex
% \library     Documents
% \author      Chris Ahlstrom
% \date        2015-11-01
% \update      2019-08-15
% \version     $Revision$
% \license     $XPC_GPL_LICENSE$
%
%     Provides some concepts and terms needed to understand Sequencer66.
%
%-------------------------------------------------------------------------------

\section{Concepts}
\label{sec:concepts}

   The \textsl{Sequencer66} program is a loop-playing machine with a 
   simple interface.  Before we describe this interface, it is useful
   to present some concepts and definitions of terms as
   they are used in \textsl{Sequencer66}.  Various terms have been used over
   the years to mean the same thing (e.g. "sequence", "pattern", "loop",
   "track", and "slot"), so it is good to clarify the terminology.

\subsection{Concepts / Terms}
\label{subsec:concepts_terms}

   This section doesn't provide comprehensive coverage of terms.  It
   covers terms that might be puzzling.

\subsubsection{Concepts / Terms / armed, muted}
\label{subsubsec:concepts_terms_armed}

   \index{armed}
   An armed sequence is a sequence
   (see \sectionref{subsubsec:concepts_terms_sequence})
   that will be heard.  "Armed" is the opposite
   of "muted".  Performing an \textsl{arm} operation in \textsl{Sequencer66}
   means clicking on an "unarmed" sequence in the patterns panel (the main
   window of \textsl{Sequencer66}).  An unarmed sequence will not be heard, and
   it has a white background.  When the sequence is \textsl{armed}, it will be
   heard, and it has a darker background.
   A sequence can be armed or unarmed in three ways:

   \begin{itemize}
      \item Clicking on a sequence/pattern box.
      \item Pressing the hot-key for that sequence/pattern box.
      \item Opening up the Song Editor and starting playback; the
            sequences arm/unarm depending on the layout of the
            sequences and triggers in the piano roll of the Song Editor.
   \end{itemize}

\subsubsection{Concepts / Terms / bank, screenset}
\label{subsubsec:concepts_terms_bank}

   \index{screen set}
   The \textsl{screen set}
   is a set of patterns that fit within the 8x4 grid of loops/patterns in the
   Patterns panel.
   \textsl{Sequencer66} supports multiple screens sets, up to 32 of them,
   and a name can be given to each for clarity.

   \index{bank}
   This term is \textsl{Kepler34}'s name for "screen set".

\subsubsection{Concepts / Terms / buss (bus or port)}
\label{subsubsec:concepts_terms_buss}

   \index{bus}
   \index{buss}
   A \textsl{buss} (also spelled "bus" these days;
   \url{https://en.wikipedia.org/wiki/Busbar}) is an entity onto which
   MIDI events can be placed, in order to be heard or to affect the
   playback.
   A \textsl{buss} is just another name for port.
   See \sectionref{subsubsec:concepts_terms_port}.

\subsubsection{Concepts / Terms / export}
\label{subsubsec:concepts_terms_export}

   \index{export}
   A \textsl{export} in \textsl{Sequencer66} is a way of writing a
   song-performance to a more standard MIDI file, so that it can be played
   exactly by other sequencers.
   An export collects all of the unmuted tracks that have
   performance information (triggers) associated with them, and creates one
   larger trigger for each track, repeating the events as indicated by the
   original performance.

\subsubsection{Concepts / Terms / group}
\label{subsubsec:concepts_terms_group}

   \index{group}
   A \textsl{group} in \textsl{Sequencer66} is a
   previously-defined mute/unmute pattern in the active screen set.
   A group is a set of patterns, in the current screen-set,
   that can arm (unmute) their playing state
   together.  Every group contains all sequences in the active screen
   set.  This concept is similar to mute/unmute groups in hardware
   sequencers.
   \index{mute-group}
   Also known as a "mute-group".

\subsubsection{Concepts / Terms / loop (pattern, track, sequence)}
\label{subsubsec:concepts_terms_loop}

   \index{loop}
   \textsl{Loop}
   is a synonym for \textsl{pattern} or \textsl{sequence}, when used
   in existing \textsl{Seq24} documentation.
   Each loop is represented by a box (pattern slot) in the Pattern (main)
   Window.

   A \textsl{Sequencer66} \textsl{pattern}
   \index{pattern}
   (also called a "sequence" or "loop")
   is a short unit of melody or rhythm in \textsl{Sequencer66},
   extending for a small number of measures (in most cases).
   Each pattern is represented by a box in the Patterns window.
   Each pattern is editable on its own.  All patterns can be layed out in
   a particular arrangement to generate a more complex song.

   \index{sequence}
   \textsl{Sequence} is
   a synonym for \textsl{pattern}.

   Note that other sequencer applications use the term "sequence"
   to apply to the complete song, and not just to one track or pattern in the
   entire song.

\subsubsection{Concepts / Terms / performance}
\label{subsubsec:concepts_terms_performance}

   In the jargon of \textsl{Sequencer66}, a
   \index{performance}
   \textsl{performance} is an organized collection of patterns.
   This layout of patterns is created using the Song Editor, sometimes
   called the "performance editor".
   This window controls the song playback in "Song Mode" (as opposed to "Live
   Mode").
   The playback of each track is controlled by a set of triggers created for
   that track.

\subsubsection{Concepts / Terms / pulses per quarter note}
\label{subsubsec:concepts_terms_pulses}

   \index{pulses}
   The concept of "pulses per quarter note", or PPQN, is very important for
   MIDI timing.  To make it a bit more confusing, sometimes these pulses are
   referred to as "ticks", "clocks", and "divisions".
   To make it even more confusing, there are separate timing concepts to
   understand, such as "tempo", "beats per measure", "beats per minute",
   "MIDI clocks", and more.

   While a full description of all these terms, and how they are calculated, is
   beyond the scope of this document, we will try to clarify the discussion
   when such confusion could be an issue.

\subsubsection{Concepts / Terms / queue}
\label{subsubsec:concepts_terms_queue_mode}

   To "queue" a pattern means to ready it for playback on the next repeat of
   a pattern.  A pattern can be armed immediately, or it can be queued to
   play back the next time the pattern restarts.
   Pattern toggles occur at the end of
   the pattern, rather than being set immediately.

   A set of queued patterns can be temporarily stored, so that a different
   set of playbacks can occur, before the original set of playbacks is
   restored.

   \index{keep queue}
   \index{queue!keep}
   The "keep queue" functionality allows the queue to be held without
   holding down a button the whole time.  Once this key is pressed,
   then the hot-keys for any pattern can be pressed, over and over,
   to queue each pattern.

\subsubsection{Concepts / Terms / snapshot}
\label{subsubsec:concepts_terms_snapshot}

   \index{snapshot}
   A \textsl{Sequencer66} \textsl{snapshot} is a briefly preserved
   state of the patterns.  One can press a snapshot key, change the state of
   the patterns for live playback, and then release the snapshot key to
   revert to the state when it was first pressed.  (One might call it a
   "revert" key, instead.)

\subsubsection{Concepts / Terms / song}
\label{subsubsec:concepts_terms_song}

   \index{song}
   A \textsl{song} is a collection of patterns (a performance)
   in a specific temporal layout, as assembled via the Song Editor window.
   See \ref{subsubsec:concepts_terms_performance}.
   See \ref{subsubsec:concepts_terms_trigger}.

\subsubsection{Concepts / Terms / trigger}
\label{subsubsec:concepts_terms_trigger}

   \index{trigger}
   A \textsl{trigger} is indicates when a sequence/pattern/loop
   should be played, and how much of the sequence (including repeats) should be
   played.  A song performance consists of a number of sequences, each
   triggered in ways that the musician can lay out.

\subsection{Concepts / Sound Subsystems}
\label{subsec:concepts_sound_subsystems}

\subsubsection{Concepts / Sound Subsystems / ALSA}
\label{subsubsec:concepts_sound_alsa}

   \textsl{ALSA} is a sound and MIDI system for Linux, with components built
   into the Linux kernel. It is the main subsystem used by
   \textsl{Sequencer66}.  The name of the library used to build
   \textsl{ALSA} projects is \texttt{libasound}.
   See reference \cite{alsa}.

\subsubsection{Concepts / Sound Subsystems / PortMIDI}
\label{subsubsec:concepts_sound_portmidi}

   \textsl{PortMIDI} is a cross-platform API (applications programming
   interface) for MIDI.  It is used in the "portmidi" C++ modules
   included with the base source-code repository of \textsl{Seq24} available
   (for example) from Debian Linux.  See reference \cite{portmidi}
   for the PortMIDI home page.
   This is the API used for Windows or Mac OSX.

%  The SubatomicGlue Windows port of \textsl{Seq24} (see reference
%  \cite{subatomicglue}) bundles a version of the PortMIDI project with the
%  source code for the port.  It also provides a complete bundle of the
%  other products (e.g. gtkmm 2.4) needed to build and run the project.
%  (By the way, the Windows port is built with
%  MingW, which provides the GNU compilers and tools.  This is a good thing,
%  as Visual Studio Community, though "free", is not "Free".)

\subsubsection{Concepts / Sound Subsystems / JACK}
\label{subsubsec:concepts_sound_jack}

   \textsl{JACK} is a cross-platform (with an emphasis on Linux)
   API and infrastructure for making it easier to connect and reroute MIDI
   and audio event between various applications and hardware ports.
   See reference \cite{jack}.

%-------------------------------------------------------------------------------
% vim: ts=3 sw=3 et ft=tex
%-------------------------------------------------------------------------------


% Building and debugging Sequencer66

\input{seq66_build}

% Discussion of MIDI formats related to Seq24 and Sequencer66

\input{seq66_midi_formats}

% Discussion of JACK support

\input{seq66_jack}

% Acknowledgments

%-------------------------------------------------------------------------------
% sequencer66-user-manual
%-------------------------------------------------------------------------------
%
% \file        seq66_kudos.txt
% \library     Documents
% \author      Chris Ahlstrom
% \date        2016-08-29
% \update      2018-10-28
% \version     $Revision$
% \license     $XPC_GPL_LICENSE$
%
%     This document provides LaTeX documentation for Sequencer66.
%
%-------------------------------------------------------------------------------

\section{Kudos}
\label{sec:kudos}

   This section gives some credit where credit is due.
   We have contributors to acknowledge, and have not caught up with all the
   people who have helped this project:

   \begin{itemize}
      \item \textsl{Tim Deagan (tdeagan)}:
         fixes to the mute-group support.
      \item \textsl{0rel}:
         an important fix to add and relink notes after a
         paste action in the pattern editor.
      \item \textsl{arnaud-jacquemin}:
         a bug report and fix for a regression in mute-groups support.
      \item \textsl{Stan Preston (stazed)}:
         ideas for some upcoming improvements based
         on his \textsl{seq32} project.  A lot of ideas.
         And a lot of code!
      \item \textsl{Animtim}:
         a number of bug reports and a new logo.
      \item \textsl{jean-emmanuel}:
         scrollable main-window support, other features and reports.
      \item \textsl{Olivier Humbert (trebmuh)}:
         French translation for the desktop files.
      \item \textsl{Oli Kester}:
         The creator of \textsl{Kepler34}, from which we are getting
         clues on porting the user-interface to Qt 5 and Windows.
   \end{itemize}

   Also some bug-reporters and testers:

   \begin{itemize}
      \item \textsl{F0rth}:
         a request for scripting support, a possible future feature.
      \item \textsl{gimmeapill}:
         testing, bug-reports, and, um, "marketing".
      \item \textsl{georgkrause}:
         a number of helpful bug reports.
      \item \textsl{goguetchapuisb}:
         found that Seq66 native JACK did not properly handle the copious
         Active Sensing messages emitted by Yamaha keyboards.
      \item \textsl{milkmiruku}:
         mainwids issues and ideas.
      \item \textsl{muranyia}:
         feature request for numbered piano keys and bug-reports.
      \item \textsl{pixelrust}:
         reports of issues with "fruity" interaction.
      \item \textsl{simonvanderveldt}:
         issues with window sizing and more.
      \item \textsl{ssj71}:
         a request for an LV2 plugin version, a possible future feature.
      \item \textsl{triss}:
         a request for OSC support, a possible future feature.
      \item \textsl{layk}:
         some bug reports, and, we are pretty sure, some nice videos that
         demonstrate \textsl{Sequencer66} on \textsl{YouTube}.  See
         \cite{layk}.
      \item \textsl{matt-bel}:
         reported a regression from \textsl{Seq24}, which could use
         a MIDI control event to mute/unmute multiple patterns at once,
         a cool feature!
         
   \end{itemize}

   ... and there are more to add to this list....

   There are a number of authors of \textsl{Seq24}.
   ideas from other \textsl{Seq24} fans),
   and some deep history,
   as one can see in \figureref{fig:seq66_menu_help_credits},
   and in \figureref{fig:seq66_menu_help_doc}.
   All of these authors, and more, have contributed to \textsl{Sequencer66},
   whether they know it or not.
   The original author is Rob C. Buse; where the word "I" occurs, that is
   probably him.  Without his work, we would never have started
   \textsl{Sequencer66}.

   From the original author:

   \begin{quotation}
      \textsl{Seq24} is a real-time MIDI sequencer. It was created to
      provide a very simple interface for editing and playing MIDI 'loops'.
      After searching for a software based sequencer that would provide the
      functionality needed for a live performance, there was little found in
      the software realm. I set out to create a very minimal sequencer that
      excludes the bloated features of the large software sequencers, and
      includes a small subset of features that I have found usable in
      performing. 

      Written by Rob C. Buse.  I wrote this program to fill a
      hole.  I figure it would be a waste if I was the only one
      using it.  So, I released it under the GPL.
   \end{quotation}

   Taking advantage of Rob's generosity,
   we've created a reboot, a refactoring, an improvement (we hope) of
   \textsl{Seq24}.  It preserves (we hope) the lean nature of \textsl{Seq24},
   while adding a few useful features.
   Without \textsl{Seq24} and its authors,
   \textsl{Sequencer66} would never have come into being.

%-------------------------------------------------------------------------------
% vim: ts=3 sw=3 et ft=tex
%-------------------------------------------------------------------------------


\section{Summary}
\label{sec:summary}

   Contact: If you have ideas about \textsl{Sequencer66} or a bug report, please
   email us (at \url{mailto:ahlstromcj@gmail.com}).
   If it's a bug report, please add \textbf{[BUG]} to the Subject, or use the
   GitHub bug-reporting interface.

% References

\input{seq66_references}

\printindex

\end{document}

%-------------------------------------------------------------------------------
% vim: ts=3 sw=3 et ft=tex
%-------------------------------------------------------------------------------
