%-------------------------------------------------------------------------------
% seq66-user-manual
%-------------------------------------------------------------------------------
%
% \file        seq66-user-manual.tex
% \library     Documents
% \author      Chris Ahlstrom
% \date        2015-11-01
% \update      2020-12-29
% \version     $Revision$
% \license     $XPC_GPL_LICENSE$
%
%     This document provides LaTeX documentation for Seq66.
%
%-------------------------------------------------------------------------------

\documentclass[
 11pt,
 twoside,
 a4paper,
 headinclude,
 footinclude,
 final                                 % versus draft
]{article}

%-------------------------------------------------------------------------------
% docs-structure
%-------------------------------------------------------------------------------
%
% \file        docs-structure.tex
% \library     Documents
% \author      Chris Ahlstrom
% \date        2015-04-20
% \update      2021-01-28
% \version     $Revision$
% \license     $XPC_GPL_LICENSE$
%
%     This "include file" provides LaTeX options for a document.
%
%     Note that enumitem is an extension of enumerate, and comes from
%     Debian's texlive-latex-recommended package.
%
%-------------------------------------------------------------------------------

\usepackage{enumitem}         % setting the whitespace between and within lists
\setlistdepth{9}
\setlist{noitemsep}           % spacing within the list

\usepackage{color}            % provide colors?
\usepackage{nameref}          % Provide references by name instead of number
\usepackage[colorlinks=true,linkcolor=webgreen,filecolor=webbrown,citecolor=webgreen]{hyperref}
\definecolor{webgreen}{rgb}{0,.5,0}
\definecolor{webbrown}{rgb}{.6,0,0}

\usepackage{ragged2e}         % For underfull boxes in the bibliography
\usepackage{verbatim}         % For the comment macro
\usepackage{url}              % Required for including URLs
\usepackage{hyperref}         % Required for including hyperlinks
\usepackage{amsthm}           % Helps avoid "destination with same
\usepackage[hypcap]{caption}  % make labels point to figure, not the caption
\usepackage[pdftex]{graphicx} % Required for including images
\graphicspath{{../images}}    % Set the default folder for images
\usepackage{float}            % For more control of location of Figures
\usepackage[T1]{fontenc}      % Remove font warnings for textleftbrace, etc.
\usepackage{geometry}         % Page & text layout
\geometry{
  letterpaper,
  top=2.5cm,
  bottom=2.5cm,
  left=2.5cm,
  right=2.5cm
}

\usepackage{longtable}        % For making multi-page tables
\usepackage{makeidx}          % For making an index

% Try to reduce the space before or after verbatim sections.
% Doesn't affect the spacing after the verbatim, though.

\usepackage{etoolbox}
\makeatletter
\preto{\@verbatim}{\topsep=10pt \partopsep=0pt}
\makeatother

% Let's try to reduce the size of quotations.

\usepackage{relsize,etoolbox}          % http://ctan.org/pkg/{relsize,etoolbox}
\AtBeginEnvironment{quotation}{\smaller}   % Step font down one size relatively

% For the MIDI Implementation Chart

\usepackage{makecell}

% This package isn't available easily on CentOS:
%
% \usepackage[subtle]{savetrees} % For tightening document vertical spacing

\hypersetup{                  % HYPERLINKS
% draft,                      % Uncomment removes links (e.g. for B&W printing)
 colorlinks=true,
 breaklinks=true,
 bookmarksnumbered,
 urlcolor=webbrown,
 linkcolor=blue,              % RoyalBlue
 citecolor=webgreen,
 pdftitle={},
 pdfauthor={\textcopyright},
 pdfsubject={},
 pdfkeywords={},
 pdfcreator={pdfLaTeX},
 pdfproducer={LaTeX with hyperref and ClassicThesis}
}

% Make an "enumber" style that makes all levels of enumerated lists show
% arabic numerals.

\newlist{enumber}{enumerate}{10}
\setlist[enumber]{nolistsep,label=\arabic*.}

% Make "paragraph" a fourth level, and make it shown in the table of
% contents.

\makeatletter
\renewcommand\paragraph{\@startsection{paragraph}{4}{\z@}%
   {-2.5ex\@plus -1ex \@minus -.25ex}%
   {1.25ex \@plus .25ex}%
   {\normalfont\normalsize\bfseries}}
\makeatother
\setcounter{secnumdepth}{4} % how many sectioning levels to assign numbers to
\setcounter{tocdepth}{4}    % how many sectioning levels to show in ToC

% Provide a way of counting user-interface items without putting them in an
% enumberation.

\newcounter{ItemCounter}

% Makes a numbered paragraph out of an item, and allows two index entries
% for it.

\newcommand{\itempar}[2] {
   \noindent
   \stepcounter{ItemCounter}
   \textbf{\arabic{ItemCounter}. #1.}
   \index{#1}
   \index{#2}
}

% Provides for two forms of an option, as might be shown in a man page.

\newcommand{\optionpar}[2] {
   \textbf{\texttt{#1}} \textbf{\texttt{#2}} \\
   \index{#1}
   \index{#2}
}

% Similar, but with no line break.

\newcommand{\optionline}[2] {
   \textbf{\texttt{#1}} \textbf{\texttt{#2}}
   \index{#1}
   \index{#2}
}

% Now deprecated in preference to \itempar

\newcommand{\settingdesc}[2] {
   \textbf{#1}
   \index{#1}
   \index{#2}
}

% Reference to a configuration file setting
%
%     \configref{xxx}{xxxxx}{xxxx}.

\newcommand{\configref}[3] {
   \index{#1!#2}
   \texttt{qseq66.#1: [#2] #3}.
}

% Make a full reference to a figure using its number, its name, and its page
% number.  Very useful if you have a hard-copy of the document to deal with.

\newcommand{\figureref}[1] {
   Figure~\ref{#1}
   "\nameref{#1}"
   on page~\pageref{#1}\ignorespaces
}

% Make a full reference to a section using its number, its name, and its page
% number.  Very useful if you have a hard-copy of the document to deal with.

\newcommand{\sectionref}[1] {%
   section~\ref{#1}
   "\nameref{#1}"
   on page~\pageref{#1}\ignorespaces
}

% Make a full reference to a "paragraph"  using its number, its name, and
% its page number.  Very useful if you have a hard-copy of the document to
% deal with.

\newcommand{\paragraphref}[1] {%
   paragraph~\ref{#1}
   "\nameref{#1}"
   on page~\pageref{#1}\ignorespaces
}

% Make a full reference to a table using its number, its name, and its page
% number.  Very useful if you have a hard-copy of the document to deal with.

\newcommand{\tableref}[1] {%
   table~\ref{#1}
   "\nameref{#1}"
   on page~\pageref{#1}\ignorespaces
}

% For lining up enumerated items.  Doesn't really work well, better
% to create a table.

\newcommand{\itab}[1]{\hspace{0em}\rlap{#1}}
\newcommand{\tab}[1]{\hspace{.1\textwidth}\rlap{#1}}

% Change the fragction of the page that can be filled with graphics from 0.7
% to 0.9.

\renewcommand\floatpagefraction{.9}
\renewcommand\dblfloatpagefraction{.9}
\renewcommand\topfraction{.9}
\renewcommand\dbltopfraction{.9}
\renewcommand\bottomfraction{.9}

\raggedbottom                          % avoid excessive vertical justification

%-------------------------------------------------------------------------------
% vim: ts=3 sw=3 et ft=tex
%-------------------------------------------------------------------------------
                 % specifies document structure and layout

% Replacing normal header/footer with a fancier version.  These two symbols of
% document class were showing up as "unused" in the log file.
%
% headinclude,
% footinclude,
%
% So we add the fancyhdr package, clear the default layout, and set it up for
% our wider pages.

\usepackage{fancyhdr}
\pagestyle{fancy}
\fancyhead{}
\fancyfoot{}
\fancyheadoffset{0.005\textwidth}
\lhead{Seq66 Live MIDI Sequencer}
\chead{}
\rhead{User Manual}
\lfoot{}
\cfoot{\thepage}
\rfoot{}

\makeindex

\begin{document}

\title{Seq66 User Manual 0.91.5}
\author{Chris Ahlstrom \\
   (\texttt{ahlstromcj@gmail.com})}
\date{\today}
\maketitle

% UPDATE THE FOLLOWING FIGURE !!!!!!!!!!!!!!!!!!!!!!!!!

\begin{figure}[H]
   \centering 
%  \includegraphics[scale=0.5]{main-window/main-window-dark-gridstyle-3-default.png}
   \includegraphics[scale=0.5]{main-window/main-window-light-gridstyle-3-default.png}
   \caption*{Button-Grid User Interface, Qt 5}
\end{figure}

\clearpage                             % moves Contents to next page

\tableofcontents
\listoffigures                         % print the list of figures
\listoftables                          % print the list of tables

% Changes the paragraph style to remove indenting and put a line between each
% paragraph.  This could be moved up into the preamble, but then would
% affect the spacing of the TOC and LOF, LOT noted above.

\setlength{\parindent}{2em}
\setlength{\parskip}{1ex plus 0.5ex minus 0.2ex}

\section{Introduction}
\label{sec:introduction}

   This document describes "Seq66", better known as
   \textsl{Seq66},
%  \cite{sequencer66},
   through version 0.91.5.
   The following projects support \textsl{Seq66} and documentation:

   \begin{itemize}
      \item \url{https://github.com/ahlstromcj/seq66.git}.
      \item \url{https://github.com/ahlstromcj/sequencer64-doc.git}.
   \end{itemize}

   (We reference the old \textsl{Sequencer64} documentation project because
   the current manual is still very incomplete.  The \textsl{Sequencer64}
   document is probably about 80 percent useful as a guide to \textsl{Seq66}.
   Just be prepared to note some significant differences in look-and-feel.)
   We would like a more lean-and-mean manual for \textsl{Seq66}.
   For now, we are concentrating on the new features in this manual.

   \textsl{Seq66} is \textsl{Sequencer64} refactored for newer versions of
   \textsl{C++} and with kruft and redundancy removed.  It drops the
   \textsl{Gtkmm} user-interface in favor of \textsl{Qt 5},
   and has better handling of sets and configuration files.
   It includes support for the \textsl{Non Session Manager}.

   We have many contributors to acknowledge.
%  Please see \sectionref{sec:kudos}.

\subsection{Seq66: What?}
\label{subsec:what_is_sequencer66}

   \textsl{Seq66} is an ongoing reboot of \textsl{Seq24},
   a live-looping sequencer with an interface similar to a hardware sequencer.
   \textsl{Seq66} is not a synthesizer.  It requires a hardware
   synthesizer or a software synthesizer.

%  such as Timidity \cite{timidity}, FluidSynth \cite{fluidsynth}, etc.

\subsection{Seq66: Why?}
\label{subsec:introduction_vs_others}

   The first reason to refactor \textsl{Sequencer64} is to take advantage of
   things learned in responding to user reports.  The second reason is to use
   the new code as an opportunity to add new functionality such as
   \textsl{Non Session Manager} support.  The last reason is to tighten the
   code by using newer features of \textsl{C++11} and beyond.

\subsection{Improvements}
\label{subsec:improvements}

   The following improvements are some that have been made in
   \textsl{Seq66} versus \textsl{Sequencer64}.

   \begin{itemize}
      \item Qt 5 as the standard user-interface.
      \item A mutes editor tab.
      \item A playlist editor tab.
      \item A sets editor tab.
      \item A better live frame (main window and external windows).
      \item Non Session Manager support.
      \item Repartition of the configuration files.
   \end{itemize}

   For developers, a \textsl{Seq66} build is customizable via C macros,
   by enabling/disabling options at 'configure' time, and by many
   command-line arguments.  We cannot show all permutations of settings in this
   document, so don't be surprised if some screenshots don't quite match
   one's setup.  Distro maintainers might create their own build
   configurations.

\subsection{Document Structure}
\label{subsec:introduction_document_structure}

   The structure of this document follows the user-interface of
   \textsl{Seq66}.
   To help the reader jump around this document, it provides
   multiple links, references, and index entries.

\subsection{Let's Go!}
\label{subsec:introduction_lets_go}

   Make sure no other sound application is running, for the first run.
   Start \textsl{Seq66} to use JACK for MIDI, or
   on \textsl{Windows}, just run it (\texttt{qseq66}, or \texttt{qpseq66.exe}
   on \textsl{Windows}).
   The port settings will depend on your system.  Provide a MIDI file.
   On our system, the synthesizer (\textsl{Yoshimi}) comes up on MIDI buss 5;
   on \textsl{Windows}, buss 0 is the "MIDI Mapper", while buss 1 is the
   built-in wavetable synthesizer.
   The \texttt{--buss} option remaps all events to the desired buss:

\begin{verbatim}
   $ qseq66 --jack-midi --buss 5 data/midi/b4uacuse-gm-patchless.midi
   C:\> qpseq66 --buss 1  data/midi/b4uacuse-gm-patchless.midi
\end{verbatim}

   The "data" directory is a installation directory, such as
   \texttt{/usr/share/seq66-0.90/} on \textsl{Linux} and
   \texttt{C:/Program Files (x86)/Seq66} on \textsl{Windows}.

   The configration directories are, by default,
   \texttt{/home/user/.config/seq66} on \textsl{Linux} and
   \texttt{C:/Users/user/AppData/Local/seq66} on \textsl{Windows}.
   These are created after the first run of \textsl{Seq66}.

   If the \texttt{--alsa} option is used instead of
   \texttt{--jack-midi}, then the \textsl{ALSA} subsystem is used
   (Linux only).  The following figure shows the main window
   using a light desktop theme.

\begin{figure}[H]
   \centering 
%  \includegraphics[scale=0.5]{main-window/main-window-light-gridstyle-3-default.png}
   \includegraphics[scale=0.65]{main-window/main-window-light-gridstyle-3-markup.png}
   \caption{Seq66 Main Screen}
   \label{fig:main_screen}
\end{figure}

   The \textsl{Seq66} main window appears, as shown above.
   This figure has some differences from the \textsl{Seq24} main window,
   but the functionality is similar.
   Most features, including the "look" of the application,
   can be configured via the 'rc', 'usr', 'ctrl', 'drums', 'playlist', 'mutes',
   and 'palette' configuration files, or via command-line options.
   There are many new front-panel items in \textsl{Seq66}.
   Many of these buttons have configurable keystrokes and configurable MIDI
   controls as well.

   For discussion, we break the discussion into sections for the following
   groups shown in the figure above:

   \begin{itemize}
      \item \textbf{Menu}
      \item \textbf{Main} (Top and Bottom panels)
      \item \textbf{Tabs} (each in their own section)
   \end{itemize}

   The \textbf{Live} tab provides a grid of patterns (also called loops or
   sequences) that display recorded MIDI data, status information, and provide
   popup-menus to further work on each pattern.  Note here that the buttons can
   be colored, and that the status of being armed is easy to see from the
   coloring.
   The \textbf{Live} tab will be discussed in more detail in its own section.

   In this section we discuss the top and bottom \textbf{Main} controls, as
   shown in the following figure:

\begin{figure}[H]
   \centering 
   \includegraphics[scale=0.75]{main-window/main-window-controls.png}
   \caption{Main Screen Controls}
   \label{fig:main_screen_controls}
\end{figure}

\subsubsection{Main Top Controls}
\label{subsubsec:introduction_main_top_controls}

   The top main control items are, from left to right:

   \begin{itemize}
      \item \textbf{PPQN Selection}
      \item \textbf{Set's Buss Override}
      \item \textbf{Global Beats Per Measure}
      \item \textbf{Global Beat Width}
      \item \textbf{BBT or HMS Time Display}
      \item \textbf{Beat Indicator}
   \end{itemize}

% Use this?
%  \setcounter{ItemCount}{0}
%  \itempar(PPQN Selection}{PPQN!selection}

\paragraph{PPQN Selection}
\label{paragraph:introduction_ppqn_selection}

   This dropdown allows one to change the pulses-per-quarter-note (PPQN) of the
   loaded tune.
   As with \textsl{Seq24}, the default PPQN is 192.  This can be changed to
   match the PPQN read from the file, or to other values.

   STILL NEEDS WORK.

   Values:
      \texttt{File, 32, 48, 96, 192, 384, 768, 960, 1920, 3840, 7680, 9600, 19200}

\paragraph{Set's Buss Override}
\label{paragraph:introduction_sets_buss_override}

   This dropdown allows for overriding the buss (port) number used by all of
   the patterns in the current set.  This modifies the file and will lead to a
   prompt to save.
   The list of
   output busses is either the existing MIDI ports on the system, or,
   if port-mapping (see \sectionref{sec:port_mapping}) is active, the list
   of mapped output ports.
   This is an easy way to redirect the set to a different output device.

\paragraph{Global Beats Per Measure}
\label{paragraph:introduction_global_beats_per_measure}

   This dropdown changes the global beats/measure for the song.

   Values: \texttt{1 to 16}

\paragraph{Global Beat Width}
\label{paragraph:introduction_global_beat_width}

   This dropdown changes the global beat width (time-signature denominator)
   for the song.

   Values: \texttt{1, 2, 4, 8, 16}

\paragraph{BBT or HMS Time Display}
\label{paragraph:introduction_time_display}

   This text simply shows the current time during playback. 
   It can be shown in BBT (bars:beats:ticks) or HMS (hours:minutes:seconds).

\paragraph{Beat Indicator}
\label{paragraph:introduction_beat_indicator}

   The beat indicator is inspired by the \textsl{Kepler34} implementation.  It
   shows the first beat in red, and the rest of the beats in white.

   STILL NEEDS WORK. It does not adapt to changes in the time-signature.

\subsubsection{Main Bottom Controls}
\label{subsubsec:introduction_main_bottom_controls}

   The bottom main control items take up two rows.  The first row contains:

   \begin{itemize}
      \item \textbf{Set Name}
      \item \textbf{Set Master Button}
      \item \textbf{Active Set Indicator}
      \item \textbf{Set Changer}
   \end{itemize}

   The second row contains:

   \begin{itemize}
      \item \textbf{Panic Button}
      \item \textbf{Stop Button}
      \item \textbf{Pause Button}
      \item \textbf{Play Button}
      \item \textbf{Live Record}
      \item \textbf{Keep Queue Button}
      \item \textbf{Mute Group Learn Button}
      \item \textbf{Developer Test Button}
      \item \textbf{Song Editor Button}
      \item \textbf{Song Mode Button}
      \item \textbf{PPQN Indicator}
      \item \textbf{BBT/HMS Toggle Button}
      \item \textbf{Tap BPM Button}
      \item \textbf{Beats Per Minute Control}
   \end{itemize}

   Many of these controls have keystrokes and MIDI-control slots that can be
   set up in the 'ctrl' file.

\paragraph{Panic Button}
\label{paragraph:introduction_panic_button}

   This button causes playback to stop, all patterns to mute, and flushes the
   MIDI buss.  It also raises a notification about an automation change.
   There is currently no keystroke, MIDI control, or MIDI-announcement (output)
   slot for this automation operation.

\paragraph{Stop Button}
\label{paragraph:introduction_stop_button}

   This button stops playback and rewinds to the beginning of the song.
   By default, the \texttt{Esc} key operates this function,
   and there is both a MIDI-control slot and a MIDI-announcement slot
   available for it.

\paragraph{Pause Button}
\label{paragraph:introduction_pause_button}

   This button stops playback, but does not rewind to the beginning of the song.
   It also resumes playback at the same point as the pause.
   By default, the \texttt{Period} key operates this function,
   and there is a MIDI-control slot and a MIDI-announcement slot available for
   it.

\paragraph{Play Button}
\label{paragraph:introduction_play_button}

   This button starts playback, either at the beginning or at the pause point.
   Also called the "start button".
   By default, the \texttt{Space} key operates this function,
   and there is both a MIDI-control slot and a MIDI-announcement slot
   available for it.

\paragraph{Live Record Button}
\label{paragraph:introduction_live_record_button}

   This button causes a live playing session to be recorded.
   That is, triggers are added to the song automatically, and they can then be
   seen in the \textsl{Song} editor.
   By default, the \texttt{P} key operates this function,

\paragraph{Keep Queue Button}
\label{paragraph:introduction_keep_queue_button}

   Puts the application into a "sticky" queue mode.
   In this mode, pressing a pattern key does not do a mute/unmute function, but
   instead turns on queuing for that pattern.
   By default, the \texttt{Backslash} key operates this function,
   and there is a MIDI-control slot available for it.

\paragraph{Mute Group Learn Button}
\label{paragraph:introduction_mute_group_learn_button}

   Also called the "L" button.
   Sets up for learn the current set of active patterns ("mute group").
   After pressing this button, the user can then press another button, which is
   automatically shifted, and the pattern set is saved, and can be recalled by
   that button later.  It can be saved in a 'mutes' file, as part of the MIDI
   tune, or in both places.

   Example:  We have 5 patterns armed in the current set. Press this button,
   and then press the "s" key.  The pattern set is saved and can be recalled
   later by the "S" ("s" shifted) key.

   By default, the \texttt{el} (lower-case "l") key operates this function,
   and there is a MIDI-control slot available for it, as well as a
   MIDI-announcement slot.

\paragraph{Developer Test Button}
\label{paragraph:introduction_developer_test_button}

   This button is always disabled.  Function is added temporarily when testing
   new features.

\paragraph{Song Editor Button}
\label{paragraph:introduction_song_editor_button}

   This button brings up an external window for editing the Song/Performance
   information.  If already up, it closes it.  Works the same as the
   \textbf{Edit / Song Editor} menu.

\paragraph{Live/Song Mode Button}
\label{paragraph:introduction_livesong_mode_button}

   This button toggles between the \textsl{Live} and \textsl{Song} performance
   mode.
   By default, the \texttt{F10} key operates this function,
   There is currently no automation control for this button.

\paragraph{PPQN Indicator}
\label{paragraph:introduction_ppqn_indicator}

   Displays the current PPQN for the current tune.

   STILL NEEDS WORK.

\paragraph{BBT/HMS Toggle Button}
\label{paragraph:introduction_time_format_toggle_button}

   Toggles the format of the current time displayed during playback. 
   It can be shown in B:B:T (bars:beats:ticks) or H:M:S (hours:minutes:seconds).

\paragraph{Tap BPM Button}
\label{paragraph:introduction_tap_bpm_button}

   Tap this button with a regular beat to determine the beats-per-minute of the
   tapping.  With each tap, the counter on the button increments and the BPM is
   recalculated.  Stop tapping for a few seconds to reset the counter.
   By default, the \texttt{F9} key operates this function, but it
   STILL NEEDS WORK.
   There is also a MIDI-control slot for this function.

\paragraph{Beats Per Minute Control}
\label{paragraph:introduction_bpm_control}

   This control can be text-edited or spun to change the beats/minute value
   used in playing back the current song.  This value is also saved to the
   file.

%     \item Log Tempo, which inserts the current tempo into the tempo track
%        as an event.
%     \item Tempo recording, which inserts all tempo changes as tempo events.

%  See \sectionref{subsec:patterns_panel_bottom}.

\rhead{\rightmark}         % shows section number and section name

% Menu

%-------------------------------------------------------------------------------
% menu
%-------------------------------------------------------------------------------
%
% \file        menu.tex
% \library     Documents
% \author      Chris Ahlstrom
% \date        2015-08-31
% \update      2024-10-31
% \version     $Revision$
% \license     $XPC_GPL_LICENSE$
%
%     Provides the Menu section of seq66-user-manual.tex.
%
%-------------------------------------------------------------------------------

\section{Menu}
\label{sec:menu}

   The \textsl{Seq66} menu structure is more complex than
   that of \textsl{Seq24}.  In particular, the \textsl{File} menu has two
   variants:  a normal file menu, and a file menu when \textsl{Seq66} is
   running under the \textsl{New/Non Session Manager}.
   (See \sectionref{subsec:sessions_nsm}.)

\subsection{Menu / File}
\label{subsec:menu_file}

   The \textbf{File} menu is used to save and load files in
   Standard MIDI Format 0 or 1, \textsl{Cakewalk} "WRK",
   and \textsl{Seq66} MIDI files.
   It also supports a list of recent files, and (new with version 0.98.3)
   sub-menus for import and export functions, which have expanded quite a bit.

   The \textsl{Seq66} \textbf{File} menu contains the sub-items shown below.
   The next few sub-sections discuss
   the sub-items in the \textbf{File} menu.
   Please note that these entries are different
   if \textsl{Seq66} is started under the control of the
   \textsl{New/Non Session Manager}.  
   See \sectionref{subsubsec:sessions_file_menu}.
   However, the import and export menus remain the same, although there are
   slight differences in how they work.

\begin{figure}[H]
   \centering 
   \includegraphics[scale=0.75]{main-menu/file/file-import-export-menus.png}
   \caption{Seq66 File Menu Plus Import/Export, Composite View}
   \label{fig:menu_file_items}
\end{figure}

   \begin{enumber}
      \item \textbf{New}
      \item \textbf{Open}
      \item \textbf{Open Playlist}
      \item \textbf{Recent MIDI files}
      \item \textbf{Save}
      \item \textbf{Save As}
      \item \textbf{Import}
      \begin{enumber}
         \item \textbf{Project Configuration...}
         \item \textbf{MIDI to Current Set...}
         \item \textbf{Playlist...}
         \index{restart!automatic}
         Once the playlist is imported,
         \textsl{Seq66} is automatically \textsl{\textbf{restarted}}
         in order to load the playlist.
         Be careful!
      \end{enumber}
      \item \textbf{Export}
      \begin{enumber}
         \item \textbf{Project Configuration...}
         \item \textbf{MIDI Only...}
         \item \textbf{Song...}
         \item \textbf{SMF 0...}
      \end{enumber}
      \item \textbf{Quit} (\textbf{Exit} in \textsl{Windows})
   \end{enumber}

   For information on the \textbf{File} menu when \textsl{Seq66} is
   running under the \textsl{Non Session Manager}, see
   \sectionref{subsubsec:sessions_file_menu}.

\subsubsection{Menu / File / New}
\label{subsec:menu_file_new}

   The \textbf{New} menu entry clears the current song.
   (A play-list or mute-groups setup, if loaded, are not affected.)
   If unsaved changes are pending, the user is prompted to save the changes.
   Prompting for changes is more comprehensive than \textsl{Seq24}.
   However, when in doubt, save!
   Keep backups of your tunes and configuration files!

\subsubsection{Menu / File / Open}
\label{subsubsec:menu_file_open}

   The \textbf{Open} menu entry opens a song (MIDI file or \textsl{Cakewalk}
   WRK file), replacing the current song (after a prompt if the song was
   modified).
   It opens up a standard file dialog:

\begin{figure}[H]
   \centering 
   \includegraphics[scale=0.65]{main-menu/file/light-menu-file-open.png}
   \caption{File / Open}
   \label{fig:menu_file_open}
\end{figure}

   This dialog lets one type a file-name, highlighting the first file
   that matches the characters typed.
   \textsl{Seq66} can open \textsl{Seq66}, MIDI SMF 0, and SMF 1 files, and
   \textsl{Cakewalk} WRK files.
   If the file is an SMF 0 file, where all channels appear on one track, the
   track is split so that each channel (0 to 15) is stored in the corresponding
   pattern, and pattern 16 contains the original track.

   Note that a MIDI file can be drag-and-dropped from a file manager onto
   the grid to open a file.

\subsubsection{Menu / File / Open Playlist}
\label{subsubsec:menu_file_open_playlist}

   The \textbf{Open Playlist...} menu entry opens a \textsl{Seq66}
   play-list file.

\begin{figure}[H]
   \centering 
   \includegraphics[scale=0.65]{main-menu/file/dark-menu-file-open-playlist.png}
   \caption{File / Open Playlist}
   \label{fig:menu_file_open_playlist}
\end{figure}

   The playlist file contains a list of "playlist sections",
   each listing a number of MIDI songs.
   These playlists and songs can be
   selected by the arrow keys or by MIDI control,
   and are displaed and editiable in the \textsl{Playlist} tab
   in the main window.
   See \sectionref{sec:playlist}.

\subsubsection{Menu / File / Recent MIDI files}
\label{subsubsec:menu_file_recent}

   This menu entry provides a list of the last few MIDI files created or opened;
   play-list selections are \textsl{not} included in this list.

\begin{figure}[H]
   \centering 
   \includegraphics[scale=0.65]{main-menu/file/menu-recent-files.png}
   \caption{Seq66 Menu File Recent Files}
   \label{fig:menu_file_recent_files}
\end{figure}

   Here is the long form when the 'rc' file's
   \texttt{[recent-files] full-paths} value is set to true:

\begin{figure}[H]
   \centering 
   \includegraphics[scale=0.65]{main-menu/file/menu-recent-files-long.png}
   \caption{Seq66 Menu File Recent Files, Full Paths}
   \label{fig:menu_file_recent_files_full_paths}
\end{figure}

   This list is saved in the \texttt{[recent-files]} section of the
   'rc' configuration file.
   In the 'rc' file, the full path to the file-name is stored.
   This path is in "UNIX" format, using the forward slash, or solidus,
   as the path separator, even in \textsl{Windows}.
   The \texttt{full-paths} option can be set to show the full path in the
   recent-files drop-down menu.
   Only unique entries are included in the recent-files list.
   The limit is 12 recent-file entries.
   This is a feature from \textsl{Kepler34} \cite{kepler34}.
   One can also set \textsl{Seq66} to load the most-recent file at startup.
   Here is an example from an 'rc' file:

\begin{verbatim}
   [recent-files]
   full-paths = false
   load-most-recent = true
   count = 3
   /home/user/git/seq66/data/b4uacuse-gm-patchless.midi
   /home/user/git/seq66/data/midi/colours.midi
   /home/user/git/Julian-data/TestBeeps.midi
\end{verbatim}

\subsubsection{Menu / File / Save and Save As}
\label{subsubsec:menu_file_open_save_as}

   The \textbf{Save} menu entry saves the song under its current file-name.
   If there is no current file-name, it opens up a standard file
   dialog to name and save the file.
   The \textbf{Save As} menu entry saves a song under a different name.
   It opens up the following standard file dialog, very similar to the 
   \textbf{File Open} dialog, with an additional \textbf{Name} text-edit field.

\begin{figure}[H]
   \centering 
   \includegraphics[scale=0.65]{main-menu/file/dark-menu-file-save-as.png}
   \caption{File / Save As}
   \label{fig:menu_file_save_as}
\end{figure}

   To save a new file or save the current file to a new name,
   enter the name in the name field, without an extension.
   \textsl{Seq66} will append a \texttt{.midi} extension to the filename.
   The file will be saved in a format that the Linux \textsl{file} command
   will tag as something like:

   \begin{verbatim}
      colours.midi: Standard MIDI data (format 1) using 16 tracks at 1/192
   \end{verbatim}

   It looks like a simple MIDI file, and yet, if one re-opens it in
   \textsl{Seq66}, one sees that the mute-groups, labeling, pattern
   information, and song layout have been preserved in this file.
   This information is saved in a way that MIDI-compliant software
   should be able to use or ignore without failure.
   After the last track in the file, a number of
   \index{SeqSpec}
   sequencer-specific (SeqSpec) items are saved, to preserve
   the extra information that \textsl{Seq66} adds to the song.
   There is no way to save a \textsl{Cakewalk} "WRK" file.
   \textsl{Seq66} can only read them, and then save them as
   \textsl{Seq66} files.

   \index{Meta events}
   Meta events are now partially handled by \textsl{Seq66}.
   Meta events \textbf{Set Tempo}
   and \textbf{Time Signature}
   are now fully supported.
   Other meta events,
   such as \textbf{Meta MIDI Channel}
   and \textbf{Meta MIDI Port}
   are now read as events, and are saved back when the file is saved.
   They cannot be edited in \textsl{Seq66}, but they are not lost.
   (Channel and port meta events are
   considered \textsl{obsolete} in the MIDI standard.)
   Lastly, various meta text events, such as \textbf{Lyric},
   can be edited and saved.

\subsubsection{Menu / File / Import / Project Configuration}
\label{subsubsec:menu_file_import_project_configuration}

   This command is useful to grab an existing project configuration
   (i.e. the set of \texttt{qseq66.*} files) and copy it
   to another directory.
   This command is most useful in importing a project into a new
   NSM session.  Previously, the "home" project would be imported automatically
   into a new NSM session, but this was deemed confusing by some users, and
   properly so!

   This command brings up a file dialog box. Navigate to the desired
   source directory and then select the desired 'rc' file.
   This saves and rereads the configuration.
   If importing into a normal (i.e. not NSM) session, \textsl{Seq66} will
   restart itself automatically to save and reread the configuration.
   Be aware!
   For more information, see
   \sectionref{subsubsec:midi_export_file_import_project}.

\subsubsection{Menu / File / Import / MIDI to Current Set}
\label{subsubsec:menu_file_import}

   The \textbf{Import} menu entry imports an SMF 0
   or SMF 1 MIDI file as one or more patterns, one pattern per track,
   into the specified screen-set.
   This functionality is explained in detail in
   \sectionref{subsubsec:midi_export_file_import}.

\subsubsection{Menu / File / Import / Playlist}
\label{subsubsec:menu_file_import_playlisMIDI to Current Sett}

   A user can create a playlist that accesses MIDI files anywhere in the file
   system.
   However, in a session manager, it is preferable to have the configuration
   self-contained.
   Even without a session manager, it can be useful to copy a playlist to a
   subdirectory in order to separate it and its MIDI files from other
   playlists.
   Once a project has been imported or saved, then a playlist can also be
   imported, along with all of the MIDI files it references.

   This command brings up a file dialog box. Navigate to the desired
   source directory and then select the desired 'playlist' file.
   This menu entry copies the playlist file and its associated
   MIDI files; see
   \sectionref{subsubsec:midi_export_file_import_playlist}.

\subsubsection{Menu / File / Export / Project Configuration}
\label{subsubsec:menu_file_export_project}

   This menu entry lets the user select a destination directory.
   Then the project files from the current "home" directory are copied
   to that destination directory. Useful for backup.
   See \sectionref{subsubsec:midi_export_configuration_export}.

\subsubsection{Menu / File / Export / Song}
\label{subsubsec:menu_file_export_song_as_midi}

   Thanks to the \textsl{Seq32} project, the ability to export songs to MIDI
   format has been added.  In this export, a complete song performance is
   recoded so that other MIDI sequencers can play the performance properly.
   This functionality is explained in detail in
   \sectionref{subsubsec:midi_export_song_export}.

\subsubsection{Menu / File / Export / MIDI Only}
\label{subsubsec:menu_file_export_midi_only}

   Sometimes it might be useful to export only the non-vendor-specific
   (non-SeqSpec) data from a \textsl{Seq66} song, in order to reduce the
   size of the file or to accomodate non-compliant sequencers.
   This functionality is explained in detail in
   \sectionref{subsubsec:midi_export_file_export_midi_only}.

\subsubsection{Menu / File / Export / SMF 0}
\label{subsubsec:menu_file_export_smf_0}

   This feature is new since version 0.97.  It allows all tracks in the song to
   be consolidated and exported in MIDI's SMF 0 format.  It follows the same
   rules as song export.
   See \sectionref{subsubsec:midi_export_file_export_smf_0}.

\subsection{Menu / Edit}
\label{subsec:menu_edit}

   The \textbf{Edit} menu has undergone some expansion in \textsl{Seq66}.

   \begin{enumber}
      \item \textbf{Preferences...}
      \item \textbf{Song Editor}
      \item \textbf{Apply Song Transpose}
      \item \textbf{Clear Mute Groups}
      \item \textbf{Reload Mute Groups}
      \item \textbf{Mute All Tracks}
      \item \textbf{Unute All Tracks}
      \item \textbf{Toggle All Tracks}
      \item \textbf{Copy Current Set}
      \item \textbf{Paste To Current Set}
   \end{enumber}

   \setcounter{ItemCounter}{0}      % Reset the ItemCounter for this list.

   \itempar{Preferences}{edit!preferences}
   This entry brings up a \textbf{Preferences} menu entry,
   to allow viewing and tweaking MIDI I/O ports, displays options, JACK
   options, and more.
   It can also be brought up by \texttt{Ctrl-P}.
   It is discussed in detail in a later section.

   \itempar{Song Editor}{edit!song editor}
   \index{performance editor}
   \index{song editor}
   This item toggles the presence of the main song/performance editor.
   Note that the song editor is also available in the
   \textbf{Song} tab in the main window.
   The song/performance editor allows specifying exact numbers of loop replays;
   this provides a canned rendition of the MIDI tune.

   \itempar{Apply Song Transpose}{edit!song transpose}
   \index{song transpose}
   Selecting this item applies the global song transposition value to
   all sequences / patterns marked as transposable.
   This actively changes the note / pitch value of all note and aftertouch
   events in the pattern.
   Normally, drum tracks are \textsl{not} transposable.
   For the setting of global song transpose, see
   \sectionref{sec:song_editor}.
   Note that transpose can be enabled in the
   in the sequence editor
   (see \sectionref{sec:pattern_editor}).

   \itempar{Clear Mute Groups}{edit!clear mute groups}
   \index{mute groups}
   A feature of \textsl{Seq66} is that the mute groups
   are saved in both the 'rc' file \textsl{and} in the "MIDI" file.
   This menu entry clears them. If this resulted in any mute-group sequences
   status being set to false, then the user is prompted to save the MIDI
   file, so that it will no longer have any
   mute-group information.  And then, if the
   application exits, the cleared mute-group information is also saved to
   the 'rc' file.

   \itempar{Reload Mute Groups}{edit!load mute groups}
   \index{rc!mute groups}
   This menu entry reloads the mute-groups from the 'rc' file.
   So, if one loads a MIDI file that has its own mute groups that one does not
   like, this command will restore one's favorite mute-grouping from the 'rc'
   file.

   \itempar{Mute All Tracks}{edit!mute all tracks}
   \index{mute all}
   This menu entry, useful mostly in \textbf{Live} mode,
   immediately mutes \textsl{all} patterns in the entire song.
   The hard-wired keyboard short-cut for this action is \texttt{Ctrl-M}.

   \itempar{Unmute All Tracks}{edit!unmute all tracks}
   \index{unmute all}
   This menu entry, useful mostly in \textbf{Live} mode,
   immediately unmutes \textsl{all} patterns in the entire song.
   The hard-wired keyboard short-cut for this action is \texttt{Ctrl-U}.

   \itempar{Toggle All Tracks}{edit!toggle all}
   \index{toggle all}
   This option toggles the mute/armed status of \textbf{all} tracks.
   It is useful mostly \textbf{Live} mode, which overrides \textbf{Song}
   mode even if the Song Editor is focussed.
   The hard-wired keyboard short-cut for this action is \texttt{Ctrl-T}.

   \itempar{Copy Current Set}{edit!copy set}
   \index{copy set}
   This item marks the current set (i.e. the play-set)
   for the copying of all its patterns to another set.
   After clicking this menu entry, one can move to another set to paste it,
   using the following menu entry.

   \itempar{Paste To Current Set}{edit!paste set}
   \index{paste set}
   Once a set has been copied into the internal set clipboard,
   then this menu item is enabled.
   Move to the desired set (whether empty or note), and then
   click this menu item.
   All of the patterns in the original set are pasted into the current set,
   \textsl{overwriting all patterns} already in the set.
   Also note that the set clipboard can be pasted after a
   \textbf{File / New} or \textbf{File / Open},
   to copy it to another file.

\subsubsection{Menu / Edit / Preferences}
\label{subsubsec:menu_edit_preferences}

   \textbf{Preferences} provides a number of settings in one
   tabbed dialog, shown in the figures that follow.
   It allows one to set MIDI clocking, MIDI Input, display tweaks, minor
   playback options, and some JACK parameters.

  Configuration items not (yet) implemented in \textbf{Preferences} are
      incoming MIDI events to control the sequencer;
      what keys are mapped to functions;
      how the mouse works, and a few other.
   The MIDI and Key controls, far more numerous than in \textsl{Seq24}, have
   been consolidated into a 'ctrl' file and are fairly easy to edit with a text
   editor.
   \textsl{Seq66} does not support the 'fruity' mouse mode at this time.
   If you want it, ask us!

\paragraph{Menu / Edit / Preferences / MIDI Clock}
\label{paragraph:menu_edit_preferences_midi_clock}

   \textbf{Note}:
   \textsl{The MIDI Clock tab is sometimes difficult to click on.}
   We are not sure why.  It seems to be theme-dependent.
   In some things the tab thumb changes color when it can work.
   Just keep clicking in various locations in the tab,
   or use the \texttt{Alt-C} key.
   Weird.

   The \textbf{MIDI Clock} tab provides a way to set MIDI clocking for
   the available MIDI output busses.
   It configures the output busses for MIDI clock and data.
   It shows the devices that can play music.
   The items that appear in this tab depend on:

   \begin{itemize}
      \item What MIDI devices are connected to the computer.
         MIDI controllers, USB MIDI cables, applications with virtual
         ports, and other connected devices will add MIDI
         output devices (ports) to the system.
         This list will generally match the output of \texttt{aplaymidi -l}
         or \texttt{aconnect -lio}.
      \item The setting of the "manual-ports" option, which tells
         \textsl{Seq66} to set up virtual MIDI ports.
         It is enabled by the
         \texttt{-{}-manual-ports} command-line option or the
         \texttt{[manual-ports]} section of the
         \texttt{qseq66.rc} configuration file,
         or in the \textbf{MIDI Input} tab described below.
      \item The setting of the \textsl{Seq66}-specific
         "reveal ALSA ports" option,
         \texttt{-{}-reveal-ports} command-line option or the
         \texttt{[reveal-ports]} section of the
         \texttt{qseq66.rc} configuration file.
   \end{itemize}

   If \texttt{-{}-manual-ports} is on, this list shows the virtual
   MIDI output busses that \textsl{Seq66} can drive.
   One needs to use a JACK or ALSA MIDI
   connection application to connect a device on each of those outputs.
   The fact that the the buss names can
   start with different numbers, depending on the system setup, can complicate
   the playing of MIDI in this manner.  Also, the 'usr' configuration file can
   change the visible names of the ports to match specific equipment attached
   to the ports.

\begin{figure}[H]
   \centering 
   \includegraphics[scale=0.50]{main-menu/edit/preferences/midi_clock_tab.png}
   \caption{MIDI Clock (Output) Tab}
   \label{fig:midi_clock_tab}
\end{figure}

   This diagram is slightly out-of-date, but we go forward!
   It shows the tab for configuring MIDI output and clocking features.

   \begin{quotation}
      \textbf{Tip}:
      With some Qt themes, it is difficult to activate this tab by clicking
      because there is little or no sensitive area on the tab.
      In this case, click \texttt{Alt-c} once or twice.
   \end{quotation}

   Port-mapping is the default, and when active, is shown in this pane.
   If there are more than about a dozen
   output ports in the system, then a vertical scrollbar appears.
   The following elements are present in this tab:

   \begin{enumber}
      \item \textbf{Ports and Clocking Table}
      \item \textbf{Clock Start Modulo}
      \item \textbf{Buss Override}
      \item \textbf{MIDI I/O Maps}
      \item \textbf{Create (maps)}
      \item \textbf{Clear (maps)}
      \item \textbf{MIDI Control Output Bus}
      \item \textbf{Meta Events}
      \item \textbf{Client Name:ID/UUID}
      \item \textbf{Restart Seq66!}
   \end{enumber}

   \setcounter{ItemCounter}{0}      % Reset the ItemCounter for this list.

   \itempar{Ports and Clocking}{output!ports and clocks}
   \index{output ports}

   This table shows the available MIDI outputs and their status.
   If the set of system MIDI devices and software devices has changed since
   the last run, this list could be in error.  Restart the application
   and see if it is now correct.  Currently, there is no way to edit the list
   except in the 'rc' file.

   The \textbf{Ports and Clocks} table contains the following elements,
   although some can be removed by specifying the
   \texttt{port-naming = short} option in the 'rc' file.

   \begin{enumber}
      \item \textbf{Index Number}
      \item \textbf{Client Number}
      \item \textbf{Port Number}
      \item \textbf{Buss Name}
      \item \textbf{Port Disabled}
      \item \textbf{Off}
      \item \textbf{On (Pos)}
      \item \textbf{On (Mod)}
      \item \textbf{Clock Start Modulo}
   \end{enumber}

   The format of the left side of the entry listing is like the following
   when the port-naming option is "long", and the
   MIDI subsystem is ALSA):

   \begin{verbatim}
      [5] 128:4 yoshimi:input
       ^  ^   ^ ^        ^
       |  |   | |        |
       |  |   | |         ---- Port/buss name
       |  |   |  ------------- Client name
       |  |    --------------- Port/buss number
       |   ------------------- Client number
        ---------------------- Index number
   \end{verbatim}

   \setcounter{ItemCounter}{0}      % Reset the ItemCounter for this list.

   \itempar{Index Number}{midi clock!index number}
   \index{index number}
   The number in square brackets is an ordinal indicating the position
   of the output buss in the list.
   For all practical purposes in \textsl{Seq66}, it \textsl{is} the
   buss/port number.  This number can be stored in a pattern in order to have
   the pattern's output go to that buss.  
   This is true even if port-mapping is in place.
   \index{port!mapping}
   \index{buss!mapping}
   \index{port!override}
   \index{buss!override}
   It can be used with the \texttt{-b},
   \texttt{-{}-buss}, or \texttt{ -{}-bus} options to redirect all
   pattern output to that buss, useful if only one buss is active or the
   \textsl{Seq66} patterns route to non-existent busses.
   (See \sectionref{subsubsec:introduction_sets_buss_override},
   and \sectionref{subsubsec:usr_file_user_midi_settings}.)

   \itempar{Client Number}{midi clock!client number}
   \index{client number}
   The number that precedes the colon is the "client number".
   It is useful mainly in ALSA, where clients can have numbers like "14",
   "128", "129", etc.  For native JACK mode, it matches the index number or is
   the name of the client (e.g. "seq66").

   \itempar{Port Number}{midi clock!port number}
   \index{port number}
   The number that follows the colon is the "port number".
   It is useful mainly in ALSA.
   For native JACK mode, it matches the index number.

   \itempar{Buss Name}{midi clock!buss name}
   \index{port name}
   \index{midi clock!port name}
   These labels indicate the output busses (ports) available.
   \textsl{Seq66} does not access devices by name, but by port number.
   However, a port-map can be created to make it possible to find the correct
   buss / port number by name lookup.

   \itempar{Port Disabled}{midi clock!port disabled}
   The \textbf{Port Disabled} clock choice marks an output port
   that the user does not want to use or that the operating system
   (\textsl{Windows} \smiley)
   is locking or disabling.
   Normally, this inaccessible port would cause \textsl{Seq66} to exit.
   With the port disabled, the inaccessible port is ignored.
   This feature also shows when a port-map cannot find a device in the system's
   device list.
   When the \textsl{Windows} version of \textsl{Seq66}
   (\texttt{qpseq66.exe}) is first started, it may error out.
   It will then write a default \texttt{qseq66.rc}
   or \texttt{qpseq66.rc} configuration file,
   which can be examined to find the offending buss, which can then be
   marked in the normal 'rc' file as disabled.

   \itempar{Off}{midi clock!off}
   Disables the MIDI \textsl{clock} for the given output buss.
   MIDI output is still sent to those ports, and
   each port that has a device connected to it will play music.
   Some synthesizers may require this setting.

   \itempar{On (Pos)}{midi clock!on (pos)}
   MIDI clock will be sent to this buss.
   MIDI Song Position and MIDI Continue will be sent if playback starts
   at greater than tick 0 in Song mode.  Otherwise, MIDI Start will be sent.
   Note: In case of trouble, see
   \sectionref{subsec:alsa_testing}.

   \itempar{On (Mod)}{midi clock!on (mod)}
   MIDI clock will be sent to this buss.
   MIDI Start will be sent, and clocking will begin
   once the Song Position has reached the start modulo of the specified size
   (see the next item's description).
   This setting is used for gear that does not respond to Song Position.

   Below the \textbf{Ports and Clocks Table} are more configuration elements.

   \setcounter{ItemCounter}{0}      % Reset the ItemCounter for this list.

   \itempar{Clock Start Modulo (ticks)}{midi clock!clock start modulo}
   This value starts at 1 and ranges up to 16384, and defaults to 64 ticks.
   It is used by the \textbf{On (Mod)} setting discussed above.
   It is the \texttt{[midi-clock-mod-ticks]} option in the \textsl{Seq66}
   'rc' file.

   \itempar{MIDI I/O Port Maps}{midi clock!port maps}
   If checked (the default), then port-mapping is employed.
   This makes it a bit easer to manage MIDI devices across systems and to store
   the numbers in each pattern.
   Note that both input and output port mappings are activated by this
   checkbox.
   If changed, the \textbf{Restart Seq66!} button is enabled.

   \itempar{Create (maps)}{midi i/o!port mapping}
   \index{port mapping}
   \index{port!mapping}
   Pressing this button saves the current set of MIDI I/O ports to sections in
   the 'rc' file.  These sections can be enabled in order to support
   port-mapping in subsequent runs of \textsl{Seq66}.
   Generally, after pressing this option, one will want to stop
   \textsl{Seq66}, rearrange the clock and input maps in the
   'rc' file with a text editor, back up this file in a safe place,
   and restart \textsl{Seq66}.

   \itempar{Clear (maps)}{midi i/o!remove mapping}
   \index{remove mapping}
   \index{port! remove mapping}
   Pressing this button removes the port mapping.
   \index{restart!manual}
   Once done, either restart \textsl{Seq66} or go to the \textbf{Session}
   tab and click the \textbf{Restart} button.
   (See \sectionref{subsec:concepts_reload_session}.)

   \itempar{MIDI Control Output Bus}{midi control!output buss}
   \index{midi control!output}
   Use this control to select the output bus used to display
   application-automation status, loop status, and mute-group status.
   Requires a \textsl{reload session} to take effect.
   The number of the buss is stored in the 'ctrl' file named in
   \sectionref{paragraph:menu_edit_preferences_session},
   as the value of \texttt{output-buss}.
   If port mapping is enabled (now the default),
   the nick-name of the bus is stored instead of the number.

   \itempar{Meta Events}{midi clock!meta events}
   \index{tempo-track-number}
   This section consists of the following items:

   \begin{enumerate}
      \item \textbf{Tempo track number}
      \item \textbf{BPM Precision}
      \item \textbf{Set Tempo Track}
   \end{enumerate}

   \textbf{Tempo track number}
   allows the user to move the tempo track from pattern 0 to
   another pattern.  Changing this option is not recommended, since track 1 (0)
   is the official track for tempo events, but \textsl{Seq66} allows the
   user to record tempo events to another track.  \textsl{Seq66} will
   process tempo events in any pattern.

   \index{usr!bpm-precision}
   \textbf{Precision}
   allows setting the number of digits past the decimal point to 0, 1, or 2.
   This is also a 'usr' setting.
   See \sectionref{subsubsec:usr_file_user_midi_settings}.
   The BPM (tempo) is stored in the MIDI file multiplied by 1000 to accommodate
   the decimal places.

   \textsl{Set Tempo Track}
   Enabled when a valid tempo track number is given.
   It makes the tempo track official if it is not zero anymore.

   \itempar{Client Name:ID/UUID}{client ID}
   This read-only text field shows two things:

   \begin{enumerate}
      \item \textbf{Client Name}.
         This is the name of the client under ALSA or JACK.  It defaults to
         \texttt{seq66}, but it can be altered by the command-line option
         \texttt{-{}-client-name} or by a session manager.
         Each instance of Seq66 run under ALSA will have a different client ID.
      \item \textbf{ID/UUID}.
         Under ALSA, the client number (client ID) is shown.
         Under JACK, the UUID that JACK assigned to \textsl{Seq66} is shown.
   \end{enumerate}

   \itempar{Restart Seq66!}{restart}
   Certain changes require a \textsl{Seq66} restart, unfortunately.
   When enabled, clicking this button does not exit \textsl{Seq66},
   but it does cause all of the internal mechanisms to be recreated
   from scratch.

%  \index{todo!manual alsa gui option}
%  There is currently no user-interface item corresponding to the "manual-ports"
%  command-line and 'rc' configuration file option.
%  We should rename this option to "virtual" eventually.

\paragraph{Menu / Edit / Preferences / MIDI Input}
\label{paragraph:menu_edit_preferences_midi_input}

   To set up \textsl{Seq66} to record MIDI from devices such as
   controllers and keyboards, the output of the ALSA MIDI recording
   command-line \texttt{arecordmidi -l} is relevant.
   Something like that listing appears in the Input tab:

\begin{figure}[H]
   \centering 
   \includegraphics[scale=0.50]{main-menu/edit/preferences/midi_input_tab-2.png}
   \caption{MIDI Input Tab}
   \label{fig:midi_input_tab}
\end{figure}

   Port-mapping is the default, and when active, is shown in this pane.
   If there are more than about a dozen
   input ports in the system, then a vertical scrollbar appears.

   Any item checked allows \textsl{Seq66} to record MIDI from that source,
   which must be connected to this input port.

   \textbf{Warning:}
   \index{warnings!usr config}
   \index{usr config}
   If the 
   \texttt{[user-midi-bus-definitions]} value in the 'usr' configuration file
   is non-zero, and the
   corresponding number of
   \texttt{[user-midi-bus-N]} settings are provided, then
   the list of existing hardware will be ignored, and those values will be
   shown instead.
   This feature can be overridden with the
   \texttt{-{}-reveal-ports} (\texttt{-r}) option.
   If you define these sections, they should match your
   hardware exactly, and your hardware should not change from session to
   session (or port-mapping should be enabled).
   If the "auto ALSA ports" option is turned on, via the \texttt{-a} or
   \texttt{-{}-auto-ports} option, then
   the input ports from the system are shown.

   \setcounter{ItemCounter}{0}      % Reset the ItemCounter for this list.

   \itempar{Input Buses}{input buses}
   \textbf{Input Buses} delineates the MIDI input devices as noted above.

   \itempar{MIDI Control Input Bus}{midi control!input buss}
   \index{midi control!input}
   Use this control to select the input bus used for MIDI control automation of 
   application actions, loop actions, and mute-group actions.
   Requires a \textsl{reload session} to take effect.
   The number of the buss is stored in the 'ctrl' file named in
   \sectionref{paragraph:menu_edit_preferences_session},
   as the value of \texttt{control-buss}.
   If port mapping is enabled (now the default),
   the nick-name of the bus is stored instead of the number.

   \itempar{Input Options}{input options}
   \index{input options}
   \textbf{Input Options} adds further refinements to MIDI input.
   Currenty it has only one setting, for recording input into patterns by the
   channel in each event.

   \itempar{Record-by-Bus}{record!by bus}
   \index{input by buss}
   \textbf{Record into patterns by buss}
   causes MIDI input from multiple busses to be distributed to
   each sequence according to MIDI input buss number.

   \itempar{Record-by-Channel}{record!by channel}
   \index{input by channel}
   \textbf{Record into patterns by channel}
   causes MIDI input with multiple channels to be distributed to
   each sequence according to MIDI output channel number.

   Only one of these record-by options can be enabled at the same time.
   The record-by-buss option takes precedence.

   When these options are disabled,
   the normal recording behavior dumps all data into the current
   sequence, regardless of channel or buss.
   See \sectionref{sec:recording}, which describes recording in more detail.

   \itempar{Input/Output Virtual Ports}{ports!virtual}
   \textbf{Use virtual (manual) I/O ports}
   This option
   allows for configuration of the manual-ports option from within the
   user-interace. 
   Once the option is enable
   A \textsl{reload session} (see \sectionref{subsec:concepts_reload_session})
   is necessary for this option to take effect.

   \itempar{Virtual Ports Auto-Enable}{ports!virtual auto-enable}
   \textbf{Auto-enable virtual I/O ports}
   If set, the ports are all automatically enabled upon a restart.
   The following figure shows that a large number of virtual ports can be
   defined.

\begin{figure}[H]
   \centering 
   \includegraphics[scale=0.95]{main-menu/edit/preferences/midi_input_tab-virtual.png}
   \caption{MIDI Virtual Inputs}
   \label{fig:midi_input_tab_virtual}
\end{figure}

   Note that the user is responsible for connecting the virtual MIDI ports,
   using something like \textsl{aconnect} (ALSA) or
   \textsl{qjackctl} (JACK).

\paragraph{Menu / Edit / Preferences / Keyboard (removed)}
\label{paragraph:menu_edit_preferences_keyboard}

   Unlike \textsl{Seq24}, \textsl{Seq66}
   \textsl{does not} provide an options tab for
   setting up the keyboard.
   There are just too many new keystroke-automation functions to fit
   in a configuration dialog box.
   The default keyboard mappings follow \textsl{Seq24} fairly well,
   but add a large number of additional controls;
   around 96 keystroke slots would need to be provided!
   The keystroke and MIDI controls are consolidated, and are easy to change by
   editing the appropriate 'ctrl' configuration file, stored in one of the
   following directories, depending on
   the operating system:
   
   \begin{verbatim}
         /home/username/.config/seq66/qseq66.ctrl           (Linux)
         C:/Users/username/AppData/Local/seq66/qpseq66.ctrl (Windows)
   \end{verbatim}

   There are also some extended examples present in the \textsl{Seq66}
   \texttt{data/linux} and
   \texttt{data/samples} directory.
   Also see \sectionref{sec:launchpad_mini}.
   For more information on keystrokes, see
   \sectionref{subsec:kbd_mouse_keyboard_control}.

   One useful enhancement, though "costly", would be support of MIDI Learn.
   Currently the only "learnable" items are the mute groups.

\paragraph{Menu / Edit / Preferences / Mouse (removed)}
\label{paragraph:menu_edit_preferences_mouse}

   Unlike \textsl{Seq24}, \textsl{Seq66}
   \textsl{does not} provide an options tab for
   the mouse-interaction method.
   It is not supported in \textsl{Seq66}...
   the \textbf{Fruity} interaction method is not available;
   only the \textbf{Seq24} interaction is available.
 
\paragraph{Menu / Edit / Preferences / Display}
\label{paragraph:menu_edit_preferences_display}

   This dialog provides a few odds and ends to enhance the user-interface.
   Some of these items (plus a few more) can be configured by editing the 'usr'
   file.

\begin{figure}[H]
   \centering 
   \includegraphics[scale=0.50]{main-menu/edit/preferences/midi_display_tab.png}
   \caption{Display Options}
   \label{fig:midi_display_tab}
\end{figure}

   \setcounter{ItemCounter}{0}      % Reset the ItemCounter for this list.

   \itempar{Editor Key Height}{key height}
   This option affects the pattern editor's piano roll.  Smaller means a wider
   range of notes can be shown.  There are also
   \textbf{-},
   \textbf{0}, and
   \textbf{+} buttons in the pattern editor that provide
   vertical zoom.

   \itempar{Grid scaling \& spacing}{window scaling}
   These three items set scale factor for width and height of the main window,
   and adjust the spacing between the grid slots..
   The lowest scale factor is 0.5, and the largest scale factor is 3.0.
   For the smallest window, the smallest practical values are 0.85 x 0.60.
   The spacing unit is pixels.

   \itempar{}{set-size}
   Provides a way to change the set size.  The default is
   \textbf{4 x 8}
   (rows by columns), but we intend to support
   \textbf{4 x 4},
   \textbf{8 x 8}, and
   \textbf{12 x 8}
   as well.

   \textbf{Warning}:
   A different set size alters the 'ctrl' file layout radically.
   We still have to work through all the implications of changing the set size,
   so back up your configuration and proceed with caution!

   \itempar{Progress Boxes}{progress-box size}
   Provides a way to change the size of the progress box in each button.
   Values are width and height fractions (up to 1.0) re the button size.
   This is a 'usr' option.

   \itempar{Progress Box Shown}{progress-box shown}
   If the \textbf{Shown} check-box
   is \textsl{unchecked}, then the progress boxes and pattern color are not
   shown.
   This is a 'usr' option.

   \itempar{Fingerprint Size}{fingerprint size}
   This value, if set from 32 to 128, indicates the number of events above
   which a "fingerprint", rather than every note, will be drawn.  It can
   save some CPU time in drawing the grid.  If set to 0, the whole
   pattern is drawn, no matter how long the pattern is.

   \itempar{Verbose Console Output}{verbose}
   This boolean makes more output appear if \textsl{Seq66} is run from a
   console/terminal. It will also increase the amount of data logged to the log
   file, if activated. It is only a temporary setting, just like
   its command-line counterpart, \texttt{--verbose}; when
   \textsl{Seq66} exits, the setting remains false.

   \itempar{Suppress startup error messages}{quiet}
   Unlike the verbose setting, this one is sticky.
   It prevents the display of error prompts at startup.
   It is useful when the system keeps flagging the same problem,
   it cannot be fixed, and can be ignored.
   It is \textsl{not} the opposite of "verbose".

   \itempar{Load Most Recent File (startup)}{load most-recent}
   If checked, the file at the top of the \texttt{[recent-files]}
   list in the 'rc' file is loaded at startup.

   \itempar{Show Full Path of Recent Files in Menu}{full paths}
   The full path of each file in the \texttt{[recent-files]} list
   is shown in the menu.  Although they can be uncomfortably long, they can
   show files that have the same name, but in different directores.

   \itempar{Long Port/Buss Name}{buss names!long}
   \index{buss names!short}
   Controls how much port information is shown in the clocks and input
   listings.  For the "portmidi" (e.g. \textsl{Windows})
   implementation, keep this option checked.

   \itempar{Lock Main Window}{main window!lock}
   This item makes the window non-resizable after startup.

   \itempar{Swap Coordinates}{grid!swap coordinates}
   Normally, \textsl{Seq66} displays the pattern and mute-groups grids
   where the pattern numbers increase fastest downward.
   Some might prefer to have pattern numbers increase fastest rightward.
   This setting make the patterns show in the more conventional manner.

   \textbf{Warning}:

      \begin{itemize}
         \item This setting requires the 'ctrl' file to be rewritten
            if one want to preserve the normal layout for the pattern hot-keys
            and the mute-group hot-keys.
         \item This setting has not been rigorously tested, so be prepared for
            some issues.
      \end{itemize}

   A 'ctrl' file for the swapped setting is provided
   in \texttt{qseq66-swapped.ctrl} in the \texttt{data/linux}
   directory, but it might not be completely correct yet.

   \itempar{Bold Slot Font}{grid!bold}
   \index{font!bold}
   \index{progress bar!thick}
   This setting makes the font in the live grid bold, and it allows
   make the progress-bar thick in the grid and in the Live and Song piano
   rolls.
   It is the same as the \texttt{progress-bar-thick = true} option in the 'usr'
   file. See \sectionref{subsubsec:usr_file_user_interface_settings}.

   \itempar{Double click for pattern editor}{grid!bold}
   \index{double-click!pattern slot}
   If set, a double-click on a grid button brings up the pattern for editing.
   Disable it if the effect is confusing.

   \itempar{Global background/scale/key}{globals!background etc.}
   \index{global pattern setting!background}
   \index{global pattern setting!key}
   \index{global pattern setting!scale}
   If set, setting the background sequence, scale to show, or the key of the
   track will apply to all pattern windows that are opened.

   \itempar{Client:port buss names}{buss!naming}
   \index{bus!naming}
   If checked the MIDI engine's "client:port" numbers are shown in the port
   listings.
 
\paragraph{Menu / Edit / Preferences / JACK}
\label{paragraph:menu_edit_preferences_jack}

   This tab sets up JACK transport, if \textsl{Seq66}
   was built with JACK support (\textsl{Linux} only).

\begin{figure}[H]
   \centering 
   \includegraphics[scale=0.50]{main-menu/edit/preferences/midi_jack_tab.png}
   \caption{Edit / Preferences / JACK}
   \label{fig:midi_jack_tab}
\end{figure}

   The main sections in this dialog are:

   \begin{enumber}
      \item \textbf{JACK Transport/MIDI}
      \item \textbf{JACK Start Mode}
      \item \textbf{JACK Transport Connect and Disconnect}
      \item \textbf{JACK Server Settings}
   \end{enumber}

   \setcounter{ItemCounter}{0}      % Reset the ItemCounter for this list.

   \itempar{Transport/MIDI}{jack sync!transport/midi}
   These settings are stored in the 'rc' file settings group
   \texttt{[jack-transport]}.
   This items collects the following settings:

   \begin{itemize}
      \item \textbf{Jack Transport}.
         \index{JACK!transport}
         Enables slave synchronization with JACK Transport.
         The command-line option is \texttt{-{}-jack-transport}.
         The behavior of this mode of operation is perhaps not quite
         correct.  Even as a slave, \textsl{Seq66} can start and
         stop playback.
         Note that this option cannot be disabled via the mouse if the
         \textbf{Transport Master} option is enabled.  Disable that one first.
      \item \textbf{Transport Master}.
         \index{JACK!transport master}
         \textsl{Seq66} will attempt to serve as the JACK Master.
         The command-line option is \texttt{-{}-jack-master}.
         If this option is enabled the \textbf{JACK Transport} option is
         automatically enabled as well.
      \item \textbf{Master Conditional}.
         \index{JACK!master conditional}
         \textsl{Seq66} will fail to serve as the JACK Master if there is
         already a Master.
         The command-line option is \texttt{-{}-jack-master-cond}.
         If this option is enabled the \textbf{JACK Transport} option is
         automatically enabled as well.
      \item \textbf{Native JACK MIDI}.
         \index{JACK!native midi}
         This option is for the \texttt{qseq66} (Linux) version of
         \textsl{Seq66}.
         If set, MIDI input and output use native JACK MIDI,
         rather than ALSA.  However, if JACK is not running on the
         system, then \texttt{seq66} will fall back to ALSA mode.
         (However, if \texttt{jackdbus} is running, but the JACK engine is not,
         then a couple of non-working manual ports are created.  To be fixed in
         the future.)
         The command-line option is \texttt{-{}-jack-midi}
         or \texttt{-{}-jack}.
      \item \textbf{JACK Auto-Connect}.
         \index{JACK!auto-connect}
         This option has been true for a long time in \textsl{Seq66}, and
         non-configurable.  Now it can be turned off, in order to let the user
         or a session manager make the connections, even when not using
         manual/virtual ports.
   \end{itemize}

   \begin{quotation}
      \textbf{Tip}:
      Seq66 generally works better as JACK Master than JACK Slave.
   \end{quotation}

   If one makes a change in the JACK transport settings, it is best to
   then press the \textbf{JACK Transport Disconnect} button, then the
   \textbf{JACK Transport Connect} button.
   Another option is to restart
   \textsl{Seq66}... the settings are automatically saved when
   \textsl{Seq66} exits.

   \itempar{JACK Start mode}{jack sync!start mode}
   This item collects the following settings, also stored in the 'rc' file
   settings group \texttt{[jack-transport]}.

   \begin{itemize}
      \item \textbf{Live Mode}.
         \index{JACK!live mode}
         \index{live mode}
         \index{non-playback mode}
         Playback will be in live mode.  Use this option to allow muting and
         unmuting of patterns.  This option might also be called "non-song
         mode".
         The command-line option is \texttt{-{}-jack-start-mode 0}.
      \item \textbf{Song Mode}.
         \index{JACK!song mode}
         \index{song mode}
         \index{playback mode}
         \index{performance mode}
         Playback will use only the Song Editor's data.
         The command-line option is \texttt{-{}-jack-start-mode 1}.
   \end{itemize}

   \textsl{Seq66} also selects the playback modes
   according to which window started the playback.
   \textsl{The main window}, or pattern
   window, causes playback to be in live mode.  The user can arm and mute
   patterns in the main window by clicking on sequences, using their hot-keys,
   and by using the group-mode and learn-mode features.
   The song editor causes playback to be in performance mode, also known as
   "playback mode", or \textbf{Song} mode.

   \itempar{Connect}{jack sync!connect}
   Connect to JACK Sync.
   This button is useful to restart JACK sync when making changes to it,
   or when \textsl{Seq66} was started in ALSA mode.

   \itempar{Disconnect}{jack sync!disconnect}
   Disconnect from JACK Sync.
   This button is useful to stop JACK sync when making changes to it.
   JACK connection and disconnection are disabled during playback, but the
   buttons don't yet reflect that status.

   \itempar{JACK Server Settings}{jack server!settings}
   This read-only section shows the current settings of the JACK
   server, as much as possible.

\paragraph{Menu / Edit / Preferences / Play Options}
\label{paragraph:menu_edit_preferences_play_options}

   This tab contains some disparate options ostensibly related to playback.

\begin{figure}[H]
   \centering 
   \includegraphics[scale=0.50]{main-menu/edit/preferences/midi_play_options_tab.png}
   \caption{Play Options}
   \label{fig:midi_play_options_tab}
\end{figure}

   \setcounter{ItemCounter}{0}      % Reset the ItemCounter for this list.

   \itempar{Resume Note Ons...}{edit!resume notes}
   \textbf{Resume Note Ons at start/stop or sequence toggle}
   allows notes that had already started
   to be resumed when playback resumes.

   \itempar{Use File's PPQN...}{edit!use file ppqn}
   \textbf{Use File's PPQN for Pre-Existing Files}, if checked, allows
   \textsl{Seq66} to run using the PPQn of the MIDI file rather than
   the default \textsl{Seq66} internal PPQN.
   This is the recommended option for most MIDI files.
   When this option is changed, the \texttt{-{}-user-save} option is turned on
   to preserve the setting when \textsl{Seq66} exits.

   \itempar{Default Seq66 PPQN...}{edit!default ppqn}
   \textbf{Default Seq66 PPQN for New or Converted Files}, if checked, allows
   the standard PPQN, 192 pulses/quarter-note, to be changed to discrete values
   from 32 to 19200.  Intermediate values, even oddball values, can be entered
   by typing the number directly.
   When this option is changed, the \texttt{-{}-user-save} option is turned on
   to preserve the setting when \textsl{Seq66} exits.
   If there is a MIDI file loaded, it is modified to use the new PPQN, and the
   user is prompted to save it at exit.
   Best to have a backup, just in case.

   \itempar{Sets Mode}{edit!sets-mode}
   This item determines how sets are handled.
   Recall that a set is a number of patterns (up to 4x8) in the pattern grid,
   and that the current set is the one visible in the pattern grid.
   The way sets work in \textsl{Seq66} is that, when a set is selected,
   all the patterns in it are loaded into what is called
   the "play-set".
   When play starts only, patterns in the play-set are handled.
   The \textbf{Sets Mode} option allows special handling of the play-set.

   \begin{enumerate}
      \item \textbf{Normal}.
         In this mode, only the current set's patterns can be unmuted.
         When switching to another set, the current set's patterns become
         muted, and the new set's patterns are shown, unmuted.
      \item \textbf{Auto-Arm}.
         Here, when the new set is loaded, it is immediately unmuted.
      \item \textbf{Additive}.
         With this option, when a new set is loaded, the previous set keeps
         playing. This allows a build-up of patterns in playback.
      \item \textbf{All Sets}.
         Here, all sets in the tune are loaded and unmuted at once.
         Try this mode with the \texttt{b4uacuse-stress.midi} file
         in the \textsl{Sequencer64} project.  It's a good test of
         \textsl{Seq66} and your hardware/software synthesizer!
   \end{enumerate}

   One can clear the out play-set, and set only the current set active, by
   clicking the exclamation point button to the left of the "Active" label at
   the bottom of the main windows.

\paragraph{Menu / Edit / Preferences / Metronome Options}
\label{paragraph:menu_edit_preferences_metronom_options}

   This tab contains options for the "metronome" and
   "background recording" features:

\begin{figure}[H]
   \centering 
   \includegraphics[scale=0.50]{main-menu/edit/preferences/midi_metro_options_tab.png}
   \caption{Metronome Options}
   \label{fig:midi_metro_options_tab}
\end{figure}

   \setcounter{ItemCounter}{0}      % Reset the ItemCounter for this list.

   The metronome feature is enabled in the main live grid via a metronome
   button.
   The metronome is a standard \textsl{Seq66} pattern that is used
   for playback of the metronome, but it is never seen
   nor directly edited by the user.
   It is not saved with a song, so changing the metronome does not modify the
   song.
   The settings shown above are saved to a "metronome" section in the 'rc'
   file.
   The metronome is a pattern that first plays a main note once, and then
   plays "sub" notes for the rest of the measure.
   Here are the settings:

   \itempar{Beats/bar}{metronome!beats/bar}
   This setting sets the beats-per-measure for the metronome only.
   It currently does not affect the time-bar in the main window.
   Should it? There is a global beats/bar as well as beats/bar for
   each pattern.

   \itempar{Beat width}{metronome!beat width}
   This setting sets the beat width for the metronome only.
   The following settings are provided for the "main" note
   (the note that occurs on the beginning of the measure)
   and the "sub" notes (the notes that occur on each beat):

   \itempar{Patch}{metronome!patch}
   This item sets the program (patch) number for the note, which sets the
   instrument to play for the notes.
   We currently do not have a drop-down box to select the patch by name.
   The default patch is 0.
   As noted below, the default channel is 10, so this
   patch is the "Standard Drum Kit" for the device.
   Thus, by default the metronome can be implemented by two different
   drums.

   \itempar{Note}{metronome!note}
   This item provides the note value to be played.  Recall that 60 is the same
   as "middle C".  By default, the main note is 75, the "Clave" for the drum
   kit, and the sub note is 76, the "High Wood Block" for the drum kit.

   \itempar{Velocity}{metronome!velocity}
   This item provides the note velocity to be played, to provide an accent on
   the main note.

   \itempar{Length Fraction}{metronome!length fraction}
   The length of the notes are specified as a fraction of the beat width, and
   this value ranges from 0.125 to 1.0 to 2.0.
   If set to 0, the length is half of the beat width.

   \itempar{Reload Metronome}{metronome!reload}
   This button pauses playback (if playing),
   loads in the new metronome settings, and
   continues playing (if it was playing).
   It is \textsl{not} enabled when the status/configuration of background
   recording changes.

   \itempar{Metro Buss}{metronome!buss}
   This value selects the output MIDI device to use to play the metronome.
   It \textsl{must} be enabled in the \textbf{MIDI Clock} list.

   \itempar{Channel}{metronome!channel}
   This value selects the channel to use to play the metronome.

   \itempar{Record Buss}{recorder!buss}
   \index{background recorder}
   This value selects the input device to use to record events into the
   background pattern.
   Note that this device \textsl{must be enabled} in the \textbf{MIDI Input}
   buss list.

   \itempar{Thru Buss}{record!thru buss}
   This value selects the output MIDI device to use to play the incoming
   background record notes.  Otherwise they will not be heard.
   It \textsl{must} be enabled in the \textbf{MIDI Clock} list.

   \itempar{Thru Channel}{recorder!thru channel}
   This value selects the channel to use to play the recorded notes as
   they come in.

   We still have some more work to do to refine the metronome, the
   background recorder, and their configuration, pending user input.
   For more information about the metronome, see
   \sectionref{paragraph:patterns_metronome}.
   For information on count-in background recording, see
   \sectionref{paragraph:patterns_background_recording}.

\paragraph{Menu / Edit / Preferences / Pattern}
\label{paragraph:menu_edit_preferences_pattern}

   This tab provides options for the status of newly-created patterns
   and for randomization of amplitude and jitter of time-stamps.

\begin{figure}[H]
   \centering 
   \includegraphics[scale=0.50]{main-menu/edit/preferences/midi_pattern_tab.png}
   \caption{Pattern Options}
   \label{fig:midi_pattern_options_tab}
\end{figure}

   \itempar{Pattern}{edit!pattern}
   \textbf{Pattern}
   This relatively new tab provides a way to configure the status of
   a newly-created pattern.
   It also provides a way to change the range of amplitude randomization
   and time jittering of events.

   The first section is \textbf{Pattern Options}.
   It defines the statuses of a newly-created or newly-opened pattern.
   This can be convenient for live-recording.
   The status setting are:

   \begin{itemize}
      \item \textbf{Armed}.
         This setting causes the pattern to be armed when a new pattern
         is created.
      \item \textbf{Record}.
         The new pattern starts in record mode.
      \item \textbf{Tighten Record}.
         The new pattern starts in tightened (partly quantized) record mode.
      \item \textbf{Quantize Record}.
         The new pattern starts in quantized record mode.
      \item \textbf{Note-map Record}.
         The new pattern starts in note-mapping record mode.
         Notes are translated live via a 'drums' file, if set active
         in the 'rc' file.
      \item \textbf{Wrap-Aound}.
         The new pattern will allow prolonged notes to wrap around so
         that the Note Off event precedes the Note On event in the
         pattern loop.
      \item \textbf{Thru}.
         The new pattern starts with MIDI Thru enabled.
      \item \textbf{Record Style}
         This setting sets how record mode works for the pattern.
         \begin{itemize}
            \item \textbf{Merge}.
               As recording and looping proceeds, new events merge with
               the existing events.
            \item \textbf{Overwrite}.
               When the pattern loops back to its beginning, any
               existing events are deleted.
               A good way to try to get the right collection of notes.
            \item \textbf{Expand}.
               When recording as notes are recorded, the pattern expands to
               accomodate them.
               This results in a longer pattern than initially specified.
            \item \textbf{Oneshot}.
               Events are entered until the end is reached.
               Useful for recording stock patterns from a drum machine.
            \item \textbf{Oneshot Reset}.
               At the end of the specified length of the pattern,
               all events are cleared.
               Normal recording is set.
               Need to look into this as we cannot rememember all the
               details :-D.
         \end{itemize}
      \item \textbf{Apply to new only}. (Not shown yet).
         If checked, only a newly-created pattern will have the options
         above automatically applied.
   \end{itemize}

   The second section is \textbf{Randomization}.
   The range of randomization is based on a range parameter, and
   goes from -range to +range.
   The concept of jitter means that the time-stamps of recorded events
   are randomized slightly.
   The concept of randomization means that the amplitudes of events
   are randomized slightly.
   The randomization settings are:

   \begin{itemize}
      \item \textbf{Jitter}.
         This value is a jitter divisor.
         It sets the fraction of of the current snap value that
         is used as the range of jittering the time.
         For example, "8" means that the range is 1/8th of the snap value.
      \item \textbf{Amplitude}.
         This value is used for various data values.
         For Notes On (but not Notes Off), this parameter affects
         the range of amplitude variation, when amplitudes are
         the standard MIDI range, 0 to 127.
   \end{itemize}

   One minor issue, which we're still trying to work around, is that
   our various randomization algorithms seem biased to emit
   negative numbers.
   If one clicks in a pattern editor piano roll, types
   \texttt{Ctrl-A} to select all notes (and aftertouch)
   and types the \texttt{r} key repeatedly to randomize
   note amplitudes, the overall velocities slowly descend to 0.
   Still not sure what's wrong with \textsl{Seq66} randomization, 
   but it is only important if randomizing a large number of times.

   The second section, not shown, is \textbf{Additional 'usr' Options}.
   It contains only one option at present,
   \textbf{Escape Can Close Pattern Editor}.
   If set (the default is false), then the \texttt{Esc}
   key can not only stop playing and exit paint mode, but
   can also close the pattern window.

\paragraph{Menu / Edit / Preferences / Session}
\label{paragraph:menu_edit_preferences_session}

   This tab contains options related to session management and the
   configuration files.

   \setcounter{ItemCounter}{0}      % Reset the ItemCounter for this list.

\begin{figure}[H]
   \centering 
   \includegraphics[scale=0.50]{main-menu/edit/preferences/midi_session_tab.png}
   \caption{Session Options}
   \label{fig:midi_session_options_tab}
\end{figure}

   This dialog has a couple of new items, to allow changing the web browser and
   PDF view to use.  See below.

   \setcounter{ItemCounter}{0}      % Reset the ItemCounter for this list.

   \itempar{Session}{edit!session}
   \textbf{Session}
   This tab provides for three modes of session management:  None, the
   Non/New Session Manager (NSM), and JACK Session management.
   None is the normal mode of operation, where the user has full control of
   where to put files, what other applications are to be run alongside
   \textsl{Seq66}, and what connections are to be made.

   NSM provides a rigorously-controlled session management, and directs
   \textsl{Seq66} what menu items to display, whether to hide the
   user-interface or not, where configuration files and MIDI files go, and what
   applications are run in a session. It can also (via \texttt{jackpatch}) keep
   a record of connections to reconstruct.

   JACK Session provides a location for file and a record of applications and
   connections, but otherwise lets the user mess things up.  It is
   provided because some people still use it.
   For more information about session management, see
   \sectionref{sec:sessions}.

   \itempar{UUID}{edit!UUID}
   \textbf{UUID} is a read-only field that shows any UUID that's relevant to a
   session. Normally has a value only in a \textsl{JACK} or
   \textsl{NSM} session.
   Also see the \textbf{Session} tab in the main window
   (\sectionref{sec:sessions}).

   \itempar{Configuration Files}{edit!configuration}
   \textbf{Configuration Files}
   shows the status of the configuration files.
   (See \sectionref{subsec:configuration_rc}).
   The 'rc' file is always active, and normally is saved at exit (even if no
   configuration changes occurred).

   The 'usr' file should also be active, but one can disable it, which is
   currently an \textsl{experimental} and \textsl{untested} option.
   Normally, it is not saved at application exit (except after the first run on
   one's system).
   (See \sectionref{subsec:configuration_usr}).

   The rest of the configuration files are optional.
   See
   \sectionref{subsec:configuration_ctrl},
   \sectionref{subsubsec:configuration_mute_group_control},
   \sectionref{subsec:configuration_drums}, and
   \sectionref{sec:playlist}.

   \itempar{Palette File Base Name}{palette}
   This text edit holds the base name of a 'palette' file, which is always
   stored in the \textsl{Seq66} configuration directory.
   (See \sectionref{sec:palettes}.)

   \itempar{Store Palette}{palette}
   Normally, there is no palette file.  Pushing this button creates one, which
   can then be modified and configured as the palette-file to use in the 'rc'
   file.

   \itempar{Browser}{browser}
   This field is text-editable, and can also be changed by using the button
   next to it to select a browser executable to use in the
   \textbf{Help / Tutorial} menu entry. If all possible browsers are available
   via one's \texttt{PATH}, then the simple name of the application
   (including \texttt{.exe} if running \textsl{Windows}) can simply be typed
   in. Otherwise, type in the complete path or use the button to bring up
   a file dialog.

   If one erases the file name, the default browser for the system will be
   used the next time \textsl{Seq66} is restarted.

   \itempar{PDF Viewer}{PDF viewer}
   This field is similar to the browser field, but specifies an alternate
   viewing application for PDFs.

\subsection{Menu / Help}
\label{subsec:menu_help}

   The usual \textbf{Help} dialog is provided.
   As of version 0.98.8, it has been beefed up with a way to access a
   tutorial and the user manual.

   These new help items are a work in progress, so please apprise
   us of any issues; include information on the operating system and,
   if \textsl{Linux}, the desktop/window manager in use.

\subsubsection{Menu / Help / About...}
\label{subsubsec:menu_help_about}

   \index{Help!about}
   This menu entry shows the "About" dialog.
   That dialog provides access to some credits for the program as well.
   authors and the project documentors, and active link to them.
   It also shows Git version-control information as well.

\subsubsection{Menu / Help / Build Info...}
\label{subsubsec:menu_help_build_info}

   \index{Help!build info}
   This menu entry shows the "Build Info" dialog.  This list of
   build options enabled in the current application is the same list
   that it generated via this command line:

   \begin{verbatim}
      $ seq66 --version
   \end{verbatim}

\subsubsection{Menu / Help / Song Summary File...}
\label{subsubsec:menu_help_song_summary_file}

   \index{Help!song summary}
   This menu entry allows one to write a summary of the song data into a text
   file. It brings up a file dialog which defaults to the name of the
   currently-loaded MIDI file, with the extenstion \texttt{.text} and
   the directory from where the MIDI file was loaded.
   It shows the filename, the information about the sets and tracks,
   MIDI format (0 or 1), and the PPQN.

   It also shows each sequence: name, channel (128 mean there is no output
   channel), the time signature, buss number (and any mapping), the length in
   pulses, the event and trigger count, transposability, key and scale, and
   color number (if any).
   For each trigger in the pattern, its start, stop, offset, and transposition
   values are shown.
   This file can be helpful for trouble-shooting or solving puzzling effects in
   the tune.

\subsubsection{Menu / Help / App Keys}
\label{subsubsec:menu_help_app_keys}

   \index{Help!app keys}
   This entry brings up a dialog that shows brief descriptions of the
   non-automation keys available in various contexts.
   These keys are almost exclusively hardwired and currently cannot be
   changed via a configuration file.  By pressing a button, the desired
   keystrokes can be quickly viewed. Note that the descriptions come from small
   HTML files that are part of the installation.

\subsubsection{Menu / Help / Tutorial}
\label{subsubsec:menu_help_tutorial}

   \index{Help!tutorial}
   This entry brings up a short tutorial of \textsl{Seq66} in the default
   browser. This tutorial is meant only to jump-start a new user of
   \textsl{Seq66}, and is a work in progress.
   It does not cover nearly as much as the user manual, so check that out in
   the next section.

   Normally, the tutorial will open a web page.  If it does not, one might need
   to set up a default browser.  On Linux, make sure that there is a "desktop"
   file for the browser, as in
   \texttt{/usr/share/applications/firefox.desktop}.
   If so, then run the following command, and then test it:

   \begin{verbatim}
      $ xdg-settings set default-web-browser firefox.desktop
      $ xdg-open https://ahlstromcj.github.io/docs/seq66/tutorial/index.html 
   \end{verbatim}

   On Windows, this procedure is still \textsl{to be determined}.

   In both systems, one can override the default applications by opening
   the specified 'usr' file (usually \texttt{qseq66.usr} or
   \texttt{qpseq66.usr} and specifying the full path to the desired
   applications (Linux paths shown here):

   \begin{verbatim}
      [user-options]
      log = "/home/user/.config/seq66/seq66.log"
      pdf-viewer = "/usr/bin/zathura"
      browser = "/usr/bin/google-chrome"
   \end{verbatim}

   Also see \sectionref{subsubsec:usr_file_user_options}.

\subsubsection{Menu / Help / User Manual}
\label{subsubsec:menu_help_user_manual}

   \index{Help!user manual}
   This menu entry first tries to locate the user manual on the internet and
   open it in the default browser. If not found, or the network is down,
   then this entry brings up the full \textsl{Seq66} user manual in the default
   PDF viewer.  It currently looks in the possible installation areas and in
   the \textsl{Seq66} source tree to find the PDF.

   On Linux, one can follow the setup procedure in the previous section and
   test it via the following command, which will show the manual in the default
   browser.:

   \begin{verbatim}
      $ xdg-open https://ahlstromcj.github.io/docs/seq66/seq66-user-manual.pdf
   \end{verbatim}

%-------------------------------------------------------------------------------
% vim: ts=3 sw=3 et ft=tex
%-------------------------------------------------------------------------------


% Patterns Panel

%%% %-------------------------------------------------------------------------------
% patterns_panel
%-------------------------------------------------------------------------------
%
% \file        patterns_panel.tex
% \library     Documents
% \author      Chris Ahlstrom
% \date        2015-08-31
% \update      2018-10-27
% \version     $Revision$
% \license     $XPC_GPL_LICENSE$
%
%     Provides the concepts.
%
%-------------------------------------------------------------------------------

\section{Patterns Panel}
\label{sec:patterns_panel}

   \textsl{Seq66} works with patterns (also known as "loops", "tracks", or
   "sequences") that are repeated throughout a song.
   One composes and edits small patterns,
   and combines them to create a full song.  This is a powerful way
   to work, and makes one productive within an hour.

   The \textsl{Seq66 Patterns Panel} is in the center of the
   \index{main window}
   \textbf{main window} of \textsl{Seq66}.
   See \figureref{fig:main_screen}.
   It is here one creates a set of patterns,
%  (see \sectionref{subsubsec:concepts_terms_screen_set}),
   manages the configuration, controls the playback rate, adds tempo events,
   and opens the pattern, song, event, or playlist editors.

   \index{live mode}
   \index{mode!live}
   \index{mode!song}
   When the Patterns Panel has the application focus,
   and \textsl{Seq66} is \textsl{not} running in \textbf{Song} mode,
   it puts \textsl{Seq66} in \textbf{Live} mode.
   The musician can
   control the playback and muting/unmuting of each pattern in
   the song, while it is playing, from within this window.

   \index{song mode}
   \index{mode!song}
   If the song editor (see \sectionref{sec:song_editor})
   has the input focus, it controls the muting/unmuting of
   each pattern, and \textsl{Seq66} runs in \textbf{Song} mode.
   (There are ways to override this behavior.)

%  However, if \textsl{Seq66} is using JACK transport, then, instead of
%  the behavior described above, live versus song mode is controlled by the
%  JACK start mode option (see the \textbf{JACK Start mode} item in
%  \sectionref{paragraph:menu_file_options_jack_sync}).

   For exposition, we break the Patterns Panel
   into a menu bar, a top panel, a pattern panel, and a bottom panel.
   The \textsl{Seq66} menu bar is discussed in
   \sectionref{sec:menu}.
   The Qt and Gtkmm user-interfaces differ in the arrangement of buttons and
   panels; and the Qt interface uses tabs.

\subsection{Patterns / Set Handling}
\label{subsec:patterns_panel_set_handling}


   Let's go through an example using the \texttt{Home} key (or whatever key is
   configured as the \textbf{Set Playing Screenset} key.)

   \begin{enumber}
      \item Load a song with more than one screen-set.
      \item Unmute the pattern(s) in the first set and start playback.
      \item Use the "\texttt{]}" (\textbf{Screenset Up}) key to move to the next
         set.  Note that the first set is still playing.  Also note that the
         now-current set is \textsl{not} playing.
      \item Press the \texttt{Home} key.
         Note that the first set turns off, and the current set turns on.
         These steps can be repeated at will.
      \item Finally, hit the \texttt{F8} (\textbf{Toggle Mutes}) key.
         Note that all tracks on all sets toggle muting each time this key is
         pressed.
   \end{enumber}

\subsection{Patterns / Main Panel}
\label{subsec:patterns_panel_main}

   The main panel of the Patterns window provides a grid of empty boxes.
   Each filled box represents a loop, track, sequence, or pattern.
   One sees only 32 loops at a time in the main panel (but many more than
   32 loops can be supported by \textsl{Seq66}).

   \index{screen-set}
   This group of 32 loops is called a "screen-set".
   One can switch between sets by using the
   \index{keys![}
   \index{keys!screenset down}
   "\texttt{[}" and
   \index{keys!]}
   \index{keys!screenset up}
   "\texttt{]}" keys on the keyboard, or by using
   the spin-widget-driven, labelled \textbf{Set} interface item, or
   \index{keys!Home}
   \index{keys!screenset play}
   by hitting the (default) \texttt{Home} key to make it the playing screenset,
   or by hitting \texttt{Page-Up} or \texttt{Page-Down} with the pattern window
   in keyboard focus.
   There are a total of 32 sets, for a total of 1024 loops/patterns. 
   Only one screen-set can be controlled at a time, in general.
%  have found; have not yet tried to verify this assertion.
   But any number of screensets can be playing at the same time.

   Note that the \texttt{Page Up} and \texttt{Page Down} keystrokes, and their
   counterparts in
   \textbf{File / Options / Keyboard / Control Keys / Screenset Up}
   and \textbf{Screenset Down}, can be used in lieu of the
   \textbf{Set} spin-button.

   It is important to note that keystroke control of the screen-set will
   wrap-around in screen-set values (i.e. screen-set down at 0 results in
   screen-set 31, and screen-set up at 31 results in screen-set 0).
   However, the spinbuttons will stop up at 31 and stop down at 0.
   We consider this a feature rather than a bug, at this time.

\begin{figure}[H]
   \centering 
%  \includegraphics[scale=0.75]{pattern-window-main-panel-items.png}
   \includegraphics[scale=0.65]{roll.png}
   \caption{Patterns Panel, Main Panel Items}
   \label{fig:pattern_window_main_panel_items}
\end{figure}

   The individual items annoted in this figure are described in
   \sectionref{subsubsec:patterns_pattern_filled}, in more detail.
   The slot at the bottom left of this figure shows some new features:

   \begin{itemize}
      \item The sequence number appears at the bottom left of the slot.
      \item The buss number (re 0) and the channel number (re 1) appears
         to the right of the sequence number, in the format "0-1".
      \item To the right of that, the time signature ("4/4") appears, at the
         bottom.
      \item The hot-key for muting/unmuting the pattern appears next,
         at the bottom right of the slot.
      \item The title of the sequence appears at the top left of the pattern
         slot.
      \item The length of the sequence, in number of measures (bars), appears
         at the upper right of the slot.
      \item Notice the default alternate font, which has a little more body
         than the \textsl{Seq24} font.
   \end{itemize}

   Observe that feature in the first figure of the next section.
   The two main items are the empty \textsl{pattern slot}, and the slot filled
   with a MIDI \textsl{pattern}:

   \begin{enumber}
      \item \textbf{Pattern Slot}
      \item \textbf{Pattern}
   \end{enumber}

\subsubsection{Pattern Slot}
\label{subsubsec:patterns_pattern_slot}

   \index{pattern!slot}
   An empty box is a slot for a pattern.
   If a pattern is present in the slot, the top line will show
   the title of the pattern, and the number of measures in the pattern.
   The latter is not shown on some of the figures in this manual, a
   lack we will ave to rectify someday.
   Also, these colors are not yet supported in the Qt user-interface.

   A pattern can show a number of different statuses based on the coloring
   of elements in the pattern slot. 

\begin{figure}[H]
   \centering 
%  \includegraphics[scale=0.75]{new/slots.png}
   \includegraphics[scale=0.65]{roll.png}
   \caption{Various Status of Pattern Slots}
   \label{fig:pattern_slots_statuses}
\end{figure}

   Not shown in the figure are the gray pattern
   colors resulting from queuing and one-shot queuing, nor is the
   number-of-measures number shown.
   The colors have meaning (in the Gtkmm user-interface only):

   \begin{itemize}
      \item \textbf{Empty background}.  Whether the classic gray pattern
         of \textsl{Seq24}, or the many patterns of \textsl{Seq66},
         including all black with yellow sequence numbers, this
         slot coloring indicates that the slot is unused.
      \item \textbf{White background}.  Unarmed (muted) patterns show black
         text on a white background.
      \item \textbf{Black background}.  Armed (unmuted) pattern.  If the text
         is yellow, it is a pattern with no MIDI events, but is armed.  Note
         that armed/unmuted patterns can be exported if they have a layout in
         the Song Editor.
      \item \textbf{Yellow background}.  A pattern with no MIDI events, just
         textual MIDI information.  If armed (uselessly), it is yellow text on
         a black background (not shown).
      \item \textbf{Cyan background, black text}.
         An unarmed pattern currently being edited in a 
         \textbf{Pattern Editor} or event
         editor. Or, if an SMF 0 MIDI file was just opened or imported, this
         color combination indicates the SMF 0 format track with all of the
         data in the song, which only occurs in slot 16 (unless the user then
         dragged it to another slot).
      \item \textbf{Black background, cyan text}.
         An armed pattern currently being edited in a 
         \textbf{Pattern Editor} or event
         editor.  Or an armed SMF 0 format MIDI sequence.
      \item \textbf{Red events}.
         Indicates a pattern for which the new transpose feature is
         disabled.  The white, black, and cyan background have the same
         meanings as in the other items for statuses of unarmed, armed, and
         currently being edited.
   \end{itemize}

   As of version 0.95, the user can also apply coloring to each sequence.
   This feature was adopted from \textsl{Kepler34} (\cite{kepler34}).
   Here is the new pattern menu for sequence color:
   \index{pattern!color menu}

\begin{figure}[H]
   \centering 
%  \includegraphics[scale=0.75]{new/seq66-sequence-color-menu.png}
   \includegraphics[scale=0.65]{roll.png}
   \caption{Sequence/Pattern Color Menu}
   \label{fig:pattern_window_sequence_color_menu}
\end{figure}

   Here is a sample of the coloration as it appears in the patterns panel and
   in the song editor:
   \index{pattern!coloring}

\begin{figure}[H]
   \centering 
%  \includegraphics[scale=0.75]{new/seq66-sequence-coloration.png}
   \includegraphics[scale=0.65]{roll.png}
   \caption{Sequence/Pattern Coloration}
   \label{fig:pattern_window_sequence_coloration}
\end{figure}

   \index{pattern!right click}
   \index{slot!empty slot right-click}
   Right-click on an empty box one brings up a menu to create
   a new loop, as well as some other operations:

\begin{figure}[H]
   \centering 
%  \includegraphics[scale=0.75]{pattern/pattern-empty-right-click-menu.png}
%  \includegraphics[scale=0.65]{new/pattern-empty-right-click-menu.png}
   \includegraphics[scale=0.65]{roll.png}
   \caption{Empty Pattern, Right-Click Menu}
   \label{fig:pattern_window_empty_right_click}
\end{figure}

   \begin{enumber}
      \item \textbf{New}
      \item \textbf{Paste}
      \item \textbf{Song}
      \begin{itemize}
         \item {Mute All Tracks}
         \item {Unmute All Tracks}
         \item {Toggle All Tracks}
         \item {Toggle Live Tracks}
      \end{itemize}
   \end{enumber}

   This menu entry is quite different for the Qt user-interface.
   See \sectionref{subsubsec:qt_portmidi_qt5_live_slot_menu}.

   \setcounter{ItemCounter}{0}      % Reset the ItemCounter for this list.

   \itempar{New}{pattern!new}
   Creates a new loop or pattern.
   Clicking this menu entry fills in the empty box with an untitled
   pattern, and brings up the Pattern Editor
   so that one can fill in the new pattern.

   In addition to right-click and select \textbf{New}, the user can
   \index{empty slot double-click}
   double-click on the empty slot, to bring up a new instance of the sequence
   editor.  For the double-click, the effect can be a bit confusing at first,
   because it also toggles the arming/mute status of the slot
   quickly twice (leaving it as it was).  It takes some getting
   used to, but we miss it when using \textsl{Seq24}.

   \index{editing shortcut}
   \index{keys!=}
   \index{keys!pattern edit}
   \index{keys!-}
   \index{keys!event edit}
   A nice feature is hitting the equals ("=") key, then hitting
   a pattern shortcut key (hot-key), to bring up a new sequence or edit an
   existing one in a 
   \textbf{Pattern Editor} .  Another feature is hitting the minus
   ("-") key, then the hot-key, to bring up the \textbf{Event Editor}.
   The configuration file settings for the the '=' and
   '-' keys can be altering in the \textbf{File / Options / Keyboard} tab.

   \index{current slot highlight}
   When an unarmed (muted) pattern is first brough up for sequence editing (or
   event editing), the slot in the main window is now highlighted (Gtkmm only),
   using black text on a cyan background, as being the "currently-edited" slot.
   (This is the same background used to indicate the original track in an
   SMF 0 to SMF 1 conversion.)

\begin{figure}[H]
   \centering 
%  \includegraphics[scale=0.75]{pattern-window-current-seq-unarmed.png}
   \includegraphics[scale=0.65]{roll.png}
   \caption{Currently-Edited Pattern, Unarmed}
   \label{fig:pattern_window_current_seq_unarmed}
\end{figure}

   If the currently-edited sequence is armed (unmuted), then the highlighting
   is reversed (cyan text on a black background), and resembles the
   highlighting for an armed sequence (which is white text on a black
   background).

\begin{figure}[H]
   \centering 
%  \includegraphics[scale=0.75]{pattern-window-current-seq-armed.png}
   \includegraphics[scale=0.65]{roll.png}
   \caption{Currently-Edited Pattern, Armed}
   \label{fig:pattern_window_current_seq_armed}
\end{figure}

   If more than one sequence or \textbf{Event Editor}
   is brought up, only the slot for
   the last one to have focus is hightlighted.
   Note that this highlighting also applies to the \textbf{Event Editor}.

   \itempar{Paste}{pattern!paste}
   Pastes a loop or pattern that was previously copied.
   Also note that there is no \texttt{Ctrl-V} key for this operation in the
   main window.

   \itempar{Song}{pattern!song}
   The \textbf{Song} items are described later, in reference to
   \figureref{fig:pattern_window_right_click_song}.
   
\subsubsection{Pattern}
\label{subsubsec:patterns_pattern_filled}

   A filled pattern slot is referred to as a \textsl{pattern}
   (or \textsl{loop}, or \textsl{sequence}).
   A pattern is shown in the Pattern window as a filled box with the
   following items of information in it.
   Examine \figureref{fig:pattern_window_main_panel_items}; it shows
   these items annotated for clarity.

   \begin{itemize}
      \item \textbf{Name}.
         \index{pattern!name}
         This line contains the name or title of the pattern, to help
         reference it when juggling a number of patters.
      \item \textbf{Pattern Length}.
         \index{pattern!length}
         If the option to show pattern hot-key is enabled, the length of the
         pattern, in measures, is shown in the upper right corner of the
         pattern slot.  This feature is useful when recording tempo events that
         will increase the length of the tempo track.
      \item \textbf{Contents}.
         \index{pattern!contents}
         The contents of the pattern provide a fairly detailed and
         distinguishable representation of the notes or events in the
         pattern.  Also, when the song is playing, a vertical bar cursor
         tracks the position of the playback of the pattern or loop; it
         returns to the beginning of the box every time that pattern starts
         over again.
         \index{empty pattern}
         With \textsl{Seq66}, an imported empty pattern will no longer
         needlessly scroll.
         However, if a pattern has even a single event (say, a program change),
         it will scroll.
%        \index{todo:one-shot pattern}
%        It might be good to have some patterns marked as one-shot patterns.
%        They play once at the start of playback, and that is it.
%        They could be marked with a cyan background.
%        Currently, it is easy enough to use the Song Editor for this purpose,
%        but then one cannot play the patterns in live mode.
      \item \textbf{Sequence Number}.
         If the option to show the sequencer number is set
         in the \textbf{File / Options / Keyboard} section
         (see \sectionref{paragraph:menu_file_options_keyboard},
         the this number is shown at the bottom left of the pattern slot.
      \item \textbf{Bus-Channel}.
         \index{pattern!bus-channel}
         This pair of numbers shows the the MIDI buss number, a dash, and
         the MIDI channel number.
         For example, "0-2" means MIDI buss 0, channel 2.
      \item \textbf{Beat}.
         \index{pattern!beat}
         This pair of numbers is the standard time-signature of the pattern,
         such as "4/4" or "3/4".  The first number is the beats-per-measure,
         and the second is the size of the beat, here, a quarter note.
      \item \textbf{Shortcut Key}.
         If the display of hot-keys is enabled (see
         \sectionref{paragraph:menu_file_options_keyboard}),
         then the key noted in the lower-right corner of the pattern can be
         pressed to toggle the mute/unmute status of that pattern.
         This action is an alternative to left-click on the pattern.
      \item \textbf{Progress Cursor}.
         At the left of each box is a vertical line, waiting for playback to
         start so that it can move through the pattern, again and again.
      \item \textbf{Armed}.
         See \figureref{fig:pattern_window_main_panel_items}; it shows a black
         and white pattern.  The black color indicates that the pattern is armed
         (unmuted), and will play if playback is initiated in the pattern
         \index{live mode}
         window in live mode.
         An item is armed/disarmed by a left-click on it.
         \index{shift left click}
         If the Shift key is held during a left-click on a pattern, then
         the armed/unarmed state of every other active pattern is toggled.
         This feature is useful for isolating a single track or pattern.
      \item \textbf{Queued}.
         That same pattern also shows that it is queued, which means that it
         will toggle its playing status when the pattern next begins again.
      \item \textbf{Alternate font}.
         Later builds of \textsl{Seq66} are now built with a new font.
         See \figureref{fig:pattern_window_main_panel_items}.  It shows the new
         font. 
         The old font can be selected in the "user" configuration file, and is
         also selected automatically if \textsl{Seq66} is run in the
         \textsl{legacy} mode.
      \item \textbf{Sequence number}.
         Later builds of \textsl{Seq66} are now built with the option to
         also show the sequence number in the pattern box, if the "show
         sequence numbers" option is on.
         This option can be set in the "user" configuration file.
         See \figureref{fig:pattern_window_main_panel_items}.  It shows an
         example of the sequence number, using the new font.
   \end{itemize}

   \index{pattern!left click}
   Left-click on an filled pattern box will toggle the status of the
   pattern between muted (white background) and unmuted (black background).
   If the song is playing via the main window, toggling this status makes
   the pattern stop playing or start playing.  The armed status
   can also be toggled using hot-keys.

   If the \textbf{Song Editor} is the active window and was used to
   start the playback, the pattern boxes will toggle between the muted/unmuted
   states as the music plays, and the pattern is active or inactive at the
   point of playback.  (The \textbf{Song Editor} acts as a list of triggers).

   \index{pattern!right click}
   A right-click on an already-filled box brings up a menu
   to allow one to edit it, or perform a few other actions
   specified in the context menu.  Here is that menu:

\begin{figure}[H]
   \centering 
%  \includegraphics[scale=0.75]{pattern/pattern-right-click-menu.png}
%  \includegraphics[scale=0.75]{new/seqmenu_menus-gtk-qt-0_96.png}
   \includegraphics[scale=0.65]{roll.png}
   \caption{Existing Pattern, Right-Click Menus, Gtkmm and Qt Versions}
   \label{fig:pattern_window_right_click}
\end{figure}

   Here one can choose to edit the pattern, cut and copy the pattern,
   set the MIDI bus/channel, and more.
   One can also clear all performance (song) data for the pattern.
   Here are the Gtkmm menu entries:
   
   \begin{enumber}
      \item \textbf{Edit...}
      \item \textbf{Event Edit...}
      \item \textbf{Cut}
      \item \textbf{Copy}
      \item \textbf{Song}
      \item \textbf{Color}
      \item \textbf{Disable/Enable Transpose}
      \item \textbf{MIDI Bus}
   \end{enumber}

   The Qt menu entries are different and more extensive:
   
   \begin{enumber}
      \item \textbf{New pattern}
      \item \textbf{Extern live frame}
      \item \textbf{Edit pattern in tab}
      \item \textbf{Edit pattern in window}
      \item \textbf{Edit events in tab}
      \item \textbf{Set pattern color}
      \item \textbf{Copy pattern}
      \item \textbf{Cut pattern}
      \item \textbf{Delete pattern}
   \end{enumber}

   See \sectionref{subsubsec:qt_portmidi_qt5_live_slot_menu}.
   It describes these additional items and how the Qt user-interface works.
   The next sections describe the Gtkmm-accessible functions.

   \setcounter{ItemCounter}{0}      % Reset the ItemCounter for this list.

   \itempar{Edit...}{pattern!edit}
   Edits an existing loop or pattern.
   Clicking this menu entry brings up the \textbf{Pattern Editor}
   so that one can modify the existing pattern by click-dragging new notes in a
   piano roll user-interface.
   See \figureref{fig:pattern_edit_window}.
   Also known as the "sequence editor".

   In addition to right-click and selecting \textbf{Edit...}, the user can
   \index{empty slot double-click}
   double-click on the slot, to bring up the \textbf{Pattern Editor}.

   \index{pattern edit}
   \index{keys!=}
   \index{keys!pattern edit}
   Another way to bring up a pattern in the 
   \textbf{Pattern Editor} is to
   click the \textbf{equal} key and then the pattern's hot-key.
   For example, "\textbf{=q}" will open up the editor for the pattern with the
   hot-key \textbf{q}.
   The Equals key (\texttt{=}) is the default key that does this action.
   This key can be changed by modifying the
   \textbf{File / Options / Keyboard / Control keys / Pattern Edit} item.

   \itempar{Event Edit...}{pattern!event edit}
   \index{pattern event edit}
   Edits an existing loop or pattern, but using a detailed \textbf{Event Editor}
   that shows events as text and numbers, and allows editing them as text and
   numbers.
   See \figureref{fig:pattern_edit_window}.

   \index{event -}
   \index{keys!-}
   \index{keys!event edit}
   Another way to bring up the \textbf{Event Editor} is to
   click the \textbf{minus} key and then the pattern's hot-key.
   For example, "\textbf{-q}" will open up the \textbf{Event Editor}
   for that pattern.
   The Minus key (\texttt{-}) is the default key that does this action.
   This key can be changed by modifying the
   \textbf{File / Options / Keyboard / Control keys / Event Edit} item.

   The \textbf{Event Editor}
   is not the same as the \textbf{event} pane in the pattern
   editor; the \textbf{Event Editor} shows all events at once, and shows them
   only in text/list format.  This editor is basic, meant for viewing
   MIDI events and making some minor edits or deletes.
   The \textbf{Event Editor} is most useful when trying to find events
   that are screwing up the performance of that pattern.
   See \sectionref{sec:event_editor}, for more information.

   To simplify the application and avoid editing a pattern in
   two different dialogs, if either the 
   \textbf{Pattern Editor} or the
   \textbf{Event Editor} is
   active for a given sequence, the right-click sequence-slot menu leaves out
   the \textbf{Edit...} and \textbf{Event Edit...} menu entries.
   This trimmed menu looks like this:

\begin{figure}[H]
   \centering 
%  \includegraphics[scale=0.75]{pattern/pattern-right-click-menu-no-edit.png}
   \includegraphics[scale=0.65]{roll.png}
   \caption{Existing Pattern, Right-Click Menu Without Edit Entries}
   \label{fig:pattern_window_right_click_no_edit}
\end{figure}

   The old functionality was to have the \textbf{Edit...} menu entry simply
   raise the existing 
   \textbf{Pattern Editor} to the top of the windows.

   \itempar{Cut}{pattern!cut}
   Deletes and copies an existing loop or pattern.
   One can also drag-and-drop a pattern into another cell (there is no outline
   box during the drag, sadly).
   Note that there is no \texttt{Ctrl-X} key for this operation in the
   main window.

% This bug was fixed:
%
%  \textbf{Bug:}
%  \index{bugs!pattern cut not dirty}
%  Although this operation works, it does not cause the user to be prompted if
%  the application is exited.

   \itempar{Copy}{pattern!copy}
   Copies an existing loop or pattern.
   The pattern can then be pasted elsewhere in the Patterns panel.
   One can also drag-and-drop a pattern into another cell (there is no outline
   box during the drag).
   See \sectionref{subsubsec:patterns_pattern_slot}.
   Note that there is no \texttt{Ctrl-C} key for this operation in the
   live (main) window.

   \itempar{Song}{pattern!song}
   Clicking this menu entry brings up a small popup menu:

\begin{figure}[H]
   \centering 
%  \includegraphics[scale=0.75]{pattern/pattern-menu-song.png}
%  \includegraphics[scale=0.75]{new/seqmenu_song_menu-0_9_15.png}
%  \includegraphics[scale=0.75]{new/seqmenu_song_menu-0_9_21.png}
   \includegraphics[scale=0.65]{roll.png}
   \caption{Existing Pattern, Right-Click Menu, Song}
   \label{fig:pattern_window_right_click_song}
\end{figure}

   \begin{enumber}
      \item \textbf{Clear This Track's Song Data}
      \item \textbf{Mute All Tracks}
      \item \textbf{Unmute All Tracks}
      \item \textbf{Toggle All Tracks}
      \item \textbf{Toggle Live Tracks}
   \end{enumber}

%  \setcounter{ItemCounter}{0}      % Reset the ItemCounter for this list.

   \index{Clear This Track's Song Data}
   \index{pattern!clear song data}
   \textbf{Clear This Track's Song Data}
   This item is not available if the pattern is empty.
   Selecting this filled-box right-click menu item causes that box's
   loop/pattern to be removed from the song editor.
   The triggers disappear from the Song Editor window, and so will not
   be played when the song plays in Song mode.

   \index{Mute All Tracks}
   \index{pattern!mute all tracks}
   \textbf{Song / Mute All Tracks}
   Selecting this filled-box right-click menu item causes
   the tracks in the Song Editor to be muted.  Sometime it takes a few seconds
   for the user-interfaces to show this big change.
   This item mutes all tracks (or loops/patterns).
   It works when one has opened the Song Editor window
   and started playing in playback
   mode by starting play using that window.

   So, let us assume the song is running in live (playback) mode.
   The patterns that are active (unmuted) in the live window are shown with a
   black background in the main patterns window.  If one right clicks on a
   pattern cell and selects \textbf{Song / Mute All Tracks}, all those patterns
   will become white and be silenced.  Eventually, the Song Editor window
   catches up and shows the "M" activated for all tracks.

   \index{Unmute All Tracks}
   \index{pattern!unmute all tracks}
   \textbf{Unmute All Tracks}
   Provides the opposite functionality, making all tracks armed and audible.
   Selecting this filled-box right-click menu item causes
   the tracks in the song to be unmuted.

   \index{Toggle All Tracks}
   \index{pattern!toggle all tracks}
   \textbf{Toggle All Tracks}
   Toggles the armed/mute status of all tracks.
   It doesn't matter if Live or Song Mode is in force.
   \index{keys!F8}
   By default, the \texttt{F8} key will also toggle all tracks.

   Note that there is also a feature where a
   \texttt{Shift-Left-Click} on a pattern slot toggles the mute
   status of the \textsl{other tracks}.

   \index{Toggle Live Tracks}
   \index{pattern!toggle live tracks}
   \textbf{Toggle Live Tracks}
   Toggles the mute status of only the armed/unmuted tracks when in Live mode.
   Works only in Live mode.  This operation unmutes all tracks that are
   currently unmuted.  The statuses of these armed tracks are saved; when
   this operation is performed again, those tracks are unmuted, turned back on.
   This menu entry provides the same function as the \textbf{Mute}
   button in the main window.

   \itempar{Color}{pattern!color}
   This menu item allows for specifying colors for the patterns.
   Colors can make it easier to find a pattern while running live.
   Note that there are some minor issues with colors, and that this feature is
   still in flux.

   \itempar{Enable/Disable Transpose}{pattern!transpose}
   This menu entry changes depending upon whether the new transpose feature is
   enabled or disabled for the sequence/pattern.  Note that, if the events
   shown in the slot are red, this denotes that transpose is currently
   \textsl{disabled} for that pattern, which might be a drum pattern.

   \itempar{MIDI Bus}{pattern!midi bus}
   Selecting this filled-box right-click menu item brings up a list
   of the up to 16 MIDI output busses that \textsl{Seq66} supports.
   For each of these buss items, another pop-up menu allows one
   to specify the MIDI output channel for that buss.

% \begin{figure}[H]
%  \centering 
%  \includegraphics[scale=0.65]{pattern/pattern-menu-midi-bus.png}
%  \caption{Existing Pattern, Right-Click Menu, MIDI Output Busses}
%  \label{fig:pattern_window_right_click_midi_bus}
% \end{figure}

\begin{figure}[H]
   \centering 
%  \includegraphics[scale=0.75]{pattern/pattern-menu-midi-bus-port.png}
   \includegraphics[scale=0.65]{roll.png}
   \caption{Existing Pattern Right-Click Menu, MIDI Output Busses/Channels}
   \label{fig:pattern_window_right_click_midi_bus}
\end{figure}

   \index{--bus option}
   Another way to specify busses is the
   \texttt{--buss n} command-line option.
   It causes \textsl{every} pattern in the MIDI
   file to be directed to that buss number, and when a new
   sequence/pattern is created.  This option is only
   for convenience in testing.  Save the file, and it will
   have that buss number as part of each track's data, which makes the song
   file less portable, so be careful.

% \begin{figure}[H]
%  \centering 
%  \includegraphics[scale=0.65]{pattern/pattern-menu-midi-bus-numbers.png}
%  \caption{Existing Pattern, Right-Click Menu, MIDI Buss Channels}
%  \label{fig:pattern_window_right_click_midi_bus_numbers}
% \end{figure}

\subsubsection{Pattern Keys and Click}
\label{subsubsec:patterns_pattern_keys_and_clicks}

   This section recapitulates all the clicks and keys that perform actions
   in the Pattern windows.  Some additional clicks and keys are noted here
   as well.

\paragraph{Pattern Keys}
\label{paragraph:patterns_pattern_keys}

   \index{keys!hot}
   \index{keys!shortcut}
   Each pattern in the patterns panel can have a hot-key or shortcut-key
   associated with it.

   \index{keys!pattern toggles}
   For each pattern, hitting its assigned hot-key will
   also toggle its status between muted/unmuted (armed/unarmed).
   Below is the default grid that is
   mapped to the loops/patterns on the screen-set.
   This grid can be changed in the
   \textbf{File / Options / Keyboard} tab, and is
   saved in the \textsl{keyboard-control} section of the
   \index{rc file}
   "rc" file.

   \begin{verbatim}
      [ 1 ][ 2 ][ 3 ][ 4 ][ 5 ][ 6 ][ 7 ][ 8 ]
      [ q ][ w ][ e ][ r ][ t ][ y ][ u ][ i ]
      [ a ][ s ][ d ][ f ][ g ][ h ][ j ][ k ]
      [ z ][ x ][ c ][ v ][ b ][ n ][ m ][ , ]
   \end{verbatim}

   These characters are shown in the lower right corner of each
   pattern, as an aid to memory.

   A "shift" functionality is available for the
   mute/unmute hot-keys when a set is larger than 32 patterns.
   \index{variset!slash key}
   Normally, pressing the \texttt{1} key will toggle
   sequence 0.  If preceded by one slash key (\texttt{/}), then sequence 32
   will be toggled.  If preceded by two slash keys, then sequence 64 will be
   toggled.  This features supports using set sizes of 32, 64, and 96 patterns.

\begin{figure}[H]
   \centering 
%  \includegraphics[scale=0.50]{new/patterns_panel_shift_toggle.png}
   \includegraphics[scale=0.65]{roll.png}
   \caption{Patterns Panel, Shift-Key Pattern Toggle}
   \label{fig:pattern_window_shift_key_pattern_toggle}
\end{figure}

   This figure shows how one presses \texttt{y}, \texttt{/y}, and \texttt{//y}
   to arm three patterns in this 96-pattern set.

   \index{keys![}
   \index{keys!decrement set}
   \index{keys!screenset down}
   The "\texttt{[}" and
   \index{keys!]}
   \index{keys!increment set}
   \index{keys!screenset up}
   "\texttt{]}" keys on the keyboard decrement or increment the set number.

   \index{keys!alt}
   \index{keys!snapshot}
   The left and right \texttt{Alt} keys are, by default, set up in the
   \textbf{File / Options / Keyboard / Snapshot 1} and
   \textbf{Snapshot 2} fields to be used as "snapshot" keys.
   Our preference is to use something that does not trigger desktop
   commands, perhaps "\texttt{F11}" or "\texttt{F12}", or one of the keys in
   the keypad.

   When a snapshot key is pressed, the state of the patterns
   (armed versus unarmed) is saved.  While the
   snapshot key is held, one can then change the state of the patterns
   (using the keyboard, \textsl{not} the mouse)
   to change how the song plays.  When the snapshot key is released, the
   original saved state of the patterns is restored.

%  \index{keys!alt}
%  \index{keys!snapshot}
%  Holding \texttt{Alt} will save the state of playing patterns and restore
%  them when \texttt{Alt} is lifted.
%  The handling of \texttt{Alt} is often taken over by the application,
%  so there could be a need to change these items to some other
%  keys.  For example, we have the \textsl{Fluxbox} window manager
%  set up so that the \texttt{Alt} keys can
%  be used for moving or resizing a window.  Therefore, we use
%  keypad keys for this purpose.

%  \index{keys!left ctrl alt}
%  Holding \texttt{Left Ctrl} and \texttt{Alt} at the same time will enable
%  one to flip over to new patterns briefly and then flip right back upon
%  lifting \texttt{Alt}.  Not yet sure exactly what this means.

   \index{keys!right ctrl}
   \index{keys!queue}
   \index{queue!temporary}
   Holding the "queue" key and then hitting a pattern hot-key
   will queue an on/off toggle for a pattern when the end of the loop is
   reached.
   This is the "queue" functionality.
   This means that the change in state of the pattern will not take hold
   immediately, but will kick in when the pattern restarts.
   This pending state is indicated by coloring the central box of the
   pattern grey, as shown in the figure below.
%  Note that queuing can also be used to turn a pattern \textsl{off}
%  at the end of a pattern.
   Please note the "keep queue" functionality and
   the "one-shot queue" functionality described below.

\begin{figure}[H]
   \centering 
%  \includegraphics[scale=0.75]{pattern/seq24-queueing-coloration.jpg}
%  \includegraphics[scale=0.75]{new/seq66-queueing-coloration.png}
   \includegraphics[scale=0.65]{roll.png}
   \caption{Pattern Coloration when Queued}
   \label{fig:queueing_coloration}
\end{figure}

   This figure shows the coloring for queuing in the top two patterns with
   dark grey event backgrounds.  At the end of the pattern, the left top
   pattern will turn off, and the right top pattern will turn on.
   The bottom two patterns show the light-grey coloring used to show
   a "one-shot" queue.  The one-shot queue can only turn a pattern on, and it
   will force the pattern off after one play.
   Queue also works for mute/unmute pattern sets ("groups"); in this case,
   every sequence will toggle its status after its individual loop ends. 

   \index{keys!avoid ctrl/alt}
   We do \textbf{not}
   recommend using \texttt{Ctrl} or \texttt{Alt}
   keys for pattern control.  They conflict with application or desktop
   settings.  However, if one insists on such hot-key combinations,
   use the \textbf{Menu} button in the main
   window to disable the menu.
   One can also use normal keys to enable queuing.
   For example, the minus key or the keypad's slash key can be used.
%  , which makes those keystrokes more
%  safely available.
%  The \texttt{Super} key can also be used, if not already used by the desktop
%  environment.  Also available are some of the function keys, and, if
%  available, the keypad keys.  These can be configured in
%  \textbf{File / Options / Keyboard} and
%  \textbf{File / Options / Ext Keys}.
%  Of course, if the \texttt{Ctrl} key is used to manage the GUI (e.g.
%  \texttt{Ctrl-Q} will unceremoniously quit the application), so one will
%  usually want to change this key to something else in the
%  \textbf{File / Options / Keyboard / Queue} field.
%  The \texttt{Super} key (i.e. the \texttt{Mod4} or \texttt{Windows key}) is a
%  good candidate to substitute for the \texttt{Ctrl} key, unless one has (like
%  the author)
%  configured the window manager to use the \texttt{Super} key to manipulate
%  windows and applications \textsl{(laughter ensues)}.
%  , with no flickering due to the keyboard repeat rate.

   \index{keys!keep queue}
   \index{queue!keep}
   Pressing the "keep queue" hot-key
   \index{rc file}
   assigned in the "rc" file activates a "sticky" queue mode.
   In this mode, pressing a pattern key immediately turns on queuing, instead
   of mute/unmute.  And multiple patterns can be handled in this way at the
   same time.
   Keep-queue persists until one clicks the normal queue function hot-key,
   or changes the active (viewed) screen-set. 
%  After pressing the "keep queue" hot-key, individual pattern
%  toggles are queued rather than happening immediately.
%  While in queue mode, pressing a patterns hot-key, or clicking on the
%  pattern, queues that pattern to change state at the beginning of the next
%  loop.
   \index{queue!cancel}
   “Keep queue” mode is cancelled by pressing the normal queue hot-key.
   This hot-key can be changed in the
   \textbf{File / Options / Keyboard / Keep queue} field.
   There is also a \textbf{Q} button for the same purpose.
   Also note the "queued replace/solo" functionality, described a bit later.

   \index{one-shot queue}
   \index{keys!one-shot queue}
   \index{queue!one-shot}
   Thanks to \textsl{Kepler34}, we have "one-shot queue"
   functionality.  This one-shot setup queues a pattern up for unmuting only,
   and, once the pattern has played, it is automatically muted.  This process
   is easier than having to unqueue the pattern manually before the next
   playback.
   This hot-key can be changed in the
   \textbf{File / Options / Ext Keys / One-shot queue} field.

   \index{keys!replace}
   The "replace" hot-key (the left \texttt{Ctrl} key by default, which 
   should be changed to something better), 
   sets a form of muting/unmuting.  When the "replace" hot-key is
   pressed and held while clicking a pattern or pressing that pattern's
   hot-key, that sequence is unmuted, and all of the other sequences are muted.
%  Again, \texttt{Ctrl}
%  is a very inconvenient default, so an alternative key mapping
%  should be set up in the "rc" file via the \textbf{File / Options / Keyboard}
%  tab.
   "Replace" is a form of "solo".
   "Replace" is also implemented via MIDI control,
   where the MIDI control can be activated, but then the user has to select
   the desired sequence.  

   \index{queue!replace}
   \index{queue!solo}
   \textsl{Seq66} provides an extension to the replace/solo functionality
   that is called "queued-replace" or "queued-solo".  In this feature, when
   the "keep queue" function is activated, the replace function is queued so
   that it does not occur until the next time the patterns loop.
   And queued-replace provides a form of snapshot, limited to the
   \textsl{current} screen-set.
   Here are the steps:

   \begin{enumber}
      \item Start playback with some patterns on. 
      \item Press and release
         the "keep queue" hot-key.  This puts the application into "queue" mode.
         It is indicated via a "\textbf{Q}" button.
      \item Press and hold the "replace" hot-key.
      \item Click the desired pattern hot-key.  Observe that it arms or
         stays on, and that the other playing patterns show the "queued" color
         (grey).  At the end of the loop, they turn off, and the "replace"
         pattern is now solo.
      \item Click the same pattern hot-key again.  Observe that the other
         patterns that were toggled off are now queued to be toggled on at the
         next loop.  Steps 4 and 5 can be repeated endlessly.
      \item To end
         \index{queue!clear}
         \index{queue!end}
         the "queued-replace" mode, click the normal "queue"
         hot-key.  Also, changing the active screen-set ends "queue-replace"
         mode.  It does \textsl{not} end normal queue mode, to preserve the
         behavior found in \textsl{Seq24}.
         One needs to clear the queue mode in order to select another pattern
         to solo.
   \end{enumber}

\begin{figure}[H]
   \centering 
%  \includegraphics[scale=0.75]{new/queued_replace.png}
   \includegraphics[scale=0.65]{roll.png}
   \caption{Queued-Replace (Queued-Solo) In Action}
   \label{fig:queued_replace}
\end{figure}

   Before pressing the "keep queue" key, patterns 33 ("\textbf{q}")
   and 34 ("\textbf{a}") are
   unmuted, while the desired replace pattern, 32 ("\textbf{1}") is off.
   Then the user presses (and holds) the "replace" key, then clicks the
   "\textbf{1}" key.
   This puts all unmuted patterns, plus the muted
   replace pattern as well, into queue mode, as shown by the grey panels.
   When the progress bar reaches the end of the pattern, pattern 32 will go on,
   and patterns 33 and 34 will go off.
   If the replace-pattern is already on, it is not queued, as
   there's no need to turn it on.

   If, while in queue mode, the replace key is held and
   "\textbf{1}" is pressed again,
   the other patterns will be queued, and will turn on again.  Thus, the
   solo status of the replace pattern can be toggled at will, until queue mode
   is exited by pressing and releasing the normal "queue" key.
   If the replace key is \textsl{not} held down, and another pattern's replace
   hot-key is pressed, that pattern will be queued normally.
   If one wants to change the solo functionality to a different pattern,
   simple hold the replace key and click on a different pattern.  The new
   arrangement of soloing is memorized.
   One can clear the queue mode by pressing the normal queue key.

   There are more keys defined in the \textbf{Keyboard} dialog, and it is
   worth figuring out what they do, if not documented here.
   For a couple of short, but good, video tutorials about using arming,
   queuing, and snapshots, see reference \cite{wootangent1}.

   \index{solo!true}
   \index{pattern!shift-left-click}
   \index{patterns panel!inverse muting}
   \index{patterns panel!solo}
   \index{shift-left-click solo}
   There is a truer "Solo" functionality in the Patterns
   Panel and the Song Editor.  To "solo" a pattern, move the mouse cursor
   over the pattern, hold the \texttt{Shift} key, and left-click the pattern.
   This will turn off all the other patterns, so that the selected pattern ins
   the only one playing.  Holding the \texttt{Shift} key and clicking the same
   pattern again will unmute all of the other patterns.

\paragraph{Pattern Clicks}
\label{paragraph:patterns_pattern_Clicks}

   \index{pattern!left click}
   \index{pattern!mute toggle}
   Left-click on a pattern-filled box will change its state
   \index{pattern!mute}
   \index{pattern!unmute}
   from muted (white background) to playing (black background), whether
   the sequencer is playing or not.

   \index{pattern!left click-drag}
   Left-click-hold-drag on a pattern, drags it to a different
   pattern on the grid.
   The box disappears while dragged, and reappears in the new location when
   dropped.  However, a pattern \textsl{cannot} be dragged if its
   \textbf{Pattern Editor} window is open.

   \index{pattern!right click}
   Right-click on a pattern brings up the appropriate context menus, as
   discussed earlier, depending on whether the pattern box is empty or
   filled.

   \index{pattern!middle click}
   Middle-click does nothing when the mouse rests inside a pattern box.

\subsection{Patterns / Bottom Panel}
\label{subsec:patterns_panel_bottom}

   The bottom panel of the Patterns window provides way to control the
   overall playback of the song.  It has changed quite a bit over the last few
   versions of \textsl{Seq66}, and we have not yet caught up with the
   diagrams. And the Qt user-interface adds more changes.
   Refer to the diagram of the whole window, for now.

\begin{figure}[H]
   \centering 
%  \includegraphics[scale=0.75]{pattern-window-bottom-panel-items.png}
%  \includegraphics[scale=0.50]{new/pattern-window-bottom-panel.png}
   \includegraphics[scale=0.65]{roll.png}
   \caption{Patterns Panel, Bottom Panel Items}
   \label{fig:pattern_window_bottom_panel_items}
\end{figure}

   This figure shows a number of new items.

   \begin{enumber}
      \item \textbf{Panic!}
      \item \textbf{Stop}
      \item \textbf{Play and Pause}
      \item \textbf{Song Record}
      \item \textbf{Song Record Snap}
      \item \textbf{BPM}
      \item \textbf{Tap Tempo}
      \item \textbf{Log Tempo}
      \item \textbf{Record Tempo}
      \item \textbf{Keep-Queue Status}
      \item \textbf{Name}
      \item \textbf{Set}
      \item \textbf{Toggle Song Editor}
   \end{enumber}

   \setcounter{ItemCounter}{0}      % Reset the ItemCounter for this list.

   \itempar{Panic!}{pattern!panic}
   This new button stops the song and sends MIDI Off messages on all notes.

   \itempar{Stop}{pattern!stop}
   The red square button stops the playback of the song and all its patterns.
   \index{keys!esc (stop)}
   The keystroke for stopping playback is the \texttt{Escape} character.
   It can be changed to \texttt{Space}, so that the space-bar then becomes
   effectively a playback toggle key.

   \itempar{Play and Pause}{pattern!Play}
   \index{pattern!Pause}
   The green triangular button starts the playback of the whole song.
   \index{keys!space (play)}
   The keystroke for starting playback is the \texttt{Space} character by
   default.

   \index{pause}
   The Play button can be used as a Pause button.
   When the Play button is clicked, the button icon changes to a Pause icon:

\begin{figure}[H]
   \centering 
%  \includegraphics[scale=1.0]{new/stop_pause_buttons.png}
   \includegraphics[scale=0.65]{roll.png}
   \caption{Patterns Panel, Pause Button}
   \label{fig:pattern_window_pause_button}
\end{figure}

   A Pause key (by default, the period) is also defined.
%  (The pause feature can be removed by rebuilding the application
%  after configuring with the \texttt{--disable-pause} option.)

   \itempar{Song Record}{pattern!song record}
   Song-recording in \textsl{Seq66} is adopted from the
   \textsl{Kepler34} project.
   This feature takes live muting changes and records them as
   triggers in the \textbf{Song Editor}.
   The default hot-key for this function is \texttt{P}.
   This feature does not honor queuing...
   rather than waiting until the end of the pattern when the queuing takes
   effect, the trigger recording starts immediately.

   \itempar{Song Record Snap}{pattern!song record snap}
   This button toggles snapping the beginning and end of a recorded trigger to
   the nearest beat.  There is no hot-key for this button at this time.

   \itempar{BPM}{pattern!BPM}
   The spin widget adjusts the "beats per minute" (BPM) value.  The
   range of this field is from 1 bpm to 600 bpm, with a default value of
   120 bpm.
   Although this field looks editable, it is not.  Most keystrokes
   that are entered actually toggle one of the pattern boxes.
   However, the following keys can also modify the BPM in small increments:
   \index{keys!semicolon}
   The \texttt{semicolon} reduces the BPM;
   \index{keys!apostrophe}
   The \texttt{apostrophe} increases the BPM.
   Also, if one right-clicks on the Up button, the BPM advances to its largest
   supported value, and if one right-clicks on the Down button, the BPM
   advances to its lowest value.
   MIDI control for this value is also available.

   The precision of the BPM value can be set to 0, 1, or 2
   decimal places, and the increment values for the step size (small)
   or page size (large) of the BPM spinner can be configured in the "usr" file.
   See \sectionref{subsec:usr_file_user_midi_settings}; it describes
   the precision and increment options more fully.
   The following figure shows the appearance of the BPM field with different
   precision values:

\begin{figure}[H]
   \centering 
%  \includegraphics[scale=0.65]{new/bpm_precision_settings.png}
   \includegraphics[scale=0.65]{roll.png}
   \caption{Patterns Panel, BPM Precisions}
   \label{fig:pattern_window_bpm_precision_settings.png}
\end{figure}

   \itempar{Tap Tempo}{pattern!tap tempo}
   This control is clicked in time with a tune, to set the
   tempo based on the tempo of the clicks.  Once clicked, the label of this
   button increments with every click, and the \textbf{BPM} field updates to
   display the calculated tempo.  If the user stops tapping for 5 seconds, the
   label reverts to 0, the BPM value keeps its final value, and the user can
   try tapping the tempo again, or accept the current value.
   Tapping can also be done using the keystroke defined
   in \textbf{File / Options / Ext Keys / Tap BPM}.
   It defaults to the "\texttt{F9}" key.

   \itempar{Log Tempo}{pattern!log tempo}
   This light-magenta button (Gtkmm only),
   logs the current tempo at the
   current playback spot as a Set Tempo meta-event, in the first
   track (pattern slot \#0) or the alternate tempo track if defined,
   and only if it is active.  According to the MIDI standard, such events
   should be present \textsl{only} in the first track,
   and so \textsl{Seq66} follows this rule, and also makes tempo events
   officially supported.  They can be edited in the 
   \textbf{Pattern Editor} or in the
   \textbf{Event Editor}.
   The main/global tempo is not written to the tempo track; it is stored in
   a SeqSpec section.
   See \sectionref{sec:meta_events}, for more information.

   \itempar{Record Tempo}{pattern!record tempo}
   This dark-magenta button (Gtkmm only)
   becomes light-magenta when activated, and turns on
   the recording of any tempo changes made in the BPM spinner.  If the spinner
   is held down, a ramping stream of tempo events is created.  If
   the time exceeds the current length of the tempo track, then the length of
   the track is automatically increased.
   These tempo events will not affect playback speed
   unless the tempo track is unmuted.

   \itempar{Keep-Queue Status}{pattern!keep-queue}
   This item is the \textbf{Q} button.
   It provides a visual way to know the current state of keep-queue, and is
   activated either by clicking on it or by pressing the assigned keep-queue
   key.

   \itempar{Name}{pattern!set name}
   Each of the 32 available screen-sets can be given a name by entering it
   into this field.  This name is saved with the MIDI file.

% This bug is fixed, as of 0.9.18, IIRC
%
%  \textbf{Bug:}
%  \index{bugs!set name has side-effect}
%  While one is typing in the name of the set in this field, the keystrokes
%  will affect the panel window, causing playback to start and pattern
%  boxes to be toggled!

   \itempar{Set}{pattern!set number}
   This spin widget selects the current screen-set.  The values in this
   field range from 0 to 31 (less if the set-size is a larger value),
   and default to 0.
%  Although this field looks editable, it is not.

% This bug is fixed, as of 0.9.18, IIRC
%
%  \textbf{Bug:}
%  \index{bugs!set number has side-effect}
%  While one is typing in the number of the set in this field, the keystrokes
%  will affect the panel window as well.

   \itempar{Toggle Song Editor}{pattern!toggle song editor}
   Pressing this button toggles the presence on-screen of the
   \textbf{Song Editor}.  The \texttt{Ctrl-E} keystroke can also be used.

\subsection{Patterns / Multiple Panels}
\label{subsec:patterns_panel_multiple}

%  If \textsl{Seq66} is built with the \texttt{--enable-multiwid}
%  \textsl{Seq66} has a "multiwid" option, this defines the 
%  \texttt{SEQ66\_MULTI\_MAINWID} macro, and allows for
   \textsl{Seq66} has a "multiwid" option, which
   allows for
   \index{multi-wid}
   multiple live panels showing more than one set at a time.
   The default is to show the usual single "mainwid".
   In the Qt user-interface, multiple live panels in separate windows
   can be opened.

\begin{figure}[H]
   \centering 
%  \includegraphics[scale=0.50]{multiwid/multiwid-mode-3x2.png}
   \includegraphics[scale=0.65]{roll.png}
   \caption{Patterns Panel, with Multiple Panels}
   \label{fig:pattern_window_bottom_panel_multiple}
\end{figure}

   In the Gtkmm user-interface, either one spinner for all, or one spinner for
   each, is available.  The latter is the "independent" mode.
   Note that it is possible, in this mode, to show the same set in two
   different "mainwids", but this is not recommended, as there may be minor
   unavoidable issues with that.

   Note that Page Up and Page Down keystrokes, as well as their
   configurable counterparts in
   \textbf{File / Options / Keyboard / Control Keys / Screenset Up}
   and \textbf{Screenset Down}, apply only to the top leftmost "mainwid".
   Of course, if the "mainwids" are synced, then all are affected by these
   keystrokes.

   In multi-wid mode, each "mainwid" frame shows the corresponding set number
   and, if present, the set notepad text for each "mainwid" set.

   See 
   \sectionref{sec:man_page}, for how the
   \texttt{-o wid=3x2,i} option can be used to set this mode, and
   \sectionref{subsec:usr_file_user_interface_settings}, for
   how these settings can be made permanent in the "usr" file.
   In that file, the options modified are \texttt{block\_rows} and
   \texttt{block\_columns}.

\subsection{Patterns / Variable Set Size}
\label{subsec:patterns_panel_variset}

   \index{variset}
   This option, informally known as "variset", allow some changes in
   the set size and layout from the default 4x8 = 32 sets layout.
   The row count can be set from 4 to 8, and the column count can be set to 8
   to 12.  Note that the set size can only be \textsl{increased} by these
   settings.

   \textbf{Warning:}
   \textsl{seq24} was fairly hardwired for supporting 32 patterns per
   set, and there are still places where that is true.  Thus,
   consider this option to be experimental.

   Also see 
   \sectionref{sec:man_page}, for how the
   \texttt{-o sets=8x8} option can be used to set this mode, and
   \sectionref{subsec:usr_file_user_interface_settings}, for
   how these settings can be made permanent in the "usr" file.
   In that file, the options modified are \texttt{mainwnd\_rows} and
   \texttt{mainwnd\_cols}.

\begin{figure}[H]
   \centering 
%  \includegraphics[scale=0.50]{multiwid/variset-mode-8x8.png}
   \includegraphics[scale=0.65]{roll.png}
   \caption{Patterns Panel, 8 x 8 Layout}
   \label{fig:pattern_window_bottom_panel_variset}
\end{figure}

   Generally, it is recommend to stick with the 4x8 (32 patterns/set),
   8x8 (64 patterns/set), and 8x12 (96 patterns/set).  This works best with the
   existing set of 32 hot-keys.

   Also note that the Qt 5 user-interface also supports "variset", whether in
   the main window or in the external live-frame.  In addition, both Qt windows
   can be resized and still show good renditions of the pattern-slots.

%-------------------------------------------------------------------------------
% vim: ts=3 sw=3 et ft=tex
%-------------------------------------------------------------------------------


% Pattern Editor

%%% %-------------------------------------------------------------------------------
% seq66_pattern_editor
%-------------------------------------------------------------------------------
%
% \file        seq66_pattern_editor.tex
% \library     Documents
% \author      Chris Ahlstrom
% \date        2015-08-31
% \update      2019-06-15
% \version     $Revision$
% \license     $XPC_GPL_LICENSE$
%
%-------------------------------------------------------------------------------

\section{Pattern Editor}
\label{sec:seq66_pattern_editor}

   The \textsl{Sequencer66 Pattern Editor} is used to edit and preview a
   pattern, as well as to configure its buss and channel settings.
   Please note that the \textbf{Qt 5} user-interface (which will eventually
   supercede the Gtkmm-2.4 user-interface) provides an \textbf{Edit Tab}, which
   has limited functionality, and an external window interface, which is on par
   with the Gtkmm-2.4 version.

%  In programmer's jargon, this window is represented by the seqroll class (and
%  the other "seq" classes).

\begin{figure}[H]
   \centering 
%  \includegraphics[scale=0.75]{pattern/pattern-edit-window.png}
%  \includegraphics[scale=0.75]{new/pattern_editor_chords-0_9_14.png}
%  \includegraphics[scale=0.75]{new/pattern_editor_transpose-0_9_15.png}
%  \includegraphics[scale=0.65]{gtkmm/pattern_editor/pattern_editor_0_96_1.png}
   \includegraphics[scale=0.65]{roll.png}
   \caption{Pattern Edit Window}
   \label{fig:pattern_edit_window}
\end{figure}

%  Not shown in this figure are:
%  \begin{itemize}
%     \item The sequence number show as part of the updated window title, which
%        now shows the application name (e.g. "seq66" versus the old
%        "Sequencer66" title.
%     \item The sequencer number is now shown to the left of the
%        sequence/track name.
%     \item The new \textbf{LFO} button in the bottom panel.
%     \item The new \textbf{Rec Type}button in the bottom panel.
%  \end{itemize}
%  Both are described later. See
%  \figureref{fig:pattern_editor_bottom_panel_items}.

   This dialog is complex.
   For exposition, we break it into some common actions, a first panel, a
   second panel, a bottom panel, and a piano-roll/events section.

   \begin{enumber}
      \item \textbf{First Panel}
      \item \textbf{Second Panel}
      \item \textbf{Piano-Roll/Events Panel}
      \item \textbf{Bottom Panel}
      \item \textbf{Common Actions}
   \end{enumber}

   Before we describe this window, there are some things to recognize.
   First, if the pattern is empty when play is started, the progress bar will
   still move, that the user can key in new notes.
   Second, to add a note, one must press the \textsl{right}
   mouse button (the pointer changes to a pencil) and,
   \textsl{while holding it}, press the left mouse button.
   Or click in the pattern editor, press the
   \index{keys!p}
   \texttt{p} key to select the "pencil" or "paint" mode, then
   \index{mouse!left-click}
   left-click to add a note or
   \index{mouse!left-click-drag}
   left-click-drag to add multiple notes as the mouse moves.
   \index{keys!x}
   Press or release the right mouse button, or press
   \texttt{x} to "eXit" or "eXscape" from paint mode.
   Third, notes are drawn only with the length selected by the "notes" button
   near the top of the pattern window.  There are tricks to
   modifying the new notes that are described later.

   \textsl{Sequencer66} automatically scrolls
   horizontally through the sequence/pattern editor window when
   playback moves the progress bar outside of the current frame of data.  This
   feature makes it easier to follow patterns that are longer than a measure or
   two.
   
   \textsl{Sequencer66} also provides a way to restart the progress
   bar within the pattern without resetting it to the beginning of the pattern.
   \index{pause}
   This "pause" feature is accessed by the
   \index{keys!.}
   \index{keys!pause}
   pause key (which defaults to the period character) or by the pause
   button, which appears when playback is underway.
%  (This feature
%  can be disabled if the application is built via source code.  Note that
%  it works only in ALSA mode at present.)

\subsection{Pattern Editor / First Panel}
\label{subsec:seq66_pattern_editor_first}

   The top bar (horizontal panel) of the Pattern (sequence) Editor
   lets one change the name of
   the pattern, the time signature of the piece, how long the loop is, and
   some other configuration items.

\begin{figure}[H]
   \centering 
%  \includegraphics[scale=0.55]{pattern/pattern-edit-first-panel-items.png}
   \includegraphics[scale=0.65]{roll.png}
   \caption{Pattern Editor, First Panel Items}
   \label{fig:pattern_editor_first_panel_items}
\end{figure}

   In recent versions of \textsl{Sequencer66}, a "sequence-is-transposable"
   feature has been added, as shown above.

% \begin{figure}[H]
%    \centering 
%    \includegraphics[scale=0.75]{new/pattern_editor_top_panel-0_9_15.png}
%    \caption{Pattern Editor, First Panel Emphasizing Transposable Button}
%    \label{fig:pattern_editor_first_panel_transposable}
% \end{figure}

   \begin{enumber}
      \item \textbf{Pattern Number} (not shown)
      \item \textbf{Pattern Name}
      \item \textbf{Beats Per Bar}
      \item \textbf{Beat Unit (Beat Width)}
      \item \textbf{Pattern Length}
      \item \textbf{Transposable (toggle)}
      \item \textbf{MIDI Out Device}
      \item \textbf{MIDI Out Port}
   \end{enumber}

   \setcounter{ItemCounter}{0}      % Reset the ItemCounter for this list.

   \itempar{Pattern Number}{pattern editor!number}
   This item shows the sequence/track/pattern/loop
   number, to make it easier to pick it out when a lot of patterns are being
   edited at once.

   \itempar{Pattern Name}{pattern editor!name}
   Provides the name of the pattern.
   This name should be short and memorable.
   It is displayed in the Patterns window, on the top line of its pattern
   box/slot.
   There is no edit cursor in this field (not sure why yet), so
   one has to edit blind.

   \itempar{Beats Per Bar}{pattern editor!beats/bar}
   Specifies the number of beat units per bar in the time signature.
   The possible values range from 1 to 16, if the drop-down menu is used.
   The numeric value can be directly edited, to achieve
   non-standard values, thanks to an update from Jean-Emmanuel.
   But be careful!

   \itempar{Beat Unit (Beat Width)}{pattern editor!beat unit}
   \index{pattern editors!beat width}
   \index{beat width}
   Specifies the size of the bottom beat unit of the time signature:
   1 for whole notes; 2 for half notes; 4 for quarter notes; 8 for eight notes;
   16 for sixteenth notes; and 32 for thirty-second notes.
   The whole time signature is display at the bottom center of a pattern
   slot.

   \itempar{Pattern Length}{pattern editor!length}
   Sets the length of the current pattern, in measures.
   The possible values range from 1 to 64.
   \textsl{However}, when opening or importing a non-\textsl{Sequencer66}
   MIDI tune, the length of each track will be used, and so other values
   are possible; they just cannot be set via the user-interface.
   The numeric value can be directly edited, to
   achieve non-standard values, thanks again to Jean-Emmanuel.

   Bringing up a short pattern (one less than one measure or bar in
   length) in the pattern editor will adjust the pattern to pad it to the
   length of one measure.
   \index{pattern editor!progress bar}
   \textsl{Sequencer66} will, when it reads such a short pattern
   from a MIDI file,
%  (whether foreign or native to \textsl{Sequencer66}),
   pre-pad it to the length of a measure, so that it will always show smooth
   progress.
   For example, a pattern containing just one program
   change will be padded to the size of a full measure.
%  This adjustment makes it show progress more smoothly in the main window when
%  the pattern is playing.

   It would be nice to have a value that represents
   "indefinite", so that the loop or pattern would be more like a track,
   and not be repeatable.  We have part of that functionality:
   a feature from user Stazed allows the pattern to expand
   indefinitely while the user inputs MIDI from a controller.
   See \sectionref{subsec:seq66_pattern_editor_bottom} below.
   (Also nice would be a "one-shot"
   pattern, useful for live intro patterns, for example.)

   \itempar{Transpose Toggle}{pattern editor!transpose toggle}
   This setting allows the sequence to be transposed by
   the global transpose selection made in the song editor.  If transpose is
   enabled for that pattern, the button will be highlighted as per the current
   desktop theme.  Patterns for drums should, in general, not be transposable.

   \itempar{MIDI Out Device (Buss)}{pattern editor!midi out device}
   This setting specifies one of the 16 MIDI output busses provided by
   \textsl{Sequencer66}, or one of the MIDI devices set up in the computer.
   The settings look a lot like
   \figureref{fig:pattern_window_right_click_midi_bus}.

   \itempar{MIDI Out Port (Channel)}{pattern editor!midi out port}
   This settings select the MIDI output channel, or port.
   The possible values range from 1 to 16.
   If instruments are defined in the "user" configuration file
   to that device and channel, their names will be shown.

\subsection{Pattern Editor / Second Panel}
\label{subsec:seq66_pattern_editor_second}

   The second horizontal panel of the Pattern Editor provides a number
   of additional settings.

\begin{figure}[H]
   \centering 
%  \includegraphics[scale=0.55]{pattern/pattern-edit-second-panel-items.png}
%  \includegraphics[scale=0.55]{new/pattern-edit-second-panel.png}
   \includegraphics[scale=0.65]{roll.png}
   \caption{Pattern Editor, Second Panel Items}
   \label{fig:pattern_editor_main_panel_items}
\end{figure}

   There is a new button to the right
   of the \textbf{Tools} ("hammer" button).  This button toggles the "follow
   progress" feature; it is the active button in the diagram above.

% \begin{figure}[H]
%    \centering 
%    \includegraphics[scale=0.55]{pattern/pattern-edit-second-panel-items-2.png}
%    \caption{Pattern Editor, Second Panel Update}
%    \label{fig:pattern_editor_main_panel_update}
% \end{figure}

   \textsl{Sequencer66} includes a chord-generation option.

%  built in (the default), then the second panel is wider to make room for an
%  addition user-interface item, shown at the right of the figure above.

% \begin{figure}[H]
%    \centering 
%    \includegraphics[scale=0.75]{new/pattern_editor_second_panel-0_9_14.png}
%    \caption{Pattern Editor, Second Panel Items Emphasizing Chord Button}
%    \label{fig:pattern_editor_main_panel_chord_button}
% \end{figure}

   \begin{enumber}
      \item \textbf{Undo}
      \item \textbf{Redo}
      \item \textbf{Quantize Selection}
      \item \textbf{Tools}
      \item \textbf{Follow Progress} (new)
      \item \textbf{Grid Snap}
      \item \textbf{Note Length}
      \item \textbf{Zoom}
      \item \textbf{Key of Sequence}
      \item \textbf{Musical Scale}
      \item \textbf{Background Sequence}
      \item \textbf{Chord Generation}
   \end{enumber}

   \setcounter{ItemCounter}{0}      % Reset the ItemCounter for this list.

   \itempar{Undo}{pattern editor!undo}
   The Undo button rolls back any changes to the pattern from this session.
   It will roll back one change each time pressed.
   It is not certain what the undo limit (if any) is, however.
   \index{keys!ctrl-z}
   Pressing \texttt{Ctrl-Z} is the same as using the \textbf{Undo} button.

   \itempar{Redo}{pattern editor!redo}
   The Redo button will restore any undone changes to the pattern from this
   session.
   It will restore one change each time it is pressed.
   It is not certain what the redo limit is, however.
%  There is no "Redo" key in the pattern editor.

   \itempar{Quantize Selection}{pattern editor!quantize}
   This button quantizes the selected events as per
   the \textbf{Grid Snap} setting.

   \itempar{Tools}{pattern editor!tools}
   This button brings up a nested menu of tools for modifying selected
   events and notes.

\begin{figure}[H]
   \centering 
%  \includegraphics[scale=0.75]{pattern/tools-first-menu.png}
   \includegraphics[scale=0.65]{roll.png}
   \caption{Tools, Context Menu}
   \label{fig:pattern_editor_tools_first_menu}
\end{figure}

   \begin{enumber}
      \item \textbf{Select}
      \item \textbf{Modify Time}
      \item \textbf{Modify Pitch}
   \end{enumber}

%  $\bullet$
   \textbf{Select} provides two sets of selections for notes:
   \begin{itemize}
      \item \textbf{All Notes}, which selects all notes in the pattern;
         Note that \index{keys!ctrl-a} \texttt{Ctrl-A} will also select
         all of the events in the pattern editor.
      \item \textbf{Inverse Notes}, which inverts the selection of notes.
   \end{itemize}

   Other event-selection actions are provided using the mouse in the piano roll:

   \begin{itemize}
      \item \index{mouse!left-click}
         \textbf{Left Click}.
         Pressing the left button on a note or a event deselects all other
         notes or events, and selects the item clicked on.
      \item \index{mouse!ctrl-left-click}
         \textbf{Ctrl Left Click}.
         Pressing the \texttt{Ctrl} key and the left button on a note or an
         unselected event \textsl{adds} that event to the selection.
      \item \index{mouse!left-click-drag}
         \textbf{Left Click Drag}.
         Pressing the left mouse button and dragging also lets one
         select ("lasso") multiple events and notes.
      \item \index{mouse!ctrl-left-click-drag}
         \textbf{Ctrl Left Click Drag}.
         \begin{itemize}
            \item Pressing the \texttt{Ctrl} while left-click-dragging
               \textsl{on unselected events} lets one make additional
               selections of multiple events and notes.
            \item Pressing the \texttt{Ctrl} while left-click-dragging
               \textsl{on an already-selected event} lets one stretch or
               compress the lengths of multiple notes in the selection.
         \end{itemize}
   \end{itemize}

   There are many things that can be done with selected notes:

   \index{modify event-data}
   $\bullet$ \textbf{Modify Event Data} offers a way to
   alter the event data values in 
   the lower pane of the pattern editor, the "data pane".
   By left-dragging the mouse in the data pane across the value lines that are
   shown, the values are chopped or set to the height of the mouse pointer at
   each event.
   When notes are selected, and the
   mouse is used to change the values (heights) of the lines in the event-data
   area,
   \textsl{only the events that are selected} are changed.  The data-values of
   \textsl{unselected} events are left unchanged.
   A cool feature from \textsl{Seq24}.

   \index{modify time}
   $\bullet$ \textbf{Modify Time} offers two ways to tweak the timing of the
   selected note:
   \index{quantize}
   \textbf{Quantize Selected Notes}, which quantizes the selected
   notes, the same way as the \textbf{Quantize} ("\textbf{Q}") button;
   \index{tighten}
   \textbf{Tighten Selected Notes}, which is merely a less
   strict form of quantization.

   \index{modify pitch}
   $\bullet$ \textbf{Modify Pitch} has only one entry by default,
   \textbf{Transpose Selected} (not shown).
   Selecting the \textbf{Transpose Selected} entry
   brings up the following sub-menu:

\begin{figure}[H]
   \centering 
%  \includegraphics[scale=0.75]{pattern/tools-transpose-selected-menu.png}
   \includegraphics[scale=0.65]{roll.png}
   \caption{Tools, Transpose Selected Values}
   \label{fig:pattern_editor_tools_transpose_selected_menu}
\end{figure}

   $\bullet$ If the user has selected a
   \textbf{Musical Scale} setting other than \textbf{Off},
   then \textbf{Modify Pitch} has two entries:
   \textbf{Transpose Selected}, discussed above, plus
   another sub-menu,
   \textbf{Harmonic Transpose Selected}, which makes sure that all
   transpositions stay on the selected scale.

   Note that one can also modify the pitch of selected notes by the
   \index{mouse!left-click-drag}
   left-click-drag action, or by moving the selection using the
   \index{keys!down-arrow}
   \index{keys!up-arrow}
   up-arrow or down-arrow keys.

\begin{figure}[H]
   \centering 
%  \includegraphics[scale=0.75]{pattern/tools-transpose-and-harmonic-selected-menu.png}
   \includegraphics[scale=0.65]{roll.png}
   \caption{Tools, Two "Transpose" Menus}
   \label{fig:pattern_editor_tools_two_transpose_menus}
\end{figure}

   Remember that only the \textbf{Transpose Selected} entry is shown if the
   \textbf{Musical Scale} setting is \textbf{Off}.
   Selecting the \textbf{Harmonic Transpose Selected} entry brings up the
   following sub-menu:

\begin{figure}[H]
   \centering 
%  \includegraphics[scale=0.75]{pattern/tools-harmonic-transpose-selected-menu.png}
   \includegraphics[scale=0.65]{roll.png}
   \caption{Tools, Harmonic Transpose Selected Values}
   \label{fig:pattern_editor_tools_harmonic_transpose_menu}
\end{figure}

   Again, the harmonic-transpose option will not be available unless a scale
   has been selected.

   \itempar{Follow Progress}{pattern editor!tools}
   This button toggles whether or not the progress bar follows
   progress in long patterns.  Turning off this feature is useful when
   one wants to concentrate on the current measure without the paging to
   subsequent measures that occurs with the "follow progess" feature.

   \itempar{Grid Snap}{pattern editor!grid snap}
   Grid snap selects where the notes will be drawn.
   The following values are supported:
   1, 1/2, 1/4, 1/8, 1/16, 1/32, 1/64, and 1/128.
   Additional values are also supported:
   1/3, 1/6, 1/12/, 1/24, 1/48, 1/96, and 1/192.

   \itempar{Note Length}{pattern editor!note length}
   Note length determines the duration of the inserted notes.
   Like the \textbf{Grid Snap} values,
   the following values are supported:
   1, 1/2, 1/4, 1/8, 1/16, 1/32, 1/64, and 1/128.
   Additional values are also supported:
   1/3, 1/6, 1/12/, 1/24, 1/48, 1/96, and 1/192.

   \itempar{Zoom}{pattern editor!zoom}
   Horizontal zoom is the ratio between MIDI pixels and ticks, written as
   "pixels:ticks", where "ticks" is the "pulses" in "PPQN".
   For example, 1:4 = 4 ticks per pixel.
   Supported values are 1:1, 1:2, 1:4, 1:8, 1:16, and 1:32, along with
   more new values to support higher PPQN tunes: 1:64, 1:128, 1:256, and
   1:512.
   The default zoom is 2 for the standard PPQN value, 192, but it
   increases for higher PPQN values, so that the zoom looks sensible.
   As the right number (ticks) goes higher,
   the effect is to zoom out, and show more of the pattern.
   In fact, it might be better to list them as ticks (pulses) per pixel.
   Legacy stuff.

%  \begin{itemize}
%     \item 1 pulse per pixel
%     \item 2 pulses per pixel (default value)
%     \item 4 pulses per pixel
%     \item 8 pulses per pixel
%     \item 16 pulses per pixel
%     \item 32 pulses per pixel (same as Song Editor)
%     \item . . .
%     \item 512 pulses per pixel
%  \end{itemize}

   \index{zoom keys}
   \index{keys!0}
   \index{keys!z}
   \index{keys!shift-z}
   After a left-click in the piano roll, the
   \textbf{z}, \textbf{Z}, and \textbf{0}
   can be used to zoom the piano-roll view horizontally.
   The \textbf{z} key zooms out (smaller),
   the \textbf{Z} key zooms in (larger),
   and the \textbf{0} key resets the zoom to the default value.
   The horizontal zoom feature also affects the time-line
   (measures indicator) and the data area.
%  If, for some reason, the data area and piano-roll get out of sync, click on
%  the horizontal scroll bar to force the views to redraw properly.

   \index{song editor!zoom}
   (Note that the Song Editor, which now has zoom functionality,
%  (through the "z", "Z", and "0" keystrokes only),
   has a default resolution of 32 pulses per pixel, so, by default, it has
   16 times the resolution of the Pattern Editor.)

   \itempar{Key of Sequence}{pattern editor!key}
   Selects the desired musical key for the pattern.  The following keys are
   supported:  C, C\#, D, D\#, E, F, F\#, G, G\#, A, A\#, and B.
   Changing the \textbf{Key} will also shifts the marked note-rows
   for the \textbf{Musical Scale} setting.

   \index{save musical key}
   The musical key that a sequence/pattern is set to is
   saved in the MIDI file along with the rest of the data for the sequence.
   \textbf{However},
   it turns out that a change made to the key, scale, or background sequence in
   the pattern editor is saved in that editor, so that opening another sequence
   will apply the same settings to that sequence.  This is an optional feature,
   now more rigorously supported, as noted below.

   \index{global-sequence}
   If the global-sequence feature is enabled, and the user selects
   a different key, scale, or background sequence in the pattern editor, 
   then \textsl{all} patterns share the selected key, scale, or background
   sequence.  Furthermore, these settings are saved in the "proprietary"
   section of the MIDI file, where they are available for all patterns.

   If the global-sequence feature is \textsl{not} enabled, and the user selects
   a different key, scale, or background sequence in the pattern editor, 
   then only that pattern will use the selected key, scale, or background.
   The key, scale, or background sequence change will be saved in the MIDI file
   only for that pattern, as a SeqSpec meta event.
   The global-sequence feature setting can be made in the "user" configuration
   file.

   \itempar{Musical Scale}{pattern editor!scale}
   Selects the desired scale for the pattern.
   When a scale is selected, the following features are supported:

   \begin{itemize}
      \item The notes that are \textsl{not}
         in the scale are shown as grey in the piano
         roll, to make it easier to key all notes in-scale.
      \item For harmonic transposition, the notes are shifted
         so that they remain in the selected scale.
      \item The exact notes that are considered "in-scale" depend also on the 
         exact value of the \textbf{Key of Sequence} setting.
   \end{itemize}

   \index{musical scales}
   The following musical scales are supported:

   \begin{itemize}
      \item \textbf{Off (Chromatic)}
      \item \textbf{Major (Ionian)}
      \item \textbf{Minor (Aeolian)}
      \item \textbf{Harmonic Minor}
      \item \textbf{Melodic Minor}
      \item \textbf{Whole Tone}
      \item \textbf{Blues}
      \item \textbf{Major Pentatonic}
      \item \textbf{Minor Pentatonic}
   \end{itemize}

   Please let us know of any mistakes in the new scales.
   Please note that the \textbf{Melodic Minor} scale is supposed to
   descend in the same was as the natural \textbf{Minor} scale, but
   there is no way to support that trick in \textsl{Sequencer66}.

% \begin{figure}[H]
%    \centering 
% %  \includegraphics[scale=0.75]{pattern/scales-menu.png}
%    \includegraphics[scale=1.0]{new/scales-menu.png}
%    \caption{Scales Currently Supported in Sequencer66}
%    \label{fig:pattern_editor_available_scales}
% \end{figure}

   One can select which \textbf{Musical Scale} and
   \textbf{Key} the piece is in,
   and \textsl{Sequencer66} will grey those keys on the piano-roll that
   are \textsl{not} in the selected scale for the selected key.
   This is purely visual; a user can still add off-key notes.
   This effect is shown for the C Major scale in the following figure:

\begin{figure}[H]
   \centering 
%  \includegraphics[scale=0.65]{pattern/major-scale-masking.png}
   \includegraphics[scale=0.65]{roll.png}
   \caption{C Major Scale Masking}
   \label{fig:pattern_editor_major_scale_masking}
\end{figure}

   This feature makes it easier to stay in key while playing and
   recording.  Note that the scale will shift when a different
   \textbf{Key} is selected.

   \index{save musical scale}
   The scale that a pattern is set to is
   now saved in the MIDI file along with the rest of the data for the pattern.
   \textbf{However},
   a change made to the key, scale, or background pattern in
   the pattern editor is saved in the editor, so that opening another pattern
   will apply the same settings to that pattern.  This is a feature.
   The feature had some quirks, which are fixed, and it is now optional.
   The user has the option of applying the key/scale/background-sequence
   either globally (all patterns) or locally, per-pattern, with each pattern
   holding its key, scale, and background-sequence settings in
   SeqSpec meta events.

   \itempar{Background Sequence}{pattern editor!background sequence}
   One can select another pattern to draw on the background to help with
   writing corresponding parts.
   The button brings up a small menu with values of \textbf{Off} and
   \textbf{[0]}.  The 0 is a set number. Sets are numbered from 0 to 31.
   Additional set numbers appear in the menu for each set that has data in it.
   Under the \textbf{0}
   entry, a menu like the following appears:

\begin{figure}[H]
   \centering 
%  \includegraphics[scale=0.75]{pattern/background-sequence-menu.png}
   \includegraphics[scale=0.65]{roll.png}
   \caption{Sample Background Sequence Values}
   \label{fig:pattern_editor_background_sequence_menu}
\end{figure}

   Once the desired pattern is selected from that list, it appears as dark cyan
   note bars, along with the normal notes that are part of the pattern.  (Also
   note the orange selected notes and events in the following figure.)

\begin{figure}[H]
   \centering 
%  \includegraphics[scale=0.75]{pattern/background-sequence-notes.png}
   \includegraphics[scale=0.65]{roll.png}
   \caption{Background Sequence Notes}
   \label{fig:pattern_editor_background_sequence_notes}
\end{figure}

   The dark cyan notes shown represent the rhythm pattern that was selected as
   the background pattern.

   \index{save background sequence}
   The background sequence that shows is saved in the MIDI file
   along with the rest of the data for the sequence/pattern.
   A change made to the key, scale, or background sequence in
   the pattern editor is saved in the editor, so that opening another sequence
   will apply the same settings to that sequence.  This is an optional feature,
   now supported, as noted earlier.

   \itempar{Chord Generation}{pattern editor!chord generation}
   The ability to insert chords with one click has been added.
   This feature comes from user "stazed"
   and his \textsl{Seq32} project (\cite{seq32}).

\begin{figure}[H]
   \centering 
%  \includegraphics[scale=1.0]{new/chords_menu-0_9_14.png}
   \includegraphics[scale=0.65]{roll.png}
   \caption{Chord Generation Menu}
   \label{fig:pattern_editor_chords_menu}
\end{figure}

   The figure shows the menu broken into two pieces.
   Once a value other than \textbf{Off} is selected, a left-click
   in drawing mode will add multiple notes representing the chord
   created, with the clicked note value as the base of the chord.

\subsection{Pattern Editor / Piano Roll}
\label{subsec:seq66_pattern_editor_piano_roll}

   The piano roll is the center of the pattern/loop/track/sequence editor.
   It is accompanied by a
   \index{event area}
   thin "event bar" ("event area", "event strip") just below it,
   \index{data area}
   and a taller "data bar" or "data area" just below that.  While the pattern
   editor is very similar to note editors in other sequencers, it is a bit
   different in feel.  A good mouse with 3 or more buttons is practically a
   necessity for editing (though \textsl{Sequencer66} is usable
   with some crummy trackpads common on modern laptops, and
   with keystrokes.) We tend to like the Logitech Marble Mouse, an ambidextrous
   USB trackball.  It has four buttons, and we use the
   \texttt{contrib/scripts/marblemouse} script to set up the left small button
   as a middle button.  The script merely makes the following call:

   \begin{verbatim}
      xmodmap -e "pointer = 1 8 3 4 9 6 7 2 5 10 11"
   \end{verbatim}

%  Editing is much easier after making that setting.   Of course, keystrokes
%  and additional mouse configuration have been added to make editing easier
%  even without a good mouse.

   One can page vertically in the piano roll using the
   \index{keys!page-up} Page Up and 
   \index{keys!page-down} Page Down keys.
   One can go to the top using the 
   \index{keys!home} Home key, and
   to the bottom using the
   \index{keys!end} End key.
   One can page left and right in the piano roll using the
   \index{keys!shift-page-up} Shift Page Up and 
   \index{keys!shift-page-down} Shift Page Down keys.
   One can go to the leftmost position using the 
   \index{keys!shift-home} Home key,
   and to the rightmost position using the
   \index{keys!shift-end} End key,
   There are more keystrokes to be described later.

   \index{step}
   \index{note step}
   Do not forget the note-step option.
   If one paints notes with the mouse,
%  the note is previewed, and
   the note position advances with each click.
   If one paints notes via an external MIDI keyboard, the notes are painted and
   advanced, but not previewed.
   To preview them, click the
   \textbf{pass MIDI in to output} button to activate so that they will be
   passed to the sound generator or software synthesizer.

\subsubsection{Pattern Editor / Piano Roll Items}
\label{subsubsec:seq66_pattern_editor_piano_roll_items}

   The center of the pattern editor consists of a time panel at the top,
   a virtual keyboard at the left, a note grid, a vertical scrollbar, an event
   panel, and a data panel at the bottom.

\begin{figure}[H]
   \centering 
%  \includegraphics[scale=0.65]{pattern/pattern-edit-piano-roll-items.png}
   \includegraphics[scale=0.65]{roll.png}
   \caption{Pattern Editor, Piano Roll Items}
   \label{fig:pattern_editor_piano_roll_items}
\end{figure}

   \begin{enumber}
      \item \textbf{Beat}
      \item \textbf{Measure}
      \item \textbf{Virtual Piano Keyboard}
      \item \textbf{Notes}
      \item \textbf{Events}
      \item \textbf{Event Values}
   \end{enumber}

   \setcounter{ItemCounter}{0}      % Reset the ItemCounter for this list.

   \itempar{Beat}{piano roll!beat}
   The light vertical lines represent the beats defined by the configuration
   for the pattern.  The even lighter lines between the beats are useful for
   snapping notes.

   \itempar{Measure}{piano roll!measure}
   The heavy vertical lines represent the measures defined by the
   configuration for the pattern.
   \index{pattern!end marker}
   Also note that the end of the pattern
   occurs at the end of a measure, and is marked by a blocky \textbf{END}
   marker.

   \itempar{Virtual Piano Keyboard}{piano roll!virtual piano keyboard}
   The virtual keyboard is a fairly powerful interface.  It shows,
   by shadowing, which note on the keyboard will be drawn. It can be
   played with a mouse, using left-clicks, to preview a short motif.
   It can show marks to indicate off-scale notes, to make them easy to
   avoid.  Every octave, a note letter and octave number are shown, as in
   "C4".  If there is a difference scale in force, then the letter changes to
   match, as in "F\#5".

   \index{virtual keyboard!right-click}
   A right-click anywhere in the virtual keyboard area toggles the display
   between the octave note letters and the MIDI note numbers (only every other
   one is displayed due to space, to avoid cramped numbering).
   The following figure shows both views, superimposed for comparison.

\begin{figure}[H]
   \centering 
   \includegraphics[scale=0.75]{pattern/pattern-edit-window-key-numbers.png}
   \caption{Pattern Editor, Virtual Keyboard Number and Note Views}
   \label{fig:pattern_editor_key_numbers}
\end{figure}

   \itempar{Notes}{piano roll!notes}
   Musical notes are indicated by thick horizontal bars with white
   centers.  Each bar provides
   a visual representation of the pitch of a note and the length of a note.

   \itempar{Events}{piano roll!events}
   \index{events strip}
   Also known as the "events pane" or "events panel".
   The narrow (a few pixels high) events strip shows discrete events,
   such as Note On and Note Off
%  Shows their velocities, or various Controller
%  items and their values.
   We recommend not editing or selecting events
   in that pane (feel free to disobey), but it is a good way to add events.
   Either
   \index{mouse!left-click}
   left-click (to add one event),
   \index{mouse!left-click-drag}
   or left-click-drag horizontally (to add a
   series of events at the current note resolution.)  One can also
   left-click in that section,
   \index{keys!p}
   then hit the \texttt{p} key to go into "paint" mode,
   \index{keys!x}
   and hit the \texttt{x} key to escape that mode.

   \itempar{Event Values}{piano roll!event values}
   Also known as the "data pane"
   \index{data pane}
   or "data panel".
   \index{data panel}

   The events values for the currently selected category of events are shown
   in this window as vertical lines of a height proportional to the value.
   These values can be easily modified by
   \index{mouse!left-click-drag}
   left-click-dragging the
   mouse past each line, to chop it off at the given value.  Easier to try
   it than explain it.
   \index{mouse!right-click-drag}
   Right-click-drag also works the same.

\subsubsection{Pattern Editor / Event Editing}
\label{subsubsec:seq66_pattern_editor_event_editing}

   When we say "editing" in the context of the piano roll, in part we mean that
   we will "draw"
   \index{draw mode}
   or "paint"
   \index{paint mode} notes.
   Drawing, modifying, copying, and deleting
   notes is actually very elegant in \textsl{Sequencer66} and in
   \textsl{Seq24}.

%  Note editing is a bit different with \textsl{Sequencer66}, since it
%  requires two mouse buttons in many cases.  There are some new
%  laptop touchpads that really have only one mouse button, and
%  use positioning to determine if the click is a left-click or a right-click.
%  Two-fingered actions are devoted to scrolling, so that there is no way
%  to generate a Linux middle-click.
%  One solution is to use a four-button USB trackball
%  configued with an easy middle-button setup.
%  It's easier than a touchpad, anyway.
%  So we've coded up a couple of solutions to this middle-click problem.

   Note that, if only a middle-button is needed,
   \texttt{Ctrl-left} will simulate that button.

%  Also note the "Mod4 mode" for the right-click action, and the
%  \texttt{p} and \texttt{x} keys for getting in and out of "paint mode",
%  discussed elsewhere.

   \index{keys!mod4}
   There is a feature to allow the Mod4
   \index{pattern editor!mod4}
   (the Super or Windows) key to keep the
   right-click in force even after it is released.  See \ref{new_mod4_mode}.
   Pressing \texttt{Mod4} before releasing the right-click that allows
   note-adding, keeps note-adding in force after the right-click is released.
   Now notes can be added at will with the left mouse button.
   Right-click again to leave the note-adding mode.

   \textbf{New:}
   \index{new!paint mode}
   Another way to turn on the paint mode has been added.
   To turn on the paint mode, first make sure that the piano roll has the
   keyboard focus by left-clicking in it, then press the
   \index{keys!p}
   \texttt{p} key while in the pattern editor.
   This is just like pressing the right mouse button, but the draw/paint mode
   sticks (as if the \texttt{Mod4} mode were in force).
   To get out of paint mode, press the
   \index{keys!x}
   \texttt{x} key while in the pattern editor, to "x-scape" (get it? get it?)
   from the paint mode.
   These keys also work while the sequence is playing.

   The \texttt{p} and \texttt{x} keys also works in the small event strip just
   above the white data area.
   The Mod4-right-click feature does not
   yet work in that area of the user-interface, but the \texttt{p} key does.

\paragraph{Editing Note Events}
\label{paragraph:seq66_pattern_editor_note_events}

   The piano roll provides for a quite sophisticated set of note-editing
   actions.  Not only is there a native mouse-interaction mode, but there is a
   "fruity" mouse-interaction mode that works more like the application
   \textsl{Fruity Loops}, its follow-on \textsl{FL Studio},
   and its Linux look-alike, \textsl{LMMS}.

   This option is currently \textsl{not available} in the
   \textbf{Qt 5} version of \textsl{Sequencer66}.
   We really want to refactor the code to be able to 
   completely reconfigure the mouse interface according to any user's
   preferences, but that is currently a low priority.

   \setcounter{ItemCounter}{0}      % Reset the ItemCounter for this list.

   \itempar{Fruity Mode}{pattern editor!fruity mode}
   \index{todo!fruity mode}
   At some point, we will add a section detailing the usage of the "fruity"
   mode of mouse-interaction.
   For now, \sectionref{paragraph:seq66_menu_file_options_mouse}, such as it
   is, will have to do.
   Study the following paragraphs carefully, ideally while
   trying them out in \textsl{Sequencer66}.

   \itempar{Enter Draw Mode}{pattern editor!draw mode}
   \index{pattern editor!right hold}
	In the note (grid/roll) panel, \textbf{holding}
	down the \textbf{right} mouse button will change the cursor
	to a pencil and put the editor into "draw" mode, 
   \index{draw mode}
   also known as "note-adding" or "paint" mode.
   \index{paint mode}
   To exit the draw mode, release the right mouse button, and the cursor will
   turn back into an arrow.
   Another way to enter paint mode is to make sure the piano roll has focus
   (left click in it), and then press the \texttt{p} key.
   To exit the draw mode, press the \texttt{x} key.

   \itempar{Add Notes}{pattern editor!add notes}
   \index{pattern editor!right hold left click}
   If using the mouse, while \textbf{holding} the \textbf{right} mouse button,
   click the \textbf{left} mouse button to \textbf{insert} new notes.
   Many people find this combination strange at first, but once one gets used
   to it, it becomes a very fast method of note manipulation.
   Holding the \texttt{Mod4} key while releasing the right button keeps the
   mouse in draw mode.
%  Another new option is to press the \texttt{p} to enter draw
%  mode, and stay in it until \texttt{Shift-p} is pressed.
   \index{notes!inserting}
   Note that this click will add a single note, and the length of the note will
   be that specified in the note-length setting (e.g. "1/16").

   \index{pattern editor!right hold left drag}
   \index{notes!duration}
   To increase the number of notes, keep dragging the mouse (with
   both buttons held).  It can be dragged rightward, leftward, upward, and
   downward.
   \index{notes!auto}
   Dragging left or right adds new notes, while dragging upward or
   downward moves the current note to a different pitch.
   \index{auto-note}
   This is the "auto-note" feature.
   Please note that the auto-note feature does \textsl{not} work with
   chord-generation.
   The draw mode has the following features:

   \begin{itemize}
      \item Notes are continually added as the mouse is dragged ("auto-notes").
      \item Notes cannot be added past the "END" marker of the pattern, which
         marks the \textbf{Sequence Length in bars} setting.
      \item As the mouse is dragged while the left button is held in draw mode,
         notes are either added, or, if already present at that note-on time,
         are moved up and down.
      \item If the draw mode is exited, and entered again, then the original
         notes will not be altered.  Instead, new ones will be added.
      \item Notes can be added while the pattern is playing, and will be heard
         the next time the progress bar passes over them.
   \end{itemize}

   Thus, one can, with some care, draw a nice chorded sequence.
   Adjustments to it can be made afterward.

   \itempar{Select Notes}{pattern editor!select notes}
   Adjustments can be made to one or more notes by selecting one or more notes,
   and then applying one or more special
   \index{selection action} "selection actions" to the selection.

   \index{pattern editor!left click}
   \index{pattern editor!select note}
   \index{selection!single note}
   To select a single note, simply \textbf{left click} on it.
   The selected note will turn orange.

   \index{pattern editor!left click drag}
   \index{selection!multiple notes}
   To select multiple notes, perform a \textbf{left click drag}
   to form a selection box that intersects (even partly) the desired notes.
   Once the mouse is released, all of the desired notes should be orange.

   \index{pattern editor!ctrl left click drag}
   \index{selection!add multiple notes}
   To add more notes to a selection of notes, move to an unselected note
   and perform a \textbf{ctrl left click drag}
   to form a selection box that intersects (even partly) the desired notes.
   Once the mouse is released, all of the desired notes should be orange.
   Be careful!  If you ctrl-left-click-drag on an already-selected note,
   the drag will change the length of all of the notes in the selection.

   \index{keys!ctrl-a}
   \index{selection!all}
   Pressing the \texttt{Ctrl-A} key will select all of the events in the
   pattern editor.

   The \textbf{Tools} button described in
   \sectionref{subsec:seq66_pattern_editor_second} can also be used to
   modify selections.

   Once one or more notes are selected, they can be modified in time,
   pitch, or length.

   \itempar{Deselect Notes}{pattern editor!deselect notes}
   \index{selection!deselect}
   To deselect the notes, click somewhere else in the piano roll, and the notes
   should change back to white.
   There is no way to deselect a single note, with, say, a shift-click
   or ctrl-click action.

%  \index{notes!lengthen}
%  For example, if one wants a long single note, first draw the
%  short note, and then use one of the operations described in the next
%  paragraph to change the length of the note.  These operations also cause
%  the note to be selected.

   \itempar{Move Notes in Pitch}{pattern editor!move notes in pitch}
   To move notes in pitch, once selected, grab one of the notes in the
   selection and drag it upward or downward.
   \index{down arrow}
   \index{up arrow}
   Also, since a selection is in force, the Up and Down arrow keys can also
   be used to change the pitch of every note in the selection.
   The smallest unit of pitch change is one MIDI note value.

   \textbf{Warning:}
   \index{warning!down arrow}
   \index{warning!up arrow}
   \index{warning!note loss}
   If one moves the selection too low or too high in pitch, whether with the
   mouse or the arrow keys, any notes that go below the lowest MIDI pitch or
   above the highest MIDI pitch \textbf{will be lost}!
   If done using the mouse, the undo feature (Ctrl-z) will work.
   If done using the arrow keys, the undo feature does not work!
   Be careful, especially if you have a fast keyboard repeat rate!

   \itempar{Move Notes in Time}{pattern editor!move notes in time}
   To move notes in time, once selected, grab one of the notes in the
   selection and drag it leftward or rightward.
   \textbf{New:}
   \index{new!left arrow}
   \index{new!right arrow}
   Also, since a selection is in force, the Left and Right arrow keys can also
   be used to change the time of every note in the selection.
   The smallest unit of time change is the \textbf{Grid snap} value,
   which might be a 16th note, for example.

   Note that there is no possibility of note loss with a change in time.  When
   a note disappears at one end of the pattern boundary, it wraps around to the
   other end.  Cool.

%  (There is another feature of the arrow keys when no selection is made, that
%  supposedly moves an "origin" for playback, but we haven't been able to
%  figure out exactly if it does anything.)

   \itempar{Change Note Length}{pattern editor!change note length}
   \index{pattern editor!ctrl left click drag}
   \index{pattern editor!middle click drag}
   Pressing the \textbf{middle} mouse button \textbf{\textsl{or}}
   pressing the \textbf{ctrl left} mouse button in tandem, while the pointer is
   hovered over a selected note, will let one
   \index{notes!duration change}
   change the length of a selected note.  If more than one note is selected,
   then the length of all selected notes is changed.

   \index{pattern editor!event stretch}
   \index{pattern editor!shift middle click drag}
   \index{event!stretch}
   \index{stretch events}
   Once a selection of notes is made, one can use the
   shift-middle-click-drag or ctrl-left-click-drag
   sequence over the selected notes to
   draw a box beyond the extent of the notes.  When the mouse is released,
   each of the events is moved and lengthed to be proportionally longer to
   fit exactly within the box one drew.
   This feature is called \textsl{event stretch}.
   \index{pattern editor!event compression}
   \index{event!compression}
   \index{compress events}
   If the box that was drawn was shorter than the original extent of the
   notes, then the notes move and shrink proportionally to occupy the
   smaller box.
   This feature is called \textsl{event compression}.
   
   \index{warning!wrap-around notes}
   \textbf{Warning}:  Reducing or increasing the length of a note selection
   by too much causes the note or notes to "wrap-around" to the end
   of the pattern boundary and grow more from the beginning of the sequence. 
   It is not clear if this new note has an ending time that is less than its
   beginning time.  If it happens, one probably ought to undo it.

%  Note that notes can be shortened below the default note length by event
%  compression.  Note that there is currently no way to change the length of
%  the note using a keystroke.

   \itempar{Copy/Paste}{pattern editor!copy/paste}
   Copying, cutting, and pasting is supported by selecting a number of events
   or notes, and using the
   \index{pattern editor!cut}
   \index{keys!ctrl-x} Cut (\texttt{Ctrl-X}), 
   \index{pattern editor!copy}
   \index{keys!ctrl-c} Copy (\texttt{Ctrl-C}),
   \index{pattern editor!paste}
   \index{keys!ctrl-v} Paste (\texttt{Ctrl-V}), and
   "drop" (\texttt{Enter})
   \index{pattern editor!drop}
   \index{keys!enter}
   keys.
   When the notes are selected,
   \index{pattern editor!delete}
   \index{keys!del}
   \index{keys!backspace}
   one can delete them with the \texttt{Delete} or \texttt{Backspace} key.
   If the events are \textsl{cut}, using the \texttt{Ctrl-X} key, then
   they can be pasted, using the \texttt{Ctrl-V} key.  However,
   once \texttt{Ctrl-V} is struck, then one must \textsl{move} the mouse
   pointer to see where to paste the events, or move it with the arrow keys.
   An orange box representing the
   selected-and-copied notes appears, and the user should move the box (note)
   to the desired location and then left-click.

%  \textsl{
%  (Warning:  We've had temporary issues where the selection box flickers, and
%  this seems to be due to updates in the graphics library used by
%  \textsl{Sequencer66}.  This issue might depend on the Linux distro one
%  uses.)
%  }

   One can move the orange box using the arrow keys, to the
   desired location, and then hit the
   \index{keys!enter} \texttt{Enter} key to
   drop the notes at that location.

   Selected notes that are cut or copied can then be
   pasted into \textsl{other} pattern editor dialogs; that is, they can be
   pasted into other sequences.

\begin{figure}[H]
   \centering 
   \includegraphics[scale=0.65]{pattern/selection-paste-box.png}
   \caption{Piano Roll, Paste-Box for Cut Notes}
   \label{fig:pattern_editor_selection_paste_box}
\end{figure}
   
%  Note that the selection box is now orange, not black.
   Move this box to where pasting is
   desired, and left-click.  The moved notes appear, still selected,
   and they can then be moved further, if desired, by using the arrow keys, or
   cut and move them again.

   For the appearance of selected events (orange), see
   \figureref{fig:pattern_editor_selected_events}.

\begin{figure}[H]
   \centering 
%  \includegraphics[scale=0.75]{pattern/pattern-edit-selected-events-0-9-10.png}
   \includegraphics[scale=0.65]{new/pattern-edit-selected-events.png}
   \caption{Piano Roll, Selected Notes and Events}
   \label{fig:pattern_editor_selected_events}
\end{figure}

   \index{selected data coloring}
   The selection, shown in
   \figureref{fig:pattern_editor_selected_events},
   illustrates the style of event selection, which colors the data
   bars as well as the event and note bars.  This selection was made by first
   selecting one set of events by the
   \index{mouse!left-click-drag}
   left-click-drag action, then
   \index{mouse!ctrl-left-click-drag}
   selecting more events by holding the \texttt{Ctrl}
   key for the next left-click drag action.
   The second selection left out some events, which are thus still
   shown as black bars in the data area.

%  As an aside, note that the event strip is gray.  \textsl{Sequencer66}
%  now shows this strip as gray when the selected event (the \textbf{Event}
%  button) is Note On, Note Off, or Aftertouch.  This color is purely a
%  reminder that moving these events can really screw up the notes, for example
%  by moving Note Off to before the Note On event.
   
\paragraph{Event and Data Panels / Editing Other Events}
\label{paragraph:seq66_pattern_editor_other_events}

%  Left-click or right-click on events in the event strip (directly under
%  the piano roll grid) will allow one to add/select/move 
%  MIDI events (including note on/off messages) somewhat like the 
%  piano grid.

   \index{event strip}
   \textsl{Note On} and \textsl{Note Off} events (and other events) can appear
   as small squares in the event strip, along with a black vertical bar with a
   height proportional to the velocity of the note event, plus a numeric
   representation of that value.
   Note events do not need to be inserted in the event strip.
   \index{warning!unterminated notes}
   \textsl{They can be inserted there, but they end up as short
   events of the lowest possible note, 0 or C1, and they don't have a Note
   Off event.  Don't do that!})

   \index{events!insert}
   Other event types can be inserted via the event strip.  To do that, first
   select the kind of event to insert using the \textbf{Event} button in the
   bottom panel.  The place the mouse cursor in the event strip.
   Right-click to make the drawing pencil appear at the exact spot where the
   event must go.  While holding the right button, click the left button.
   A small square for the event should appear.

   Should one want more of the same event, continue to hold both buttons and
   drag the mouse.  One event should appear at each beat position (e.g. at
   each 16th note position) that is crossed.

   To move the event(s) to a different spot, select it or them via the left
   button.  Then drag it or them to where one wants them.
   \index{todo!high precision events}
   it is currently not possible to move them to positions smaller than the
   beat size.  The work-around is to temporarily reduce the beat size,
   but this requires caution.

   Once the event positions are set, the next step is to modify the
   data values of the events.
   \index{event data}
	The event value (data) editor (directly under the event strip) is used 
	to change note velocities, channel pressure, control codes,
	patch select, etc.
   \index{event data editor!draw}
   \index{event data editor!left click}
   \index{event data editor!right click}
   \index{event data editor!middle click}
   Just left-click+drag the mouse across the window to draw a line.  The
   values will match that line.  
   middle-click+drag and right-click+drag also
   draw the value line.

   \textbf{Bug:}
   \index{bugs!event editing can fail}
   Sometimes the editing of event values in the event data section will not work.
   The workaround is to do a \texttt{Ctrl-A}, and the click in the roll
   to deselect the selection; that makes the event value editing work again.
   
   \index{event data editor!mouse wheel}
   Any events that are selected in the piano roll or event strip can have
   their values modified with the mouse wheel.

\paragraph{Editing Note Events the "Fruity Way"}
\label{paragraph:seq66_pattern_editor_note_events_fruity}

   This mode is a lot different, and we have yet to do the exhaustive testing
   needed to understand how this mode works.  Input from actual users of this
   mode would be welcome.
   For now, see \sectionref{paragraph:seq66_menu_file_options_mouse}.

\subsection{Pattern Editor / Bottom Panel}
\label{subsec:seq66_pattern_editor_bottom}

   The bottom horizontal panel of the Pattern Editor provides for
   selecting events for viewing and editing, MIDI playback,
   pass-through, and recording.

\begin{figure}[H]
   \centering 
%  \includegraphics[scale=0.75]{pattern/pattern-edit-bottom-panel-items.png}
   \includegraphics[scale=0.50]{new/pattern-edit-bottom-panel.png}
   \caption{Pattern Editor, Bottom Panel Items}
   \label{fig:pattern_editor_bottom_panel_items}
\end{figure}

   Missing from this diagram is the "Existing Event Selector" which zeroes
   in only on events already present in the pattern.

%  Until we can get a completely labelled screenshot, here is the
%  latest bottom panel, with the new \textbf{LFO}
%  and \textbf{Recording Type} buttons, and the popup menu for the latter.
%
%\begin{figure}[H]
%   \centering 
%   \includegraphics[scale=0.75]{new/lfo_and_rectype_buttons.png}
%   \caption{Pattern Editor, Additional Bottom Panel Items}
%   \label{fig:pattern_editor_added_bottom_panel_items}
%\end{figure}

   \begin{enumber}
      \item \textbf{Event Selector}
      \item \textbf{Existing Event Selector}
      \item \textbf{Event Selection}
      \item \textbf{LFO}
%     \item \textbf{Time Scroll}
      \item \textbf{Data To MIDI Buss}
      \item \textbf{MIDI Data Pass-Through}
      \item \textbf{Record MIDI Data}
      \item \textbf{Quantized Record}
      \item \textbf{Recording Type} (Merge, Replace, Expand)
      \item \textbf{Select Recording Volume}
   \end{enumber}

   \setcounter{ItemCounter}{0}      % Reset the ItemCounter for this list.

   \itempar{Event Selector}{pattern editor!event selector}
   This button brings up the following context menu, so that the user can
   select the category of events to view and edit.

\begin{figure}[H]
   \centering 
%  \includegraphics[scale=0.75]{pattern/event-context-menu.png}
   \includegraphics[scale=0.65]{roll.png}
   \caption{Pattern Editor, Event Button Context Menu}
   \label{fig:pattern_editor_bottom_event_context_menu}
\end{figure}

   Note the squares.  Some of filled (black), others are empty.  The filled
   squares indicate that the sequence does indeed have some events of that
   type.  Otherwise, there are no such events in the sequence.
   Useful in deciding if it is worth selecting the event.

   The sub-menus of this context menu show 128 controller messages,
   so we won't try to show all of them here.  They also use the squares to
   indicate if there are any events of the type shown in the menu.
   These sub-menus can be modified by editing the file
   
   \begin{verbatim}
      $HOME/.config/sequencer66/sequencer66.usr
   \end{verbatim}

   to make it match one's instrument.  See \sectionref{sec:seq66_usr_file}.

   \itempar{Existing Event Selector}{pattern editor!existing events}
   Not shown in the figure above is a small button that will contain either
   an empty square (no editable events) or a black square (at least one
   editable event).  Unlike the event-selector described above, this one
   shows only the actual events existing in the track, for quicker selection.
%  The event-selector above is more useful for creating new events.

   \itempar{Event Selection}{pattern editor!event selection}
   Shows the selection event, with its number shown in hexadecimal notation,
   and the name of the event shown.

   \itempar{LFO}{pattern editor!LFO}
%  If \textsl{Sequencer66} is built with the \texttt{--enable-lfo} option, then
   A low-frequency oscillator allows data events
   can be modulated by some rudimentary wave functions.
%  This feature is now enabled by default.
   By clicking on the \textbf{LFO} button or using the \texttt{Ctrl-L} key,
   the following window appears, shown as the set of 5 vertical sliders:

\begin{figure}[H]
   \centering 
%  \includegraphics[scale=0.65]{new/pattern_editor_with_LFO.png}
   \includegraphics[scale=0.65]{roll.png}
   \caption{Pattern Editor, LFO Support}
   \label{fig:pattern_editor_bottom_lfo_support}
\end{figure}

%  Note the controls in this window:

   \begin{enumber}
      \item \textbf{Value}:
         Provides a kind of DC offset for the data value. Starts at 64, and
         ranges from 1 to 127.
      \item \textbf{Range}:
         Controls the depth of modulation. Starts at 64, and ranges from 1 to
         127.
      \item \textbf{Speed}:
         Indicates the number of periods per pattern (divided by beat width,
         normally 4).  For long patterns, this parameter needs to be set high,
         to even show an effect.  It is also subject to an 'anti-aliasing'
         effect in some parts of the range, especially for short patterns.
         Try it!  For short patterns, try a value of 1 at first.  For a pattern
         of one measure in length, this will create four periods of the wave.
      \item \textbf{Phase}:
         Provides the phase shift within a period of the LFO wave.
         A value of 1 is a phase shift of 360 degrees (or maybe it is one
         radian?).
      \item \textbf{Wave Type}:
         Selects the kind of wave to use for the LFO:
         \begin{enumber}
            \item \textbf{Sine wave}.
            \item \textbf{Ramp (up) sawtooth}.
            \item \textbf{Decay (down) sawtooth}.
            \item \textbf{Triangle wave}.
         \end{enumber}
   \end{enumber}

   We may have more to explain about this dialog at some point.  For now,
   try it out on the file \texttt{one-measure.midi}, and be sure to hover over
   each control to see the tooltips.
   Note that it works best with short patterns.

   \itempar{Time Scroll}{pattern editor!time scroll}
   Allows one to pan through the whole pattern, if it is too long to fit in
   the window horizontally.

   \itempar{Data To MIDI Buss}{pattern editor!data to midi buss}
   This button causes the pattern to be output to the
   selected MIDI output buss,
   which will normally be connected to a software or hardware
   synthesizer, to be heard.
   Generally, this control should always be activated.

   \itempar{MIDI Data Pass-Through}{pattern editor!midi data pass-through}
   This button routes incoming MIDI data through
   \textsl{Sequencer66}, which then writes it to the MIDI output buss.

   \itempar{Record MIDI Data}{pattern editor!record midi data}
   This button routes incoming MIDI data into
   \textsl{Sequencer66}, which then saves the data to its buffer, and also
   displays the new information (notes) in the piano roll view.

   \itempar{Quantized Record}{pattern editor!quantized record}
   This button will causes MIDI data to be recorded, but be
   quantized on the fly before recording it.
   The quantization is to the current snap value.

   \itempar{Recording Type}{pattern editor!recording type}
   In \textsl{Seq24}, the pattern recording worked by merging new notes played
   as the pattern to be recorded was looped.  This method allows a loop to be
   built up bit-by-bit.  \textsl{Sequencer66} adds two more methods from
   Stazed's \textsl{Seq32} project.  The three methods are:

   \begin{enumber}
      \item \textbf{Merge}. \index{merge}
         \index{recording type!merge}
         This is the "legacy" style of recording loops, where notes can
         accumulate.
      \item \textbf{Replace.}
      \index{replace}
      \index{recording type!replace}
         When the loop starts over, and a note is pressed,
         then the existing notes in that loop are erased,
         and the new note is added.
         This provides a good way of correcting major mistakes, live.
         It will not work if adding notes while not recording.
         This mode can cause incomplete notes if one
         holds the note and releases it in the next iteration, leaving a
         partially-drawn note behind.  The workaround is to try again.
      \item \textbf{Expand}.
         \index{expand}
         \index{recording type!expand}
         Once the end of the loop is near, whether or
         not any notes are being input, another measure is added to the length of
         the loop.  This continues indefinitely, whether or not any notes are
         being played/recorded.
   \end{enumber}

   \itempar{Vol}{pattern editor!vol}
   This button allows setting the volume of the recording.
   The velocity of the notes will be set to the selected value upon recording.
   If the \textbf{Free} item is selected, then the incoming note velocity is
   preserved.

\begin{figure}[H]
   \centering 
%  \includegraphics[scale=0.75]{pattern/vol-context-menu-new.png}
   \includegraphics[scale=0.65]{roll.png}
   \caption{Pattern Recording Volume Menu}
   \label{fig:pattern_edit_recording_volume_menu}
\end{figure}

   The velocity values are shown at the right side of each menu entry.
   These values correspond to MIDI volume levels from 127 down to 16, as
   shown in the figure.

%  One thing fixed in the 0.90.x version is the ability to store MIDI note-on
%  events with the actual velocity provided by the MIDI keyboard used to
%  generate the notes.  Previously, even in \textsl{seq24},
%  the \textbf{Free} option in the \textbf{Vol} menu option did not work.
%  This is fixed.

\subsection{Pattern Editor / Common Actions}
\label{subsec:seq66_pattern_editor_common}

   This section is a catch-all for actions not described above.

\subsubsection{Pattern Editor / Common Actions / Scrolling}
\label{subsec:seq66_pattern_editor_scrolling}

   Let us describe the actions that can be performed with a
   scroll wheel, or with the scrolling features of multi-touch touchpads.
   There are three major scrolling actions available when using mouse
   scrolling, with the mouse hovering in the piano-roll area:

   \begin{itemize}
      \item \textbf{Vertical Panning (Notes Panning)}
         \index{scroll!normal scroll}
         \index{scroll!vertical pan}
         \index{scroll!notes pan}
         \index{pan!seqroll notes}
         Using the vertical scroll action of a mouse or touchpad moves the
         view of the sequence/pattern notes up and down.
         One can also click in the piano roll, and then use the
         \texttt{Page-Up} \index{keys!page-up}
         and \texttt{Page-Down} \index{keys!page-down}
         keys to move the view up and down in pitch.
      \item \textbf{Horizontal Panning (Timeline Panning)}
         \index{scroll!shift scroll}
         \index{scroll!horizontal pan}
         \index{scroll!timeline pan}
         \index{pan!seqroll time}
         Holding the Shift key, and then using the vertical scroll action of a
         mouse or touchpad moves the view of the sequence/pattern time forward
         and backward.
         One can also click in the piano roll, and then use the
         \texttt{Shift Page-Up} \index{keys!shift page-up}
         and \texttt{Shift Page-Down} \index{keys!shift page-down}
         keys to move the view left and right in time.
      \item \textbf{Horizontal Zoom (Timeline Zoom)}
         \index{scroll!ctrl scroll}
         \index{scroll!horizontal zoom}
         \index{scroll!timeline zoom}
         \index{zoom!seqroll time}
         Holding the Ctrl key, and then using the vertical scroll action of a
         mouse or touchpad zooms the view of the sequence/pattern time to
         compress it or expand it.
         One can also click in the piano roll, and then use the
         \texttt{z} \index{keys!z},
         \texttt{Z} \index{keys!Z}, and
         \texttt{0} \index{keys!0} keys to change the timeline zoom.
   \end{itemize}

   The actions of this scrolling are smooth and fast.
   If an event is selected in the piano-roll area or the (thin) event area,
   then the scrolling increases or decreases the value of the event.
   In the case of a note, this increases or decreases the velocity of the note.
   For all events, this increases or decreases the length of the vertical line
   that represents the value of the event.

\subsubsection{Pattern Editor / Common Actions / Close}
\label{subsec:seq66_pattern_editor_close}

   There is no \textbf{Close} button in the pattern editor.  One can use
   window-manager actions, such as clicking on the X button of the window
   frame, or pressing the exit key defined in the window manager.
   \index{window!close}
   \textsl{Sequencer66} also provides the Ctrl-W key to close the pattern
   editor window.
   However, be aware that this convention does not apply to the other
   application windows of \textsl{Sequencer66}.

%-------------------------------------------------------------------------------
% vim: ts=3 sw=3 et ft=tex
%-------------------------------------------------------------------------------


% Song Editor

%%% %-------------------------------------------------------------------------------
% song_editor
%-------------------------------------------------------------------------------
%
% \file        song_editor.tex
% \library     Documents
% \author      Chris Ahlstrom
% \date        2015-08-31
% \update      2018-10-28
% \version     $Revision$
% \license     $XPC_GPL_LICENSE$
%
%     Provides the concepts.
%
%-------------------------------------------------------------------------------

\section{Song Editor}
\label{sec:song_editor}

   The \textsl{Seq66 Song Editor} combines all patterns
   into a complete tune with specified repetitions of each pattern.
   It shows one row per pattern/loop/sequence in numbered columns,
   with the placement of each pattern at various time locations in the song.
   \index{performance}
   In \textsl{Seq66} parlance, the song editor creates a
   \textsl{performance}, and the performance is implemented by a set of
   triggers (see \sectionref{subsubsec:concepts_terms_trigger}).

   \index{song mode}
   \index{new!dual song editors}
   As an option in the \texttt{[user-interface]}
   section of the "user" configuration file, two song editor windows can be
   brought onscreen, as a convenience for arranging projects with a large
   number of sequences/patterns.
   (In the Qt user-interface, a Song tab and a Song window can be shown at the
   same time.)
   The Song window, when in focus, also activates
   the \textbf{Song} mode of \textsl{Sequencert64},
   as opposed to the \textbf{Live} mode activated by the Patterns panel when it
   is in focus.
%  , when \textsl{Seq66} is running in ALSA mode.
%  (In JACK mode, the live versus song mode is controlled by the JACK
%  start-mode flag.)

   When the song editor has the focus of the application, it
   takes over control from the patterns panel.  The song editor then
   controls playback.
   Once playback is started in the song editor, some actions
   in the patterns panel no longer have effect, effectively disabling live
   mode.  The song editor takes over the arming/unarming (unmuting/muting)
   shown in the patterns panel.  The highlighting of armed/unarmed patterns
   changes according to whether the pattern is playing in the song editor, or
   not.  If one tries to change the muting using a hot-key (or even a click) in
   the patterns panel, the song editor immediately returns the pattern to the
   state it has in the song editor.  The only way to manually change the muting
   then is to click the pattern's label in the song editor.
   Both the song editor and the patterns panel both reflect the change in
   muting in the user-interface, though with \textsl{opposite colors}.

\begin{figure}[H]
   \centering 
%  \includegraphics[scale=0.75]{song-editor/song-editor-window-new.png}
%  \includegraphics[scale=0.75]{new/song_editor_transpose-0_9_15.png}
%  \includegraphics[scale=0.75]{song-editor/song-editor-window.png}
   \includegraphics[scale=0.65]{roll.png}
   \caption{Song Editor Window}
   \label{fig:song_editor_window}
\end{figure}

   Here are some features for the song editor, as
   seen above and in the following figure:

\begin{figure}[H]
   \centering 
%  \includegraphics[scale=0.75]{new/song-editor-0_9_10_1.png}
   \includegraphics[scale=0.65]{roll.png}
   \caption{Song Editor Window, Features}
   \label{fig:song_editor_window_new_features}
\end{figure}

   \begin{itemize}
      \index{shift left click}
      \item Toggling of the mute state of multiple patterns by holding the
         Shift key while left-clicking on the M or a pattern name.
      \index{pause}
      \item Pause button functionality.
      \index{progress bar}
      \item Optional coloring (selected in the Patterns panel)
         and thickening of the progress bar.
      \index{redo}
      \item A Redo button (not shown).
      \index{transpose}
      \item A Transpose button (not shown).
      \index{untransposable color}
      \item Red coloring of events for patterns that are not transposable, such
         as drum tracks.
   \end{itemize}

   \index{empty pattern}
   This window (in \textsl{Seq66}) shows any empty patterns
   highlighted in yellow (Gtkmm only).
   An empty pattern is one that exists, but
   contains only meta information, and contains no MIDI events that
   can be played.  For example, some tracks just serve as name tracks or
   information tracks.
   
   The song editor is not too complex, but for exposition, we break it into
   the top panel, the bottom panel, and the rest of the window.

%  There are still more features of the song editor in
%  pending version 0.9.15, as seen in the following
%  figure:
%
% \begin{figure}[H]
%    \centering 
%    \includegraphics[scale=0.75]{new/song_editor_top_panel_transpose-0_9_15.png}
%    \caption{Song Editor Window, Transpose Song}
%    \label{fig:song_editor_window_transpose_song}
% \end{figure}

   \index{transpose}
   There is a button that allows the whole song (except for
   exempt, non-transposable sequences) to be transposed up or down by up to an
   octave in either direction.

\subsection{Song Editor / Top Panel}
\label{subsec:song_editor_top}

   The top panel provides quick access to song-playback actions and
   configuration.

\begin{figure}[H]
   \centering 
%  \includegraphics[scale=0.55]{song-editor/song-editor-top-panel-items.png}
%  \includegraphics[scale=0.55]{new/song-editor-top-panel-items.png}
   \includegraphics[scale=0.65]{roll.png}
   \caption{Song Editor / Top Panel Items}
   \label{fig:song_editor_top_panel_items}
\end{figure}

   \begin{enumber}
      \item \textbf{Stop}
      \item \textbf{Play}
      \item \textbf{Play Loop}
      \item \textbf{Beats Per Bar}
      \item \textbf{Beat Unit}
      \item \textbf{Grid Snap}
      \item \textbf{Transpose}
      \item \textbf{Toggle JACK Sync}
      \item \textbf{Toggle Following JACK Transport}
      \item \textbf{Redo}
      \item \textbf{Undo}
      \item \textbf{Collapse}
      \item \textbf{Expand}
      \item \textbf{Expand and copy}
   \end{enumber}

   \setcounter{ItemCounter}{0}      % Reset the ItemCounter for this list.

   \itempar{Stop}{song editor!stop}
   Stops the playback of the song.
   \index{keys!esc (stop)}
   The keystroke for stopping playback is the \texttt{Escape} character.
   It can be configured to be another character (such as \texttt{Space}, which
   would make the space-bar toggle the playback status).

   \itempar{Play}{song editor!play}
   \index{L marker}
   Starts the playback of the song, starting at the \textbf{L marker}.
   The \textbf{L marker} serves as the start position for playback
   in the song editor.  One can change the start position only when the
   performance is not playing.
   \index{keys!space (play)}
   The default keystroke for starting playback is the \texttt{Space} character.
   \index{keys!esc (stop)}
   The default keystroke for stopping playback is the \texttt{Escape} character.
   \index{keys!period (pause)}
   The default keystroke for pausing playback is the \texttt{Period} character.
%  which currently works only in ALSA mode.

   \itempar{Play Loop}{song editor!play loop}
   \index{loop mode}
   Activates loop mode. When Play is activated,  play the song and loop
   between the
   \index{L marker}
   \index{R marker}
   \textbf{L marker} and the \textbf{R marker}.
   This button is a state button, and its appearance indicates when it is
   depressed, and thus active.
   If this button is deactivated during playback, then playback
   continues past the \textbf{R marker}.

   \itempar{Beats Per Bar}{song editor!beats/bar}
   Part of the time signature, and specifies the number of beat units per bar.
   The possible values range from 1 to 16.

   \itempar{Beat Unit}{song editor!beat unit}
   Also called "beat width".
   Part of the time signature, and specifies the size of the beat unit:
   1 for whole notes; 2 for half notes; 4 for quarter notes; 8 for eight notes;
   and 16 for sixteenth notes.

   \itempar{Grid Snap}{song editor!grid snap}
   Grid snap selects where the patterns are drawn.
   Unlike the \textbf{Grid Snap} of the pattern editor, the units
   of the song editor snap value are in fractions of a measure length.
   The following values are supported:
   1, 1/2, 1/4, 1/8, 1/16, 1/32, and 1/3, 1/6, 1/12, and 1/24.

   \itempar{Redo}{song editor!redo}
   The Redo button reapplies the last change undone by
   the Undo button.  It is inactive if there is nothing to redo.

   \itempar{Undo}{song editor!undo}
   The Undo button rolls back the last change in the layout of a
   pattern.  Each time it is clicked, the most recent change is undone.
   It rolls back one change each time pressed.
%  It is not certain what the undo limit is, however.
   It is inactive if there is nothing to undo.

   \itempar{Collapse}{song editor!collapse}
   This button collapses the song between the \textbf{L marker} and the
   \textbf{R marker}.
   What this means is that, if there is song material (patterns) before the
   \textbf{L marker} and after the \textbf{R marker},
   and the \textbf{Collapse} button is
   pressed, any song material between the L and R markers is erased, and
   the song material after the \textbf{R marker} is moved leftward to
   the \textbf{L marker}.
   Collapsing occurs in all tracks present in the song editor.

   \itempar{Expand}{song editor!expand}
   This button expands the song between the
   \textbf{L marker} and the \textbf{R marker}.
   It inserts blank space between these markers, moving the song material
   that is after the \textbf{R marker}
   to the right by the duration of the blank space.
   Expansion occurs in all tracks present in the song editor.

   \itempar{Expand and copy}{song editor!expand and copy}
   This button expands the song between the \textbf{L marker} and the
   \textbf{R marker} much like the \textbf{Expand} button.
   However, it also copies the original data that is present after the
   \textbf{R marker}, and pastes it into the newly-available space between
   the L and R markers.

\subsection{Song Editor / Arrangement Panel}
\label{subsec:song_editor_arrangement_panel}

   The arrangement panel is the middle section shown in
   \figureref{fig:song_editor_window}.  It is also known as the
   "piano roll" of the song editor. Here, we zero in on its many
   features.

   Keystrokes and additional mouse configuration have been added to make
   editing easier even without a good mouse.
   For example, one can page up and down vertically in the arrangement
   panel using the
   \index{keys!page-up} Page Up and 
   \index{keys!page-down} Page Down keys.
   One can go to the top using the 
   \index{keys!home} Home key,
   to the bottom using the
   \index{keys!end} End key.
   One can page left and right horizontally in the arrangement
   panel using the
   \index{keys!shift-page-up} Shift Page Up and 
   \index{keys!shift-page-down} Shift Page Down keys.
   One can go to the leftmost position using the 
   \index{keys!shift-home} Home key,
   and to the rightmost position using the
   \index{keys!shift-end} End key,

   The following figure is taken from a conventional MIDI file, imported,
   with a few long tracks, rather than a large number of smaller patterns.
   In other words, the patterns used here are very long, and used only once
   in the song.
%  (We will provide an example that shows off \textsl{Seq66}'s
%  pattern features better, at some point.)

   If playback is started with the song editor as the
   active window, then the pattern boxes in the patterns panel
   show as armed/unarmed (unmuted/muted) depending upon whether or not the
   pattern is shown as playing (or not) at the current playback position in
   the song editor piano roll.

   The following figure highlights the main features of the center panel of the
   song editor.

\begin{figure}[H]
   \centering 
%  \includegraphics[scale=0.75]{song-editor/song-editor-window-full-items.png}
   \includegraphics[scale=0.65]{roll.png}
   \caption{Song Editor Arrangement Panel, Annotated}
   \label{fig:song_editor_window_full_items}
\end{figure}

   \index{transpose}
   One new feature of the song editor is that, if the new Transpose feature is
   built into \textsl{Seq66}, any patterns that are marked as exempt from
   transposition (common with drum tracks) have their events shown in red
   instead of black.

\begin{figure}[H]
   \centering 
%  \includegraphics[scale=1.0]{new/perf-non-transposable.png}
   \includegraphics[scale=0.65]{roll.png}
   \caption{Song Editor for Non-Transposable Patterns}
   \label{fig:song_editor_non_transposable_items}
\end{figure}

   \index{measures ruler}
   The measures ruler (measures strip)
   consists of a \textsl{measures ruler} (bar indicator) at the top, a
   numbered patterns column at the left with a muting indicator, and the
   grid or roll section.  There are a lot of hidden details in the
   arrangement panel, as the figure shows.  Here are the main sections:

   \begin{enumber}
      \item \textbf{Patterns Column}
      \item \textbf{Piano Roll}
      \item \textbf{Measures Ruler}
   \end{enumber}

   These items are discussed in the following sections.

\subsubsection{Song Editor / Arrangement Panel / Patterns Column}
\label{subsubsec:song_editor_arrangement_panel_patterns_column}

   Here are the items to note in the patterns column:

   \begin{enumber}
      \item \textbf{Number}.
         Not yet sure what the number on the left means.
         The number of the screen set?
      \item \textbf{Title}.
         \index{pattern!title}
         \index{pattern!name}
         The title is the name of the pattern, for easy reference.
      \item \textbf{Channel}.
         \index{pattern!channel}
         The channel number appears (redundantly)
         at the right of the title.
      \item \textbf{Buss-Channel}.
         \index{pattern!buss-channel}
         This pair of numbers shows the MIDI buss number used in the pattern
         and the channel used for the pattern.
      \item \textbf{Beat/Measure}.
         \index{pattern!beat}
         This pair of numbers is the standard time-signature of the pattern.
      \item \textbf{Mute Indicator}.
         \index{song editor!mute indicator}
         The letter M is in a black box if the track/pattern is muted, and a
         white box if it is unmuted.
         Left-clicking on the "M" (or the name of the pattern)
         mutes/unmutes the pattern.
         \index{shift left click}
         If the Shift key is held while left-clicking on the M or the pattern
         name, then
         the mute/unmute state of every other active pattern is toggled.
         This feature is useful for isolating a single track or pattern.
      \item \textbf{Empty Track}.
         Completely empty tracks (no track events or meta events)
         are indicated by a dark-gray filling in the pattern column.
         Tracks that have only meta information, but no playable event, are
         indicated by a yellow filling in the pattern column.
   \end{enumber}

   The patterns column shows a list of all of the patterns that have been
   created in the current song.  Each pattern in this list has a track of
   pattern layouts associated with it in the piano roll section.

   \index{patterns column!left-click}
   \index{patterns column!ctrl-left-click}
   \index{song editor!muting}
   Left-clicking on the pattern name or the "M" button toggles the muting
   (arming) status of the track.
   It does the same thing if the \texttt{Ctrl} key is held at the same time.

   \index{pattern!shift-left-click}
   \index{song editor!inverse muting}
   \index{song editor!solo}
   \index{shift-left-click solo}
   Shift-left-clicking on the pattern name or the "M" button toggles the muting
   (arming) status of \textsl{all other tracks} except the track that was
   selected.  This action is useful for quickly listening to a single sequence
   in isoloation.

   \index{patterns column!right-click}
   Right-clicking on the pattern name or the "M" button brings up the same
   pattern editing menu as discussed in
   \sectionref{subsubsec:patterns_pattern_filled}.
   Recall that this context menu has the following entries:
   \textbf{Edit...}, \textbf{Event Edit...}, \textbf{Cut}, \textbf{Copy},
   \textbf{Song}, \textbf{Disable Transpose}, and \textbf{MIDI Bus}.

\subsubsection{Song Editor / Arrangement Panel / Piano Roll}
\label{subsubsec:song_editor_arrangement_panel_roll}

   The "Piano Roll" section of the arrangement panel is where patterns or
   subsections are inserted, deleted, shrunk, lengthened, or moved.
   Here are features to note in the annotated piano roll area
   shown in \figureref{fig:song_editor_window_full_items}:

   \begin{enumber}
      \item \textbf{Single}.
         In the diagram, under the word "Single", is a very small pattern.
         It is small because it consists only of some MIDI Program Change
         messages meant to set the programs on a Yamaha PSS-790 keyboard.
      \item \textbf{Multiple}.
         This item is the same pattern as in "Single", but dragged out for
         multiple repetitions, simply to show how even the shortest patterns
         can be replicated easily.
      \item \textbf{Pattern Subsection}.
         \index{song editor!middle click}
         \index{pattern subsection}
         Middle-clicking inside a pattern inserts a selection position
         marker in it, breaking the pattern into two equal pieces.
         We call each piece a \textsl{pattern subsection}.
         This division can be done over and over.
         In the song editor, a middle-click
          \textsl{cannot} be simulated by ctrl-left-click.)
      \item \textbf{Selection Position}.
         A selection position is a marker that divides a pattern into two
         pieces, called \textsl{pattern subsections}.  This makes it easy to
         select smaller portions of a pattern for editing or deleting.  It
         is especially useful for making holes in a pattern.
%        There may be
%        other uses of a selection position that we have not yet discovered.
      \item \textbf{Selection}.
         Clicking inside a pattern or a pattern subsection darkens it
         (gray) to denote that it is selected.
         A pattern subsection can be deleted by the
         \index{keys!delete}
         Delete key, copied by the
         \index{keys!ctrl-c}
         \index{keys!copy}
         \texttt{Ctrl-C} key, and then inserted (one or more times) by the
         \index{keys!ctrl-v}
         \index{keys!paste}
         \texttt{Ctrl-V} key.  When inserted, each insert goes immediately
         after the current item or the previous insertion.  The same can be
         done for whole patterns.
      \item \textbf{Section Length ("handle")}.
         \index{song editor!handle}
         \index{song editor!section length}
         Looking closely at the diagram where the arrows point, small
         squares in two corners of the patterns can be seen.
         Call them "handles".
         By grabbing
         a handle with a left-click, the handle can be moved horizontally
         to either lengthen or shorted the pattern or pattern subsection, if
         there is room to move in the desired direction.
         It doesn't matter if the item is selected or not.
      \item \textbf{Section Movement}.
         \index{song editor!section movement}
         If, instead of grabbing the section length handle, one grabs inside
         the pattern or pattern subsection, that item can be moved
         horizontally, as long as there is room.  A left-click
         inside the item shows it as selected.
         One can also highlight a pattern section (making it gray),
         then click the \texttt{p} key to enter "paint" mode, and move the
         pattern left or right with the arrow keys.
%        In the near future, movement of selected trigger segments will not
%        require the paint mode to be active just to move the segments left or
%        right.
      \item \textbf{Expansion}.
         \index{song editor!section expansion}
         Originally, all the long patterns of this sample song were continuous.
         But, by setting the L and R markers, and using the \textbf{Expand}
         button, we opened up some silent space in the song, just to be able
         to show it off.
   \end{enumber}

   \index{song editor!split pattern}
   A useful feature is to split a pattern section in the
   song editor, either in half, or at the nearest snap point to the mouse
   pointer.
   The location of the split is determined by the
   setting of the \textbf{File / Options / Mouse / Seq66 Options /
   Middle-click splits song trigger at nearest snap (instead of halfway point)}
   setting.  To split a pattern, left-click it to highlight it, move
   the mouse (if not splitting in half) to the desired pointer, and
   ctrl-left-click with the mouse.

   The \textsl{Seq24} help files refer to work in the song editor as the
   "Performance Editor" or "Performance Mode".  Adding a pattern in this
   window is a bit like adding a note in the pattern editor.
   One clicks, holds, and drags the mouse to insert a copy or copies of the
   pattern associated with the row in which one is dragging.
   The longer one drags, the more copies of the pattern that are inserted.

   \index{song editor!right-click-hold}
   \index{song editor!draw}
	Right-click on the arrangement panel (roll) to enter
   draw mode, and hold the button.
   \index{new!mod4 mode}
   \index{keys!mod4}
   \index{song editor!mod4}
   Just like the patterns panel, there is a feature to allow the Mod4 (the
   Super or Windows) key to keep the right-click in force even after it is
   released.  See \ref{new_mod4_mode}.  Pressing Mod4 before
   releasing the right-click that allows pattern-adding ("painting"), keeps
   pattern-adding in force after the right-click is released.  Now pattern
   can be entered at will with the left mouse button.  Right-click again to
   leave the pattern-adding mode.

   \index{new!paint mode}
   Another way to turn on painting:
   make sure that the performance editor piano roll has the
   keyboard focus by left-clicking in it, then press the
   \index{keys!p}
   \texttt{p} key.
%  This is just like pressing the right mouse button, but the draw/paint mode
%  sticks (as if the Mod4 mode were in force).
   While in the paint mode, one can add pattern clips with the left mouse
   button, via click or drag, and one can highlight a pattern clip and move it
   with the left and right arrow keys.

%  In the near future, movement of selected trigger segments will not
%  require the paint mode to be active just to move the segments left or
%  right.

   To get out of the paint mode, press the
   \index{keys!x}
   \texttt{x} key while in the sequence editor, to "x-scape"
   from the paint mode.
   These keys, however, do not work while the sequence is playing.

   \index{zoom keys}
   \index{keys!0}
   \index{keys!z}
   \index{keys!shift-z}
   The song editor supports zoom in the piano roll.
   This feature is not accessible via a button or a menu
   entry -- it is accessible only via keystrokes.
   After one has left-clicked in the piano roll, the \texttt{z}, \texttt{Z},
   and \texttt{0} can be used to zoom the piano-roll view.  The \texttt{z} key
   zooms out, the \texttt{Z} key zooms in, and the \texttt{0} key resets the
   zoom to the default value.  The zoom feature also modifies the time-line.

   \index{song editor!left-click-right-hold}
   \index{song editor!insert}
   A left-click with a simultaneous right-click-hold inserts one copy of the
   pattern.  The inserted pattern shows up as a box with a tiny
   representation of the notes visible inside.  Some patterns can
   be less than a measure in length, resulting in a tiny box.
   \index{song editor!right-left-hold-drag}
   \index{song editor!multiple insert}
   To keep adding more copies of the pattern, continue to hold both buttons
   and drag the mouse rightward.

   \index{song editor!middle-click}
   Middle-click on a pattern to drop a new selection position into the
   pattern,
   \index{song editor!pattern subsection}
   which breaks the pattern into two equal \textsl{pattern subsections}.
   Each middle-click on the pattern adds a new selection position,
   halving the size of the subsections as more pattern subsections are
   added.

   \index{song editor!left-click}
   \index{song editor!selection}
   When a pattern or a pattern subsection is left-clicked in the piano
   roll, it is marked with a dark gray filling.
   \index{song editor!right left hold drag}
   \index{song editor!deletion}
   When a right-left-hold-drag action is done in this gray area, the result
   is to \textsl{delete} that pattern section or subsection.
   \index{keys!delete}
   One can also hit the Delete key to \textsl{delete} that pattern section
   or subsection.

\subsubsection{Song Editor / Arrangement Panel / Measures Ruler}
\label{subsubsec:song_editor_arrangement_panel_measures_ruler}

   The \textsl{measures ruler} is the ruled and numbered section at the top
   of the arrangement panel.  It provides a place to put the left and right
   markers.  In the \textsl{Seq24} documentation, it is called the "bar
   indicator".

   \index{measures ruler!left-click}
   Left-click in the measures ruler to move and drop an
   \index{L anchor}
   \index{L marker}
   \textbf{L marker} (\textbf{L anchor}) on the measures ruler.
   \index{measures ruler!right-click}
   Right-click in the measures ruler to drop an
   \index{R anchor}
   \index{R marker}
   \textbf{L marker} (\textbf{R anchor}) on the measures ruler.

   Once these anchors are in place, one can then use
	the \textsl{Collapse} and \textsl{Expand} buttons to modify the
   placement of the pattern events.

   Note that the \textbf{L marker} serves as the start position for playback
   in the song editor.  One can change the start position only when the
   performance is not playing.

   \index{new!marker mode}
   \index{new!movement mode}
   Another way to move the "L" and "R" markers has been added.
   To select which marker will move, first click the upper half of the time
   strip (otherwise, the "L" will move, prematurely) to give it keyboard focus.
   Then press the lower-case
   \index{keys!l}
   \texttt{l} key or the lower-case
   \index{keys!r}
   \texttt{r} key.
   \textsl{There is no visual feedback that one is in the movement mode.}
   Then press the left or right arrow key to move the selected ("L" or "R")
   key marker by one snap value at a time.

\subsection{Song Editor / Bottom Panel}
\label{subsec:song_editor_bottom}

   The bottom panel is simple, consisting of a stock horizontal scroll bar
   and a small button, called the \textbf{Grow} button, labelled with a
   "\textbf{$>$}".

   \index{grow button}
   \index{song editor!grow}
   The \textbf{Grow} button adds to the number of measures that exist
   in the song editor. The visual effect is very subtle, resulting only
   in a small change in the thumb of the horizontal scroll-bar, unless one
   is at the right end of the piano roll.  Then, one can see the added
   measures.  Usually about 128 at a time are added, but this depends on the
   value of PPQN in force.

%-------------------------------------------------------------------------------
% vim: ts=3 sw=3 et ft=tex
%-------------------------------------------------------------------------------


% Event Editor

%%% %-------------------------------------------------------------------------------
% event_editor
%-------------------------------------------------------------------------------
%
% \file        event_editor.tex
% \library     Documents
% \author      Chris Ahlstrom
% \date        2016-01-02
% \update      2025-06-07
% \version     $Revision$
% \license     $XPC_GPL_LICENSE$
%
%-------------------------------------------------------------------------------

\section{Event Editor}
\label{sec:event_editor}

   The \textsl{Seq66} \textbf{Event Editor} tab is used to view and edit,
   in detail, the events present in a loop / sequence / pattern / track.
   It is accessed by right-clicking on a pattern in the \textbf{Live} frame,
   then selecting the \textbf{Edit pattern in tab} menu entry.
   The default keystroke combination for this action is to use the
   \textsl{minus} key followed by the desired pattern's hot-key.

   The event editor is not very sophisticated.
   It is a basic editor for simple edits, viewing, and trouble-shooting.
   It is disabled if recording a pattern, to avoid refresh issues while
   recording.

   Viewing and scrolling work;
   editing, deleting, and inserting events work.
   But there are many possible interactions between event links
   (Note Off events linked to Note On events, for example),
   performance triggers, and the pattern,
   performance, and event editor dialogs.
   Surely some bugs still lurk.
   If anything bad happens, do \textsl{not} press the
   \textbf{Save to Sequence} button!
   If the application aborts, let us know!
   Here are the major "issues":

   \begin{enumerate}
      \item It requires the user to know the details
         about MIDI events and data values.
      \item For safety, it does not detect any changes made to the sequence in
         the pattern editor; we might add a refresh button.
      \item It does not have an undo function. Just don't save!
      \item It cannot mark more than one event for deletion or modification.
         However, if one note event is deleted, the corrsponding linked note
         event is also deleted.
      \item There is no support for dragging and dropping of events.
   \end{enumerate}

   The event editor is a good way to see the events in a sequence,
   and to delete or modify problematic events.
   Additionally, it can be used to add \textbf{Set Tempo} and other
   meta events.
   \index{sequence extension}
   \index{pattern extension}
   If an event is added that has a time-stamp beyond the current
   length of the sequence, then the length of the sequence is extended.
   Unlike the event pane in the pattern editor, the event-editor
   dialog shows all types of events at once.

\begin{figure}[H]
   \centering
   \includegraphics[scale=0.65]{event-editor/event-editor-tab.png}
   \caption{Event Editor Window}
   \label{fig:event_editor_window}
\end{figure}

   The event-editor dialog is fairly complex.
   For exposition, we break it down into a few sections:

   \begin{enumber}
      \item \textbf{Event Frame}
      \item \textbf{Info Panel}
      \item \textbf{Edit Fields}
      \item \textbf{Bottom Buttons}
   \end{enumber}

   The event frame is a list of events, which can be traversed, and edited.
   The fields in the right panel show the name of
   the pattern containing the events and other information about the
   pattern.  The edit fields provide text fields for viewing and entering
   information about the current event, and buttons to delete, insert, and
   modify events.  The bottom buttons allow changes to be saved and the editor
   to be closed.  
   The following sections described these items in detail.

\subsection{Event Editor / Event Frame}
\label{subsec:event_editor_frame}

   The event frame is the event-list shown on the left side of the
   event editor.  It is accompanied by a vertical scroll-bar, for moving one
   line or one page at a time.
   Mouse or touchpad scrolling can be used to move up and down
   in the event list.
   Depending on the window manager theme, the currently-selected event
   is highlighted.
   (We have been trying to get this table to auto-stretch vertically when the
   main window is vertically maximized, but have not succeeded so far. Qt!)

\subsubsection{Event Frame / Data Items}
\label{subsec:event_frame_data}

   The event frame shows a list of numbered events, one per line.
   The currently-selected event is shown in the edit fields.
   Here is an example of the data line for a MIDI event:

   \begin{verbatim}
      17   003:3:128 Note On -- 3  69 107 003:4:96
   \end{verbatim}

   This line consists of the following parts:

   \begin{enumber}
      \item \textbf{Index Number}
      \item \textbf{Time Stamp}
      \item \textbf{Event Name}
      \item \textbf{Bus Number} (new)
      \item \textbf{Channel Number}
      \item \textbf{Data Bytes D0 and D1}
      \item \textbf{Link}
   \end{enumber}

   \setcounter{ItemCounter}{0}      % Reset the ItemCounter for this list.

   \itempar{Index Number}{event editor!index number}
   Displays the index number of the event.
   This number is purely for reference, and is not part
   of the event.  Events in the pattern are numbered from 1 to the number of
   events in the pattern.

   \itempar{Time Stamp}{event editor!time stamp}
   Displays the time stamp of the event,
   which indicates the cumulative time of the event in the pattern.
   It is displayed in the format of "measure:beat:divisions" (e.g. B:B:T).
   The measure values start from 1, and range up to the number of measures in
   the pattern.
   The beat values start from 1, and range up to the number of beats in the
   measure.
   The division values range from 0 up to one less than the
   \index{ppqn}
   PPQN (pulses per quarter note) value for the whole song.
   \index{ppqn!\$ shortcut}
   As a shortcut, one can use the dollar sign ("\$") to represent
   PPQN-1.
   If the \textbf{Tick Time} box is checked, then the timestamps are
   shown in units of "ticks" (MIDI pulses, also called "divisions").

   \itempar{Event Name}{event editor!event name}
   Displays the name of the event.
   The event name indicates what kind of MIDI event it is. 
   See \sectionref{subsec:event_editor_fields}.

   \itempar{Bus Number}{event editor!bus number}
   Shows the bus that the event came in on, or a hyphen if not applicable.
   This value is currently not saved with the event,
   so is of limited utility.

   \itempar{Channel Number}{event editor!channel number}
   Shows the channel number (for channel-events only) re 1, not 0.
   (For the user, MIDI channels always range from
   1 to 16.  Internally, they range from 0 to 15.)

   \itempar{Data Bytes D0 and D1}{event editor!data bytes}
   Shows the one or two data bytes for the event.
   The byte is shown in one two formats, hexadecimal and decimal, depending
   on the setting of the \textbf{Hex} check-box.
   For text events, a small portion of the text is shown.
   The full text is shown in the text field below the D0 and D1 values.

   Note Off, Note On, and Aftertouch events requires a byte for the key (0 to
   127) and a byte for the velocity (also 0 to 127).
   Control Change events require a control code and a value for that control
   code.  Pitch wheel events require two bytes to encode the full range of
   pitch changes.
   Program change events require only a byte value to pick the patch or program
   (instrument) to be used for the sequence.  The Channel Pressure event
   requires only a one-byte value.
   Tempo requires a number (e.g. "120.3") to be typed in.

   \itempar{Links}{event editor!links}
   Note events are linked together; each Note On is linked to the corresponding
   Note Off, and vice versa.

\subsubsection{Event Frame / Navigation}
\label{subsec:event_frame_navigation}

   Moving about in the event frame is straightforward, but has some
   wrinkles to note.
   Navigation with the mouse is done by moving to the desired event and
   clicking on it.  The event becomes highlighted, and its data items are shown
   in the "info panel".
   There is no support for dragging and dropping events in the event frame.
   There is no support for selecting multiple events.

   The scrollbar can be used to move within the frame, either by one line at a
   time, or by a page at a time.  A page is defined as one frame's worth of
   lines, minus 5 lines, for some overlap in paging.

   Navigation with keystrokes is also supported, for the Up and Down arrows and
   the Page-Up and Page-Down keys.  Note that using the Up and Down arrows by
   holding them down for awhile causes autorepeat to kick in.
   Use the scrollbar or page keys to
   move through multiple pages.  Home and End also work.

\subsection{Event Editor / Info Panel}
\label{subsec:event_editor_info}

   The "info panel" is a read-only list of properties on the top right
   of the event editor.  It serves to remind the used of the pattern being
   edited and some characteristics of the pattern and the whole song.
   It also includes one button.
   Five items are shown:

   \begin{enumber}
      \item \textbf{Sequence Number and Name}.
         A bit redundant, as the window caption or the pattern
         also shows the pattern name.
         It can be set here or in the pattern editor.
      \item \textbf{Sel Linked}. (Not shown in picture.)
         If this button is checked, then selecting a Note On or Off event
         also selects the opposite, linked note.
         This allows for easy deletion of a complete Note pair.
      \item \textbf{Channel Number}.
         Shows the output channel number.
      \item \textbf{Time Signature}.
         A pattern property, shown only as a reminder.
         It can be set in the pattern editor.
      \item \textbf{PPQN}.
         Shows the "parts per quarter note", or resolution of the
         whole song.  The default PPQN of \textsl{Seq66} is 192.
      \item \textbf{Sequence Channel}.
         In \textsl{Seq66}, the channel number is a property of the
         pattern.  All channel events in the pattern get routed to the same
         channel, even if somehow the event itself specifies a different
         channel.
      \item \textbf{Sequence Count}.
         Displays the current number of events in the pattern.
         This number changes as events are inserted or deleted.
   \end{enumber}

\subsection{Event Editor / Edit Fields}
\label{subsec:event_editor_fields}

   The edit fields show the values of the currently-selected event.  They allow
   changing an event, adding a new event, or deleting the currently-selected
   event. These have been greatly updated for user convenience in the last
   few releases of \textsl{Seq66}.

   \begin{enumber}
      \item \textbf{Category}.
         This dropdown shows what kind of event is selected, and can
         also be used to insert specific events. The event categories are:
         \begin{itemize}
            \item \textbf{Channel Message}.
            \item \textbf{System Message}.
            \item \textbf{Meta Event}.
            \item \textbf{SecSpec Event}.
         \end{itemize}
         See below for more information about supported events.
      \item \textbf{Time}.
         Shows the event timestamp in B:B:T notation.
         It can be edited to move an event in time.
      \item \textbf{Event}.
         This dropdown shows the event name.
         It can also be used to select an event that is appropriate for
         the selected event category.
      \item \textbf{D0}.
         Shows data byte 1, e.g. the note number for a note event.
         It can also be edited.
      \item \textbf{D1}.
         Shows data byte 2, e.g. the velocity for a note event.
         It can also be edited.
      \item \textbf{Tick Time}.
         If check-marked, the \textbf{Time} column in the event list
         is shown as ticks (pulses, divisions) instead of B:B:T.
      \item \textbf{Hex}.
         If check-marked, the \textbf{D0} and \textbf{D1} columns
         in the event list, as well as the editable
         \textbf{D0} and \textbf{D1} fields, as shown in hexadecimal notation.
         For example, D0 = 67 is shown as "0x43".
      \item \textbf{Text}.
         This field is for the display of certain events, such as
         pitch data. It is also used for entering text data for various
         text events.
   \end{enumber}

   \textbf{Important}: changes made in the event editor
   are \textsl{not written} to the sequence until the \textbf{Save}
   button is clicked.  If one messes up an edit field, just click on the event
   again; all the fields will be filled in again.
   That's as much "undo" as the event-editor offers at this time, other than
   closing without saving.

   \setcounter{ItemCounter}{0}      % Reset the ItemCounter for this list.

   \itempar{Category}{event editor!event category}
   Displays the event category of the event.
   All channel events, most system messages
   and many meta events can be handled,
   even system-exclusive events.

\begin{figure}[H]
   \centering
   \includegraphics[scale=0.65]{event-editor/message-category-dropdown.png}
   \caption{Event Category List}
   \label{fig:event_editor_category_dropdown}
\end{figure}

   There are four types of messages handled by \textsl{Seq66}:
   Channel, System, Meta, and SeqSpec. 
   Seqspec messages are stored when a pattern is saved, but they are
   not part of the pattern's event list.
   Hence this item is grayed out; these message values are shown in various
   parts of the user interface.
   See the table in \sectionref{subsec:midi_format_meta_format}.

   \itempar{Time}{event editor!event timestamp}
   Displays the timestamp of the selected event in B:B:T format.
   "measure:beat:division" format is fully supported.
   We allow editing (but not display) of the timestamp in
   pulse (divisions) format and "hour:minute:second.fraction" format, but
   there may be bugs to work out.

   If one wants to delete or modify an event, this field does not need to be
   modified. If this field is modified, and the \textbf{Modify}
   button is pressed, then the event will be moved  This field can locate
   a new event at a specific time.  If the time is not in the current frame,
   the frame will move to the location of the new event and make it the current
   event.

   \itempar{Time}{event editor!event time}
   Shows the time of the event in the format of "measure:beat:divisions" (only).
   This field can be edited to change the time of the event.

   \itempar{Event}{event editor!event name}
   Displays the name of the event, and allows entry of an event name.
   The event name indicates what kind of MIDI event it is. 
   The following event names are supported for Channel events:

   \begin{enumber}
      \item \textbf{Note Off}
      \item \textbf{Note On}
      \item \textbf{Aftertouch}
      \item \textbf{Control}
      \item \textbf{Program}
      \item \textbf{Channel Pressure}
      \item \textbf{Pitch Wheel}
   \end{enumber}

   Selecting one of these names from the dropdown changes the kind of event if
   the event is modified.
%  Abbreviations and case-insensitivity can be used to
%  reduce the effort of typing.
%  Also, if \textbf{Control} or
%  \textbf{Program} are selected, then a data drop-down box is available
%  to select either the controller or
%  the instrument patch (program), and it can fill in the data values.
   If \textbf{Control} or \textbf{Program} are selected,
   then the name of item-number entered into \textbf{D0} is
   shown in the text area.

   In the future, we may make these configurable and add selection
   drop-downs, with the control-change
   following the settings in the 'usr' file, and the program-change being made
   from a (new) 'patches' file.
   Currently the programs are General MIDI.

   If the \textbf{Meta Event} category is selected, the following events are
   shown:

\begin{figure}[H]
   \centering
   \includegraphics[scale=0.75]{event-editor/meta-event-dropdown.png}
   \caption{Meta Event List}
   \label{fig:event_editor_meta_dropdown}
\end{figure}

   If the \textbf{System Message} category is selected, the following events are
   shown:

\begin{figure}[H]
   \centering
   \includegraphics[scale=0.85]{event-editor/system-event-dropdown.png}
   \caption{System Message List}
   \label{fig:event_editor_system_dropdown}
\end{figure}

   At present, many of these messages, especially the textual messages,
   should be editable.
   Report any issues.

   \itempar{Channel}{event editor!channel}
   This dropdown shows the channel of the current event, ranging from 1 to 16.
   If the event is not a channel message, then \textbf{None} is shown.

   \itempar{D0}{event editor!data byte 1}
   Allows modification of the first data byte of the event.
   One must know what one is doing.
   The scanning of the digits is very simple:  start with the first digit, and
   convert until a non-digit is encountered.  The data-byte value can be
   entered in decimal notation, or, if prepended with "0x", in hexadecimal
   notation.
   For \textbf{Control} or \textbf{Program},
   the name of item-number in \textbf{D0} is
   shown in the text area.

   For a Tempo setting only this field is used; Data Byte 2 is ignored.
   Enter a Tempo value, such as "120", and then click \textbf{Insert}. The
   value is converted to the 3 bytes of a tempo event, and then
   added at the given timestamp.  (Screen refresh is not perfect yet, but
   reloading the pattern shows the correct tempo.)

   \itempar{D1}{event editor!data byte 2}
   Allows modification of the second data byte of the event (if applicable
   to the event).
   One must know what one is doing.
   The scanning of the digits is as noted above.

   \itempar{Tick time}{event editor!tick time}
   Selecting this check-box
   shows the time-stamps in the frame in tick (pulses) format.

   \itempar{Hex}{event editor!hex data}
   Selecting this check-box
   shows all number items in hex format.

   \itempar{Text data}{event editor!text data}
   This unlabelled pane shows the text of a meta text event.
   This field can be edited to add or change a meta text event.
   The events supported are:
   \textsl{Text Event},
   \textsl{Copyright},
   \textsl{Instrument Name},
   \textsl{Lyric},
   \textsl{Program Name}, and
   \textsl{Device Name}.
   It also shows translated versions of the control or program numbers in
   \textbf{D0}.

%  \itempar{Delete}{event editor!delete event}
%  Causes the selected event to be deleted.
%  The frame display is updated to move following events upward.

\subsection{Event Editor / Bottom Buttons}
\label{subsec:event_editor_buttons}

   These are the buttons that act on the edit fields or current event
   selection:

   \begin{enumber}
      \item \textbf{Delete}.
         Delete the selected event.
      \item \textbf{Insert}.
         Create a new event based on the data in the fields.
      \item \textbf{Modify}.
         Officially modify the selected event.
      \item \textbf{Clear}.
         Clear all events.
      \item \textbf{Save}.
         Save all the changes back to the pattern.
      \item \textbf{Dump}.
         Dump the events to a file.
   \end{enumber}

   Note that
   \index{bugs!event delete key}
   \index{bugs!event insert key}
   \textsl{Seq66} does not support using the
   \texttt{Delete} and \texttt{Insert} keys to
   supplement the buttons; the \texttt{Delete}
   key is needed for editing the event data fields.

   \setcounter{ItemCounter}{0}      % Reset the ItemCounter for this list.

   \itempar{Insert}{event editor!insert event}
   Inserts a new event, described by the 
   \textbf{Event Timestamp},
   \textbf{Event Name},
   \textbf{Data Byte 1}, and
   \textbf{Data Byte 2} fields.
   The new event is placed in the appropriate location for the given timestamp.
   If the timestamp is at a time that is not visible in the frame, the frame
   moves to show the new event, so be careful.

   \itempar{Modify}{event editor!modify event}
   Deletes the current event, and inserts the modified event,
   which is placed in the appropriate location for the given
   timestamp.  (This feature does not work with linked Note Ons and Note Offs).

   \itempar{Clear}{event editor!clear events}
   Deletes all of the events in the event table.
   As with all edits, does not become official until the \textbf{Save} button
   is clicked.

   \itempar{Save}{event editor!save events}
   Saves all of the events in the event table into the original sequence.
   There is no way to undo this action.
   This button does not close the dialog; further
   editing can be performed.  The Save button is enabled only if
   some unsaved changes to the events exist.
   Any sequence/pattern editor that is open should be reflected
   in the pattern editor once this button is pressed.

   \itempar{Dump}{event editor!dump events}
   Write the events to a text file in the same directory as the MIDI file, very
   useful for troubleshooting.  The name of the file is of the form:

   \begin{verbatim}
      midi_file_name-pattern-#.text
   \end{verbatim}

   where '\#' is the pattern number.  For example, if the loaded file is

   \texttt{/home/user/miditunes/The\_Wild\_Bull.midi}

   and the pattern is 9, then the resulting dump-file is

   \texttt{/home/user/miditunes/The\_Wild\_Bull-pattern-9.text}.

%-------------------------------------------------------------------------------
% vim: ts=3 sw=3 et ft=tex
%-------------------------------------------------------------------------------


% Session Management

%-------------------------------------------------------------------------------
% seq66 sessions
%-------------------------------------------------------------------------------
%
% \file        sessions.tex
% \library     Documents
% \author      Chris Ahlstrom
% \date        2020-10-03
% \update      2022-06-02
% \version     $Revision$
% \license     $XPC_GPL_LICENSE$
%
%  Provides a discussion of how Seq66 supports session management, specifically
%  the Non Session Manager.
%
%-------------------------------------------------------------------------------

\section{Session Management}
\label{sec:sessions}

   Session management helps recreate complex setups and provide some uniformity
   of application control.
   The first thing to do for session management is to make sure that the
   application is capable of various levels of session management, from
   \textsl{UNIX} signals to
   a complete session manager like the \textsl{Non Session Manager}.
   Basic session management consist of being able to properly start the
   application and let it run properly during its life-cyle, whether it is a
   command-line application or a graphical application.
   \textsl{Seq66} supports session management in three ways:

   \begin{enumber}
      \item \textbf{Signals}.
         During a normal run, \textsl{Seq66} will respond
         to signals to save and to quit.
         The normal configuration files and command-line options will be used.
         This mode is useful with \textsl{nsm-proxy}, a way to script
         applications that don't have \textsl{NSM} support.
      \item \textbf{JACK Session}
         Deprecated, but implemented nonetheless.
         This allows for the configuration files to be stored in
         a separate directory, for \textsl{Seq66} to be started, and files to
         be saved.  No restrictions on where the MIDI files can be stored.
      \item \textbf{Non Session Manager}
         Known as \textsl{NSM}, and in the form of a fork of that project,
         \textsl{New Session Manager},
         it provides a replacement for \textsl{JACK Session}.
         It requires all files to be
         stored in a session directory, and provides commands for saving,
         quitting, hiding/showing the user-interface, and more.
         Like \textsl{JACK Session}, it allows control over the startup of
         multiple applications, the process of saving a session, and provides a
         way to save their patching (connections) in \textsl{JACK}.
         However, it supports more functionality and has
         strict requirements the application must follow.
         Development of NSM has, for various reasons, been suspended, but
         offshoots such as \textsl{Agordejo} (\cite{agordejo})
         and \textsl{RaySession} (\cite{raysession}), which are front ends for
         the \textsl{New/Non Session Manager} (\cite{nsm}),
         continue to advance.
   \end{enumber}

   If one desires session management, \textsl{NSM} is the way to go.
   \textsl{JACK} session management is provided for those who still use it.
   There are other session solutions, such as \textsl{aj-snapshot},
   \textsl{Claudia}, and \textsl{Chino}.
   For now, we do not discuss them.

   The desired session can be set in the \textbf{Edit / Preferences /
   Session} tab.  But note that, if started by \textsl{NSM}, \textsl{Seq66}
   will still set up for NSM usage.  The \textsl{NSM} setting is
   useful for attaching to a pre-existing known session.
   \textsl{JACK} session management events are processed
   only if \textsl{JACK} is selected.
   \textsl{JACK} session management will still start \textsl{Seq66} in
   an existing session, if \textsl{JACK} is
   not selected, but that's it.

   Also note that sometimes one will want the session manager to make the JACK
   connections.  In this case, go to
   \textbf{Edit / Preferences / JACK / Jack Auto-Connect}, uncheck that option,
   and restart \textsl{Seq66}.  This option (\texttt{jack-auto-connect}
   can also be changed in the 'rc' file.
   The \textsl{NSM} tool called \textsl{jackpatch} can also be used to manage
   connections.

\subsection{Session Management / Signals}
\label{subsec:sessions_signals}

   \index{sessions!signals}
   By default, the basic form of session management in
   \textsl{Seq66} occurs by signals.  A
   session manager can start \textsl{Seq66}, and it can tell \textsl{Seq66} to
   save or stop.  Starting is done by a system call to spawn the application.
   The save and stop actions are supported by sending the following signals to
   the application:

   \begin{itemize}
      \item \texttt{SIGINT}.
         This signal stops \textsl{Seq66}. It corresponds
         to using \texttt{Ctrl-C} from the command-line to stop \textsl{Seq66}.
         This signal should work for both the graphical and command-line
         application.  As \textsl{Seq66} shuts down, it does its normal saving
         of the current state of the configuration.
      \item \texttt{SIGTERM}.
         This signal also stops \textsl{Seq66}.  It can
         be sent by an application to exit \textsl{Seq66}.
      \item \texttt{SIGUSR1}.
         This signal tells \textsl{Seq66} to save.  This
         action will save the current MIDI file.
   \end{itemize}

   One application that can control \textsl{Seq66} via these signals, when not
   in session mode, is \textsl{nsm-proxy}:

      \url{https://non.tuxfamily.org/wiki/nsm-proxy}

   \textsl{NSM-Proxy} is a simple \textsl{NSM} client for wrapping non-NSM
   capable programs. It enables the use of programs supporting LADISH Level 0
   and 1, and programs which accept their configuration via command-line
   arguments.  There is a command-line version and a graphical version.
   More to come on how to use \texttt{nsm-proxy}.

\subsection{Session Management / JACK Session}
\label{subsec:sessions_jack}

   Although deprecated by the \textsl{JACK} authors in favor of \textsl{NSM},
   we are implementing \textsl{JACK} session (JS) management for the benefit of
   people who either do not know of \textsl{NSM} or do not want to implement or
   use it.

   \textsl{Seq66}, as a JS-aware applications, is set up to

   \begin{enumerate}
      \item Register with a JS manager.
      \item Respond to messages from the JS manager.
      \item Be startable with session information.
   \end{enumerate}

   A response to a JS message will do one of the following:

   \begin{itemize}
      \item Save the application's state into a file, where the directory is
         supplied by the session manager.
      \item Reply to the session manager with a command-line that starts the
         application, with information to restore its state, such as
         the name of the file holding its state information.
   \end{itemize}

	JS-aware clients identify themselves to the session manager by a UUID
	(unique universal identifier). The session manager provides it to
	the client application as an integer represented as a string.
   This can be passed to the session manager when registering, but
   \textsl{Seq66} just uses the value given to it (for now).

%  but should also be passed back to the client when it is restarted
%by the session manager. This is done by a command line argument to the
%application, and the format of the command line is also up to the client.

   For this discussion, we will use the \textsl{JACK} session implementation in
   the \textsl{QjackCtl} application.
   Also, read the script stored in
   \texttt{seq66/data/linux/startqjack} to set up
   \textsl{QjackCtl} to run \textsl{JACK} and kick off
   \textsl{a2jmidid}; it should be added to the \textsl{QjackCtl}
   configuration.

   Once that setup is made (installing the script and configuring
   \texttt{qjackctl}, then start \texttt{qjackctl}.
   Verify that there are a number of system audio and MIDI playback and capture
   port, \textsl{PulseAudio JACK} sinks and sources if the system uses
   \textsl{PulseAudio}, and that there are "a2j" MIDI ports for all of your USB
   hardware devices.

   Then start \textsl{Qsynth} so it uses \textsl{jack} for MIDI and
   \textsl{jack} (or \textsl{pulseaudio}) for audio.
   Then run \textsl{Seq66} with \textsl{JACK} for slave transport and for MIDI
   (either command works the same):

   \begin{verbatim}
      $ qseq66 --jack-slave --jack-midi
      $ qseq66 --jack-slave --jack
   \end{verbatim}

   Load a file, make sure its MIDI output goes to "fluidsynth" or "qsynth", and
   plays.
   In your desired location (e.g. \texttt{~/.config/seq66/sessions},
   create a new session directory (e.g. \texttt{qtest}).

   In \textsl{qjackctl}, open the \textbf{Sessions} dialog.
   Click \textbf{Save}, and choose the directory just created.
   In the dialog should appear entries for MIDI capture and playback for
   "fluidsynth" and "seq66", all the "a2j" USB devices,
   plus an entry for \textsl{JACK} client
   \textsl{seq66master} or \textsl{seq66slave} that shows
   something like:

   \begin{verbatim}
qseq66 --jack --jack-master --jack-session-uuid 84670 --home ${SESSION_DIR}
qseq66 --jack --jack-slave --jack-session-uuid 84670 --home ${SESSION_DIR}
   \end{verbatim}

   In the \textbf{Connections} dialog of \textsl{QjackCtl}, all of these ports
   will be shown in the MIDI tab, auto-connected appropriately.
   In the sessions directory that was created, will be seen an
   empty \texttt{seq66master}
   or \texttt{seq66slave} directory, and a
   \texttt{sessions.xml} configuration file containing the information shown in
   the sessions dialog.
   Exit \textsl{Seq66}, \textsl{QSynth} and \textsl{QjackCtl}
   (in that order).

   One issue is that \textsl{QSynth}
   does not support \textsl{JACK Session}.
   Try it with \textsl{Yoshimi}, which does support it.

\subsection{Seq66 Session Management / NSM}
\label{subsec:sessions_nsm}

   \index{sessions!nsm}
   The \textsl{Non Session Manager} is an API implementation for session
   management for Linux audio/MIDI.
   \textsl{NSM} clients use a well-defined
   \index{sessions!OSC}
   \textsl{OSC} protocol (\cite{osc})
   to communicate with the session management daemon.

   Note that \textsl{Non Session Manager} is in a state of suspended
   development, and has been reimplemented as a \textsl{GitHub} project,
   the \textsl{New Session Manager}.

   The applications it manages should be installed normally (that is,
   for system-wide usage, in
   \texttt{/usr/bin/} or \texttt{/usr/local/bin}).

\subsubsection{Session Management / NSM / First Run Without NSM}
\label{subsec:sessions_nsm_first_run_without_nsm}

   This section discusses what happens when \textsl{Seq66} is installed, then
   run outside of any session from the console or an application menu.
   For a discussion where \textsl{Seq66} is run for the first time under
   \textsl{NSM},
   see \sectionref{subsec:sessions_nsm_first_run_in_nsm}.

   Generally, after installing \textsl{Seq66}, or when creating a new setup
   (such as a play-list) it is good to run it normally first, to simplify
   trouble-shooting.
   This action creates the configuration files in the default location,
   \texttt{/home/user/.config/seq66}:

\begin{verbatim}
   $ qseq66 
   No 'rc' file, will create: qseq66.rc/ctrl/midi/mutes
   No 'usr' file, will create: /home/user/.config/seq66/qseq66.usr
   File exists: /home/user/.config/seq66/qseq66.rc
   Saving initial config files to session directory!
   Writing 'rc': /home/user/.config/seq66/qseq66.rc
   Writing 'ctrl': /home/user/.config/seq66/qseq66.ctrl
   Writing 'mutes': /home/user/.config/seq66/qseq66.mutes
   Writing 'usr': /home/user/.config/seq66/qseq66.usr
   . . .
\end{verbatim}

   Then exit \textsl{Seq66} to ensure the configuration files are created.
   Optionally, in this initial setup,
   one can also create a 'playlist' file and a 'drums' file, or
   copy them from:

   \begin{verbatim}
      /usr/share/seq66-0.91/data/samples
   \end{verbatim}

   to

   \begin{verbatim}
      /home/user/.config/seq66
   \end{verbatim}

   and modify them appropriately.
   Another first-time modification to consider is setting up \textsl{Seq66} to
   use the \textsl{JACK} audio/MIDI subsystem (on \textsl{Linux}).
   In the 'rc' file, look for the following line:

   \begin{verbatim}
      [jack-transport]
      jack-midi = false
   \end{verbatim}

   And change it to:

   \begin{verbatim}
      [jack-transport]
      jack-midi = true
   \end{verbatim}

   Another first-time modification to consider is using virtual ports (option
   \texttt{--manual-ports}) versus the automatic port connections
   \textsl{Seq66} normally makes.
   This setup allows the user to manually make connections between
   \textsl{Seq66} and other MIDI applications.
   In the 'rc' file, look for the following lines:

\begin{verbatim}
   [manual-ports]
   virtual-ports = false   # 'true' = manual (virtual) ALSA or JACK ports
   output-port-count = 8   # number of manual/virtual output ports
   input-port-count = 4    # number of manual/virtual input ports
\end{verbatim}

   And change the virtual-ports line to:

\begin{verbatim}
   [manual-ports]
   virtual-ports = true    # 'true' = manual (virtual) ALSA or JACK ports
\end{verbatim}

   It is then important to start \texttt{qseq66} in the normal manner again,
   and verify that everything works as expected.

   We have added a command, \textbf{File / Import / Import Project}
   command to import the configuration files from one directory into the
   current \textsl{NSM} session configuration directory.
   This import used to be automatic, but is too surprising to an unsuspecting
   user.

\subsubsection{Seq66 Session Management / NSM / Run in NSM}
\label{subsec:sessions_nsm_first_run_in_nsm}

   Note: When \textsl{Seq66} is run in \textsl{NSM} for the first time,
   a stock default configuration is saved when
   \textsl{Seq66} exits.
   This is different from earlier behavior, where the home configuration was
   imported automatically.
   Now, the user must use the
   \textbf{File / Import / Import Project...}
   command.

   For illustration, we run \textsl{NSM} from a terminal window, which can be
   very helpful when problems occur.

\begin{verbatim}
   $ non-session-manager
   [non-session-manager] Starting daemon...
   [nsmd] Session root is: /home/user/NSM Sessions
   NSM_URL=osc.udp://mycomputer.mls:19625/
   [nsmd] Listing sessions
\end{verbatim}

   \index{sessions!non-starter}
   \index{sessions!liblo library}
   If \textsl{NSM} refuses to start, make sure that the \texttt{liblo} library
   from the OSC project is installed.
   \index{sessions!/etc/hosts}
   \index{sessions!loopback interface}
   If it is installed, then check the
   \texttt{/etc/hosts} file to make sure that a loopback interface is
   defined. In some versions of \textsl{Linux}, it isn't defined properly,
   and the \textsl{NSM} daemon (\texttt{nsmd}) will not start.
   Here is an example of the loopback installed in \textsl{Debian Sid};

\begin{verbatim}
   127.0.0.1   localhost
   127.0.1.1   mycomputer.mls mycomputer
\end{verbatim}

   The NSM user-interface (not shown here) that comes up is empty at first.
   So create a session by clicking the \textsl{NSM}
   \textsl{New} button, and entering a session name
   (here, "\textbf{Seq66}") in the
   prompt that comes up.  In the console window, a couple of 
   \texttt{/nsm/server/new} \textsl{OSC} messages
   about the creation of the session appear.

\begin{verbatim}
   [non-session-manager] Sending new for: Seq66
   [nsmd] Creating new session "Seq66"
   [non-session-manager] /nsm/server/new says Created.
   [non-session-manager] /nsm/server/new says Session created
\end{verbatim}

   Next, click the \textsl{Add Client to Session}, and, since
   \texttt{qseq66} has been installed system-wide, it is in the \texttt{PATH}
   and its executable name can be entered simply: "\texttt{qseq66}".
   A number of console messages from
   \textsl{Seq66} appear, plus some messages from \textsl{NSM}.

\begin{verbatim}
   [non-session-manager] Sending add for: qseq66
   [nsmd] Process has pid: 2797436
   [nsmd] Launching qseq66
   [nsmd] Got announce from seq66
   [nsmd] Client was expected.
   [nsmd] Process has pid: 2797436
   [nsmd] The client "seq66" at "osc.udp://127.0.0.1:13318/" informs us it's
    ready to receive commands.
\end{verbatim}

   Important: the \textsl{Seq66} user-interface will not show at first.
   It is hidden so that the screen is not inundated with the windows of all the
   applications that are (eventually) running under the session.
   This is especially annoying with tiled window managers.
   In order to see the \textsl{Seq77} user-interface, click on
   the \textbf{GUI} button in the session line shown in the \textsl{NSM} window .

   Once \textsl{Seq66} is running under \textsl{NSM},
   then click the \textbf{Save}
   button at the top of the \textsl{NSM} interface in order
   to save the session information.
   This is an \textsl{important} step.
   After \textsl{Seq66} exits,
   one can see what has been created to support the session;
   the directory that \textsl{NSM}
   creates by default is \texttt{/home/user/NSM Sessions}.

\begin{verbatim}
   $ pwd
   /home/user/NSM Sessions
   $ lstree Seq66
   Seq66/
     +-- seq66.nGJDW/
     |   +-- config/
     |   |   +-- qseq66.ctrl
     |   |   +-- qseq66.drums
     |   |   +-- qseq66.mutes
     |   |   +-- qseq66.palette
     |   |   +-- qseq66.playlist
     |   |   +-- qseq66.rc
     |   |   +-- qseq66.usr
     |   +-- midi/
     +-- session.nsm
\end{verbatim}

   So \textsl{NSM} has created a directory with the session name we gave it:
   \texttt{Seq66}.  Under that directory is a file, \texttt{session.nsm}, which
   contains information like the following:

\begin{verbatim}
   seq66:qseq66:nXYZT
\end{verbatim}

   The format of this text is \texttt{appname:exename:nXYZT}, where
   \texttt{XYZT} is a 4-letter randomly-generated token
   generated by \textsl{NSM}.
   Also created is a directory, \texttt{seq66.nXYZT}, which is the root of the
   \texttt{Seq66} session.

   The rest of the directories,
   \texttt{config} and \texttt{midi},
   are generated by \textsl{Seq66}
   The \texttt{config} directory is used instead of
   \texttt{/home/user/.config/seq66}) and \texttt{midi} directory
   contains new MIDI files, imported MIDI files,
   or MIDI files from a play-list.
   The new \texttt{config} directory
   contains versions of the various configuration files that will always be
   used to start up \textsl{Seq66} during the session.
   One can also add valid play-list, palette, and drums/note-mapping files to
   that directory later.

   If before running \textsl{NSM},
   one had set up a play-list file and provided the proper "MIDI
   base directory" in the 'rc' file, then all the MIDI files are copied to
   the \textsl{NSM} session \texttt{midi} directory,
   preserving all relative directories.
   When the \textsl{Non Session Manager} is started the next time, and the
   "Seq66" session is clicked, this starts \textsl{Seq66}, and the play-list can
   be seen in the \textsl{Playlist} tab.

   Note that the \textbf{Save} button on the session's row in the
   \textsl{NSM} user-interface sends a message to \textsl{Seq66}
   to tell it to save its state.

   One last thing to note is that, when viewing the MIDI ports created by
   \textsl{Seq66}, they will be named "seq66" when not in session management,
   and "seq66.nXYZT" (for example) when under session management.  This makes
   it possible to run multiple instances of \textsl{Seq66}.

\subsubsection{Session Management / NSM / Run with Remote NSM}
\label{subsec:sessions_nsm_before_using_nsm}

   As described in the \textsl{NSM} documentation, the \texttt{nsmd} daemon can
   be run stand-alone, and can also be ran on a remote computer.
   The \texttt{qseq66.usr} file can be edited to allow \textsl{Seq66} to
   use a pre-planned \textsl{NSM} and specify the URL to connect.
   Look for the following lines in the 'usr' file:

   \begin{verbatim}
      [user-session]
      session = none
      url = ""
   \end{verbatim}

   Now assume we've run the daemon as follows:

   \begin{verbatim}
      $ nsmd --osc-port 9999
      [nsmd] Session root is: /home/user/NSM Sessions
      NSM_URL=osc.udp://mycomputer.mls:9999/
   \end{verbatim}

   Change the \texttt{session} lines to allow the usage of
   \textsl{NSM} at that URL:

\begin{verbatim}
   [user-session]
   session = nsm
   url = "osc.udp://mycomputer.mls:9999"
\end{verbatim}

   The \texttt{url} is not used if running \textsl{Seq66} from the \textsl{NSM}
   GUI... the application will get the URL from the \textsl{NSM} environment.
   Note that \texttt{qseq66} can still be run outside of a
   session manager.  It will detect the absense of the session manager and run
   normally.

\subsubsection{Session Management / Sessions Tab}
\label{subsubsec:sessions_tab}

   The \textsl{Session} tab is a \textsl{read-only} tab
   provided to orient the user to the setup supported by the session.
   When not running in a session, the normal configuration directory and files
   are shown.  When running in an \textsl{NSM} section, the configuration
   information received from \textsl{NSM} is displayed.
   It is meant to display information to
   help the user understand what is happening in the run.
   This tab is not yet fully functional.
   In particular, \textbf{Session Log} does not yet work.

\begin{figure}[H]
   \centering 
   \includegraphics[scale=0.65]{tabs/session/qseq66-session-tab.png}
   \caption*{Session Tab Under NSM}
\end{figure}

   \index{sessions!ui}
   This section describes the \textsl{Session} tab in the main
   \textsl{Seq66} window.  This tab is mostly informative and
   \textsl{read-only}.
   It displays the following bits of information that \textsl{Seq66} has received
   from \textsl{NSM} via the \texttt{nmsd} daemon:

   \begin{itemize}
      \item Name of the session manager.
      \item Session path for the session, the root directory of the session.
         All data goes into this directory. If not running in a session,
         the active configuration directory (which can be modified via
         command-line arguments) is shown.
      \item The OSC URL of the session, which includes the port number.
         Generally, the port number is selected at run-time, but it is also
         possible to configure \textsl{NSM} to use a specific port number.
      \item Display-name for the session.
      \item The generated client ID for the session.
      \item The log of action of the session manager. Not yet supported,
         though one can see what is happening by running in a console window.
      \item Macro Execution.
         This drop-down contains all of the macros defined in the 'ctrl' file's
         \texttt{[macro-control-out]} section.
         By selecting one, it is automatically sent out via the
         \texttt{[output-buss]} port defined in the 'ctrl' file.
         The "startup" and "shutdown" macros, if defined,
         are sent automatically. "Startup" is useful to put
         a MIDI controller into the proper mode for controlling and displaying
         information in \textsl{Seq66}, and "shutdown" can return the controller
         to its normal operating mode.
      \item Reload Session.
         \index{restart!manual}
         After editing some of the preferences in the \textbf{Edit / Preferences}
         dialog, one can visit this tab and press this button to essentially
         restart \textsl{Seq66}, reloading the new configuration.
         Be careful!
   \end{itemize}

\subsubsection{Seq66 Session Management / NSM / File Menu}
\label{subsubsec:sessions_file_menu}

   The author of \textsl{NSM} has provided documentation for session-management
   which provides very strict instructions on how an application must behave
   under session management.  \textsl{Seq66} tries very hard to stick to these
   instructions.  One major adjustment an application must make is to adhere to
   the "File menu" guidelines.

\begin{figure}[H]
   \centering 
   \includegraphics[scale=0.65]{tabs/session/nsm-qseq66-menus-2.png}
   \caption*{File Menu Under NSM, Composite View}
\end{figure}

   This has been changed for 0.98.6; the \textbf{Quit} menu entry becomes
   \textbf{Hide}, as per the NSM protocol.  Also have fixed a bug that disables
   the load-most-recent option under NSM.
   We will update the figure above eventually.
   The following items describe the menu entries.

   \begin{itemize}
      \item \textbf{New MIDI File}.
         This function prompts for the name of a
         new MIDI file and clears the current MIDI file.  The file-name must not
         include a full-path to the file.  The path is hardwired by the
         session.  A relative path can be included.  This name is needed
         because there is no "Save As" option when running in an \textsl{NSM}
         session.
      \item \textbf{Import / Import Project Configuration...}
         Imports a whole project configuration into the current NSM session.
         This functionality used to be automatic (importing the "home"
         configuration), but it is better left to the user to do.
         \index{restart!automatic}
         However, the restart of \textsl{Seq66} after this operation is
         automatic.  Be careful!
      \item \textbf{Import / Import MIDI to Current Set...}
         This action works the same as in normal mode.
         This item allows the user to grab a MIDI file from anywhere and import
         it into the current set.
         The default directory that comes up in the
         prompt is the "last-used directory" from the session 'rc' file.
      \item \textbf{Import / Import Playlist...}
         This action works the same as in normal mode.
         The destination is the NSM session directory.
         \index{restart!automatic}
         Once the playlist is imported,
         \textsl{Seq66} is automatically \textsl{\textbf{restarted}}
         in order to load the playlist.
         Be careful!
      \item \textbf{Import / Import into Session...}
         Prompts the user for a MIDI file to
         be imported (copied) into the current session.  The path to the file
         is then adjusted to use the \textsl{NSM} \texttt{midi} subdirectory.
      \item \textbf{Export / Export Song...}
         Allows exporting the current song as a stock MIDI file, using the
         performance information (triggers) to write the MIDI data as it would
         be played in "song" mode.
         The default directory that comes up in the
         prompt is the "last-used directory" from the session 'rc' file.
      \item \textbf{Export / Export MIDI Only...}
         Allows exporting the current song as a stock MIDI file.
         The "proprietary" SeqSpec data is \textsl{not} written.
         The default directory that comes up in the
         prompt is the "last-used directory" from the session 'rc' file.
      \item \textbf{Export / Export SMF 0...}
         This action works the same as in normal mode.
         It converts the destination file to SMF 0 format.
      \item \textbf{Hide}.
         This menu item replaces the \textbf{Quit} item.
         It hides the main window and tells NSM about it.
   \end{itemize}

   At some point we would like to present a small tutorial showing a session
   under \textsl{JACK}.
   Also note that NSM can invoke or kill applications via
   \textsl{signals}, as explained in 
   \sectionref{subsec:sessions_signals}.

\subsubsection{Seq66 Session Management / NSM / Debugging}
\label{subsubsec:sessions_debugging}

   This section is oriented towards advanced users who found a problem running
   \textsl{Seq66} and want to track it down themselves.  The issue is that we
   need to start the application under the debugger, or start it under NSM and
   somehow attach to \textsl{Seq66} before it starts running.  Another issue is
   that we have found that, at least on the same host, an NSM session
   \textsl{must} be open before \textsl{Seq66} can attach to it, even if the
   correct \texttt{NSM\_URL} is provided.
   So we have to open a session, get the proper URL, configure it in the 'usr'
   file, and then start \textsl{Seq66} under the debugger.
   Here are the steps:

   \begin{enumerate}
      \item Start \textsl{non-session-manager} from a command-line console.
         Write down the URL that it advertises.
      \item Prepare a session for the executable as per earlier instructions.
         Once \texttt{qseq66} starts, immediately exit it, and leave the session
         open.
      \item Open the proper 'usr' file (usually \texttt{qseq66.usr}) in a 
         text editor.  Set variable "session = nsm", and set the variable "url"
         to the value that was advertised.
      \item Now start \texttt{qseq66} in a debugger.
      \item Set a breakpoint in \texttt{clinsmanager::detect\_session()}.
   \end{enumerate}
   
   Now you can step through and see where NSM and Seq66 are getting mixed up.
   Also check the session directory afterward to make the configuration
   (and any MIDI files) are in good shape.

\subsection{Seq66 Session Management / LASH}
\label{subsec:sessions_lash}

   \index{sessions!lash}
   LASH support has been removed.  Use the \textsl{NSM Session Manager} or
   the \textsl{JACK Session Manager}.

%-------------------------------------------------------------------------------
% vim: ts=3 sw=3 et ft=tex
%-------------------------------------------------------------------------------


% Import/Export

%%% %-------------------------------------------------------------------------------
% midi_export
%-------------------------------------------------------------------------------
%
% \file        midi_export.tex
% \library     Documents
% \author      Chris Ahlstrom
% \date        2018-10-20
% \update      2021-04-14
% \version     $Revision$
% \license     $XPC_GPL_LICENSE$
%
%     This section discusses the details of the import/export functionality.
%
%-------------------------------------------------------------------------------

\section{Import/Export}
\label{sec:midi_export}

   This section explains the details of the MIDI import and export
   functionality, accessed by the main menu as noted in sections
   \ref{subsubsec:menu_file_import},
   \ref{subsubsec:menu_file_export}, and
   \ref{subsubsec:menu_file_export_midi_only}, on page
   \pageref{subsubsec:menu_file_import}.

\subsection{Import MIDI}
\label{subsec:midi_export_file_import}

   The \textbf{Import} menu entry imports an SMF 0
   or SMF 1 MIDI file as one or more patterns, one pattern per track, and
   imports them into the currently-active set.
   Even long tracks, that aren't short loops, are imported.
   The difference from \textbf{File / Open} is that the destination screen-set
   (bank) for the import can be specified, and the existing data in the
   already-loaded MIDI file is preserved.
   If the imported file is a
   \textsl{Seq66} MIDI file, it's proprietary sections will
   \textsl{not} be imported, in order to preserve the performance setup.
   The \textbf{Import} dialog is similar to the \textbf{Open} dialog.

   When imported, each track, whether music or information,
   is entered into its own loop/pattern box (slot).
   The import operation can handle reasonably complex files.
   When the file is imported, the sequence number for each track is
   adjusted to put the track into the desired screen-set.
   The import can place the imported data into any of the 32 available
   screen-sets.  Quite large songs can be built by importing patterns.

   Import also handles SMF 0 MIDI files.  It parcels out the SMF 0 data
   into sequences/patterns for each of the 16 MIDI channels.  It also puts
   all of the MIDI data into the 17th pattern (pattern 16), in case it is
   needed.  Note that this slot is used no matter which screen-set one imports
   the file into.  Bug, or feature?

\subsection{Export Song as MIDI}
\label{subsec:midi_export_file_export}

   Thanks to \textsl{Seq32}, exporting song performances (see the
   \textbf{Song Editor}) to standard MIDI format has been added.
   The \textbf{Export Song as MIDI} operation modifies the song in the
   following ways:

   \begin{itemize}
      \item Only tracks (sequences, loops, or patterns)
         \index{exportable}
         that are "exportable" are written.  To be exportable, a
         track must have triggers present
         in the \textbf{Song Editor}, and, \textsl{in the song editor}, it
         must not be muted.
      \item Each trigger generates the events, including repeats,
         offset-play of the events, and transposition.
         If there is a gap in the layout
         (e.g. due to an \textbf{Expand} operation in the
         \textbf{Song Editor}),
         then the corresponding gap in the events is exported.
         The result is a track that reconstructs the original
         playback/performance layout of that pattern.
         The events themselves are sufficient to play the performance exactly
         in any MIDI sequencer.
         The triggers are useful for further editing of the song/performance,
         so they are preserved in the triggers \textsl{SeqSpec} section, but
         they cover the whole song.
      \item Empty pattern slots between tracks are removed.
      \item No matter what set the original track was in, it ends up in the
         first set; sets are consolidated.
      \item Other additions, such as time signature and tempo meta events, are
         written in the same manner as for a normal \textbf{File / Save}
         operation.
   \end{itemize}

   The Export dialog is similar to the Open dialog; one will likely want to
   change the name of the file so as not to overwrite it.
   If there are no exportable tracks, the following message is shown:

\begin{figure}[H]
   \centering 
   \includegraphics[scale=0.65]{main-menu/file/light-menu-file-song-unexportable.png}
   \caption{MIDI File Unexportable}
   \label{fig:midi_export_file_unexportable}
\end{figure}

   Once the file is exported, reopen it to see the results of the export.
   The following figure shows a before and after picture of the export, as
   seen in the song editor.

\begin{figure}[H]
   \centering 
   \includegraphics[scale=0.75]{song-editor/song-layout-sample-2.png}
   \caption{MIDI File Layout Before/After Export}
   \label{fig:midi_export_file_before_after}
\end{figure}
   
   The gaps in layouts in the song/performance data are reflected in the
   consolidated triggers.
   Here is the before/after triggers for pattern \#0, which was
   layed out with \textbf{Record Snap} \textsl{on}:

   \begin{verbatim}
       BEFORE                              AFTER
      Sequence #0 'One Bar'               Sequence #0 'One Bar'
      Length (ticks): 768                 Length (ticks): 5375
      trigger: 768 to 1535 at 768         trigger: 0 to 5375 at 5375
      trigger: 3840 to 5375 at 768        
   \end{verbatim}

   Note that 768, at PPQN = 192, is 4 beats (1 measure), while 5375 is 28
   beats (7 measures).
   For each of these triggers, the first number is the start of the trigger in
   PPQNs, the second is the end of the triger, and the third, called the
   "offset", is actually the length of the pattern.
   Note how the "AFTER"
   trigger consolidates the "BEFORE" triggers, starts at time 0, extends to
   the end of the last trigger, and has a length equal to the end of the
   trigger.

   Now here is the before/after triggers for pattern \#5, which was
   layed out with \textbf{Record Snap} \textsl{off}:

   \begin{verbatim}
       BEFORE                              AFTER
      Sequence #5 'Two Bars'              Sequence #5 'Two Bars'
      Length (ticks): 1536                Length (ticks): 6911
      trigger: 2344 to 3879 at 1536       trigger: 0 to 6655 at 6911
      trigger: 4944 to 6655 at 0
   \end{verbatim}

   6655 is a little over 34.5 beats, which is what the bottom grey trigger
   shows.
   6911 is almost 36 beats (9 measures).  Something to figure out.

\subsection{Export MIDI Only}
\label{subsec:midi_export_file_export_midi_only}

   Sometimes it might be useful to export only the non-sequencer-specific
   (non-SeqSpec) data from a \textsl{Seq66} song.
   For example, some buggy sequencers
   (hello \textsl{Windows Media Player})
   might balk at some SeqSpec item in the song, and refuse to load the MIDI
   file.
   For such cases,
   the \textbf{Export MIDI Only} menu item writes a file that does not contain
   the SeqSpec data for each track, and does not include all the SeqSpec data
   (such as mute groups) that is normally written to the end of the
   \textsl{Seq66} MIDI file.

%-------------------------------------------------------------------------------
% vim: ts=3 sw=3 et ft=tex
%-------------------------------------------------------------------------------


% Tables of keyboard and mouse actions

%%% %-------------------------------------------------------------------------------
% kbd_mouse
%-------------------------------------------------------------------------------
%
% \file        kbd_mouse.tex
% \library     Documents
% \author      Chris Ahlstrom
% \date        2016-04-07
% \update      2021-01-13
% \version     $Revision$
% \license     $XPC_GPL_LICENSE$
%
%     Provides tables for keyboard and mouse support in Seq66.
%
%-------------------------------------------------------------------------------

\section{Seq66 Keyboard and Mouse Actions}
\label{sec:kbd_mouse_actions}

   This section presents some tables summarizing keyboard and mouse actions
   available in \textsl{Seq66}.
   It does not cover mute keys and group keys, which are well
   described in the keyboard options for the main window.
%  See \sectionref{paragraph:menu_file_options_keyboard}).
   It does not cover the "fruity" mouse actions, as this mode of mouse-handling
   is not supported at present in \textsl{Seq66}.
%  , though they are touched
%  on in \sectionref{paragraph:menu_file_options_mouse}.

%  Any volunteers to fill in the table?

   This section describes the keystrokes that are currently hardwired
   in \textsl{Seq66}.
   This description only includes items not defined in the 'ctrl' file.
   That is, hardwired values.
   "KP" stands for "keypad".
   \index{keys!focus}
   The effect that keystrokes have depends upon
   which window has the keyboard/mouse focus.
   \index{keys!qt}
   It must be noted that the Qt 5 user-interface still has a few missing pieces
   in keystroke support.

\subsection{Keyboard Control}
\label{subsec:kbd_mouse_keyboard_control}

   \textsl{Seq66} provides a plethora of keyboard controls for user-interface
   actions, note-modification, zooming, and pattern control.
   These controls are easy to change by editing the appropriate 'ctrl'
   configuration file, stored in one of the following directories, depending on
   the operating system:
   
   \begin{verbatim}
         /home/username/.config/seq66/qseq66.ctrl
         C:/Users/username/AppData/Local/seq66/qpseq66.ctrl
   \end{verbatim}

   There are also some extended examples present in the \textsl{Seq66}
   \texttt{data/linux} and
   \texttt{data/samples} directory.
   Also note that keyboard and MIDI control settings have been consolidated
   into a single table in the 'ctrl' file.

   \textbf{Warning:}
   \index{keys!gotchas}
   There are a number of "gotchas" to be aware of when assigning keys to the
   fields in the \textbf{Keyboard} tab:

   \begin{itemize}
      \item Some of the keystrokes are hard-wired, such as 
         "arrow" keys (for controlling play-lists) or "page up/down" keys.
      \item \textsl{Seq66} has appropriated the
         \index{keys!shift} Shift key so that a a Shift-left-click on a pattern
         slot opens up the corresponding set-number in an external live grid.
%        modifies a click on a pattern so that all of the other patterns are
%        \textsl{toggled}.
         \index{auto-shift}
         Also, for the group-learn feature, the \texttt{Shift} key is 
         automatically enabled, using an "auto-shift" feature.
         Thus, using characters that require the Shift
         key while clicking, such as \texttt{\{} and \texttt{\}},
         becomes surprising.
         Instead, look to the remaining keys: \texttt{F11}, \texttt{F12},
         and the "keypad" keys if more options are wanted.
   \end{itemize}

   \texttt{[keyboard-control]}.
   We won't attempt to cover every key-control item,
   just the categories.  Some items might be discussed in other parts
   of this manual. Remember that key and MIDI control have been consolidated.
   Also remember that the 'ctrl' file contains comments and an orderly layout to
   make it easier to understand and to edit.

   The \texttt{[mute-group]} section has been moved to it's own 'mutes' file.

   \index{pause}
   An additional key definition is shown for the pause key.
   By default, the pause key is the period
   ("\texttt{.}."), but that can be changed.

   \index{pattern edit}
   New features try to achieve being able to edit a pattern using mainly the
   keyboard.  \textsl{Seq66} now supports two modifier keys.
   The first modifier key causes the usual pattern-toggle key (hot-key) for a
   given slot to instead bring up the pattern editor.  By default, this key is
   the equals ("\texttt{=}") key.
   \index{event edit}
   The second modifier key causes the usual
   pattern-toggle key (hot-key) for a given slot to instead bring up the event
   editor.  By default, this key is the minus ("\texttt{-}") key.
   Both of these keys are configurable.

   Some of the keys have positional mnemonic value.  For example,
   for BPM control, the semicolon is at the left (down), and the apostrophe
   is at the right (up).

   \index{slot-shift}
   \index{keys!slot-shift}
   The \textbf{slot shift} key is useful when using pattern grids larger
   than 8 x 4 patterns.  Pressing the slot-shift key basically adds 32 to the
   pattern number of the slot-key that is pressed.
   The default key is the forward slash ("\texttt{/}") key.

   \index{snapshot}
   \index{keys!snapshot}
   A \textbf{snapshot} is a briefly-preserved state of the patterns.
   One can press a snapshot key, change the state of the patterns for live
   playback, and then release the snapshot key to revert to the state when
   the snapshot key was first pressed.
   The default key is the \texttt{Ins} key.

%   Holding 'Alt' will save the state of playing sequences
%   and restore them when 'Alt' is lifted.
%
%   Holding 'Left Ctrl' and 'Alt' at the same time will enable
%   you to flip over to new sequences briefly and then
%   flip right back upon lifting 'Alt'.
%
%	Is this Snapshot 1 versus Snapshot 2?  In Seq24's code, either key
%  does exactly the same thing!

   \index{queue}
   \index{keys!queue}
   To \textbf{queue}
   a pattern means to ready it for playback upon the next repeat
   of a pattern.  A pattern can be armed immediately with a hot-key,
   or it can be queued to play back the next time the pattern repeats.
   A pattern can be queued by holding the queue key (defined in
   \textbf{File / Options / Keyboard / queue}) and pressing a pattern-slot
   hot-key.  Instead of the pattern turning on
   immediately, it turns on at the next repeat of the pattern.
   The default key is the "\texttt{o}" key.

   \index{keep queue}
   \index{keys!keep queue}
   \index{queue!keep}
   \textbf{Keep queue}
   allows the queue to be held without holding
   down the queue button the whole time.  First, press the keep-queue key.
   Next, hitting
   any of the slot hot-keys, no matter how many, sets up the corresponding
   pattern slot to be queued.  Also, in keep-queue mode, clicking on the
   pattern slot will queue the pattern.  The keep-queue mode is disabled by
   hitting the "queue" key again (any currently active queues remain active
   until finished).
   The default key is the backslash key, "\texttt{\\}" key.
   There is also a "Q" button to toggle the keep-queue
   status.

   \index{one-shot}
   \index{keys!one-shot}
   \textbf{One-Shot}
   causes a slot to be queued for only a single playback.
   The default key is the pipe, "\texttt{|}" key.

   \itempar{Sequence toggle keys}{keyboard!sequence toggle keys}
   Each of these keys toggles the playing/muting of one of the 32
   loop/pattern boxes.
   These keys are layed out logically on the keyboard by default,
   and can also be shown in each loop/pattern box.
   Please note that we often call them "shortcut keys" or "hot-keys"
   where the context
   makes it clear that they apply to the armed/unarmed state of a pattern.

   \itempar{Mute-group slots}{keyboard!mute-group slots}
   There can be up to 32 mute-groups.
   \index{playing set}
   When activated, a mute-group
   sets the muted/unmuted status of the current "playing set"
   to the pattern-muting statuses of the selected mute-group.
   Each of these keys operates on the mute-grouping of one of the 32
   stored mute groups.
   These keys are layed out logically on the keyboard by default, and consist
   of \texttt{Shift} versions of the sequence-toggle (hot) keys.
   Note that a mute-group key will be memorized only when
   \textsl{Seq66} is in
   \index{group-learn}
   \textsl{group-learn} mode.

%  \index{mute-group}
%  One thing to explain is just what mute-grouping means.
%  \textsl{Mute groups} are shortcuts to play a defined group of patterns
%  on the current set, while stopping other patterns from the current set, and
%  all patterns from other sets.

   \itempar{Learn}{keyboard!learn}
   \index{group!learn}

   To define the group of patterns for one mute group, press and hold the
   configured Learn key (the "el", "\texttt{l}" key by default,
   the hard-wired \texttt{Ctrl-L} key, or the "\textbf{L}"
   button in the user-interface.
   Simultaneously (not needed with the "L" button),
   pressing one of the mute group keys: \textsl{Seq66}
   will save the currently-playing pattern slots into the corresponding mute
   group.
   \index{auto-shift}
   The default mute group keys must be the shifted version of the key,
   but one does not need the \texttt{Shift} key while pressing
   \texttt{Insert} to learn the group, only to trigger it.
   \textsl{Seq66} will automatically assign the corresponding key with
   \texttt{Shift} activated.  Try pressing the \texttt{Shift} key in Learn mode
   and see what happens!

   Group-mute can be globally enabled or disabled (with default keys apostrophe
   \texttt{'} \index{grave} \index{igrave} and igrave or grave \texttt{`}).
   So make sure it is enabled before trying to use it.

   \itempar{Disable}{keyboard!disable}
   \index{keys!apostrophe}
   It is the inverse \textbf{apostrophe} key by default.
   \index{group!off}
   \index{keyboard!group off}
   This key is the \textsl{group off} key.

   \itempar{Enable}{keyboard!enable}
   \index{keyboard!igrave}
   It is the \textbf{igrave} (back-tick) key by default.
   \index{group!on}
   \index{keyboard!group on}
   This key is the \textsl{group on} key.

   \texttt{[extended-keys]}.
   A number of additional functions have been added to \textsl{Seq66},
   and keystrokes have been provided for those new functions.

   \setcounter{ItemCounter}{0}      % Reset the ItemCounter for this list.

   \index{song mode}
   Note the \textbf{Song/Live toggle} key.
   The \textsl{song mode} normally is in effect only when playback is started
   from the \textbf{Song Editor}.  Now this mode can be used from any
   window, if enabled by pressing this key.  There is also
   a button in the main window for this function, which shows the current state
   of this flag.  Note that this flag is also stored in the 'rc' configuration
   file, as well as this hot-key value, which defaults to \texttt{F10}.

   \index{toggle JACK}
   \index{JACK toggle}
   The \textsl{JACK mode} is set via the
   \textbf{File / Options / JACK / JACK Connect} or 
   \textbf{JACK Disconnect} buttons.
   This keystroke will toggle between JACK connect and JACK disconnect.
   The \textbf{Song Editor} will also have a \textbf{JACK} button.
   The hot-key for this function defaults to \texttt{F2}.

   \index{menu mode}
   The \textsl{menu mode} indicates if the main menu of the
   main window is accessible or not.  It is disabled during playback
   so that more hot-keys can be used without triggering menu functions.
   It can also be disabled by the user; the default hot-key is \texttt{F3}.
   This feature is needed because the original \textsl{Seq24} had numerous
   conflicts between the menu key bindings and the default key bindings for the
   main window.

%  Here is Stazed's explanation of the feature, mildly edited:
%
%  \begin{quotation}
%     \textsl{"why disabling is needed when playing"}
%     The original seq24 had numerous conflicts between the menu key binding
%     and the default seq24 key binding for the mainwind sequence triggers.
%     For example: Ctrl-q (quits the program without prompt). If you place a
%     sequence in the default 'q' slot, you cannot use it with Ctrl-l or Ctrl-r
%     (default replace or queue) because the menu grabs the keys. Same goes for
%     the Alt-l or Alt-r (default snapshot 1 or 2). Try same as above with
%     Alt-f, Alt-v, Alt-h, Ctrl-n, Ctrl-o...  etc. So I just shut off all the
%     menus by default when playing because it seems that they should not be
%     needed then... especially in a live performance.
%
%     \textsl{"why a button?"}
%     On occasion I wanted to use the mainwnd key binding when stopped to set
%     the sequences to be ready before starting. It's also a sort of safety
%     feature as well, just toggle the menus off before going live so that you
%     don't hit Ctrl-q, Ctrl-n etc. forgetting things are not playing....
%  \end{quotation}

   \index{follow jack}
   \textsl{Follow JACK} is a feature ported from \textsl{Seq32}.
   The default key is \texttt{F4}.
   It determines if \textsl{Seq66} follows JACK transport.

   \index{fast forward}
   \textsl{Fast forward} is a feature ported from \textsl{Seq32}.
   The default key is \texttt{F6}.
   While this key is held, the song pointer will fast-forward
   through the song.
   This feature does not have a corresponding button.

   \index{rewind}
   \textsl{Rewind} is a feature ported from \textsl{Seq32}.
   The default key is \texttt{F5}.
   While this key is held, the song pointer will rewind.
   This feature does not have a corresponding button.

   \index{pointer position}
   \textsl{Pointer position} is a feature ported from \textsl{Seq32}.
   The default key is \texttt{F7}.
   When this key is pressed, the song pointer will move to the
   current position of the mouse, snapped.
   This feature does not have a corresponding button.

   \index{toggle mutes}
   \textsl{Toggle mutes} toggles the mute status of every
   pattern on every screen-set.  It corresponds to the
   \textbf{Edit / Toggle mute all tracks} or the 
   \textbf{Song / Toggle All Tracks}
   menu entries.  There is also a button in the main window for this function,
   which shows the current state of this flag.  Note that this
   hot-key value is stored in the 'rc' configuration file, and
   defaults to \texttt{F8}.

   \index{tap bpm}
   \textsl{Tap BPM} allows the user to "tap" in time with some
   other music, and see the tap sequence translated into beats/minute (BPM).
   There is also a "0" button for this function.
   After 5 seconds, this feature resets automatically, so the user can try
   again if not satisfied.  At least two taps are needed for the
   BPM to be registered.

% VERIFY and the UNCOMMENT
%
%  Tap BPM causes events to be logged to the tempo track which is the first
%  track (track 0) by default.

\subsection{Main Window}
\label{subsec:kbd_mouse_main_window}

   The main window keystrokes are all defined via the options dialog
   and "rc" configuration file, or are stock Gtk window-management keystrokes.
   The main window has a very complete setup for live control of the MIDI tune
   via keystrokes.  These actions are not included in
   \tableref{table:main_window_support}.
%  There may be some other keystrokes to be documented at some point.

   \begin{table}[H]
      \centering
      \caption{Main Window Support}
      \label{table:main_window_support}
      \begin{tabular}{l l l l l l}
         \textbf{Action} & \textbf{Normal} & \textbf{Double} &
            \textbf{Shift} & \textbf{Ctrl} & \textbf{Mod4} \\
         \textbf{e} & --- & --- & --- & Open song editor & --- \\
         \textbf{l} (el) & --- & --- & --- & Enter Learn mode & --- \\
         Left-click slot & Mute/Unmute & New/Edit & Toggle other slots &
            --- & --- \\
         Right-click slot & Edit menu & --- & Edit menu & Edit Menu &
            --- \\
      \end{tabular}
   \end{table}

   The new mouse features of this window for \textsl{Seq66},
   as noted in \sectionref{sec:patterns_panel}, are:

   \begin{itemize}
      \item \textsl{Shift-left-click}:
         Over one pattern slot, this action toggles the mute/unmute
         (armed/unarmed) status of all other patterns
         (even the patterns in other, unseen sets).
      \item \textsl{Left-double-click}:
         Over a pattern slot, this action quickly toggles the mute/unmute status,
         which is confusing.  But it ultimately brings up the pattern editor
         (sequence editor) for that pattern.
%        It acts like Ctrl-left-click.
   \end{itemize}

\subsection{Performance Editor Window}
\label{subsec:kbd_mouse_performance_editor_window}

   The "performance editor" window is also known as the "song editor" window.
   It's main sections are the "piano roll" (perfroll) and the "performance
   time" (perftime) sections, discussed in the following sections.
   Also, some keystrokes are handled by the frame of the window.

   \begin{itemize}
      \item \texttt{Ctrl-z}. Undo.
      \item \texttt{Ctrl-r}. Redo.
   \end{itemize}

\subsubsection{Performance Editor Piano Roll}
\label{subsubsec:kbd_mouse_performance_editor_piano_roll}

%  \begin{itemize}
%     \item \texttt{Ctrl-x}. Cut.
%     \item \texttt{Ctrl-c}. Copy.
%     \item \texttt{Ctrl-v}. Paste.
%     \item \texttt{Ctrl-z}. Undo.
%     \item \texttt{Ctrl-r}. Redo.
%     \item \texttt{Shift-Up}.   Move backward one small unit (which is...?)
%     \item \texttt{Shift-Down}.   Move forward one small unit (which is...?)
%     \item \texttt{Shift-Page Up}.   Move backward one frame.
%     \item \texttt{Shift-Page Down}.   Move forward one frame.
%     \item \texttt{Shift-Home, Shift-KP Home}.  Move to beginning of piano roll.
%     \item \texttt{Shift-End, Shift-KP End}.  Move to end of piano roll.
%     \item \texttt{Shift-z (Z)}.  Zoom in.
%     \item \texttt{0}.  Set default zoom.
%     \item \texttt{z}.  Zoom out.
%     \item \texttt{Left}.  Move item left one snap unit.
%     \item \texttt{Right}.  Move item right one snap unit.
%     \item \texttt{Up}.  Move frame up one small scroll unit.
%     \item \texttt{Down}.  Move frame down one small scroll unit.
%     \item \texttt{Home}.  Move to top of piano roll.
%     \item \texttt{End}.  Move to bottom of piano roll.
%     \item \texttt{Page Up}.  Move up one frame (page-increment).
%     \item \texttt{Page Down}.  Move down one frame (page-increment).
%  \end{itemize}

   Note that the keystrokes in this table
   (see \tableref{table:perf_window_piano_roll})
   require that the focus first be
   assigned to the piano roll by left-clicking in an empty area within it.
   Otherwise, another section of the performance editor might receive the
   keystroke.

   \begin{table}[H]
      \centering
      \caption{Performance Window Piano Roll}
      \label{table:perf_window_piano_roll}
      \begin{tabular}{l l l l l l}
         \textbf{Action}   & \textbf{Normal} & \textbf{Double}    & \textbf{Shift}     & \textbf{Ctrl}   & \textbf{Mod4}      \\
         Space             & Start playback  & ---                & ---                & ---             & ---                \\
         Esc               & Stop playback   & ---                & ---                & ---             & ---                \\
         Period (.)        & Pause playback  & ---                & ---                & ---             & ---                \\
         Del               & Cut section     & ---                & ---                & ---             & ---                \\
         c key             & ---             & ---                & ---                & Copy            & ---                \\
         p key             & Paint mode      & ---                & ---                & ---             & ---                \\
         v key             & ---             & ---                & ---                & Paste           & ---                \\
         x key             & Escape paint    & ---                & ---                & Cut             & ---                \\
         z key             & Zoom out        & ---                & ---                & Undo            & ---                \\
         0 key             & Reset zoom      & ---                & ---                & ---             & ---                \\
         Z key             & Zoom in         & ---                & ---                & Undo            & ---                \\
         Left-arrow        & Move earlier    & ---                & ---                & ---             & ---                \\
         Right-arrow       & Move later      & ---                & ---                & ---             & ---                \\
         Left-click        & Select section  & ---                & ---                & ---             & ---                \\
         Right-click       & Paint mode      & ---                & Paint mode         & Paint mode      & Lock Paint mode    \\
         Scroll-up         & Scroll up       & ---                & Scroll Left        & Scroll Up       & ---                \\
         Scroll-down       & Scroll down     & ---                & Scroll Right       & Scroll Down     & ---                \\
      \end{tabular}
   \end{table}

   This section of the performance editor also handles the start, stop, and
   pause keys.  These can be modified in the \textbf{Options / Keyboard} page.
   A "section" in the performance editor is actually a box that
   specifies a trigger for the pattern in that sequence/pattern slot.
   Note that the "toggle other slots" action occurs only if shift-left-clicked
   in the "names" area of the performance editor.
   Left-click is used to select performance blocks if clicked within
   a block, or to deselect them if clicked in an empty area of the piano roll.
   Also note that all scrolling is done by the internal horizontal and vertical
   step increments.
   Some features of this window for \textsl{Seq66},
   as noted in \sectionref{sec:song_editor}, are explained here:

   \begin{itemize}
      \item \textsl{p}:  Enters the paint mode, until right-click is pressed or
         until the "x" key is pressed.
      \item \textsl{x}:  Exits the paint mode.  Think of the made-up term
         "x-scape".
      \item \textsl{z}:  Zooms out the performance view.  It makes the view
         look smaller, so that more of the performance can be seen.
         Opening a second performance view is another way to see more
         of the performance.
      \item \textsl{0}:  Resets the zoom to its normal value.
      \item \textsl{Z}:  Zooms in the performance view, making the view
         larger, so that more details of the performance can be seen.
%     \item \textsl{.}:  The period (configurable) is a new key devoted to the
%        new pause functionality.
      \item \textsl{Left Arrow}:  Moves the selected item to the left (earlier
         in time) in the performance layout.
      \item \textsl{Right Arrow}:  Moves the selected item to the right (later
         in time) in the performance layout.
      \item \textsl{Mod4-right-click, release}:  Locks the paint mode,
         until right-click is pressed.
      \item Once selected (rendered in grey), a pattern section (trigger)
         can be moved by the mouse.
         To move it using the left or right
         arrow keys, the paint mode must be entered, but only via the "p"
         key.
%        -- the right mouse button deselects the greyed pattern.
%        Too tricky, we might try fixing it later.
   \end{itemize}

\subsubsection{Performance Editor Time Section}
\label{subsubsec:kbd_mouse_performance_editor_time_section}

   \begin{itemize}
      \item \texttt{l}.  Set to move L marker.
      \item \texttt{r}.  Set to move R marker.
      \item \texttt{x}.  Escape ("x-scape") the movement mode.
      \item \texttt{Left}.  Move the selected marker left.
      \item \texttt{Right}.  Move the selected marker right.
   \end{itemize}

   This section of the performance editor is also known as the "measure ruler"
   or the "bar indicator".

   \begin{table}[H]
      \centering
      \caption{Performance Editor Time Section}
      \label{table:performance_editor_time_section}
      \begin{tabular}{l l l l l l}
         \textbf{Action}   & \textbf{Normal} & \textbf{Double}    & \textbf{Shift} & \textbf{Ctrl}   & \textbf{Mod4}      \\
         l                 & Move L [1]      & ---                & ---            & ---             & ---                \\
         r                 & Move R [1]      & ---                & ---            & ---             & ---                \\
         x                 & Escape Move     & ---                & ---            & ---             & ---                \\
         Left-Click        & Set L [2]       & ---                & ---            & ---             & ---                \\
         Middle-Click      & ---             & ---                & ---            & ---             & ---                \\
         Right-Click       & Set R [2]       & ---                & ---            & ---             & ---                \\
      \end{tabular}
   \end{table}

   \begin{enumerate}
      \item Activates movement of this marker using the left and right arrow
         keys.  Movement is in increments of the snap value.  This mode is
         exited by pressing the 'x' key.  Also see note [2].
      \item Controlled in the pertime section.
   \end{enumerate}

   The features of this window for \textsl{Seq66} are:

   \begin{itemize}
      \item \textsl{l}:  Enters a mode where the left and right arrow keys move
         the L marker, until the "x" key is pressed.
      \item \textsl{r}:  Enters a mode where the left and right arrow keys move
         the R marker, until the "x" key is pressed.
      \item \textsl{x}:  Exits the marker-movement  mode.
   \end{itemize}

\subsubsection{Performance Editor Names Section}
\label{subsubsec:kbd_mouse_performance_editor_names_section}

   \begin{table}[H]
      \centering
      \caption{Performance Editor Names Section}
      \label{table:performance_editor_names}
      \begin{tabular}{l l l l l l}
         \textbf{Action}   & \textbf{Normal}    & \textbf{Double}    & \textbf{Shift}        & \textbf{Ctrl}   & \textbf{Mod4}      \\
         Left-Click        & Toggle track       & ---                & Toggle other tracks   & ---             & ---                \\
         Middle-Click      & ---                & ---                & ---                   & ---             & ---                \\
         Right-Click       & New/Edit menu      & ---                & ---                   & ---             & ---                \\
      \end{tabular}
   \end{table}

\subsection{Pattern Editor}
\label{subsec:kbd_mouse_pattern_editor}

   The pattern/sequencer editor piano roll is a complex and powerful event
   editor;
   \tableref{table:pattern_editor_piano_roll},
   doesn't begin to cover its functionality.
   Here are some keystrokes handled by the main frame of the piano roll:

   \begin{itemize}
      \item \texttt{Ctrl-L}.  Bring up the LFO event modulation editor.
      \item \texttt{Ctrl-W}.  Exit the sequence (pattern) editor.
      \item \texttt{Ctrl-Page Up}.  Zoom in.
      \item \texttt{Ctrl-Page Down}.  Zoom out.
      \item \texttt{Shift-Page Up}.  Scroll leftward.
      \item \texttt{Shift-Page Down}.  Scroll rightward.
      \item \texttt{Shift-Home}.  Scroll leftward to the beginning.
      \item \texttt{Shift-End}.  Scroll rightward to the end.
%     \item \texttt{Shift-z (Z)}.  Zoom in.
%     \item \texttt{0}.  Set default zoom.
%     \item \texttt{z}.  Zoom out.
      \item \texttt{Page Down}.  Scroll downward.
      \item \texttt{Page Up}.  Scroll upward.
      \item \texttt{Home}.  Scroll upward to the beginning.
      \item \texttt{End}.  Scroll downward to the end.
      \item \texttt{Delete}.  Deletes (not cuts) the currently-selected notes
         in the piano roll; can be undone with the \textbf{Undo} button.
   \end{itemize}

\subsubsection{Pattern Editor Piano Roll}
\label{subsubsec:kbd_mouse_pattern_editor_piano_roll}

   Here are the keystrokes handled by the piano roll:
   These keystrokes require that the focus be set to the piano roll by clicking
   in it with the mouse.

   \begin{itemize}
%     \item \texttt{Ctrl-x}. Cut.
%     \item \texttt{Ctrl-c}. Copy.
%     \item \texttt{Ctrl-v}. Paste.
%     \item \texttt{Ctrl-z}. Undo.
      \item \texttt{Ctrl-r}. Redo.
      \item \texttt{Ctrl-a}. Select all.
      \item \texttt{Ctrl-Left}.  Shrink selected notes.
      \item \texttt{Ctrl-Right}.  Grow selected notes.
      \item \texttt{Delete}.  Remove selected notes.
      \item \texttt{Backspade}.  Remove selected notes.
      \item \texttt{Home.  Set sequence to beginnging of sequence}.  (Verify!)
%     \item \texttt{Left}.  Move selected notes one snap left.
%     \item \texttt{Down}.  Move selected notes one pitch downward.
%     \item \texttt{Up}.  Move selected notes one pitch upward.
      \item \texttt{Enter, Return}.
         Paste the selected notes at the current position.
%     \item \texttt{p}.  Enter "paint" (also known as "adding") mode.
%     \item \texttt{x}.  Escape ("x-scape") the paint mode.
   \end{itemize}

   And here is the table, which includes items not described above:

   \begin{table}[H]
      \centering
      \caption{Pattern Editor Piano Roll}
      \label{table:pattern_editor_piano_roll}
      \begin{tabular}{l l l l l l}
         \textbf{Action}   & \textbf{Normal} & \textbf{Double}    & \textbf{Shift} & \textbf{Ctrl}   & \textbf{Mod4}      \\
         Del               & Delete Selected & ---                & ---            & ---             & ---                \\
         c                 & ---             & ---                & ---            & Copy            & ---                \\
         p                 & Paint mode      & ---                & ---            & ---             & ---                \\
         v                 & ---             & ---                & ---            & Paste           & ---                \\
         x                 & Escape Paint    & ---                & ---            & Cut             & ---                \\
         z                 & Zoom Out        & ---                & Zoom In        & Undo            & ---                \\
         0                 & Reset Zoom      & ---                & ---            & ---             & ---                \\
         Left-Arrow        & Move Earlier [1] & ---               & ---            & ---             & ---                \\
         Right-Arrow       & Move Later [1]  & ---                & ---            & ---             & ---                \\
         Up-Arrow          & Increase Pitch  & ---                & ---            & ---             & ---                \\
         Down-Arrow        & Decrease Pitch  & ---                & ---            & ---             & ---                \\
         Left-Click        & Deselect        & ---                & ---            & ---             & ---                \\
         Right-Click       & Paint mode      & ---                & Edit Menu      & Edit/Edit Menu  & Lock Paint mode    \\
         Left-Middle-Click & Grow Selected   & ---                & Stretch Sel.   & ---             & ---                \\
         Scroll-Up         & Zoom Time In    & ---                & Scroll Left    & Zoom Time In    & ---                \\
         Scroll-Down       & Zoom Time Out   & ---                & Scroll Right   & Zoom Time Out   & ---                \\
      \end{tabular}
   \end{table}

   \begin{enumerate}
      \item Once selected (and thus rendered in grey), a pattern segment
         can be moved by the mouse.  To move it using the left or right
         arrow keys, the paint mode must be entered, but only via the
         \texttt{p} key -- the right mouse button deselects the greyed pattern.
         Too tricky, we might try fixing it later.
   \end{enumerate}

   Features of this window section for \textsl{Seq66}, as noted in
   \sectionref{subsubsec:pattern_editor_piano_roll_items}, are:

   \begin{itemize}
      \item \textsl{p}:  Enters the paint mode, until right-click is pressed or
         until the \texttt{x} key is pressed.  Notes are added
         by clicking or click-dragging.
      \item \textsl{x}:  Exits ("x-scapes") the paint mode.
      \item \textsl{z}:  Zooms out.
      \item \textsl{0}:  Resets zoom to its normal value.
      \item \textsl{Z}:  Zooms in.
      \item \textsl{.}:  The period (configurable) does the pause function.
      \item \textsl{Left Arrow}:  Moves selected events to the left.
      \item \textsl{Right Arrow}:  Moves selected events to the right.
      \item \textsl{Up Arrow}:  Moves selected notes upward in pitch.
      \item \textsl{Down Arrow}:  Moves selected notes downward in pitch.
      \item \textsl{Mod4-Right-Click}:  Locks the paint mode, until right-click
         is pressed again.
   \end{itemize}

\subsubsection{Pattern Editor Event Panel}
\label{subsubsec:kbd_mouse_pattern_editor_event_panel}

   \begin{itemize}
      \item \texttt{Ctrl-x}. Cut.
      \item \texttt{Ctrl-c}. Copy.
      \item \texttt{Ctrl-v}. Paste.
      \item \texttt{Ctrl-z}. Undo.
      \item \texttt{Delete}.  Delete (not cut!) the selected events.
      \item \texttt{p}.  Enter "paint" (also known as "adding") mode.
      \item \texttt{x}.  Escape ("x-scape") the paint mode.
   \end{itemize}

\subsubsection{Pattern Editor Data Panel}
\label{subsubsec:kbd_mouse_pattern_editor_data_panel}

   Currently, no keystroke support is provided in the data panel.
   One potential upgrade would be the ability to change the value of the event
   with the Up and Down arrow keys.

\subsubsection{Pattern Editor Virtual Keyboard}
\label{subsubsec:kbd_mouse_pattern_editor_virtual_keyboard}

   \begin{table}[H]
      \centering
      \caption{Pattern Editor Virtual Piano Keyboard}
      \label{table:pattern_editor_virtual_piano_keyboard}
      \begin{tabular}{l l l l l l}
         \textbf{Action}   & \textbf{Normal} & \textbf{Double}    & \textbf{Shift} & \textbf{Ctrl}   & \textbf{Mod4}      \\
         Left-Click        & Play note       & ---                & ---            & ---             & ---                \\
         Right-Click       & Toggle labels   & ---                & ---            & ---             & ---                \\
      \end{tabular}
   \end{table}

\subsection{Event Editor}
\label{subsec:kbd_mouse_event_editor}

   \begin{itemize}
      \item \texttt{Down}.  Move one slot down.
      \item \texttt{Up}.  Move one slot up.
      \item \texttt{Page Down}.  Move one frame down.
      \item \texttt{Page Up}.  Move one frame up.
      \item \texttt{Home}.  Move to top frame.
      \item \texttt{End}.  Move to bottom frame.
      \item \texttt{Asterisk, KP Multiply}.  Delete the currently-selected event.
   \end{itemize}

%-------------------------------------------------------------------------------
% vim: ts=3 sw=3 et ft=tex
%-------------------------------------------------------------------------------


% Meta-event support

%%% %-------------------------------------------------------------------------------
% meta_events
%-------------------------------------------------------------------------------
%
% \file        meta_events.tex
% \library     Documents
% \author      Chris Ahlstrom
% \date        2017-07-23
% \update      2023-05-02
% \version     $Revision$
% \license     $XPC_GPL_LICENSE$
%
%     Provides a discussion of the MIDI GUI meta_events that Seq66
%     supports.
%
%-------------------------------------------------------------------------------

\section{Seq66 Meta Event / SysEx Support}
\label{sec:meta_events}

   \textsl{Seq66} attempts better support
   for MIDI Meta and System Exclusive events and a Tempo track.
   It supports the display of Set Tempo and Time Signature events.
   They can also be added and edited, in
   various ways.  For example, see \sectionref{sec:event_editor}.

   For the internal format of Meta events,
   see \sectionref{subsec:midi_format_meta_format}.

   Only the first Time Signature event is used to modify playback.
   System Exclusive support is also still in progress, but very incomplete.
   This section consolidates the description of the meta-event support.
   The following topics apply:

   \begin{enumerate}
      \item Tempo display min/max in 'usr' settings.
      \item Tempo display in main window.
      \item Tempo display in pattern editor.
      \item Tempo display in song editor.
      \item Tempo and Time signature display and editing in the event editor.
   \end{enumerate}

   First, we need to note \textsl{how} the tempo track is
   implemented in \textsl{Seq66}.  Rather than make a SeqSpec track for
   the tempo events, we use the MIDI specification mandate that
   Tempo events should occur only in the first track.
   \textsl{Seq66} treats Set Tempo and Time Signature as full-fledged
   MIDI events that can be viewed (and later, edited) in the existing
   user-interface.  Notes and other events can occur in the same
   track.
%  To reiterate, track 1 (pattern 0) is the only track where tempo events
%  can be placed and edited.

\subsection{'usr' BPM Display Settings}
\label{subsec:meta_events_usr}

   \textsl{Seq66} allows the tempo to range from 1 to 600 BPM
   (beats per minute).
   This range is hardwired into the application.
   To display tempo with a little more granularity,
   \textsl{Seq66} provides scaling for the tempo
   displays.  These values are found in the 'usr' file:

   \begin{verbatim}
		0         # midi_bpm_minimum
		360       # midi_bpm_maximum
   \end{verbatim}

   This setting can only be made by editing the 'usr' file
   while \textsl{Seq66} is not running.
   Note that this setting affects the global BPM setting ("c\_bpmtag").

\begin{comment}

\subsection{Composite Display of Tempos}
\label{subsec:meta_events_composite_display}

The following figure shows a composite picture of the various representations
of Set Tempo events.

\begin{figure}[H]
   \centering 
   \includegraphics[scale=0.65]{roll.png}
   \caption{Various Tempo Displays}
   \label{fig:meta_events_tempo_displays}
\end{figure}

The \textsl{top} of the figure shows the magenta tempo lines in a pattern slot
that is currently being edited.  This view edited, but the
event editor and the main window's BPM settings can be used to add, delete, or
adjust the tempo.
The \textsl{middle} panel shows the very similar representation of the tempo in
the song editor.  This view does not allow editing of the tempo events.
The \textsl{bottom} shows tempo as an event (in the event strip) and a data
value in the data pane.  A tempo event can be added here by holding the Ctrl
key and painting an event in the event strip, and it can then be modified by
same method that note velocities can be edited.  Tempo events are
\textsl{always} shown in the event strip and the data pane, no matter what
other \textbf{Event} type has been selected.

\end{comment}

\begin{comment}

\subsection{Tempo in the Main Window}
\label{subsec:meta_events_mainwnd}

The tempo is shown as a solid magenta-colored line at the relative height
for the tempo,
based on the minimum and maximum values configured in the 'usr' file as
discussed above.
This pattern-slot tempo display is rudimentary.  It doesn't allow for ramping
of the tempo at present (except by recording while holding the BPM
spin-control), and cannot be directly edited in this window.
However, tempos can be logged or recorded via magenta-colored controls at the
bottom of the main window.

\begin{figure}[H]
   \centering 
   \includegraphics[scale=0.65]{roll.png}
   \caption{Tempo Recording Controls}
   \label{fig:meta_events_mainwnd_tempo_recording}
\end{figure}

The 0th pattern slot shown in the figure is Track 1, the
MIDI Tempo track.  The magenta lines show the tempos already in that track.
Now look at the BPM control.  The first button to its right ("0") is the
tempo-tap button, used for setting a tempo by tapping in time to music.
The light-magenta button that comes next, when pressed while playback is
occurring, logs a tempo event at the current progress location and the
current BPM value in the BPM spin-field.  The dark magenta button to the right
of that toggles the mode of recording the changes to the BPM spin-button while
playback is occurring.

Although pattern 0 might start out with a length of only a
measure or two, the timer continually ticks upward, and tempo events that
are recorded after the end of the track at still recorded, and
\textsl{they will extend the length of the tempo track}.
The length of each track, in measures, is shown at the top right of each main
window pattern slot, so it can be tracked by the user.

Once tempo events have been recorded, they can be tweaked (or deleted)
either in the pattern editor or in the event editor.  Generally, they are
treated like control events that are always available.  Deleting all tempo
events will not reduce the (possibly new) length of the sequence.
The Tempo track will \textsl{not} change tempo unless that track is unmuted.
This behavior is a feature, not a bug.

\end{comment}

%-------------------------------------------------------------------------------
% vim: ts=3 sw=3 et ft=tex
%-------------------------------------------------------------------------------


% Configuration file

%%% %-------------------------------------------------------------------------------
% seq66_rc_file
%-------------------------------------------------------------------------------
%
% \file        seq66_rc_file.tex
% \library     Documents
% \author      Chris Ahlstrom
% \date        2015-08-31
% \update      2018-11-11
% \version     $Revision$
% \license     $XPC_GPL_LICENSE$
%
%     Provides the rc_file.
%
%-------------------------------------------------------------------------------

\section{Seq66 "rc" Configuration File}
\label{sec:seq66_rc_file}

   There are two \textsl{Seq66} configuration files:
   \texttt{sequencer66.rc} and \texttt{sequencer66.usr}.
   See \sectionref{sec:seq66_usr_file}; it describes the "usr" file,
   is handled a bit differently than the "rc" file.

   \index{sequencer66.rc}
   The \textsl{Seq66} configuration file originally was
   named \texttt{.seq24rc},
   and it was stored directly in the user's \texttt{\$HOME} directory,
   following the convention of \textsl{Seq24}.
   To avoid interference with one's existing installation of 
   \textsl{Seq24}, we created a new file
   to take its place, with a fall-back to the original file-name if the new
   file does not exist, or if \textsl{Seq66} is running in
   \index{legacy mode}
   legacy mode.
   In addition, one can change the configuration directory and the base name of
   the "rc" and "usr" files from the command-line.

   After you run \textsl{Seq66} for the first time (in non-legacy
   mode), it will generate a \texttt{sequencer66.rc} file in your home
   \textsl{configuration} directory:

   \begin{verbatim}
      /home/ahlstrom/.config/sequencer66/sequencer66.rc
   \end{verbatim}

   It contains the the data for remote MIDI control, computer keyboard
   control, MIDI clock, JACK transport, and a few other settings.

   \textsl{Seq66} will
   \textsl{always} overwrite the \texttt{sequencer6.rc} file upon
   quitting.  One must therefore quit \textsl{Seq66} before making
   manual modifications to the \texttt{sequencer66.rc} file.
   Note that many of
   its settings can be modified in the \textbf{Options} dialog
   (see \sectionref{subsubsec:seq66_menu_file_options}).
   There is an old, but complete, example of the \textsl{Seq24}
   "rc" file at \cite{seq24launchpadmapper}.
   It includes a setup for the
   Novation Launchpad device.

   Now let's covered each section of the "rc" file in order.

\subsection{"rc" File / Comments}
\label{subsec:seq66_rc_file_midi_comments}

   The very top of the "rc" file that \textsl{Seq66} generates is a stock
   banner showing the version of \textsl{Seq66} to which this file
   applies, the name of the configuration file, and when it was written.  The
   user can also add a comments section that explain what the user's setup is
   in brief:

   \begin{verbatim}
   [comments]

   Comments added to this section are preserved.  Lines starting with
   a '#' or '[', or that are blank, are ignored.  Start lines that must
   be blank with a space.
   \end{verbatim}

   A blank line (not even a space) ends the comment section.

\subsection{"rc" File / MIDI Control}
\label{subsec:seq66_rc_file_midi_control}

   Like \textsl{Seq24}, \textsl{Seq66} provides a way to control the
   application to some extent via a MIDI controller, such as a MIDI keyboard or
   a MIDI pad device.  The current section describes this feature;
   additional resources and ideas can be found at \url{linuxaudio.org}
   (\cite{midicontrol}).

   \index{[midi-control-file]}
   New with version 0.96 of \textsl{Seq66} is the ability
   to offload the MIDI control section to a separate file.  Simply move
   the whole \texttt{[midi-control]} section to a separate file in
   the \textsl{Seq66} configuration directory, and add the following
   snippet:

   \begin{verbatim}
   [midi-control-file]
      nanomap.rc        # contains a whole [midi-control] section
   \end{verbatim}

   As with the normal "rc", this file is rewritten upon exit, so
   don't bother trying to add comments to it.  The rest of this section
   applied either to that "rc" or the normal "rc" file.

   \index{[midi-control]}
   The MIDI control section begins with the following "INI"-style
   group marker tag:

   \begin{verbatim}
   [midi-control]
      74      # MIDI controls count (74/84/96)
   \end{verbatim}

   The number (74) is the number of lines in the MIDI Control section.  The
   extended automation values bring this number up to 84, and the most recent
   version of \textsl{Seq66} brings this up to 96.  Note that earlier
   version will complain about numbers higher than what they can handle.  If
   so, edit the "rc" file to reduce that number.

   Even in the latest version of \textsl{Seq66}, 
   not all of these new values are yet usable, and there are also some values
   reserved for future expansion.  Currently, the
   "start", "pause", "stop", and "bpm" page controls, and the "performance
   record", "MIDI THRU", "MIDI RECORD", "MIDI Quantized RECORD",
   and play-list control have been implemented.
   Each MIDI control line has the following format:

   \begin{verbatim}
      74     [0 0 0 0 0 0]   [0 0 0 0 0 0]   [0 0 0 0 0 0]
   \end{verbatim}

   The first number is an \textsl{internal} control number ranging from
   0 to 73, 83, or 95 depending on the version of \textsl{Seq66}.
   The three bracketed sections represent MIDI controls to
   \index{[midi-control]!toggle}
   toggle,
   \index{[midi-control]!on}
   turn on, and
   \index{[midi-control]!off}
   turn off a pattern or mute group.
   However, some MIDI controls extend the meanings of these brackets
   for additional functionality.

   \begin{verbatim}
               ------------------ on/off
              | ----------------- inverse
              | |  -------------- MIDI status (event) byte (e.g. note on)
              | | |  ------------ data 1 (e.g. note number)
              | | | |  ---------- data 2 min
              | | | | |  -------- data 2 max
              | | | | | |
              v v v v v v
      74     [0 0 0 0 0 0]   [0 0 0 0 0 0]   [0 0 0 0 0 0]
    Index:    Toggle          On              Off
    Playback: Pause           Start/Play      Stop
    Playlist: By-Value        Next            Previous
   \end{verbatim}

   The first number is an index number, starting at 0.  It indicates what
   function the control line will affect.
   The numbers in the leftmost brackets define a \textsl{toggle} filter;
   the numbers in the middle brackets define a \textsl{on} filter;
   the numbers in the rightmost brackets define a \textsl{off} filter.
   Additional functions are shown that extend these basic functions,
   such as changing a selection, controlling playback, or activating a feature.

   The numbers inside the brackets define six values that set up the control.
   The layout of each filter inside the brackets is as follows:

      \textbf{[OPR INV STAT D1 D2min D2max]}

   \begin{itemize}
      \item \textbf{OPR} = \textbf{on/off}
      \item \textbf{INV} = \textbf{inverse}
      \item \textbf{STAT} = \textbf{MIDI status byte} (channel ignored) 
      \item \textbf{D1} = \textbf{data1}
      \item \textbf{D2min} = \textbf{data2 min}
      \item \textbf{D2max} = \textbf{data2 max}
   \end{itemize}

   If \textbf{OPR (on/off)} is set to 1, it will match the incoming MIDI
   against the \textbf{STAT (MIDI status byte)} pattern.
   and perform the action (on/off/toggle) if the data
   falls in the range specified.  All values are in decimal.

   \textbf{Note}: In legacy versions (\textsl{Seq24} and early versions
   of \textsl{Seq66}), the channel nybble of the MIDI control (and all
   other incoming MIDI events) were stripped off.
   This is no longer the case, and thus opens up many more events useful for
   MIDI control.   But do note that events that are actually recorded end up
   getting the channel number of the pattern into which they are recorded.

   The \textbf{INV (inverse)} field will make the pattern perform the opposite
   action (\textsl{off} for \textsl{on}, \textsl{on} for \textsl{off}) if the
   data falls outside the specified range.  This is cool because one can map
   several sequences to a knob or fader.

   The \textbf{STAT (MIDI status byte)} field is a MIDI status byte number in
   decimal notation.  The channel nybble of this byte is ignored.  One can look
   up the possible status values up in the MIDI messages tables; the relevant
   data can be found at \cite{midicontroltable}.  As the channel on which the
   events are sent is ignored, it is sufficient to use the values for channel
   1; that is, 0.

   The last three fields describe the range of data that will match.  The
   \textbf{D1 (data1)} field provides the actual MIDI event message number to
   detect, in decimal.  This item could be a Note On/Off event or a
   Control/Mode change event, for example.

   The \textbf{D2min (data2 min)} field is the minimum value of the event for
   the filter to match. For Note On/Off events, this would be the velocity
   value, for example.

   The \textbf{D2max (data2 max)} field is the maximum value of the event for
   the filter to match.

%  This set of values is explained below.

   For each pattern, we can set up MIDI events to turn a 
   pattern on, off, or to toggle it, or to control some other function.
   If the incoming MIDI event value matches a value present in the filter, it
   will \textsl{toggle} (first field), \textsl{enable} (second field) or
   \textsl{disable} (third field) the sequence, or perform some kind of automation
   control.

   As a quick example, let us set up a Note On event on channel 2 and key value
   48 that will increment \textsl{Seq66}'s BPM value each time it is
   pressed.  A Note On event is 0x90 hex, or 144 decimal.  Channel 2 is 0x1 hex
   or 1 decimal.   Adding the Note On value to the the channel number yields
   0x91 hex, or 145 decimal.  Then we pick a note value of 48, which is one of
   the C keys.  We don't care about the velocity, so we allow all values (0 to
   127).  We add the following entry to
   \texttt{\textasciitilde/.config/sequencer66/sequencer66.rc}):

   \begin{verbatim}
      [midi-control]
       . . .
      # bpm up:
      64 [0 0   0   0   0   0] [1 0 145  48   0 127] [0 0   0   0   0   0]
   \end{verbatim}

   Now, whenever we press key 48 (C), we see that the BPM value in the main
   patterns panel increments by 1.

   The MIDI control setup resembles a matrix.  This matrix is divided into a
   number of sections depending on the overall functionality of the MIDI
   controls in the section:

   \index{rc!pattern-group}
   \index{rc!mute-in-group}
   \index{rc!automation-group}
   \index{rc!extended-group}
   \begin{enumerate}
      \item \textbf{Pattern Group} (rows 0 to 31).
      \item \textbf{Mute-In Group} (rows 32 to 63).
      \item \textbf{Automation Group} (rows 64 to 73).
      \item \textbf{Extended Automation Group} (row 74 to 95).
   \end{enumerate}

\subsubsection{"rc" File / MIDI Control / Pattern Group}
\label{subsubsec:seq66_rc_file_midi_ctrl_pattern}

   The pattern group consists of 32 lines (0 to 31), one for each
   pattern slot shown in the Pattern window.
   It provides a way to control the arming/disarming (muting/unmuting) of
   each pattern shown in the main window.  Note that the main window
   shows the \textsl{active} screen-set.  These MIDI controls affect the
   \textsl{active} screen-set.

   This block of matrix elements, numbered from 0 to 31,
   represent control functions (toggle, mute, unmute) for the 32 patterns
   of the active screen-set.
   These 32 rows correspond to the hot-keys assigned in
   the \textbf{File / Options / Keyboard / Control keys [keyboard-group]} 
   configuration panel.

   Here is an example of setting mute/unmute control for the first four
   patterns of the active screen-set.  Note that the numbers are decimal
   numbers, and that 144 is 90 hex (a Note On event) and 128 is 80 hex (a Note
   Off event).

   \begin{verbatim}
                          on (enabled)------------
                            |  Note On            | Note Off
                            |    |  Note #        |    | Note #
                            |    |    |   Vel     |    |    |
       --- off (disabled)   |    |    |   Range   |    |    |
      |                     |    |    |   |  |    |    |    |
      v                     v    v    v   v  v    v    v    v
   0 [0 0   0   0   0   0] [1 0 144   0   0 127] [1 0 128   0   0 127]
   1 [0 0   0   0   0   0] [1 0 144  16   0 127] [1 0 128  16   0 127]
   2 [0 0   0   0   0   0] [1 0 144  32   0 127] [1 0 128  32   0 127]
   3 [0 0   0   0   0   0] [1 0 144  48   0 127] [1 0 128  48   0 127]
   #    Toggle                 On                      Off
   \end{verbatim}

   The "toggle" section is empty and disabled.  The "on" section specifies that
   a Note On event of any velocity will unmute a pattern, and the note number
   determines which pattern.  A Note Off of any velocity will mute a pattern.
   These settings are for the control grid of a Novation Launchpad.

   Look at line "0.
   The first number, 0, indicates the first pattern (pattern numbering starts
   from 0).
   The first section, \textbf{Toggle}, is off (inactive).  All values are 0.
   There is no setup to use MIDI control to toggle pattern 1 here.
   
   On to the second section, \textbf{On}:

   \begin{itemize}
      \item The \textbf{On} section starts with \textbf{OPR} = 1,
         so it is on (1 = active).
      \item The \textbf{inverse} value is off (0 = inactive).
      \item The \textbf{MIDI status byte}, 144, which is 0x90 (hex), which
         is a Note On event on channel 0.  However, the channel is ignored.
      \item The \textbf{data1} values sets the actual Note value to 0,
         meaning the lowest possible MIDI note (pitch) value.
      \item \textbf{data2 min} value sets the minimum value to 0.
      \item \textbf{data2 max} sets the maximum value to 127.
   \end{itemize}

   Thus, receiving any Note On velocity for note 0 will turn sequence
   1 \textsl{on}.  This is the second pattern; in the default setup, key
   \texttt{q} would operate on this pattern as well.
   
   On to the \textbf{Off} section:

   \begin{itemize}
      \item The \textbf{Off} field is on (active).
      \item The \textbf{inverse} value is off (0 = inactive).
      \item The \textbf{MIDI status byte}, 128, which is 0x80 (hex), which
         128, which is 0x80 (hex), which is a Note Off event on channel 0.
      \item The \textbf{data1} values sets the actual Note value to 0,
         meaning the lowest possible MIDI note (pitch) value.
      \item \textbf{data2 min} value sets the minimum value to 0.
      \item \textbf{data2 max} sets the maximum value to 127.
   \end{itemize}

   Thus, receiving any Note Off velocity for note 0 will turn sequence
   1 \textsl{off}.

   So, basically, pattern 1 starts when any Note On for MIDI note 0
   is received, and it stops when any Note Off for MIDI note 0 is received.  
   One can easily extend this so that Note On/Off values from 0 to 31
   control the corresponding pattern slot.

   Obviously, one might not want Note On/Off events from any channel to trigger
   events, so some other event would likely be more useful.
   Here's a little table of the decimal numbers for some commonly-used MIDI
   controls:

   \begin{itemize}
      \item \textbf{128} or \textbf{129} for any Note On or Note Off events.
      \item \textbf{160} Polyphonic aftertouch.
      \item \textbf{176} Control Change event.
      \item \textbf{192} Program change.
      \item \textbf{208} Aftertouch.
      \item \textbf{224} Pitch wheel.
   \end{itemize}

   The following example would map a row of sequences to one knob sending
   out changes for Control Code 1:

   \begin{verbatim}
        #    Toggle                 On                      Off
        0 [0 0 0 0 0 0]      [1 1 176 1   0   15]     [0 0 0 0 0 0]
        1 [0 0 0 0 0 0]      [1 1 176 1  16   31]     [0 0 0 0 0 0]
        2 [0 0 0 0 0 0]      [1 1 176 1  32   47]     [0 0 0 0 0 0]
        3 [0 0 0 0 0 0]      [1 1 176 1  48   63]     [0 0 0 0 0 0]
        4 [0 0 0 0 0 0]      [1 1 176 1  64   79]     [0 0 0 0 0 0]
        5 [0 0 0 0 0 0]      [1 1 176 1  80   95]     [0 0 0 0 0 0]
        6 [0 0 0 0 0 0]      [1 1 176 1  96  111]     [0 0 0 0 0 0]
        7 [0 0 0 0 0 0]      [1 1 176 1 112  127]     [0 0 0 0 0 0]
   \end{verbatim}

   The \textbf{on} field is on (active).  Inverse is active.  The
   \textbf{MIDI status byte}, 176, is 0xB0 (hex), which is a Control Change
   event (channel ignored).  \textbf{data1} is 1, which is the controller
   number for a Modulation Wheel.  The \textbf{data2} ranges are set so
   that, as the controller data increases (as the modulation-wheel knob is
   turned, so to speak), patterns 0 through 7 come on one at a time until
   all are running.

\subsubsection{"rc" File / MIDI Control / Pattern Group Multiples}
\label{subsubsec:seq66_rc_file_midi_ctrl_pattern_mult}

   This section describes a feature of the pattern-group that needs it own
   section for emphasis.  This section describes using a single MIDI control to
   control a number of operations at one time.  Essentially, if a particular
   MIDI control row is repeated, each repetition has its own effect on the
   patterns, which permits one MIDI control event to control multiple patterns
   at once.

   This control can be used, for example, to emulate the \textsl{Ableton Live
   row control} functionality.  Here is a sample that uses the lowest range of
   MIDI notes to control the muting and unmuting of patterns:

   \begin{verbatim}
      # Pattern-group section:
      0 [0 0 0 0 0 0]   [1 0 144  0   0 127]  [1 0 128    0   0 127]
      1 [0 0 0 0 0 0]   [1 0 144  1   0 127]  [1 0 128    1   0 127]
      2 [0 0 0 0 0 0]   [1 0 144  2   0 127]  [1 0 128    2   0 127]
      3 [0 0 0 0 0 0]   [1 0 144  3   0 127]  [1 0 128    3   0 127]
      4 [0 0 0 0 0 0]   [1 0 144  0   0 127]  [1 0 128    0   0 127]
      5 [0 0 0 0 0 0]   [1 0 144  1   0 127]  [1 0 128    1   0 127]
      6 [0 0 0 0 0 0]   [1 0 144  2   0 127]  [1 0 128    2   0 127]
      7 [0 0 0 0 0 0]   [1 0 144  3   0 127]  [1 0 128    3   0 127]
      8 [0 0 0 0 0 0]   [1 0 144  0   0 127]  [1 0 128    0   0 127]
      9 [0 0 0 0 0 0]   [1 0 144  1   0 127]  [1 0 128    1   0 127]
      10 [0 0 0 0 0 0]  [1 0 144  2   0 127]  [1 0 128    2   0 127]
      11 [0 0 0 0 0 0]  [1 0 144  3   0 127]  [1 0 128    3   0 127]
      12 [0 0 0 0 0 0]  [1 0 144  0   0 127]  [1 0 128    0   0 127]
      13 [0 0 0 0 0 0]  [1 0 144  1   0 127]  [1 0 128    1   0 127]
      14 [0 0 0 0 0 0]  [1 0 144  2   0 127]  [1 0 128    2   0 127]
      15 [0 0 0 0 0 0]  [1 0 144  3   0 127]  [1 0 128    3   0 127]
   \end{verbatim}

   Observer that MIDI On (144) and Off (128) events appear four times for
   each note value of 0, 1, 2, and 3.  Each note value thus controls four
   patterns -- one whole row in a 4x8 pattern.  When note 0 is pressed,
   patterns 0, 4, 8, and 12 turn on.  When note 0 is released, they turn off.

\subsubsection{"rc" File / MIDI Control / Mute-In Group}
\label{subsubsec:seq66_rc_file_midi_ctrl_mutein}

   \index{mute-in group}
   \index{[midi-control]!mute-in group}
   This section controls 32 groups of mutes.
   A group is a set of patterns that can toggle their playing state
   together.  Every group contains all 32 sequences in the active screen set.
   So, this part of the MIDI Control section is used for muting and unmuting
   (and toggling) a group of patterns.

   The mute-in group onsists of 32 lines (32 to 63), one for each
   pattern box shown in the Pattern window.
   It provides a way to control the mute groups.
   A group is a set of sequences that can arm their playing state
   together; every group contains all 32 sequences in the
   \textsl{active} screen-set.

   These 32 values represent the same actions as the
   the \textbf{File / Options / Keyboard / Mute-group slot [mute-group]} 
   configuration panel.
   See \sectionref{paragraph:seq66_menu_file_options_keyboard}.

   What is the different between the \textbf{mute-in group}
   section and the \textbf{mute group} section?  The former defines the MIDI
   control values that can affect the muting of a group, while the latter
   specifies the armed patterns that are part of a group.

\subsubsection{"rc" File / MIDI Control / Automation}
\label{subsubsec:seq66_rc_file_midi_ctrl_automation}

   The automation control keys occupy the entries from 64 to 73.
   These entries control
   \textsl{Seq66} actions like changing the BPM value, screen-set, etc.
   \textsl{Seq66} adds some more entries, from 74 to 83, to control
   additional \textsl{Seq66} functions such as performance
   record, solo, etc.
   
   One issue with this group is that the original control functions are a bit
   wasteful.  For example there are two control lines for BPM:  BPM up and BPM
   down.  These could have been combined into one line, with the "on" group
   meaning "up", and the "off" group meaning "down".  
   \textsl{Seq66} at present does not change this setup, to avoid
   breaking existing MIDI control sections.

   Each item in this group consists of one line.  Each line
   specifies a MIDI event that can cause a given
   \textsl{Seq66} user-interface operation to occur.
   Here are the original automation-group settings:

   \begin{enumerate}
      \item \textbf{bpm up}
      \item \textbf{bpm down}
      \item \textbf{screen-set up}
      \item \textbf{screen-set down}
      \item \textbf{mod replace}
      \item \textbf{mod snapshot}
      \item \textbf{mod queue}
      \item \textbf{mod gmute}
      \item \textbf{mod glearn}
      \item \textbf{screen-set play}
   \end{enumerate}

\paragraph{Automation / BPM Up and Down}
\label{paragraph:seq66_rc_file_midi_ctrl_bpmupdn}

   The BPM Up MIDI control increments the beats-per-minute setting, as if
   the up-arrow has been clicked, or the up-arrow key pressed, in
   the BPM user-interface control.
   This increment is the
   \index{bpm!step increment}
   \index{usr!step increment}
   "step increment" which defaults to 1, but can be modified by
   changing the "bpm\_step\_increment" value in the "usr"
   configuration file.
   See \sectionref{subsec:seq66_usr_file_user_midi_settings}.

   Similarly, the BPM Down MIDI control decrements the beats-per-minute
   setting, as if the down-arrow has been clicked/pressed.

   Here is a sample group for changing the BPM in both directions.  The value
   176 is B0 hex, which is a Control Change event.  104 is 68 hex, and
   represents an undefined control.  105 is 69 hex, and is also undefined. So
   both MIDI events can be used without interfering with playback.

   Question: Why are both the "on" and "off" sections defined?

   \begin{verbatim}
   # bpm up
   64 [0 0   0   0   0   0] [1 0 176 104 127 127] [1 0 176 104 127 127]
   # bpm down
   65 [0 0   0   0   0   0] [1 0 176 105 127 127] [1 0 176 105 127 127]
   \end{verbatim}

   Note that these controls correspond to the hot-keys assigned in
   the \textbf{File / Options / Keyboard / Control keys [keyboard-group]} 
   "BPM Up" and "BPM Down" configuration panel items.

\paragraph{Automation / Screen-Set Up and Down}
\label{paragraph:seq66_rc_file_midi_ctrl_ssupdn}

   The Screen-Set Up MIDI control increments to the next screen-set. 
   Once the screen-set has been altered, mute-groups and other
   actions apply to that screen set.

   Similarly, the Screen-Set Up MIDI control decrements to the previous
   screen-set.

   \begin{verbatim}
   # bpm up
   66 [0 0   0   0   0   0] [1 0 176 104 127 127] [1 0 176 104 127 127]
   # bpm down
   67 [0 0   0   0   0   0] [1 0 176 105 127 127] [1 0 176 105 127 127]
   \end{verbatim}

   Note that these controls correspond to the hot-keys assigned in
   the \textbf{File / Options / Keyboard / Control keys [keyboard-group]} 
   "BPM Up" and "BPM Down" configuration panel items.

\paragraph{Automation / Mod Replace}
\label{paragraph:seq66_rc_file_midi_ctrl_modrep}

   The Mod Replace MIDI control sets the "replace" status flag.
   Then, when the user manually clicks a pattern slot,
   that pattern is unmuted, and all the rest are muted.
   Thus, this MIDI control is kind a of "Solo" function.
   It works whether in "Live" or "Song" mode.

   Note that this control corresponds to the hot-key assigned in
   the \textbf{File / Options / Keyboard / Control keys [keyboard-group]} 
   "Replace/Solo" configuration panel item.

\paragraph{Automation / Mod Snaphot}
\label{paragraph:seq66_rc_file_midi_ctrl_modsnap}

   The Mod Snapshot MIDI control causes the playing statuses of all active
   (i.e. having data) patterns to be saved.  When turned off, the
   original playing status is restored.  Thus, two MIDI events
   need to be allocated to this functionality. Compare
   to section \sectionref{paragraph:seq66_patterns_pattern_keys}
   for a better idea of how it works.

   Note that this control corresponds to the hot-keys assigned in
   the \textbf{File / Options / Keyboard / Control keys [keyboard-group]} 
   "Snapshot 1" and "Snapshot 2" configuration panel items.

\paragraph{Automation / Mod Queue}
\label{paragraph:seq66_rc_file_midi_ctrl_modqueue}

   The Mod Queue MIDI control sets up the "queue" status flag.
   Then, when the user manually clicks a pattern slot,
   that pattern is queued, and will play at the next cycle of the
   pattern.

   Here is an example from \cite{midicontrol}, which shows how to set up
   the "Sustain" control-change event to queue or un-queue a sequence:
   The \textsl{Akai MPK Mini} has a Sustain button and we can set the
   Sustain MIDI event (with MIDI status byte 176 [0xB0] to represent a
   Controller event, and control/mode change number 64 [0x40] to
   represent the Sustain or Pedal control) up as the queue modifier in
   the \texttt{mod queue} entry:

   \begin{verbatim}
   # mod queue
   #    Toggle                 On                      Off
   70 [0   0   0   0   0   0  ] [1   0   176 64 127 127] [1   0  176 64  0  0]
   #   OPR INV STA D1  mn mx     OPR INV STA D1 mn  mx   OPR INV STA D1  mn mx
   #                                      ^  ^                    ^  ^
   #                                      |  |                    |  |
   #                                      |   ----Sustain---------|--
   #                                       -------Control Change--
   \end{verbatim}

   So when the Sustain button is held down, and one presses one of the pads
   on the \textsl{MPK Mini}, the corresponding sequence gets queued.

   Note that this control corresponds to the hot-key assigned in
   the \textbf{File / Options / Keyboard / Control keys [keyboard-group]} 
   "Queue" configuration panel item.  There is also a "Q" button on the
   Gtkmm-2.4 user-interface.

\paragraph{Automation / Mod Mute Group}
\label{paragraph:seq66_rc_file_midi_ctrl_modgmute}

   The Mod Group Mute MIDI control sets up a "mute group".
   More to come on this one,
   see \sectionref{subsec:seq66_patterns_panel_top}.

   This control sets an internal "mode-group" flag on.

   When activated, \textsl{Seq66} loops through all of the 32
   screen-sets, and through the active pattern in each screen-set, and
   sets each pattern as muted or unmuted, depending on the saved mute state.

\paragraph{Automation / Mod Mute Group}
\label{paragraph:seq66_rc_file_midi_ctrl_modgmute}

   The Mod Group Learn MIDI control sets up a "group learn".
   However, as the group-learn key is a modifier key that needs to
   be held, we're not quite sure how this works with MIDI control.

   This control sets two internal flags on : "mode-group" and "group-learn".
   The first flag indicates that we will be handling mute-groups.
   The second flag indicates that we are learning these mute-groups,
   effectively recording the current status of all the patterns in all of the
   screen-sets.

   These statuses ultimately get saved in the \texttt{[mute-group]} section of
   the "rc" file.

   \index{L button}
   Note that this control corresponds to the "L" button in the main window
   user-interface.
   \index{keys!Ctrl-L}
   It can also be accessed by the hard-wired hot-key, Ctrl-L.

\paragraph{Automation / Screen-Set Play}
\label{paragraph:seq66_rc_file_midi_ctrl_ssplay}

This MIDI control sets the playing screen-set, 
but we're not completely sure how it works yet.

\subsubsection{"rc" File / MIDI Control / Extended Automation}
\label{subsubsec:seq66_rc_file_midi_ctrl_automationex}

   These additional control items were requested by users, to control
   additional features of the application.
   Each item in this group consists of one line.
   The following sections show how to set up extended automation controls.

      \begin{enumber}
         \item \textbf{Stop/Pause/Start}.  Emulate the Stop, Pause, and
            Start keys, using Toggle for pause, Off for stop, and On for
            start.
         \item \textbf{Record}.  For recording a live performance by
            recording the mute/unmute states that the musician played.
            Not yet functional, thought the feature has been added as of
            version 0.94.
         \item \textbf{Solo on/off}.
            Not yet functional.  See the "Mod Replace" functionality.
         \item \textbf{Thru toggle}.
            Not yet functional.
         \item \textbf{Reserved for expansion} (a few of these are reserved).
      \end{enumber}

\paragraph{Ext Automation / Stop/Pause/Start}
\label{paragraph:seq66_rc_file_midi_ctrl_ex_stopps}

   \index{rc!start/stop control}
   Here, we will set up \textsl{Seq66} so that the first three MIDI
   white keys (notes 0, 2, and 4) will become "Stop", "Pause", and "Start"
   buttons.

   The first step, if not already done, is to install the newest version
   (\textbf{0.96.0} and above) of \textsl{Seq66},
   run it, and then exit.
   Verify in the regenerated "rc" file
   (\texttt{\textasciitilde/.config/sequencer66/sequencer66.rc}) that the
   following line exists:

   \begin{verbatim}
      96      # MIDI controls count (74/84/96)
   \end{verbatim}

   Then go to line 74, which is the Stop/Pause/Play MIDI control.
   Replace it with:

   \begin{verbatim}
      # start playback (pause, start, stop):
      74 [1 0 144   2   0 127] [1 0 144   4   0 127] [1 0 144   0   0 127]
   \end{verbatim}

   This sets up MIDI Note On (144) with note numbers 2 (toggle/pause),
   4 (on/start), and 0 (off/stop).
   Any velocity (0 to 127) will work to trigger these actions.
   Now we are ready to test this feature.  One can use a MIDI keyboard to do
   so, but here we will use the \textsl{VMPK} (\cite{vmpk}) virtual MIDI
   piano keyboard application for this test.  Refer to the figure below.
   In addition, let's set up the standard and the new, extended, BPM
   (beats/minute) MIDI control values.  

   \begin{verbatim}
      # bpm up: Note On 9
      64 [0 0   0   0   0   0] [1 0 144   9   0 127] [0 0   0   0   0   0]
      # bpm down: Note On 7
      65 [0 0   0   0   0   0] [1 0 144   7   0 127] [0 0   0   0   0   0]
   \end{verbatim}

   The section above sets up the standard (step-size) BPM controls, which
   correspond to the fine control possible with the up and down arrows of the
   BPM spinner in the main window.  It sets up Note On 7 to be the BPM-down
   control, and Note On 9 to be the BPM-up control.  These two keys are the
   dark-cyan keys shown in the figure.

   \begin{verbatim}
      # bpm page up:  Note On 11
      78 [0 0   0   0   0   0] [1 0 144  11   0 127] [0 0   0   0   0   0]
      # bpm page down:  Note On 5
      79 [0 0   0   0   0   0] [1 0 144  05   0 127] [0 0   0   0   0   0]
   \end{verbatim}

   That section uses the extended controls (74 to 83) set up the coarse
   (page-size) BPM controls, which correspond to the larger jumps possible with
   the Page Up and Page Down keys when the BPM spinner has focus.  It sets up
   Note On 5 to be the BPM-page-down control, and Note On 11 to be the
   BPM-page-up control.  These two keys are
   the bright-cyan keys shown in the figure.

\begin{figure}[H]
   \centering 
%  \includegraphics[scale=0.50]{new/stop_pause_start_test_setup.png}
   \includegraphics[scale=0.65]{roll.png}
   \caption{Stop/Pause/Start ALSA Test Setup}
   \label{fig:rc_file_stop_pause_start_alsa_test_setup}
\end{figure}

   One can copy these settings from the sample file
   \texttt{contrib/simple-midi-control-section.rc}, if one wants to
   try them.  It also illustrates some other setups that we use for testing
   purposes, but are not described here.

   Set up \textsl{VMPK} to use the lowest octave by setting
   \textbf{Base Octave} to \texttt{0}.  The red, yellow, and green
   keys shown will be our stop, pause, and start keys.
   Also set the velocity to a value ranging from 1 to 127.
   Do not use a zero velocity, as it seems that VMPK will not transmit Note On
   messages with a zero velocity.

   Next, run \textsl{Seq66}, using the following command line to make
   sure that it is using ALSA and using automatic mode to connect the ALSA MID
   ports:

   \begin{verbatim}
      $ seq66 -A -a
   \end{verbatim}
   
   Then open a MIDI file.  Next,
   open the \textbf{File / Options / MIDI Input} tab, and make sure that
   the \textbf{VMPK Output} is check-marked as shown in the figure.
   If desired, also connect up to some kind of synthesizer so that the song can
   be heard.

   Finally, press the third white key (shown as green in the figure) to start
   playback.  The second white key (yellow in the figure) will pause and resume
   playback.  The first white key (red in the figure) will stop (and rewind)
   playback.

   Then play with the BPM MIDI control keys.  Note that the size of the
   BPM step-increment and the BPM page-increment are configurable in the
   \index{usr!user-midi-settings}
   \texttt{[user-midi-settings]} section of the "usr" configuration file,
   using the following values in that section:

   \begin{verbatim}
      1        # bpm_precision
      1.0      # bpm_step_increment
      10       # bpm_page_increment
   \end{verbatim}

   See \sectionref{subsec:seq66_usr_file_user_midi_settings}; it has
   information about the usage and enabling of these settings.

   Obviously, this setup is not useful for performance, but serves as a good
   example to verify this MIDI control.

   One thing we noticed while implementing this functionality is that there
   is really no need to have two lines for pairs such as BPM up/down and
   screen-set up/down.  Also, is screen-set play now partly redundant?
   No matter, we will not break the user's existing setup.

\paragraph{Ext Automation / Performance Record}
\label{paragraph:seq66_rc_file_midi_ctrl_ex_precord}

   To do.

\paragraph{Ext Automation / Solo}
\label{paragraph:seq66_rc_file_midi_ctrl_ex_solo}

   To do.

\paragraph{Ext Automation / MIDI Thru}
\label{paragraph:seq66_rc_file_midi_ctrl_ex_thru}

   To do.

\paragraph{Ext Automation / BPM Page Up}
\label{paragraph:seq66_rc_file_midi_ctrl_ex_bpmpageup}

   To do.

\paragraph{Ext Automation / BPM Page Down}
\label{paragraph:seq66_rc_file_midi_ctrl_ex_bpmpageup}

   To do.

\paragraph{Ext Automation / Screen-Set By Number}
\label{paragraph:seq66_rc_file_midi_ctrl_ex_ssnumber}

   To do.

\paragraph{Ext Automation / MIDI Record}
\label{paragraph:seq66_rc_file_midi_ctrl_ex_mrecord}

   To do.

\paragraph{Ext Automation / MIDI Quantized Record}
\label{paragraph:seq66_rc_file_midi_ctrl_ex_qrecord}

   To do.

\paragraph{Ext Automation / Fast Forward}
\label{paragraph:seq66_rc_file_midi_ctrl_ex_fforward}

   To do.

\paragraph{Ext Automation / Rewind}
\label{paragraph:seq66_rc_file_midi_ctrl_ex_rewind}

   To do.

\paragraph{Ext Automation / Top}
\label{paragraph:seq66_rc_file_midi_ctrl_ex_top}

   To do.

\paragraph{Ext Automation / Select Playlist}
\label{paragraph:seq66_rc_file_midi_ctrl_ex_sellist}

   To do.

\paragraph{Ext Automation / Select Song}
\label{paragraph:seq66_rc_file_midi_ctrl_ex_selsong}

   To do.

\subsection{"rc" File / Mute-Group Section}
\label{subsec:seq66_rc_file_mute_group}
     
   This section is delimited by the \texttt{[mute-group]} construct.
   It controls 32 groups of mutes in the same way as defined for
   \texttt{[midi-control]}. A group is set of sequences that can toggle their
   playing state together.  Every group contains all 32 sequences in the
   active screen set.

   \begin{verbatim}
      [mute-group]
      1024    # group mute value count
      0 [0 0 0 0 0 0 0 0] [0 0 0 0 0 0 0 0] [0 0 0 0 0 0 0 0] [0 0 0 0 0 0 0 0]
      1 [0 0 0 0 0 0 0 0] [0 0 0 0 0 0 0 0] [0 0 0 0 0 0 0 0] [0 0 0 0 0 0 0 0]
      2 [0 0 0 0 0 0 0 0] [0 0 0 0 0 0 0 0] [0 0 0 0 0 0 0 0] [0 0 0 0 0 0 0 0]
      ...      ...               ...               ...               ...
      31 [0 0 0 0 0 0 0 0] [0 0 0 0 0 0 0 0] [0 0 0 0 0 0 0 0] [0 0 0 0 0 0 0 0]
   \end{verbatim}

   The initial number, 1024 is probably the total count of 32 x 32 sequences.
   In this group are the definitions of the state of the 32 sequences
   in the playing screen set when a group is selected.
   Each set of brackets defines a group:
   
   \begin{verbatim}
      [state of the first 8 sequences] [second 8] [third 8] [fourth 8]
   \end{verbatim}

   After the list of sequences and their MIDI events, one can 
   set \textsl{Seq66} to handle MIDI events and change some more settings
   in \texttt{sequencer66.rc}.

   What is the different between the \textbf{mute-in group}
   section and the \textbf{mute group} section?  The former defines the MIDI
   control values that can affect the muting of a group, while the latter
   specifies the patterns that are part of a group.

\subsection{"rc" File / MIDI-Clock Section}
\label{subsec:seq66_rc_file_midi_clock}

   \index{[midi-clock]}
   The MIDI Clock fields will contain the clocking state from the last 
   time \textsl{Seq66} was run.  Turn off the clock with a 0, or on
   with a 1 (which means to send MIDI Song Position, and MIDI Continue if
   starting after tick 0), or on with positioning with a 2, which sends MIDI
   Start and then begins clocking after the position reaches a modulo of the
   \textbf{Clock Start Modulo value)}.  Luckily, the user-interface makes it
   easy to select the desire value, and has tool-tips to instruct the user.
   This section has 16 entries, one for each MIDI output buss that
   \textsl{Seq66} supports.

   This configuration item is the same as the 
   \textbf{MIDI Clock} tab described in
   \paragraphref{paragraph:seq66_menu_file_options_midi_clock}
   
   Here is the format:

   \begin{verbatim}
      [midi-clock]
      16
       0 0  #  [1] seq24 1
       1 0  #  [2] seq24 2
       2 0  #  [3] seq24 3
       3 0  #  [4] seq24 4
       4 0  #  [5] seq24 5
       5 0  #  [6] seq24 6
       6 0  #  [7] seq24 7
       7 0  #  [8] seq24 8
       8 0  #  [9] seq24 9
       9 0  # [10] seq24 10
      10 0  # [11] seq24 11
      11 0  # [12] seq24 12
      12 0  # [13] seq24 13
      13 0  # [14] seq24 14
      14 0  # [15] seq24 15
      15 0  # [16] seq24 16
   \end{verbatim}

   That sample would be written one had started up \textsl{Seq66} in
   manual-mode.  On our system, where we have Timidity running, and
   erroneously have also specified 3 MIDI busses that we do not have, in the
   \texttt{sequencer66.usr} file:

   \begin{verbatim}
      [midi-clock]
      5    # number of MIDI clocks/busses
      # Output buss name: [0] 14:0 2x2 A (SuperNova,Q,TX81Z,DrumStation)
      0 0  # buss number, clock status
      # Output buss name: [1] 128:0 2x2 B (WaveStation,ESI-2000,MV4,ES-1,ER-1)
      1 0  # buss number, clock status
      # Output buss name: [2] 128:1 PCR-30 (303)
      2 0  # buss number, clock status
      # Output buss name: [3] 128:2 TiMidity port 2
      3 0  # buss number, clock status
      # Output buss name: [4] 128:3 TiMidity port 3
      4 0  # buss number, clock status
   \end{verbatim}

\subsection{"rc" File / MIDI-Meta-Events Section}
\label{subsec:seq66_rc_file_midi_meta_events}

   \index{[midi-meta-events]}
   The new MIDI Meta events section is the start of additional options
   supporting meta events as normal events in \textsl{Seq66}.
   \index{tempo-track-number}

   \begin{verbatim}
      [midi-meta-events]
      10      # tempo_track_number
   \end{verbatim}

   Normally, as per the MIDI specification, the first track (track 1 in track
   numbering, or pattern 0 in \textsl{Seq66} numbering) is \textsl{the}
   official track for certain MIDI meta events, such as Set Tempo and Time
   Signature.  However, to accommodate existing tunes and their set
   arrangement, we allow the user to go into \textbf{File / Options / MIDI
   Clock} and change the tempo track to another pattern.

   Please note that the user can insert Set Tempo events into any track via the
   pattern editor or the event editor.  But, when recording tempo events, they
   will always be written to the patten having the tempo-track number.

\subsection{"rc" File / Keyboard Control Section}
\label{subsec:seq66_rc_file_keyboard_control}
        
   \index{[keyboard control]}
   The keyboard control is a dump of the keys that \textsl{Seq66}
   recognises, and each key's corresponding sequence number.
   Note that the first number corresponds to the number of sequences in
   the active screen set.

   \begin{verbatim}
      [keyboard-control]
      32     # number of keys
      # Key #  Sequence #   Key name
      44  31        # comma
      49  0         # 1
      50  4         # 2
      51  8         # 3
      52  12        # 4
      53  16        # 5
      54  20        # 6
      55  24        # 7
      56  28        # 8
      97  2         # a
      98  19        # b
      99  11        # c
      100  10       # d
      101  9        # e
      102  14       # f
      103  18       # g
      104  22       # h
      105  29       # i
      106  26       # j
      107  30       # k
      109  27       # m
      110  23       # n
      113  1        # q
      114  13       # r
      115  6        # s
      116  17       # t
      117  25       # u
      118  15       # v
      119  5        # w
      120  7        # x
      121  21       # y
      122  3        # z
   \end{verbatim}

\subsection{"rc" File / Keyboard Group Section}
\label{subsec:seq66_rc_file_keyboard_group}

   \index{[keyboard-group]}
   This section is the same as
   \textbf{[keyboard-control]}, but to control groups of patterns, rather than
   individual patterns, using keystrokes.
   The keyboard group specifies more automation for the application.  The
   first number specifies the key number, and the second number specifies
   the Group number.

   Additional control items:

   \begin{enumber}
      \item \textbf{\# bpm up and down}.
         Keys to control BPM (beats per minute).
      \item \textbf{\# screen set up and down}.
         Keys for changing the active screenset.
      \item \textbf{\# group functionality on, off, learn}.
         \index{group learn}
         Note that the group learn key is a modifier key to be held while 
         \index{group toggle}
         pressing a group toggle key.
      \item \textbf{\#replace, queue, snapshot\_1, snapshot\_2, keep queue}.
         These are the other modifier keys explained in section 3a.
   \end{enumber}

   To see the required key codes when pressed, run \texttt{seq24} with
   the \texttt{--show-keys}.

   Some keys should not be assigned to control sequences in
   \textsl{Seq66} as they are already assigned in the
   \textsl{Seq66} menu (with \texttt{Ctrl}). 

   This configuration item is the same as the 
   \textbf{Keyboard} tab described in
   \sectionref{paragraph:seq66_menu_file_options_keyboard}.

   \begin{verbatim}
      [keyboard-group]
      # Key #, group # 
      32
      33  0         # exclam
      34  1         # quotedbl
      35  2         # numbersign
      36  3         # dollar
      37  4         # percent
      38  5         # ampersand
      40  7         # parenleft
      47  6         # slash
      59  31        # semicolon
      65  16        # A
      66  28        # B
      67  26        # C
      68  18        # D
      69  10        # E
      70  19        # F
      71  20        # G
      72  21        # H
      73  15        # I
      74  22        # J
      75  23        # K
      77  30        # M
      78  29        # N
      81  8         # Q
      82  11        # R
      83  17        # S
      84  12        # T
      85  14        # U
      86  27        # V
      87  9         # W
      88  25        # X
      89  13        # Y
      90  24        # Z
      39 59         # bpm up, down: apostrophe semicolon
      93 91 65360   # screen set up, down, play: bracketright bracketleft Home
      236 39 65379  # group on, off, learn: igrave apostrophe Insert
      # replace, queue, snapshot_1, snapshot 2, keep queue:
      65507 65508 65513 65514 92  # Control_L Control_R Alt_L Alt_R backslash
      1             # show_ui_sequence_key and pattern measures (1=true/0=false)
      32            # space start sequencer
      65307         # Escape stop sequencer
      0 #  show sequence numbers (1 = true / 0 = false);  ignored in legacy mode
   \end{verbatim}

   Note that most of these group-control keys are shifted versions of the
   keystrokes that control the individual sequences.  Also note the
   \texttt{Control\_L} and \texttt{Control\_R} notations a few lines above.
   \index{keys!no ctrl/alt please}
   Please avoid using any Control key combinations in the "rc"/Keyboard
   configuration.  Control keys are the province of the user-interface
   (\textsl{Gtk+}) and assigning them can cause surprising behavior!
   It is also wise to avoid the \texttt{Alt} key.

   \index{auto-shift}
   \index{group-learn!auto-shift}
   When in group-learn mode, the \texttt{Shift} key cannot be hit, so the
   group-learn mode automatically converts the keys to their shifted versions.
   \index{shift-lock}
   \index{group-learn!shift-lock}
   This feature known as \textsl{shift-lock} or \textsl{auto-shift}.

\subsection{"rc" File / JACK Transport}
\label{subsec:seq66_rc_file_jack_transport}

   This section holds the settings for both JACK transport and for native JACK
   MIDI mode.

   \index{[jack-transport]}
   The JACK Transport options are also command-line options, as indicated in
   the comments below.

   This configuration item is the same as the 
   \textbf{Jack Sync} tab described in
   \sectionref{paragraph:seq66_menu_file_options_jack_sync}.

   \index{--jack-transport}
   \index{--jack-master}
   \index{--jack-master-cond}
   \index{--jack-start-mode}
   \begin{verbatim}
      [jack-transport]
      # jack_transport - Enable slave sync with JACK Transport.
      0
      # jack_master - Seq66 attempts to serve as JACK Master.
      0
      # jack_master_cond - Seq66 is master if no other master exists.
      0
      # song_start_mode (applies mainly if JACK is enabled)
      # 0 = Playback in live mode. Allows muting and unmuting of loops.
      # 1 = Playback uses the song editor's data.
      1
   \end{verbatim}

   An additional item, new, specifies if native JACK MIDI input/output is to be
   used.

   \index{--jack-midi}
   \begin{verbatim}
      # jack_midi - Enable JACK MIDI, which is a separate option from
      # JACK Transport.
      1
   \end{verbatim}

   Please note that only \textsl{one} of
   jack\_transport, jack\_master, and jack\_master\_cond should be selected
   (set to 1) at a time.
   Also note that JACK transport is separately configurable from
   JACK MIDI, and each uses a different JACK client internally.

\subsection{"rc" File / MIDI Clock Mod Ticks}
\label{subsec:seq66_rc_file_midi_cmt}

   \index{[midi-clock-mod-ticks]}
   This configuration item is the same as the
   \textbf{Clock Start Modulo} option described in
   \paragraphref{paragraph:seq66_menu_file_options_midi_clock}.

   \begin{verbatim}
      [midi-clock-mod-ticks]
      64
   \end{verbatim}

\subsection{"rc" File / MIDI Meta Events}
\label{subsec:seq66_rc_file_midi_meta}

   This section defines some features of MIDI meta-event handling.  Normally,
   tempo events are supposed to occur in the first track (pattern 0).  But one
   can move this track elsewhere to accomodate one's existing body of tunes.
   If affects where tempo events are recorded.  The default value is 0, the
   maximum is 1023.  A pattern must exist at this number for it to work.

   \begin{verbatim}
      [midi-meta-events]
      0    # tempo_track_number
   \end{verbatim}

\subsection{"rc" File / MIDI Input}
\label{subsec:seq66_rc_file_midi_input}

   \index{[midi-input]}
   This configuration item is the same as the 
   \textbf{MIDI Input} tab described in
   \paragraphref{paragraph:seq66_menu_file_options_midi_input}.
   The "1" is a line count, and would equal the number of
   supported input ports.
   This "rc" entry here has two variables; the first is the record number or
   port number, and the second number indicates whether it is disabled (0),
   or enabled (1).

   \begin{verbatim}
      [midi-input]
      1   # number of MIDI busses
      # The first number is the port number, and the second number
      # indicates whether it is disabled (0), or enabled (1).
      # [0] 0:0 seq66 midi in
      0 0
      # If set to 1, this option allows the master MIDI bus to record
      # (filter) incoming MIDI data by channel, allocating each incoming
      # MIDI event to the sequence that is set to that channel.
      # This is an option adopted from the Seq32 project at GitHub.
      0   # flag to record incoming data by channel
   \end{verbatim}

   There is no user-interface item for the following value, but
   it does correspond to the \texttt{--manual-ports} command-line
   option.

\subsection{"rc" File / Manual ALSA Ports}
\label{subsec:seq66_rc_file_manual_ports}

   The name of this setting is a bit of a misnomer in a couple of ways:

   \index{ports!virtual}
   \index{ports!manual}
   \begin{enumerate}
      \item It actually refers to the usage of \textsl{virtual} MIDI ports.
         These are ports that are set up by the application so that other
         devices or applications can connect to the MIDI application.
         Where the "manual" idea comes in is that the user can manually choose
         the connections to be made.
      \item This option is not just for ALSA.  It can also be used when
         \textsl{Seq66} is running in native JACK mode, so support
         virtual JACK ports that can be connected manually (e.g. in the
         \textsl{QJackCtl} application.)
   \end{enumerate}

   \index{[manual-ports]}
   \begin{verbatim}
      [manual-ports]
      # Set to 1 to have sequencer66 create its own ALSA ports and not
      # connect to other clients.  Use 1 to expose all 16 MIDI ports to
      # JACK (e.g. via a2jmidid).  Use 0 to access the ALSA MIDI ports
      # already running on one's computer, or to use the autoconnect
      # feature (Seq66 connects to existing JACK ports on startup.
      1
   \end{verbatim}

   \index{--auto-ports}
   The opposite of \texttt{--manual-ports}
   is \texttt{--auto-ports}.  The auto-ports option
   forces \textsl{Seq66} to use the system's existing ALSA ports.
   This is necessary in order to play tunes through software synthesizers that
   use ALSA MIDI.

   \index{jack!manual-ports}
   Turning on the manual-ports option is necessary if one
   wants to use the legacy \textsl{Seq66} (\texttt{sequencer66})
   with JACK.
   It is \textsl{not} necessary if using the native JACK MIDI version,
   \texttt{seq66}.
   However, if one needs to avoid the auto-connect feature of \texttt{seq66},
   then the manual option is necessary.

   It will create ports as per the settings in the "user" configuration file's
   \texttt{user-midi-bus-definitions} and \texttt{user-midi-bus-N} sections.
   These definitions can be used by JACK for connection, and these definitions
   can be used to specifically rename the ports that exist in the system.
   However, this option is misleading if one wants to have access to the
   actual ALSA ports that exist on the system.
   The next option gets around that issue.

\subsection{"rc" File / Reveal ALSA Ports}
\label{subsec:seq66_rc_file_reveal_ports}

   Again, this option applies to both ALSA and native JACK.

   \index{[reveal-ports]}
   \begin{verbatim}
      [reveal-ports]
      # Set to 1 to have sequencer66 ignore any system port names
      # declared in the 'user' configuration file.  Use this option if
      # you want to be able to see the port names as detected by ALSA.
      1   # flag for reveal ALSA ports
   \end{verbatim}

   \index{jack!reveal-ports}
   Turning on the reveal-ports option is necessary if one
   wants to see the actual ALSA port names defined by the system.
   It will ignore the settings in the "user" configuration file's
   \texttt{user-midi-bus-definitions} and \texttt{user-midi-bus-N} sections.
   If this option is turned on, the definitions in the
   "user" configuration file are \textsl{not} read from that file.

\subsection{"rc" File / Interaction Method}
\label{subsec:seq66_rc_file_interaction}

   \index{[interaction-method]}
   This configuration item is the same as the 
   \textbf{Mouse} tab described in
   \paragraphref{paragraph:seq66_menu_file_options_mouse}.

   \begin{verbatim}
      [interaction-method]
      # 0 - 'seq24' (original seq24 method)
      # 1 - 'fruity' (similar to a certain fruity sequencer we like)
      0   # interaction_method
   \end{verbatim}

   \index{qt!fruity}
   Note that the "fruity" mouse support is not available in the Qt
   user-interface.

   \index{[allow-mod4-mode]}
   \index{new!mod4 edit-lock}
   There is an option to use the Mod4 (Super, or Windows) key in the
   Pattern Editor to lock the editing of a note.  When this mode is enabled,
   and Mod4 is pressed while the mouse right-button is released, the
   editing pencil icon remains, and notes can be added.  This feature is
   useful for crippled trackpads and trackpad drivers that cannot provide
   two simultaneous button presses.

   \begin{verbatim}
      # Set to 1 to allow seq24 to stay in note-adding mode when
      # the right-click is released while holding the Mod4 (Super or
      # Windows) key.
      1   # allow_mod4_mode
   \end{verbatim}

   \index{qt!mod4 edit-lock}
   Note that the "edit-lock" support is not available in the Qt
   user-interface.  It is just as easy to click in that pattern editor and
   press the 'p' key to enter "paint" (edit) mode.

   \index{[allow-snap-split]}
   \index{new!snap-split}
   This option comes from the \textsl{seq32} project.  It allows for
   pattern-splitting in the Song editor at snap points, rather than just
   at the middle of the pattern.

   \begin{verbatim}
      # Set to 1 to allow Seq66 to split performance editor
      # triggers at the closest snap position, instead of splitting the
      # trigger exactly in its middle.  Remember that the split is
      # activated by a middle click.
      0   # allow_snap_split
   \end{verbatim}

   \index{qt!snap-split}
   Not sure that this support is available in the Qt user-interface.

   \index{[allow-click-edit]}
   \index{new!click-edit}
   This option allows one to enable/disable the ability to double-click
   in a pattern slot in the main window to bring it up for editing.  This
   can interfere with a live performance where muting/unmuting come fast enough
   to be seen as a double-click.

   \begin{verbatim}
      # Set to 1 to allow a double-click on a slot to bring it up in
      # the pattern editor.  This is the default.  Set it to 0 if
      # it interferes with muting/unmuting a pattern.
      1   # allow_click_edit
   \end{verbatim}

   \index{qt!click-edit}
   Not sure that this support is available in the Qt user-interface.

\subsection{"rc" File / LASH Session}
\label{subsec:seq66_rc_file_lash_session}

   \index{[lash-session]}
   The following configuration item is the same as the
   \texttt{--lash} or \texttt{--no-lash} options described in
   \sectionref{sec:seq66_man_page}.
   If set to 0, LASH session support is disabled.
   If set to 1, LASH session support is enabled.
   However, if LASH support is not built into the application, neither option
   has any effect -- there is no LASH support.  
   To determine if LASH support is built in, run sequencer66 from the command
   line with the \texttt{--version} option, and see if LASH is mentioned.

   \begin{verbatim}
      [lash-session]
      # Set the following value to 0 to disable LASH session management.
      # Set the following value to 1 to enable LASH session management.
      # This value will have no effect is LASH support is not built into
      # the application.  Use the --help option to see if LASH is part of
      # the options list.
      1     # LASH session management support flag
   \end{verbatim}

\subsection{"rc" File / Auto Option Save}
\label{subsec:seq66_rc_file_auto_rc_save}

   \index{[auto-option-save]}
   This new item determines if the "rc" configuration file is saved
   upon exit of \textsl{Seq66}.  The legacy behavior is to save it,
   which can sometimes be inconvenient when one is just trying out some
   command-line options.

   \begin{verbatim}
      [auto-option-save]
      # Set the following value to 0 to disable the automatic saving of the
      # current configuration to the 'rc' file.  Set it to 1 to
      # follow legacy seq24 behavior of saving the configuration at exit.
      # Note that, if auto-save is set, many of the command-line settings,
      # such as the JACK/ALSA settings, are then saved to the configuration,
      # which can confuse one at first.  Also note that one currently needs
      # this option set to 1 to save the configuration, as there is not a
      # user-interface control for it at present.
      0     # auto-save-options-on-exit support flag
   \end{verbatim}

\subsection{"rc" File / Last Used Directory}
\label{subsec:seq66_rc_file_last_used_dir}

   The following item refers to the last directory in which one opened or
   saved a MIDI file.

   \index{[last-used-dir]}
   \begin{verbatim}
      [last-used-dir]
      /home/ahlstrom/Home/ca/mls/git/sequencer66/contrib/midi/
   \end{verbatim}

\subsection{"rc" File / Recent Files}
\label{subsec:seq66_rc_file_recent_files}

   The following item preserves a list of the last few MIDI files loaded.
   It is not filled when a MIDI file is loaded via a play-list.

   \index{[recent-files]}
   \begin{verbatim}
      [recent-files]
      4
      /home/ahlstrom/Home/ca/mls/git/sequencer66-alternate/contrib/midi/2Bars.midi
      contrib/midi/b4uacuse-seq24.midi
      contrib/midi/Bars.midi
      contrib/midi/b4uacuse-GM-format.midi
   \end{verbatim}

\subsection{"rc" File / Play-List}
\label{subsec:seq66_rc_file_playlist}

   This is a feature new with version 0.96.0 of \textsl{Seq66}.
   It provides a configured set of named play-lists in a play-list file,
   and a flag to activate it.
   
   \index{[playlist]}
   \begin{verbatim}
      [playlist]
      0     # playlist_active, 1 = active, 0 = do not use it
      # Provides the name of a play-list.  If there is none, use '""'.
      # Or set the flag above to 0.
      /home/ahlstrom/.config/sequencer66/sample.playlist
   \end{verbatim}

   See \sectionref{sec:playlist}.
   It describes the setup, layout, and usage of a
   \textsl{Seq66} playlist file containing one or more playlists.

%-------------------------------------------------------------------------------
% vim: ts=3 sw=3 et ft=tex
%-------------------------------------------------------------------------------


% Discussion of JACK support

%%% %-------------------------------------------------------------------------------
% jack
%-------------------------------------------------------------------------------
%
% \file        jack.tex
% \library     Documents
% \author      Chris Ahlstrom
% \date        2016-01-28
% \update      2018-02-02
% \version     $Revision$
% \license     $XPC_GPL_LICENSE$
%
%     Provides the JACK page section of seq24-user-manual.tex.
%
%-------------------------------------------------------------------------------

\section{Seq66 JACK Support}
\label{sec:jack}

   This section describes some details concerning the JACK support of
   \textsl{Seq66}.
   As with \textsl{Seq24}, \textsl{Seq66} has JACK transport support.
   But, if one wants to use the older version of \textsl{Seq66} (versions
   0.9.x) with JACK MIDI, one needs to expose the ALSA ports to JACK using
   \texttt{a2jmidid --export-hw} and connect the resultant MIDI JACK ports
   oneself, using \textsl{QJackCtl}, for example.

   To enable the JACK transport support at run-time, the options
   \texttt{-j}/\texttt{--jack-transport}, \texttt{-J}/\texttt{--jack-master},
   and \texttt{-C}/\texttt{--jack-master-cond} are available.

   With version 0.90, \textsl{Seq66} can be built to support the legacy
   ALSA interface, the PortMIDI interface, or, best of all, the native JACK
   MIDI interface, loosely based on the RtMIDI project
   (see \cite{rtmidi}).  This mode also supports fallback-to-ALSA if the JACK
   server is not running.

   \begin{itemize}
      \item \textsl{Seq66}.
         \index{Seq66}.
         This application is built when the
         \texttt{--enable-alsamidi} option is specified at "configure" time.
         It is basically the 0.9.x version of \textsl{Seq66} application.
         However, this build is no longer the default.
      \item \textsl{seq66}.
         \index{seq66}.
         This application is built when the
         \texttt{--enable-rtmidi} option is specified at "configure" time.
         This build is now the default build.
      \item \textsl{seq66portmidi}.
         \index{seq66portmidi}.
         This application is built when the
         \texttt{--enable-portmidi} option is specified at "configure" time.
         This build is \textsl{deprecated}.  It works with Linux, but, for
         Windows support, we will instead add Windows API calls to the "rtmidi"
         build of the project.  We won't discuss this version at all.  We've
         tested it for playback, but nothing else.
   \end{itemize}

   The following sections discuss the JACK transport support and the native
   JACK MIDI support.

\subsection{Seq66 JACK Transport}
\label{subsec:jack_transport}

   This section is just underway.  Here are some of the topics to be discussed:

   \begin{enumerate}
      \item What JACK functions are supported for JACK Transport.
      \item Exposing the ALSA MIDI ports to JACK, when using the legacy
         ALSA version of \textsl{Seq66}.
      \item Fixes to JACK Master mode.
      \item Interactions with the Klick and Hydrogen applications.
      \item Patches from the new (!) version of Seq24, 0.9.3, to correct
         for MIDI Clock drift over long durations.
   \end{enumerate}

   In the meantime, the text files in the project's \texttt{contrib/notes}
   directory provide some useful setup notes.

   JACK transport support is \textsl{separate} from native JACK MIDI support.
   The JACK transport client is an invisible client with the
   name "seq66-transport", while the JACK MIDI client is visible in
   \textsl{QJackCtl}, and the ports created are part of the
   "seq66" client.

   The first thing to note about JACK transport with \textsl{Seq66} is
   that the progress bars will not move unless \textsl{Seq66} is
   connected to a JACK client, such as \textsl{Hydrogen} (in JACK MIDI mode)
   or \textsl{Yoshimi}.  Currently, \textsl{Seq66} will connect to a JACK
   client automatically only at startup, where it will connect to all JACK
   clients that it finds.  If it can't find a JACK client, then it will
   fail to register a JACK port, and cannot play.

   The second thing is that \textsl{Seq66} still has the issue where it
   must be JACK Master to follow transport.  More on this issue later.

\subsection{Seq66 Native JACK MIDI}
\label{subsec:jack_native_midi}

   This section discusses the new \textsl{seq66} application, which supports
   native JACK MIDI.  It is now the \textsl{official} version of
   \textsl{Seq66}, and new bugs will be fixed mainly in this version.

   The first thing to note about \textsl{seq66}
   is that it supports both ALSA and JACK
   MIDI.  If one runs it to support JACK, and JACK is not present, then
   \textsl{seq66}
   falls back to ALSA support.

   To run seq66 to support JACK, for now, one must add the
   \texttt{-t} or \texttt{--jack-midi}
   option.  Why \texttt{-t}?  We are running out of option letters.
   And eventually we will make the \texttt{-t} option the default.
   If \textsl{Seq66} (in its native JACK mode, using the
   \texttt{-t} or \texttt{--jack-midi} option)
   is run in JACK mode \textsl{without JACK running} on
   the system, it will take awhile to come up (in ALSA mode).  If run from the
   console, one will see:

\begin{verbatim}
	$ ./Seq66rtmidi/seq66 -t
	[Activating native JACK MIDI]
	[Reading rc configuration /home/ahlstrom/.config/sequencer66/sequencer66.rc]
	[Reading user configuration /home/ahlstrom/.config/sequencer66/sequencer66.usr]
	[Activating native JACK MIDI]
	Cannot connect to server socket err = No such file or directory
	Cannot connect to server request channel
	jack server is not running or cannot be started
     . . .
	[JACK server not running?]
	connect: JACK server not running?
	rtmidi_info: no compiled support for specified API     (???)
	[Initialized, running without JACK sync]
	9 rtmidi ports created:
	Input ports (3):
	  [0] 0:1 system:announce (system)
	  [1] 14:0 Midi Through:Midi Through Port-0
	  [2] 24:0 nanoKEY2:nanoKEY2 MIDI 1
	Output ports (6):
	  [0] 14:0 Midi Through:Midi Through Port-0 
	  [1] 24:0 nanoKEY2:nanoKEY2 MIDI 1 
	  [2] 128:0 TiMidity:TiMidity port 0 
	  [3] 128:1 TiMidity:TiMidity port 1 
	  [4] 128:2 TiMidity:TiMidity port 2 
	  [5] 128:3 TiMidity:TiMidity port 3 
\end{verbatim}

   To enable native JACK MIDI, use the
   \texttt{-t}/\texttt{--jack-midi} options.
   When enabled, the "transport" button in the main window will show the word
   "JACK" (no matter whether JACK transport is also enabled or not).
   Please note that, if you
   select the JACK-MIDI option and JACK is not available,
   \textsl{seq66} will start in
   ALSA mode and \textsl{it will save} that mode when exiting.

	\index{sticky options}
   The JACK (\texttt{-t}) and ALSA (\texttt{-A}) options are sticky options.
   That is, they are saved to the "rc" configuration file at exit,
   so one does not have to specify them in subsequent \textsl{seq66} sessions.

\subsubsection{Seq66 JACK MIDI Output}
\label{subsubsec:jack_midi_output}

   By default (or depending on the "rc" configuration file), the new
   \texttt{seq66} version of \textsl{Seq66} will
   \index{jack!auto-connect}
   \index{auto-connect}
   automatically connect the ports that it finds to \texttt{seq66},
   as shown in the following figure:

\begin{figure}[H]
   \centering 
%  \includegraphics[scale=0.75]{jack/jack-nano-yosh-midi-auto-seq66.png}
   \includegraphics[scale=0.65]{roll.png}
   \caption{JACK MIDI Ports and Auto-Connect}
   \label{fig:jack_nano_yosh_midi_auto}
\end{figure}

   The \texttt{seq66:system midi\_playback\_1} output port shown
   in the left panel is created by \texttt{seq66}.  It connects it to the 
   \texttt{system:midi\_playback\_1} port in the right panel, which
   is actually the input for the \textsl{Korg nanoKEY2} controller.
   ALSA detects the real name of this USB MIDI device, but JACK does not.  
   Thus, the MIDI tab shows the "system" name of the USB MIDI port, while
   the ALSA tab does show the name.

\begin{figure}[H]
   \centering 
%  \includegraphics[scale=0.75]{jack/jack-nano-yosh-alsa-pre-seq66.png}
   \includegraphics[scale=0.65]{roll.png}
   \caption{ALSA MIDI Ports}
   \label{fig:jack_nano_yosh_alsa_pre}
\end{figure}

	Note that the input connection is not useful unless \texttt{seq66} could
   send setup information to the \textsl{nanoKEY2}.
   Korg provides a configuration application for \textsl{Windows}.
   For \textsl{Linux}, a Python application called \textsl{Nano-Basket}
	(\cite{nanobasket}) is available.

   More useful is the automatic connection between
   \texttt{seq66:yoshimi midi in} in the left (output) panel and
   \texttt{yoshimi:midi in} in the right (input) panel.  With it it,
   \textsl{Seq66} patterns with the proper output-buss setting can play
   to the \textsl{Yoshimi} software synthesizer.

	The output ports available are shown in \textsl{seq66}'s
	\textbf{File / Options / MIDI Clock} tab, shown here:

\begin{figure}[H]
   \centering 
%  \includegraphics[scale=0.75]{jack/jack-nano-yosh-midi-clock-seq66.png}
   \includegraphics[scale=0.65]{roll.png}
   \caption{JACK MIDI Ports in Seq66}
   \label{fig:jack_nano_yosh_midi_clock}
\end{figure}

   Note that the index, client, and buss numbers are all the same.
   There's actually a bug here, since all \texttt{seq66} ports should have the
   same client number.  However, in JACK, clients and ports are referred to by
   name, not number, and so functionality is not affected.

   Entry 0 (\texttt{seq66:system midi\_playback\_1}) is,
   as already noted, not useful unless the \textsl{nanoKEY2} can
   accept input control.  Entry 1 allows \texttt{seq66} to send MIDI
   to \textsl{Yoshimi}.  A pattern must specify output buss 1 in order for the
   MIDI to reach \textsl{Yoshimi}.  Another option, normally for testing only,
   is to specify the "bus" option on the command line:

   \begin{verbatim}
      $ seq66 --jack-midi --bus 1
   \end{verbatim}

   With that option, all patterns send to buss 1.

\subsubsection{Seq66 JACK MIDI Input}
\label{subsubsec:jack_midi_input}

   One more connection to note is the input connection to \texttt{seq66}.
   Referring back to 
   \figureref{fig:jack_nano_yosh_midi_auto}
   we see that
   \texttt{system:midi\_capture\_1} in the left (output) panel is connected to
   \texttt{seq66:system midi\_capture\_1} in the right (input) panel.
   This allows the \textsl{nanoKEY2} MIDI output port to feed \texttt{seq66},
   which can then record the input notes, and also forward them to
   \textsl{Yoshimi} so that they can be heard.

   This input port is also shown in the \textbf{File / Options / MIDI Input}
   tab, shown here:

\begin{figure}[H]
   \centering 
%  \includegraphics[scale=0.75]{jack/jack-nano-yosh-midi-input-seq66.png}
   \includegraphics[scale=0.65]{roll.png}
   \caption{JACK MIDI Input Ports}
   \label{fig:jack_nano_yosh_midi_input}
\end{figure}

   When the check-box for that buss is selected, the input can be captured by
   \texttt{seq66}.

\subsubsection{Seq66 JACK MIDI Virtual Ports}
\label{subsubsec:jack_midi_virtual_ports}

   \index{ports!manual}
   \index{ports!virtual}
   The manual-versus-normal port support for JACK MIDI is essentially the same
   as that for ALSA.  Currently, the same option name is used (we will provide
   a more generic option-name soon).
   The \texttt{-m}/\texttt{--manual-ports} option actually provides what
   are known as "virtual" ports.  These are ports that do not represent
   hardware, but are created by applications to allow them to connect to other
   applications or MIDI devices.

   The difference between manual/virtual ports and normal ports is that, while
   normal ports are automatically connected to the remote ports that exist in
   the system, the manual/virtual ports are just created, and one must
   manually connect them via, for example, the
   \textsl{QJackCtl} connections dialog.

   So, if one wants \textsl{seq66} to automatically connect to all existing
   JACK MIDI ports, \textsl{do not} use the
   \texttt{-m}/\texttt{--manual-ports} option... use the
   \texttt{-a}/\texttt{--auto-ports} option.  Both options apply to both
   ALSA and JACK, but we do not want to change the option-names at this time.

   If one wants the freedom to make the connections oneself, or with a session
   manager, then use the manual/virtual option.
   Here are the ports created in manual/virtual mode:

\begin{figure}[H]
   \centering 
%  \includegraphics[scale=0.75]{jack/jack-nano-yosh-midi-manual-seq66.png}
   \includegraphics[scale=0.65]{roll.png}
   \caption{JACK MIDI Manual Ports}
   \label{fig:jack_nano_yosh_midi_manual}
\end{figure}

   One sees that \texttt{seq66} creates 16 output ports (busses), and one input
   port (buss).  One also sees that \texttt{seq66} \textsl{does not} connect
   the ports automatically.  The user or the session manager will have to make
   those connections.

   The \textbf{MIDI Clock} and \textbf{MIDI Input} tabs reflect in an obvious
   manner what is seen in \textsl{QJackCtl}, so we won't bother to show those
   tabs.

\subsubsection{Seq66 JACK MIDI and a2jmidid}
\label{subsubsec:jack_midi_a2jmidid}

   One more thing to show is that \texttt{seq66} can deal with the odd naming
   of JACK ports created by the \textsl{a2jmidid} application.

\begin{figure}[H]
   \centering 
%  \includegraphics[scale=0.75]{jack/a2jmidid-jack-midi.png}
   \includegraphics[scale=0.65]{roll.png}
   \caption{JACK MIDI a2jmidid Ports}
   \label{fig:a2jmidid_jack_midi}
\end{figure}

   One can see in the right (input) panel that that the \texttt{a2j} client
   creates 5 entries, one for "Midi Through", and four for the
   \textsl{TiMidity} client.
   In the left (output) panel, one sees (in blue) the output
   ports that \texttt{seq66} creates to connect to the ports created by
   \textsl{a2jmidid}.

   Also note the true JACK output port,
   \texttt{seq66:yoshimi midi in} to connect to the input port
   \texttt{yoshimi:midi in}.

   Again, if these automatic connections get in the way, run \texttt{seq66} in
   manual/virtual mode.

   When recording, do not forget the step option.  If one paints notes with the
   mouse, the note is previewed, and the note position advances with each
   click.  If one paints notes via an external MIDI keyboard, the notes are
   painted and advanced, but they are not previewed.  To preview them, click
   the "pass MIDI in to output" button in the pattern editor window to activate
   so that they will be passed to your sound generator.
	Be careful of MIDI loops!

%-------------------------------------------------------------------------------
% vim: ts=3 sw=3 et ft=tex
%-------------------------------------------------------------------------------


% Port-Mapping

%%% %-------------------------------------------------------------------------------
% port_mapping
%-------------------------------------------------------------------------------
%
% \file        port_mapping.tex
% \library     Documents
% \author      Chris Ahlstrom
% \date        2020-12-29
% \update      2020-12-29
% \version     $Revision$
% \license     $XPC_GPL_LICENSE$
%
%     Provides a discussion of the MIDI GUI port_mapping that Seq66
%     supports.
%
%-------------------------------------------------------------------------------

\section{Port Mapping}
\label{sec:port_mapping}

   \textsl{Seq66}, like \textsl{Seq24}, bases its I/O port scheme on buss
   numbers (also called "port numbers").  This numbering scheme applies whether
   \textsl{ALSA}, \textsl{JACK}, or \textsl{Windows Multimedia}
   are used as the MIDI engine, and whether \textsl{Seq66} is running with
   "automatic" ports or "manual" (virtual, software-created) ports.
   These buss numbers range from 0 on upward
   based on the input or output MIDI ports active in the system.
   In "automatic" (non-virtual, non-manual) mode
   these ports represent the hardware ports and ports created by
   other applications.  In "manual" mode, these ports represent virtual ports
   that can be connected through other software under \textsl{ALSA} or
   \textsl{JACK}.

   A given pattern/loop/sequence can be assigned to output to a given port via
   a buss number that is saved with the pattern.  Thus, when a tune is loaded,
   each sequence can automatically output to the desired MIDI device.

   The problem is that the list of MIDI devices can change, with devices being
   reordered, removed, or added to the set of MIDI devices available on the
   system.  Port mapping provides a partial solution to this issue.  It allows
   the buss number stored with a pattern to be remapped to another buss number,
   based on the name of the port.

   As with the normal port listings, the port-mappings are managed in the
   \textsl{Seq66} 'rc' file.

\subsection{Output Port Mapping}
\label{subsec:output_port_mapping}

   Assume that the system has the following set of ports.  These busses are
   stored in the 'rc' file when \textsl{Seq66} exits.

   \begin{verbatim}
      [midi-clock]
      6      # number of MIDI clocks (output busses)
      0 0    "[0] 14:0 Midi Through Port-0"
      1 0    "[1] 28:0 nanoKEY2 MIDI 1"
      2 0    "[2] 36:0 E-MU XMidi1X1 Tab MIDI 1"
      3 0    "[3] 40:0 USB Midi MIDI 1"
      4 0    "[4] 44:0 Launchpad Mini MIDI 1"
      5 0    "[5] 128:0 yoshimi:input"
   \end{verbatim}

   If some items are unplugged, then this list will change, so save it.
   Click the
   \textbf{Save Clock/Input Maps} button in the
   \textbf{Edit / Preferences/ MIDI Clock} dialog. 
   The result is a new section in the 'rc' file:

   \begin{verbatim}
      [midi-clock-map]
      1   # map is/not active
      0   "Midi Through Port-0"
      1   "nanoKEY2 MIDI 1"
      2   "E-MU XMidi1X1 Tab MIDI 1"
      3   "USB Midi MIDI 1"
      4   "Launchpad Mini MIDI 1"
      5   "input"
   \end{verbatim}
   
   It is simpler, containing only an index number and shorter versions of the
   port names, called "nick-names".  These index numbers can be used like buss
   numbers: they can be stored in a pattern, and used to direct output to a
   device by name.  Let's say we've unplugged some devices, so that the MIDI
   clocks list is shorter:

   \begin{verbatim}
      [midi-clock]
      4      # number of MIDI clocks (output busses)
      0 0    "[0] 14:0 Midi Through Port-0"
      1 0    "[1] 32:0 USB Midi MIDI 1"
      2 0    "[2] 36:0 Launchpad Mini MIDI 1"
      3 0    "[3] 128:0 yoshimi:input"
   \end{verbatim}

   So, if a pattern has stored item 3 "USB Midi MIDI 1" as its output buss,
   and the output port map is active, the "3" is looked up in the map, the
   nick-name "USB Midi MIDI 1" grabbed, and looked up in the system list, which
   returns "1" as the buss number to use for output.

   On the other hand, if a pattern has stored item 2 "E-MU XMidi1X1 MIDI 1" as
   its output buss, this item will not be found in the system list, so that the
   pattern will need to be routed to an existing port.

   Note that the mapping can be disabled by setting the first value to 0.  In
   that case, \textsl{Seq66} uses buss numbers in the normal way.
   In the user interface dropdowns for output buss, if a map is active, it is
   put into the dropdown; any missing items are noted and are shown as
   disabled.
   If the map is not active, then only the actual system output ports are shown.

\subsection{Input Port Mapping}
\label{subsec:input_port_mapping}

   The input ports are handling somewhat similarly.  Here's the initial system
   input setup:

   \begin{verbatim}
      [midi-input]
      6      # number of input MIDI busses
      0 1    "[0] 0:1 system:announce"
      1 0    "[1] 14:0 Midi Through Port-0"
      2 0    "[2] 28:0 nanoKEY2 MIDI 1"
      3 0    "[3] 36:0 E-MU XMidi1X1 Tab MIDI 1"
      4 0    "[4] 40:0 USB Midi MIDI 1"
      5 0    "[5] 44:0 Launchpad Mini MIDI 1"
   \end{verbatim}

   Note that the "system:announce" buss is always disabled, as \textsl{Seq66}
   does not use it.  Here is the stored input port-map:

   \begin{verbatim}
      [midi-input-map]
      0   "announce"
      1   "Midi Through Port-0"
      2   "nanoKEY2 MIDI 1"
      3   "E-MU XMidi1X1 Tab MIDI 1"
      4   "USB Midi MIDI 1"
      5   "Launchpad Mini MIDI 1"
   \end{verbatim}

   And here is the system input map with some devices unplugged.

   \begin{verbatim}
      [midi-input]
      1   # map is/not active
      0   "announce"
      1   "Midi Through Port-0"
      2   "USB Midi MIDI 1"
      3   "Launchpad Mini MIDI 1"
   \end{verbatim}

   Note that the mapping can be disabled by setting the first value to 0.  In
   that case, \textsl{Seq66} uses buss numbers in the normal way.
   In the user interface dropdowns for input buss, if a map is active, it is
   put into the dropdown; any missing items are noted and are shown as
   disabled.
   If the map is not active, then only the actual system input ports are shown.

%-------------------------------------------------------------------------------
% vim: ts=3 sw=3 et ft=tex
%-------------------------------------------------------------------------------


% User file

%%% %-------------------------------------------------------------------------------
% usr_file
%-------------------------------------------------------------------------------
%
% \file        usr_file.tex
% \library     Documents
% \author      Chris Ahlstrom
% \date        2015-08-31
% \update      2020-12-07
% \version     $Revision$
% \license     $XPC_GPL_LICENSE$
%
%     Provides the usr_file.
%
%-------------------------------------------------------------------------------

\section{Seq66 'usr' Configuration File}
\label{sec:usr_file}

   There are many \textsl{Seq66} configuration files:
   \texttt{qseq66.rc}
   \texttt{qseq66.usr},
   \texttt{qseq66.mutes},
   \texttt{qseq66.drums}, and
   \texttt{qseq66.playlist}.

%  See \sectionref{sec:rc_file}; it describes the 'rc' file.
%  It is a bit different in how it is handled.

   This section describes the \textsl{Seq66} 'usr' (or "user") file.
   The \textsl{Seq66} 'usr'
   configuration file provides a way to give more
   informative names to the MIDI busses, MIDI channels, and MIDI controllers of
   a given system setup.  This configuration overrides the default values
   of the \textbf{Event} drop-down list and menu items in the Pattern editor,
   and make them reflect the names of the MIDI Control (CC) values of one's
   devices.

   In \textsl{Seq66} it, also includes some items that affect the
   user-interface's look, and many other new configuration items.
   At some point we will likely split this file into another configuration file
   ("qseq66.ui"?)

   \index{usr!-u}
   \index{usr!--user-save}
   Unlike the 'rc' file, the 'usr' file is \textsl{not} written every time
   \textsl{Seq66} exits.  If the 'usr' files does not exist, one is
   created, but it is normally not overwritten thereafter.  To
   cause it to be overwritten at exit, run \textsl{Seq66} with the
   \texttt{-u} or \texttt{--user-save} option:

   \begin{verbatim}
      $ seq66 --user-save
   \end{verbatim}

   This option is recommended when one installs a new version of
   \textsl{Seq66}, which might add new options to the 'usr' file.
   See \sectionref{sec:man_page}; it discusses more options involving the
   'usr' file.

   Another difference between the 'rc' file and the 'usr' file is that
   the 'usr' file currently has no graphical user-interface dialog to
   configure the 'usr' settings.  One has to edit the file manually.
   There are a few items that can be tweaked in the \textsl{Seq66} application, and
   if they are modified, the user-save flag is turned on.

   By default, the list of MIDI devices that \textsl{Seq66} shows depends
   on one's system setup and whether the manual-port option is specified
   or not.  Here's our system, with the
   the \texttt{[manual-port]} option turned off, shown in a
   composite view with all menus one can look at for MIDI settings:

\begin{figure}[H]
   \centering 
   \includegraphics[scale=0.65]{usr/manual-0-buss-dropdown.png}
   \caption{Seq66 Composite View of Native Devices}
   \label{fig:manual_0_buss_dropdown}
\end{figure}

   At the top center, the dropdown menu contains the MIDI busses/ports
   supported by this computer.  At right, the MIDI channel shows
   the channels numbers that can be picked for buss 0.  At bottom left, we see
   the default controller values that \textsl{Seq66} includes.  We have
   no idea if these correspond to any controllers that the selected MIDI buss
   supports.  We \textsl{can} use this dropdown to see if any such controller
   events are in the loaded MIDI file, of course; a solid black square
   indicates that such an event was found in the pattern.

   To change the defaults, we can create a 'usr' file to set them up.
   The discussion here relies on the reader opening the file
   \texttt{sample.usr}, which is included in the shared \texttt{data/samples}
   directory provided once \textsl{Seq66} is installed.

   Assume we have 3 MIDI "buss" devices hooked to our system:
   two Model "2x2" MIDI port devices, and an old PCR-30 MIDI controller
   keyboard.  Let's number them:

   \begin{enumerate}
      \item Model 2x2 A
      \item Model 2x2 B
      \item PCR-30
   \end{enumerate}

   Then assume that we have nine different MIDI instruments in our kit.
   Let's number them, too:

   \begin{enumerate}
      \item Waldorf Micro Q
      \item SuperNova
      \item DrumStation
      \item TX81Z
      \item WaveStation
      \item ESI-2000
      \item ES-1
      \item ER-1
      \item TB-303
   \end{enumerate}

   The \textsl{Waldorf Micro Q},
   the \textsl{SuperNova},
   and the \textsl{DrumStation} all have a large
   number of special MIDI controller values for modifying the sound they
   produce.
   The \textsl{DrumStation} accepts MIDI controllers that change various
   features of the sound of each type of drum it supports.

   The buss devices can be configured to route certain
   MIDI channels to certain MIDI devices.  Assume we have them
   set up this way:

   \begin{enumerate}
      \item Bus 0: Model 2x2 A
      \begin{itemize}
         \item SuperNova: channels 1 to 8
         \item TX81Z: channels 9 to 11
         \item Waldorf Micro Q: channels 12 to 15
         \item DrumStation: channel 16
      \end{itemize}
      \item Bus 1: Model 2x2 B
      \begin{itemize}
         \item WaveStation: channels 1 to 4
         \item ESI-2000: channels 5 to 14
         \item ES-1: channel 15
         \item ER-1: channel 16
      \end{itemize}
      \item Bus 2: PCR-30
      \begin{itemize}
         \item TB-303: channel 1
      \end{itemize}
   \end{enumerate}

   We use the \textbf{'usr' configuration file}.
   to show these items with the proper
   names associated with each device, channel, and controller value

   \index{seq66.usr}
   The \textsl{Seq24} configuration file was called
   \texttt{.seq24usr}, and it was stored in the user's \texttt{\$HOME}
   directory.
   \textsl{Seq66} uses a new file-name
   to take its place.
   After one runs \textsl{Seq66} for the first time (or after deleting
   the configuration files), it will generate a
   \texttt{seq66.usr} file in one's HOME directory:

   \begin{verbatim}
      /home/user/.config/seq66/seq66.usr
   \end{verbatim}

   It allows you to give an alias to 
   each MIDI bus, MIDI channel, and MIDI control 
   codes, per channel.
   The process for setting up the 'usr' file is to:

   \begin{enumber}
      \item Define one or more MIDI busses, the name of each, and what
         instruments are on which channels.  Each buss is configured in a
         section of the form "\texttt{[user-midi-bus-X]}", where "X" ranges
         from 0 on up.  Each buss then defines up to 16 channel entries.
         Each entry includes the channel number and the number of a
         section in the user-instrument section described next.
      \item Define all of the instruments and their controller
         names, if they have them.  Each instrument is configured in a
         section of the form "\texttt{[user-instrument-X]}", where "X"
         ranges from 0 on up.  Up to 128 controllers can be defined.
   \end{enumber}

   Let's walk through the structure of this setup, since it is a little bit
   tricky.  Here is a diagram of the relationships between the buss definitions
   and instrument definitions:

\begin{figure}[H]
   \centering 
   \includegraphics[scale=0.50]{usr/user-busses-and-instruments.png}
   \caption{Busses and Instruments in the 'usr' File}
   \label{fig:manual_user_busses_and_instruments}
\end{figure}

   The first section in the 'usr' file (after \texttt{[comments]})
   is \texttt{[user-midi-bus-definitions]}.  The solid diamond link, with the
   "*" marker, indicates that this section contains an arbitrary number ("*")
   of \texttt{[user-midi-bus-N]} sections, where "N" ranges from 0 on upward.
   These correspond to the MIDI busses expected to be in the system, ignoring
   the ALSA "announce" buss.

   Each of the busses contains 16 (0 to 15) channel entries.
   These channels are referred to as "instrument numbers", and are
   represented as and linked to "instruments" in the
   \texttt{[user-instrument-definitions]} section.  Each instrument contains up
   to 128 controller values; these controller values are available in the
   \textbf{Event} button in the Pattern Editor, and their names are shown.

   So, each instrument is setup as a "channel" in a particular "buss".
   In the Pattern Editor, when a particular buss and channel is selected,
   the \textbf{Event} menu entries should match the controller entries set up
   in the 'usr' file.

   The list of devices and channels shown earlier
   can be seen in the \textsl{Seq66} sample file
   \texttt{data/samples/sample.usr}.
   Deducting 1 from each device number and channel number (so that numbering
   starts from 0), and consulting the device manuals to determine the
   controller values supported, one can assemble a 'usr' configuration file
   that makes the setup visible in \textsl{Seq66}.

   Peruse the next couple of sections to understand a bit about the format of
   this file, following along in the sample 'usr' file.
   Once satisfied, go to
   \sectionref{subsec:usr_file_midi_bus_results}, and 
   see what it all looks like.

\subsection{'usr' File / MIDI Bus Definitions}
\label{subsec:usr_file_midi_bus_definitions}

   \index{usr!user-midi-bus-definitions}
   \index{[user-midi-bus-definitions]}
   This section begins with an
   "INI" group marker \texttt{[user-midi-bus-definitions]}.
   It defines the number of user busses that will be configured in this file.

   \begin{verbatim}
      [user-midi-bus-definitions]
      3     # number of user-defined MIDI busses
   \end{verbatim}

   \index{usr!user-midi-bus-n}
   \index{[user-midi-bus-n]}
   This means that the 'usr' file will have three MIDI buss
   sections: [user-midi-bus-0], [user-midi-bus-1], and [user-midi-bus-2].
   Here's is an example of one such section:

   \begin{verbatim}
      [user-midi-bus-0]
      2x2 A (SuperNova,Q,TX81Z,DrumStation)
      16
      0 1
      1 1      # Instrument #1 of the [user-instrument-definitions] section
      2 1
      . . .
      8 3      # Instrument #3 of the [user-instrument-definitions] section
      9 3
      10 3
      11 0     # Instrument #0 of the [user-instrument-definitions] section
      12 0     # This is the Waldorf Micro Q device defined below
      13 0
      14 0
      15 2     # Instrument #2 of the [user-instrument-definitions] section
   \end{verbatim}

   Here's an example of one that needs only one override:

   \begin{verbatim}
      [user-midi-bus-2]
      PCR-30 (303)
      1
      0 8
      # The rest default to -1... General MIDI
   \end{verbatim}

   These user-instrument entries can be quickly disabled by changing
   the count values to 0.
   These instrument-definition
   sections are read from the 'usr' configuration file only if
   the \texttt{--reveal-ports} option is \textsl{off} ("0");
   this option can also be specified in the
   \texttt{[reveal-ports]} section of the 'rc' file,
   see \sectionref{subsec:rc_file_reveal_ports}.
   Otherwise, the actual port names reported by ALSA are shown.
   The \texttt{user-midi-bus-definitions} and \texttt{user-midi-bus-N} sections
   can be misleading if one wants to have access to the
   actual ALSA ports that exist on the system.

   Try these combinations of options to see what they look like:

   \begin(itemize)
      \item Clocks View, -m (--manual-ports)
      \item Clocks View, -m (--manual-ports) and -R (--hide-ports)
      \item Clocks View, -r (--reveal-ports)
      \item Clocks View, -R (--hide-ports)
      \item Inputs View, -m (--manual-ports)
      \item Inputs View, -r (--reveal-ports)
      \item Inputs View, -R (--hide-ports)
   \end(itemize)

\iffalse

   The following figures show the results of various settings with an
   active 'usr' file.  They have been clipped to save space.

\begin{figure}[H]
   \centering 
   \includegraphics[scale=0.50]{user/seq66-clock-m.png}
   \caption{Clocks View, -m (--manual-ports)}
   \label{fig:clock_m}
\end{figure}

   Above, the virtual (manual) output ports are shown just as created by
   \textsl{Seq66}.
   The \texttt{--reveal-ports} option is \textsl{off} here.

\begin{figure}[H]
   \centering 
   \includegraphics[scale=0.50]{user/seq66-input-m.png}
   \caption{Inputs View with -m (--manual-ports) Option}
   \label{fig:input_m}
\end{figure}

   Above, the single virtual (manual) input port is shown just as created by
   \textsl{Seq66}.
   Again, the \texttt{--reveal-ports} option is \textsl{off} here.

\begin{figure}[H]
   \centering 
   \includegraphics[scale=0.50]{user/seq66-clock-m-R.png}
   \caption{Clocks View, -m (--manual-ports) and -R (--hide-ports)}
   \label{fig:clock_m_R}
\end{figure}

   Above, by adding the "hide" ports option, the system port labels are
   replaced by the labels from the 'usr' file.

\begin{figure}[H]
   \centering 
   \includegraphics[scale=0.50]{user/seq66-clock-r.png}
   \caption{Clocks View, -r (--reveal-ports)}
   \label{fig:clock_r}
\end{figure}

   Above, the "reveal" ports option overrides the device names given in the
   'usr' file, so that the native system names of the output ports are shown.

\begin{figure}[H]
   \centering 
   \includegraphics[scale=0.50]{user/seq66-input-r.png}
   \caption{Inputs View with -r (--reveal-ports) Option}
   \label{fig:input_r}
\end{figure}

   Above, the "reveal" ports option overrides the device names given in the
   'usr' file, so that the native system names of the input ports are shown.
   However, \textsl{Seq66} no longer overrides the
   names of the input ports via the 'usr' file.  This is done to
   save some trouble in displaying the input port names, which are shown
   only in this dialog.  We may consider offering a separate override section
   for the input ports in the future.

\begin{figure}[H]
   \centering 
   \includegraphics[scale=0.50]{user/seq66-clock-R.png}
   \caption{Clocks View with -R (--hide-ports) Option}
   \label{fig:clock_R}
\end{figure}

   The figure above shows how hiding the system port names shows the names
   defined in the 'usr' file.  But notice that the actual port names are shown
   in square brackets, for reference.

\begin{figure}[H]
   \centering 
   \includegraphics[scale=0.50]{user/seq66-input-R.png}
   \caption{Inputs View with -R (--hide-ports) Option}
   \label{fig:input_R}
\end{figure}

   Although the "hide" ports option is specified above, this view is
   currently also the normal view of the input ports, even with device names
   defined in the 'usr' file.
   At this time, there's no real need to show the user-instrument names
   on the input port.  If there turns out to be such a need, the definitions
   would likely need to be different from the definitions for the output ports.
   Another complexity is the possible existence, under ALSA, of the
   \texttt{system:announce} port.

\fi

\subsection{'usr' File / MIDI Instrument Definitions}
\label{subsec:usr_file_midi_instrument_definitions}

   \index{usr!user-instrument-definitions}
   \index{[user-instrument-definitions]}
   This section begins with an
   "INI" group marker \texttt{[user-instrument-definitions]}.
   It defines the number of user instruments that will be configured in this
   file.  This section defines characteristics, such as
   the meanings of MIDI controller values, of the instruments themselves,
   not the MIDI busses to which they attached.

   \begin{verbatim}
      [user-instrument-definitions]
      9     # number of user instrument
   \end{verbatim}

   \index{usr!user-instrument-n}
   \index{[user-instrument-n]}
   So this 'usr' file will define 9 instruments.  We provide only one section
   as an example.  Note that items without text default to the values
   prescribed by the General MIDI (GM) specification.

   \begin{verbatim}
      [user-instrument-0]
      Waldorf Micro Q                     # name of instrument
      128                                 # number of MIDI controllers
      0                                   # first controller value, unnamed
      1 Modulation Wheel
      2 Breath Control
      3 
      4 Foot Control
         . . .
      119
      120 All Sound Off (0)
      121 Reset All Controllers (0)
      122 Local Control (0-127) (Off,On)
      123 All Notes Off (0)
      124                                 # defaults to GM
      125 Unsupported
      126 Unsupported
      127                                 # defaults to GM
   \end{verbatim}

   Note the unnamed control numbers above.
   An unnamed control number might be an unsupported control number.
   It is termed to be "inactive".  In this case, the \textbf{Event} menu of
   the Pattern editor will show the default name of this controller.
   Again, though, the function denoted by this name might not be supported by
   the device.  In that case, it might be better to call it "Unsupported".
   See the examples above.

\subsection{'usr' File / User Interface Settings}
\label{subsec:usr_file_user_interface_settings}

   \index{usr!user-interface-settings}
   \index{[user-interface-settings]}
   This section, new to \textsl{Seq66}, begins with an
   "INI" group marker \texttt{[user-interface-settings]}.

   It provides for a feature we will hopefully be able to complete some day:
   the complete specificition of the appearance of the user-interface.
   There is plenty of room to change the appearance of
   \textsl{Seq66} already!  Please try the settings and see what looks good.
   Refer to either the sample file or the file generated when \textsl{Seq66} first
   runs.

   \index{usr!grid-style}
   \begin{verbatim}
      [user-interface-settings]
      1       # grid_style
   \end{verbatim}

   \index{usr!grid-brackets}
   \begin{verbatim}
      2       # grid_brackets
   \end{verbatim}

   \index{usr!mainwnd-rows}
   \index{variset}
   \begin{verbatim}
      4       # mainwnd_rows
   \end{verbatim}

   \index{usr!mainwnd-cols}
   \index{variset}
   \begin{verbatim}
      8       # mainwnd_cols
   \end{verbatim}

   \index{usr!max-sets}
   \begin{verbatim}
      32      # max_sets
   \end{verbatim}

   \index{usr!mainwid-border}
   \begin{verbatim}
      0      # mainwid_border
   \end{verbatim}

   \index{usr!mainwid-spacing}
   \begin{verbatim}
      2      # mainwid_spacing
   \end{verbatim}

   \index{usr!control-height}
   \begin{verbatim}
      0      # control_height
   \end{verbatim}

   \index{usr!zoom}
   \begin{verbatim}
      2      # zoom
   \end{verbatim}

   \index{usr!global-seq-feature}
   \begin{verbatim}
      1      # global_seq_feature
   \end{verbatim}

   \index{usr!use-new-font}
   \begin{verbatim}
      1      # use_new_font
   \end{verbatim}

   \index{usr!allow-two-perfedits}
   \begin{verbatim}
      1      # allow_two_perfedits
   \end{verbatim}

   \index{usr!perf-h-page-increment}
   \begin{verbatim}
      4      # perf_h_page_increment
   \end{verbatim}

   \index{usr!perf-v-page-increment}
   \begin{verbatim}
      8      # perf_v_page_increment
   \end{verbatim}

   \index{usr!progress-bar-colored}
   \begin{verbatim}
      6      # progress_bar_colored
   \end{verbatim}

   \index{usr!progress-bar-thick}
   \begin{verbatim}
      1      # progress_bar_thick
   \end{verbatim}

   \index{usr!inverse-colors}
   \begin{verbatim}
      0      # inverse_colors
   \end{verbatim}

   \index{usr!window-redraw-rate}
   \begin{verbatim}
      40     # window_redraw_rate
   \end{verbatim}

   \index{usr!use-more-icons}
   \begin{verbatim}
      0      # use_more_icons (currently affects only main window)
   \end{verbatim}

   \index{usr!block-columns}
   \begin{verbatim}
      2      # block_columns (number of columns of set blocks/wids)
   \end{verbatim}

   \index{usr!block-independent}
   \begin{verbatim}
      1      # block_independent (separate set spinner for blocks/wids)
   \end{verbatim}

   Note that the window-redraw rate option is meant more for experimentation
   than anything else.  It probably doesn't affect CPU usage much, but might
   provide a smoother-running cursor on some systems.

\subsection{'usr' File / User MIDI Settings}
\label{subsec:usr_file_user_midi_settings}

   \index{[user-midi-settings]}
   This section begins with an
   "INI" group marker \texttt{[user-midi-settings]}.
   It supports files with different PPQN, and and allows one to specify the
   global defaults for tempo, beats per measure, and so on.

   \index{usr!midi-ppqn}
   \begin{verbatim}
      [user-midi-settings]
      192     # midi_ppqn
   \end{verbatim}

   \index{usr!midi-beats-per-measure}
   \begin{verbatim}
      4       # midi_beats_per_measure/bar
   \end{verbatim}

   \index{usr!midi-beats-per-minute}
   \begin{verbatim}
      120     # midi_beats_per_minute
   \end{verbatim}

   \index{usr!midi-beat-width}
   \begin{verbatim}
      4       # midi_beat_width
   \end{verbatim}

   \index{usr!midi-buss-override}
   \begin{verbatim}
      -1     # midi_buss_override
   \end{verbatim}

   \index{usr!velocity-override}
   \begin{verbatim}
      80     # velocity_override (-1 = 'Free')
   \end{verbatim}

   \index{usr!bpm-precision}
   \begin{verbatim}
      1      # bpm_precision
   \end{verbatim}

   \index{usr!bpm-step-increment}
   \begin{verbatim}
      0.1    # bpm_step_increment
   \end{verbatim}

   \index{usr!bpm-page-increment}
   \begin{verbatim}
      5.0    # bpm_page_increment
   \end{verbatim}

   \index{usr!midi-bpm-minimum}
   \begin{verbatim}
		0       # midi_bpm_minimum
   \end{verbatim}

   \index{usr!midi-bpm-maximum}
   \begin{verbatim}
		360       # midi_bpm_maximum
   \end{verbatim}

   \index{usr!velocity-override}
      The \texttt{velocity-override} option fixes a long standing (from
      \textsl{Seq24}) bug where the actual incoming note velocity was always
      replaced by a hard-wired value.

   \index{usr!bpm-step-increment}
   \index{usr!bpm-page-increment}
      The \texttt{bpm-precision}, \texttt{bpm-step-increment}, and
      \texttt{bpm-page-increment} values allow more precise control over tempo,
      which makes it easier to match the tempo of external music sources.  Note
      that the step-increment is used by the up/down arrow buttons, the up/down
      arrow keys, and the MIDI BPM control values.  The page-increment is used
      if the BPM field has focus and the Page-Up/Page-Down keys are pressed,
      and new MIDI control values have been added to support coarse MIDI
      control of tempo.

   \index{usr!midi-bpm-minimum}
   \index{usr!midi-bpm-maximum}
		The \texttt{midi-bpm-minimum} and \texttt{midi-bpm-maximum} settings
		are used in scaling the display of Tempo events.
      By adjusting these values, one can more easily see the variations in
      tempo.  In a main window pattern slot, or in the song editor tempo track,
      this range is scaled to the full range of note values, 0 to 127.
      Generally, one wants to select a range that keeps the main tempo line at
      the middle height of the pattern display.

   To obtain these new settings, remember to backup the existing
   \textsl{seq66.usr}, then run \textsl{Seq66} with the
   \texttt{--user-save} option, and then do a "diff" on the new file and the
   original to merge any old values that need to be preserved.  Then make any
   further tweaks to the new values.

\subsection{'usr' File / User Options}
\label{subsec:usr_file_user_options}

   \index{[user-options]}
   This section begins with an
   "INI" group marker \texttt{[user-options]}.
   It provides for additional options keyed by the
   \texttt{-o}/\texttt{--option} options.
   This group of options serves to expand the options that are available, since
   \textsl{Seq66} is  running out of single-character options.
   This group of options are shown below.

   \index{usr!option-daemonize}
   \begin{verbatim}
		0       # option_daemonize
   \end{verbatim}

   If this option is not used when running \texttt{seq66cli}, then the
   application stays in the console window and dumps informational output to
   it.  If this option is in force, then the only way to affect
   \texttt{seq66cli} is to send a signal (e.g. SIGKILL) to it, or use
   MIDI control.

   \index{usr!option-logfile}
   \begin{verbatim}
      "seq66.log"
   \end{verbatim}

   This log-file is written to the same directory as the 'rc' and 'usr' files.
   If this file-name is empty, then a valid file-name must be specified
   in the "--option log=filename.log" option.  Note that this file
   is always written to the \textsl{Seq66} configuration directory.

\subsection{'usr' File / Device and Control Names}
\label{subsec:usr_file_midi_bus_results}

   Okay, now we have this file copied to our home directory:

   \begin{verbatim}
      /home/user/.config/seq66/seq66.usr
   \end{verbatim}

   If we'd already run \textsl{Seq66} at least once, we'd have
   overwritten the skeleton sample file that \textsl{Seq66}
   writes by default.  We now have a full-fledged 'usr' file.

   However, because we don't actually have all that equipment (we got the
   example from the Web, for cryin' out loud), let's see what we end up with
   when we run \textsl{Seq66} this time and show the pattern editor
   settings:

\begin{figure}[H]
   \centering 
   \includegraphics[scale=0.65]{usr/manual-0-userfile-buss-dropdown.png}
   \caption{Seq66 Composite View of Non-Native Devices}
   \label{fig:manual_0_userfile_buss_dropdown}
\end{figure}

   Compare that diagram to \figureref{fig:manual_0_buss_dropdown}.
   If the original figure, we saw the 5 native busses (ports) on our system,
   their bare-bones channel numbers, and the default controller values.  In
   this new figure, we see the three buss devices (ports), plus the two
   Timidity ports.  If we stopped the Timidity service, these would go away.

   Look at the selected buss, "[0]".  It's 16 channels are now associated with
   the devices to which the channels have been assigned.
   Thus, when we have a new pattern we've created in \textsl{Seq66},
   we can assign it to exactly the buss and device we want.

   If we don't have port-mappers installed, and thus have only one playback
   device plugged into the buss, we can still create a setup that
   shows the device and a specific program setup.  Doing so would be tedious,
   but perhaps there's some automated way to do it?
   Lastly, note the following figure.

\begin{figure}[H]
   \centering 
   \includegraphics[scale=0.65]{usr/manual-0-userfile-seq-buss-dropdowns.png}
   \caption{The MIDI Bus Menu for a Specific Pattern}
   \label{fig:manual_0_userfile_seq_buss_dropdown}
\end{figure}

   This figure shows that we can also select the desired port and channel
   directly from the main window.
   There's more to the 'usr' configuration file than we've exposed here.
%  but finding more information about this file has proven a bit tricky.

%  Sometime we would like to create a 'usr' that sets up the
%  \textsl{Yoshimi} 1.3.5+ software synthesizer as a device and instrument.

%-------------------------------------------------------------------------------
% vim: ts=3 sw=3 et ft=tex
%-------------------------------------------------------------------------------


% Playlists

%-------------------------------------------------------------------------------
% playlist
%-------------------------------------------------------------------------------
%
% \file        playlist.tex
% \library     Documents
% \author      Chris Ahlstrom
% \date        2018-09-15
% \update      2018-11-11
% \version     $Revision$
% \license     $XPC_GPL_LICENSE$
%
%     Provides a discussion of the MIDI GUI playlist that Seq66
%     supports.
%
%-------------------------------------------------------------------------------

\section{Seq66 Play-Lists}
\label{sec:playlist}

   \textsl{Seq66} supports play-lists.
   A play-list is a variation on the 'rc' file, conventionally ending with the
   extension \texttt{.playlist}.  It contains a number of "playlist" sections,
   each with a human-readable title, selectable via a MIDI data number,
   or by moving to the next or previous playlist in the list using the
   \texttt{Up} and \texttt{Down} arrow keys.
   Each playlist section contains a list of songs, also selectable via a MIDI
   data number, or by moving to the next or previous song in the list using the
   \texttt{Left} and \texttt{Right} arrow keys.

   Movement between the playlists and the songs is accomplished via 
   MIDI control; see
   \sectionref{subsec:rc_file_midi_control},
   it describes the general usage of the [midi-control] section.
   Using MIDI control makes it possible to use the \texttt{seq66cli}
   headless version of \textsl{Seq66} in a live setting.
   In the normal user-interface, play-list movement
   can also be done manually via the four arrow keys on the computer
   keyboard.

   The playlist file can be specified on the command-line, in
   the 'rc' file (see \sectionref{subsec:rc_file_playlist}), or be loaded
   from the \textbf{File / Open Playlist} menu.
   If it is specified on the command line, that playlist setup will
   be written to the 'rc' file.  It can be removed by specifying a blank (i.e.
   two double-quotes, "") play-list name.
   The file extension is \texttt{.playlist}.

   The Qt user-interface supports editing of the play-list, though not yet
   perfected.
   The user can use a text editor to edit the play-list file, if careful.

   The play-list format is defined in the following section.
   Later sections describe the user-interface.

\subsection{Seq66 Play-Lists / Format}
\label{subsec:playlist_setup}

   The play-list file, by convention, has a file-name of the form
   \texttt{sample.playlist}.
   The play-list file starts with a hardwired top banner that the user can edit
   with a text editor.  It can also have an optional comments section, much
   like the 'rc' and 'usr' files.  It is \textsl{not} overwritten
   when \textsl{Seq66} exits.

   \begin{verbatim}
   [comments]
   Comments added to this section are preserved....
   \end{verbatim}

   A blank line (without even a space) ends the comment section.
   Following the comments section is a \texttt{[playlist-options]} section.

   \begin{verbatim}
   [playlist-options]
   1   # If set to 1, when a new song is selected, unmute all its patterns
   0   # If set to 1, every MIDI song is opened to verify it.
   \end{verbatim}

   The first option allows the load of the next song to enable the patterns in
   that song.
   The second option causes each MIDI file to be opened to verify that it is an
   error-free play-list.  This process can be time-consuming for large
   playlists.

   Following the options section are one or more \texttt{[playlist]} sections.
   Here is the layout of a sample playlist section.

   \begin{verbatim}
   [playlist]

   # Playlist number, arbitrary but unique. 0 to 127 recommended
   # for use with MIDI playlist control.
   126

   # Display name of this play list.
   "Music for Serious Dogs"

   # Storage directory for the song-files in this play list.
   contrib/midi/

   # Provides the MIDI song-control number, and also the
   # base file-name (tune.midi) of each song in this playlist.
   # The playlist directory is used, unless the file-name contains its
   # own path.
   70 allofarow.mid
   71 CountryStrum.midi
   72 contrib/wrk/longhair.wrk
   \end{verbatim}

   \index{playlist!tag}
   A play-list file can have more than one \texttt{[playlist]} section.  This
   allows for partitioning songs into various groups that can be easily
   selected (e.g. based on the mood of the musician or the audience).

   \index{playlist!number}
   After the \texttt{[playlist]} tag comes the play-list number.
   This number can be any non-negative value.
   However, in order to use MIDI control to select the playlist, this number
   should be limited to the range 0 to 127.
   If there is more than one \texttt{[playlist]} section, they are ordered by
   this number, regardless of where they sit in the play-list file.

   \index{playlist!title}
   Next comes a human-readable name for the playlist, which is meant to be
   displayed in the user-interface.  If surrounded by quotes, the quotes are
   removed before usage.

   \index{playlist!song-storage directory}
   Next is the song-storage directory.
   This directory is the default location in which to find the songs.
   It can be an absolute directory or a relative directory.
   However, be wary of using relative directories, since they depend on where
   \textsl{Seq66} is run.
   Also, if a song's file-name  has its own directory component, that overrides
   the default song-storage directory.

   Lastly, there is a list of MIDI song file-names, preceded by their numbers.
   As with the playlist numbers, it is recommended to keep them between 0 and
   127, for usage with MIDI control.  And the songs are ordered by this number,
   rather than by their position in the list.

\subsection{Seq66 Play-Lists / 'rc' File}
\label{subsec:playlist_rc_file}

   The most consistent way to specify a play-list is to add an entry like the
   following to the 'rc' file:

   \begin{verbatim}
   [playlist]
   # Provides a configured play-list and a flag to activate it.
   0     # playlist_active, 1 = active, 0 = do not use it
   # Provides the name of a play-list.  If there is none, use '""'.
   # Or set the flag above to 0.
   /home/ahlstrom/.config/sequencer66/sample.playlist
   \end{verbatim}

   This setup allows a play-list file to be specified and activated.
   If the name of the play-list file does \textsl{not} contain a directory,
   then the play-list file is search for in the user's \textsl{Seq66}
   configuration directory.

   If the play-list file-name is empty (i.e. set to \texttt{""}, then there is
   no play-list active.

\subsection{Seq66 Play-Lists / 'rc' File / [midi-control]}
\label{subsec:playlist_rc_file_midi_ctrl}

   The MIDI control stanzas for play-list and song-selection don't quite follow
   the toggle/on/off convention of the \texttt{[midi-control]} section, though
   the layout is the same:

   \begin{verbatim}
      # MIDI Control to select playlist (value, next, previous)
      88 [1 0 144   2   1 127] [1 0 144   4   1 127] [1 0 144   0   1 127]
      # MIDI Control to select song in current playlist (value, next, previous)
      89 [1 0 144   5   1 127] [1 0 144   3   1 127] [1 0 144   1   1 127]
   \end{verbatim}

   Both lines specify setting the next playlist or song according to a number,
   or via "next" and "previous" controls.  The "next" and "previous" controls
   can be implemented by any MIDI event, including \textsl{Note On} or
   \textsl{Program Change}.  However, the "value" section requires a MIDI event
   that provides a \texttt{d1} (second data byte) value, because this value is
   used as the MIDI control number to select a playlist or song item.
   So, the following setting,

   \begin{verbatim}
      88 [1 0 144   2   1 127]
   \end{verbatim}

   specifies that a \textsl{Note On} event with channel 0 (144 = 0x90) on note
   \#2 with a velocity between the range 1 to 127 will select a play-list.
   However, this selection will be made only if the velocity ranges from 1 to
   127, and there exists a selection with that velocity in the play-list file.
   This control requires a controller device that can be configured to provide
   the exact \textsl{Note On} event, including the exact velocity.

\subsection{Seq66 Play-Lists / Command Line Invocation}
\label{subsec:playlist_cmd_line}

   The command-line options to specify (and activate) the play-list feature
   are:

   \begin{verbatim}
      -X playlist_file
      --playlist playlist_file
   \end{verbatim}

   The play-list file is either a base-name (e.g. \texttt{sample.playlist})
   or a name that includes the full path to the play-list file
   (e.g. \texttt{data/sample.playlist}).
   If no path is specified, the directory is the currently set
   \textsl{Seq66} configuration-file directory.

   Please note that any play-list file specified on the command line
   will be written into the 'rc' file's \texttt{[playlist]} section when
   \textsl{Seq66} exits.

\subsection{Seq66 Play-Lists / Verification}
\label{subsec:playlist_verify}

   When \textsl{Seq66} loads a play-list file, an option allows every
   song in the play-list file to be verified by loading it.  If any load fails,
   then the playlist will fail to load.  This check can be slow when there are
   many large MIDI files specified in the play-list file.

\subsection{Seq66 Play-Lists / User Interfaces}
\label{subsec:playlist_uis}

   Playlists and songs can be selected or moved-to via keystrokes or
   user-interface actions, in addition to MIDI control.

   The \texttt{Up} and \texttt{Down} arrows move forward or backward through
   the list of play-lists, and the
   The \texttt{Right} and \texttt{Left} arrows move forward or backward through
   the list of songs for the currently-selected play-list.

\subsubsection{Seq66 Play-Lists / User Interfaces / Qt 5}
\label{subsubsec:playlist_ui_qt}

   The Qt 5 user-interface supports the full display, selection, and editing of
   the play-lists and the song-list for each play-list.
   There are still some minor issues to work out.  If encountered, close
   \textsl{Seq66} and edit the \texttt{.playlist} file manually.
   It is self-documenting.

\begin{figure}[H]
   \centering 
   \includegraphics[scale=0.65]{playlist/personal-playlist-light.png}
   \caption*{Qt 5 Playlist Tab}
\end{figure}

   There is a lot to talk about in this tab.

   \begin{enumber}
      \item \textbf{Playlist File}.
         This field displays the path to the loaded
         play-list file.  It is not editable.  Remember that
         a play-list file can contain multiple play-lists.
      \item \textbf{Playlist Selection}.
         These fields display the main MIDI-file directory,
         the MIDI control number, and the name of the selected play-list.
         The \textbf{Directory} is where the MIDI files reside by default.
         A file-name can include a different path, however.
         These fields are editable, with the intent to use them to add a new
         play-list or modify the current one.
      \item \textbf{Song Selection}.
         These fields display the MIDI-file directory,
         the MIDI control number, and the file-name of the selected play-list.
         Note that the directory is normally the play-list directory, but a
         path present in the MIDI file-name overrides that directory.
         These fields are editable, with the intent to use them to add a new
         song or modify the "meta information" of the current one.
      \item \textbf{List Names}.
         This table shows the MIDI-control number and
         the name of each play-list.
      \item \textbf{List Buttons}.
         These buttons are described below.
         Please note that, in some cases, the exact functionality is still
         being worked out or perfected.
      \item \textbf{Song Files in List}.
         This table shows the MIDI-control number and
         the name of each song.
      \item \textbf{Song Files in List Buttons}.
         These buttons are described below.
         Please note that, in some cases, the exact functionality is still
         being worked out or perfected.
   \end{enumber}

\paragraph{Seq66 Play-Lists / User Interfaces / Playlist Buttons}
\label{paragraph:playlist_ui_qt_playlist_buttons}

   This section briefly describes the "List" buttons to the right of the
   play-list table.

   \begin{enumber}
      \item \textbf{Load List}.
   \end{enumber}

%-------------------------------------------------------------------------------
% vim: ts=3 sw=3 et ft=tex
%-------------------------------------------------------------------------------


% Palettes

%-------------------------------------------------------------------------------
% palettes
%-------------------------------------------------------------------------------
%
% \file        palettes.tex
% \library     Documents
% \author      Chris Ahlstrom
% \date        2020-12-29
% \update      2021-02-19
% \version     $Revision$
% \license     $XPC_GPL_LICENSE$
%
%     Provides a discussion of the MIDI GUI palettes that Seq66
%     supports.
%
%-------------------------------------------------------------------------------

\section{Palettes for Coloring}
\label{sec:palettes}

   Many user-interface elements in \textsl{Seq66} are drawn independently of
   the Qt theme in force, and they have their own coloring.  Also, patterns can
   be colored, and the color is stored (as a color number) in the pattern when
   the tune is saved.

   There are four palettes:

   \begin{itemize}
      \item \textbf{Pattern}.  This palette contains 32 color entries, and each
         can be used to add color to a pattern in the \textsl{Live} grid or in
         the \textsl{Song} editor.  The color of a pattern, if used, is saved
         with the pattern in the MIDI file.
      \item \textbf{Ui}.  This palette contains 16 color entries.  These
         color entries are used in drawing text, backgrounds, grid lines,
         background patterns, drum notes, and more.  These colors each have a
         counterpart that is used with the \texttt{-{}-inverse} option is applied
         to a run of \textsl{Seq66}.
      \item \textbf{Inverse Ui}.  This palette contains 16 color entries.
         These colors are used when the \texttt{-{}-inverse} option is applied
         to a run of \textsl{Seq66}.
      \item \textbf{Brushes}.  This "palette" provides a way to specify the
         fill type for the drawing of notes, the scale (if shown) in the
         pattern editor, and the background sequence (if shown).  It allows the
         user to select solid file, hatching, and some other fill patterns.
   \end{itemize}

   All palettes have default values built into the application.  However, the
   user can also include 'palette' files to change the colors used.  For
   example, the normal colored palette can be changed to a gray-scale palette.
   The name of the palette file is specified in the 'rc' file by lines like the
   following:

   \begin{verbatim}
      [palette-file]
      1     # palette_active
      qseq66-alt-gray.palette
   \end{verbatim}

   If this palette file is active, it is loaded, changing all of the palettes,
   and thus the coloring of \textsl{Seq64}.

\subsection{Palettes Setup}
\label{subsec:palettes_setup}

   The palette file is a standard \textsl{Seq66} configuration file with a name
   something like \texttt{qseq66.palette}, plus two sections:

   \begin{verbatim}
      [palette]
      [ui-palette]
   \end{verbatim}

   The first section is the "Pattern" palette, and the second section is the
   "Ui" palette, which includes the inverse palette as well.

\subsubsection{Palettes Setup / Pattern}
\label{subsubsec:palettes_setup_pattern}

   The following shows the pattern palette, with some entries elided for
   brevity:

   \begin{verbatim}
      [palette]
       0            "Black" [ 0xFF000000 ]      "White" [ 0xFFFFFFFF ]
       1              "Red" [ 0xFFFF0000 ]      "White" [ 0xFFFFFFFF ]
       2            "Green" [ 0xFF008000 ]      "White" [ 0xFFFFFFFF ]
       3           "Yellow" [ 0xFFFFFF00 ]      "Black" [ 0xFF000000 ]
       4             "Blue" [ 0xFF0000FF ]      "White" [ 0xFFFFFFFF ]
       ...            ...       ...             ...       ...
      29      "Dark Violet" [ 0xFF9400D3 ]      "Black" [ 0xFF000000 ]
      30       "Light Grey" [ 0xFF778899 ]      "Black" [ 0xFF000000 ]
      31        "Dark Grey" [ 0xFF2F4F4F ]      "Black" [ 0xFF000000 ]
project.
   \end{verbatim}

   The names are color names, and these names are what show up in the popup
   color menus for the pattern buttons in the \textsl{Live} grid.
   The colors on the left are the background colors, and the colors on the
   right are the foreground colors, which are chosen for contrast with the
   background.  The colors are in \texttt{\#AARRGGB} format, with the "\#"
   replaced by "0x" because "\#" starts a comment in \textsl{Seq66}
   configuration files.  Note that all the alpha values are "FF" (opqque); we
   have not yet experimented with changing them.
   Lastly, only 32 entries are accepted.

\subsubsection{Palettes Setup / Ui and Inverse Ui}
\label{subsubsec:palettes_setup_ui}

   The following shows the pattern palette, with some entries elided for
   brevity:

   \begin{verbatim}
      [ui-palette]
       0       "Foreground" [ 0xFF000000 ] "Foreground" [ 0xFFFFFFFF ]
       1       "Background" [ 0xFFFFFFFF ] "Background" [ 0xFF000000 ]
       2            "Label" [ 0xFF000000 ]      "Label" [ 0xFFFFFFFF ]
       3        "Selection" [ 0xFFFFA500 ]  "Selection" [ 0xFFFF00FF ]
       4             "Drum" [ 0xFFFF0000 ]       "Drum" [ 0xFF000080 ]
             ...            ...       ...       ...       ...
      13        "Beat Line" [ 0xFF2F4F4F ]  "Beat Line" [ 0xFF2F4F4F ]
      14        "Step Line" [ 0xFF778899 ]  "Step Line" [ 0xFF808080 ]
      15            "Extra" [ 0xFF778899 ]      "Extra" [ 0xFFBD6BB7 ]
   \end{verbatim}

   Here, the names are feature names, not color names.  The first color is the
   normal color, and the second color is the inverse color.  Only 16 entries
   are accepted.

\subsubsection{Palettes Setup / Brushes}
\label{subsubsec:palettes_setup_brushes}

   The last palette is small, allowing the fill-pattern of a few pattern-editor
   items to be changed.

   \begin{verbatim}
      [brushes]
      empty = nobrush
      note = solid
      scale = dense3
      backseq = dense2
   \end{verbatim}

   On the left of the equals sign is the item than can be filled, and on the
   right side is the \textsl{Qt} brush to be used.  The defaults for most are
   solid fill.

   The entry \texttt{empty} isn't used yet.
   The entry \texttt{note} affects the fill of normal/selected notes.
   The entry \texttt{note} affects the fill for the piano roll scale.  The
   hatching used here makes it easier to recognize that the scale is just there
   for orientation.
   The entry \texttt{note} affects the fill of the background sequence.  The
   hatching used here helps further distinguish the real notes from the
   background notes.

\subsection{Palettes Summary}
\label{subsec:palettes_summary}

   There are some obvious enhancements to this scheme, including increasing the
   number of palette items, synchronizing the palette with the current desktop
   them semi-automatically, and providing a user interface to drag-and-drop
   colors.

%-------------------------------------------------------------------------------
% vim: ts=3 sw=3 et ft=tex
%-------------------------------------------------------------------------------


% Qt / PortMidi / Windows version

%-------------------------------------------------------------------------------
% qt_portmidi
%-------------------------------------------------------------------------------
%
% \file        qt_portmidi.tex
% \library     Documents
% \author      Chris Ahlstrom
% \date        2018-05-28
% \update      2018-10-05
% \version     $Revision$
% \license     $XPC_GPL_LICENSE$
%
%     Provides a discussion of the Qt 5 GUI and portmidi, along with a Windows
%     implementation that Seq66 supports.
%
%     However, the Qt user-interface can be combine with the more powerful
%     Rtmidi framework in the Linux version.
%
%-------------------------------------------------------------------------------

\section{Seq66 Qt 5 and Windows}
\label{sec:qt_portmidi}

   With version 0.95.1 and above,
   \textsl{Seq66} provides builds using
   the \textsl{Qt 5} GUI framework and a modified version of the
   PortMidi framework (it cleans up the code and removes the need for
   \textsl{Java}).
   With this additional work, we now have ...
   \index{Windows}
   \index{Seq66 for Windows}
   \index{qpseq66.exe}
   \textsl{Seq66 for Windows}.

   In \textsl{Linux}, the Qt 5 user-interface can be used with either the
   modified \textsl{PortMidi} library or the more powerful modified
   \textsl{RtMidi} library.

   The Qt 5 user-interface is similar
   the \textsl{Kepler34} project \cite{kepler34}.
   But it uses the internal functions of \textsl{Seq66} and adds a
   number of features not present in \textsl{Kepler34}.
   The Qt version has a tabbed interface, plus some separate windows,
   and different ways of handling the main
   window (the "Live" tab), the performance window (the "Song" tab), and the
   editing window (the "Edit" tab).
   It adds an "Events" tab and a "Playlists" tab.
   For portability reasons, the Qt user-interface will eventually be the
   \textsl{official} version for \textsl{Seq66}.
%  It will eventually be the main version of \textsl{Seq66},
%  simply because Qt 5 has fewer portability issues than Gtkmm-3.

   There is still some \textsl{Seq66} functionality that
   the Qt 5 version lacks:

   \begin{enumerate}
      \item No support for the special coloring of empty pattern slots
         or the pattern currently being edited. A low priority.
      \item A reduced-functionality options/preferences dialog, lacking an
         editor for keystroke commands.
      \item Keystroke support is not as comprehensive as the Linux version.
      \item Modified support for multi-wid.  The Qt interface allows
         for any number of external live frames.
         An arbitrary number of sets (banks) can be shown at once,
         each in its own window.  The window in focus is the active set.
   \end{enumerate}

   We are incorporating existing \textsl{Seq66} user-interface
   features into this Qt 5 version as time and bug-fixes allow.
   With version 0.96.0, we have improved the Qt/Windows version by adding the
   following features:

   \begin{enumerate}
      \item An external pattern editor has been added that matches
         the Gtkmm-2.4 \textsl{Seq66} pattern editor.  It supports
         background sequences, chord entry, event merging and extension,
         scales, snap, etc.  The tabbed pattern editor is still in place, but,
         due to space, it will never support all of the features of the
         external window.
      \item The LFO (low-frequency oscillator) window can be
         called up from the external pattern editor.
      \item The song/performance/trigger editor has been improved, and also
         made available as an external window.
      \item The pattern and song editors now have better support for zooming
         via keystrokes.
      \item A playlist tab has been added to support the new play-list
         functionality.  This tab is still a work in progress.
         See section \sectionref{sec:playlist}.
   \end{enumerate}

   Some ill-advised features from \textsl{Seq66} will not be migrated.
   Ultimately, we want to replace the Gtkmm-2.4 user-interface completely.
   But this is awhile down the road.
   There are a number of alternate versions of \textsl{Seq66} using
   PortMidi:

   \begin{enumerate}
      \item A Gtkmm-2.4 user-interface and PortMidi.  This version is
         Linux-only, and is useful for debugging the internal PortMidi code
         without worrying about the Qt 5 user-interface.
      \item A Qt 5 user-interface and PortMidi on Linux.  This version is
         useful for people who like to experiment, and for debugging Qt 5 and
         PortMidi issues at the same time.
      \item A Qt 5 user-interface and PortMidi on Windows.  This version is
         basically working, but takes some special setup.
         See \sectionref{subsec:qt_portmidi_windows_setup} for details.
   \end{enumerate}

   We also have code in place to support
   \index{Mac OSX}
   \textsl{Mac OSX},
   but currently have no OSX system on which to build and test this
   code.  \textbf{HELP WANTED!}

   In the following sections, we cover the basic differences between the
   Gtkmm-2.4 and Qt 5 versions of the application, as well as how to set up
   \textsl{Seq66 for Windows}.
   We will not show all of the user-interface elements, but we will mention
   important differences one will find.

\subsection{The Qt 5 User Interface}
\label{subsec:qt_portmidi_qt5_user_interface}

   The Qt 5 version of \textsl{Seq66} has an executable name of
   \texttt{qpseq66} (Linux) or \texttt{qpseq66.exe} (Windows).
   It is also possible to build a Qt 5 version with the RtMidi interface
   (\texttt{qseq66}), and
   that will ultimately become the default version for Linux.
   To keep explanations simple, we will refer to the Qt 5 version of
   \textsl{Seq66}, using the \textsl{PortMidi} engine,
   as \textbf{qpseq66}.
   The Gtkmm-2.4/RtMidi version (Linux only) will be referred to as
   \textbf{seq66}.

   Here is a screenshot of the main user-interface of \textsl{qpseq66}, with
   some of the patterns colored for emphasis.

\begin{figure}[H]
   \centering 
%  \includegraphics[scale=0.75]{kepler34/qt5-main-window-linux.png}
   \includegraphics[scale=0.65]{roll.png}
   \caption{Qt 5 Main Window, Linux}
   \label{fig:qt5_main_window_linux}
\end{figure}

   Note the difference in layout from the Gtkmm-2.4 version.
   Also notice that some controls from the Gtkmm-2.4 version are missing.
   Some of these missing items will ultimately be added.

   There are some issues, in our opinion, with the \textsl{Kepler34} coloring
   under various situations (e.g. muting and queuing).  So we have adapted the
   \texttt{grid\_style} option in the "usr" configuration file so that, if set
   to "1", the slots look more like the Gtkmm-2.4 version of the slots:

\begin{figure}[H]
   \centering 
%  \includegraphics[scale=0.75]{kepler34/qt5-main-window-slots-gtkstyle.png}
   \includegraphics[scale=0.65]{roll.png}
   \caption{Qt 5 Slots in Gtk Style}
   \label{fig:qt5_main_window_slots_gtkstyle}
\end{figure}

   In this style, the outer border is either white (muted) or black (unmuted).
   Also, queuing is indicated in the legacy manner (the central box is colored
   gray), rather than with a black frame drawn around the slot.
   Choose which style you like, and send feedback about what changes would
   be desirable for either style.

%  Here is a similar screenshot in Windows 10.
%
% \begin{figure}[H]
%  \centering 
%  \includegraphics[scale=0.75]{kepler34/qt5-main-window-windows.png}
%  \caption{Qt 5 Main Window, Windows 10}
%  \label{fig:qt5_main_window_windows}
% \end{figure}

   Apart from different window decoration, the Windows version looks
   the same.  Coding an application for both Windows and Linux is very
   instructive, and can improve the overall code significantly.  It certainly
   forced us to add some new features to handle the operating-system
   differences!

\subsubsection{Qt 5 Live Slot Menu}
\label{subsubsec:qt_portmidi_qt5_live_slot_menu}

   The Qt 5 version of the right-click pattern/slot menu is slightly different
   from the Gtkmm-2.4 version.

\begin{figure}[H]
   \centering 
%  \includegraphics[scale=0.75]{kepler34/qt5-live-frame-slot-menu.png}
   \includegraphics[scale=0.65]{roll.png}
   \caption{Qt 5 Slot Menu}
   \label{fig:qt5_main_window_slot_menu}
\end{figure}

   It lets one add an external window for the pattern slots (live-frame),
   and lets one open a pattern editor in a tab or in an external window.
   It lets one edit events textually in a tab.  Here are the menu entries to
   describe:
   
   \begin{enumber}
      \item \textbf{New pattern}
      \item \textbf{External live frame}
      \item \textbf{Edit pattern in tab}
      \item \textbf{Edit pattern in window}
      \item \textbf{Edit events in tab}
      \item \textbf{Copy pattern}
      \item \textbf{Cut pattern}
      \item \textbf{Delete pattern}
   \end{enumber}

   Let's describe them.

   \setcounter{ItemCounter}{0}      % Reset the ItemCounter for this list.

   \itempar{New pattern}{qt!new pattern}
      Creates a new pattern in the slot.
      If there is already a pattern there, the user is prompted about a
      sequence already present, with a yes/no question about overwriting it and
      creating a new blank sequence.

   \itempar{External live frame}{qt!external live frame}
      With Qt 5, we decided not to reimplement the Gtkmm-2.4 version's
      "multiwid" features, which allows one to specify a matrix of screen-sets.
      This option creates an additional "Live" frame, which is external to the
      Live frame tab.  This live-frame shows the set/bank corresponding to the
      slot number (tricky!) over which the menu was opened.  This won't work if
      used on a higher set.  When an external live-frame has focus, it
      represents the \textsl{active set/bank}.  The active set is shown in the
      main window in the \textbf{Active Set} field, just to be clear.

\begin{figure}[H]
   \centering 
%  \includegraphics[scale=0.75]{kepler34/qt5-external-live-frame.png}
   \includegraphics[scale=0.65]{roll.png}
   \caption{Qt 5 External Live Frame}
   \label{fig:qt5_main_window_external_live_frame}
\end{figure}

   \itempar{Edit pattern in tab}{qt!edit pattern in tab}
      Sets up a pattern editor in the "Edit" tab.  This editor is not as
      versatile as the external-window pattern editor described in the next
      section.

   \itempar{Edit pattern in window}{qt!edit pattern in window}
      Sets up a pattern editor a separate window.  This pattern editor works a
      lot like the Gtkmm-2.4 version, though it does not support the "fruity"
      mode of editing.

   \itempar{Edit events in tab}{qt!edit events in tab}
      Sets up an event editor in the "Edit" tab.  This editor is a basic editor
      and mostly useful for finding and deleting unwanted events, or adding
      events that are not otherwise available.

   \itempar{Copy pattern}{qt!copy pattern}
      Copies the selected pattern.

   \itempar{Cut pattern}{qt!cut pattern}
      Cuts (and copies) the selected pattern.

   \itempar{Delete pattern}{qt!delete pattern}
      Deletes the selected pattern.

\subsubsection{Qt 5 Live Tab}
\label{subsubsec:qt_portmidi_qt5_live_tab}

   The \textbf{Live} tab of \textsl{qpseq66}, as shown above, is very similar
   to the \textsl{seq66} version.  It supports the
   varisets mode (the alternate row-by-column sizes of \textsl{seq66}).
   In addition, external versions of the Live frame can be instantiated outside
   of the main window, so that the user can
   view and work with a number of set/banks at the same time.

   The size of this window can be changed to a certain degree via
   the command-line option \texttt{--option scale=x.y}, where \texttt{x.y} can
   range from 0.4 to 3.0.  This can be made permanent via the
   \texttt{window-scale} setting in the "usr" file.
   For example, these two options should be used together:
   \texttt{-o sets=8x8 -o scale=1.5}.  Experiment with varying values for these
   options.

   In fact, all of the tabs except for the \textbf{Events} and
   \textbf{Playlist}
   tabs will expand properly to
   use additional space if the application is resized, whether by a
   command-line option or by dragging a corner of the window.

\subsubsection{Qt 5 Song Tab}
\label{subsubsec:qt_portmidi_qt5_song_tab}

\begin{figure}[H]
   \centering 
%  \includegraphics[scale=0.75]{kepler34/qt5-song-window-linux.png}
   \includegraphics[scale=0.65]{roll.png}
   \caption{Qt 5 Song Window, Linux}
   \label{fig:qt5_song_window_linux}
\end{figure}

   This window is very similar to the \textsl{seq66} version.
   Additional features may be added as time goes on.  In addition,
   this frame can be opened in its own external window by
   pressing the "Song Editor" button at the bottom of the main window.

\subsubsection{Qt 5 Edit Tab}
\label{subsubsec:qt_portmidi_qt5_edit_tab}

\begin{figure}[H]
   \centering 
%  \includegraphics[scale=0.75]{kepler34/qt5-edit-window-linux.png}
%  \includegraphics[scale=0.65]{roll.png}
   \includegraphics[scale=0.65]{roll.png}
   \caption{Qt 5 Edit Window, Linux}
   \label{fig:qt5_edit_window_linux}
\end{figure}

   This version of the pattern editor is a bit behind
   \textsl{seq66} for features.  There really isn't enough space in the
   main window for all the features of the \textsl{Seq66} pattern editor.
   Note that the event-data editor panel is not visible here.
   One must scroll down to the bottom of the window to see it and use it.
   Note that that the \textbf{LFO} window is not accessible here.
   Finally, note that the note heights and vertical grid spacing are
   modifiable via a key-height option, which is a feature the Gtkmm-2.4
   user-interface does not offer.

   With version 0.95.1, the pattern/slot popup menu has an option to open the
   pattern in an external window that has a lot more features than the
   pattern editor in the tab.  It also has a different scrolling mechanism,
   more like that of \textsl{seq66}.
   Note that only one of the tab/external pattern
   editors can be shown at the same time for a given pattern.
   This pattern editor is pretty much identical to the Gtkmm-2.4 version,
   except that it does not provide a "fruity" mode of mouse usage.
   Sorry.

\subsubsection{Qt 5 Events Tab}
\label{subsubsec:qt_portmidi_qt5_events_tab}

\begin{figure}[H]
   \centering 
%  \includegraphics[scale=0.75]{kepler34/qt5-event-editor.png}
   \includegraphics[scale=0.65]{roll.png}
   \caption{Qt 5 Events Window}
   \label{fig:qt5_events_window}
\end{figure}

   This window works a lot like \sectionref{sec:event_editor}; please see
   that section for user instructions.  Please remember that, at the present
   time, this view is meant only for viewing events and making minor tweaks to
   a pattern, such as eliminating problematic events.  It is not intended to be
   a full-featured event editor.  In particular, note that editing within each
   cell in the event table is not supported.  Event editing must be done in the
   right panel.  An event must be explicitly selected by clicking on it, before
   it can be modified or deleted.

   Also, if one adds or modifies an event to show up an the end of the pattern,
   by modifying the time stamp, the event will appear there, but the
   user-interface will remain at the current event.

\subsubsection{Qt 5 Playlist Tab}
\label{subsubsec:qt_portmidi_qt5_playlist_tab}

\begin{figure}[H]
   \centering 
%  \includegraphics[scale=0.75]{kepler34/qt5-playlist-editor.png}
   \includegraphics[scale=0.65]{roll.png}
   \caption{Qt 5 Playlist Window}
   \label{fig:qt5_playlist_window}
\end{figure}

   The play-list functionality can be viewed in this tab.
   Currently, this tab only shows the playlist items.  One can
   select a playlist, and view the song-titles in it, but
   editing and loading the play-lists and songs does not yet
   work.  One must edit the play-list file manually for now,
   and load it from the \textbf{File / Open Playlist} menu.

   Also note that the play-list title can be seen in the main live frame, 
   and that play-lists and songs-can be selected with the arrow keys in that
   window.

   See section \sectionref{sec:playlist}; there are many things that can be
   done with the playlist feature, including controlling it via MIDI events.

\subsubsection{Qt 5 Edit / Preferences}
\label{subsubsec:qt_portmidi_qt5_edit_prefs}

\begin{figure}[H]
   \centering 
%  \includegraphics[scale=0.75]{kepler34/qt5-prefs-clock-windows.png}
   \includegraphics[scale=0.65]{roll.png}
   \caption{Qt 5 Clock Preferences, Windows}
   \label{fig:qt5_prefs_clock_windows}
\end{figure}

   In a newer version of this document, we will show the other preferences
   tabs.  There are many tabs we still need to add to this dialog.
   In particular, be sure to go to the \textbf{MIDI Input} tab and
   make sure that your desired input device (e.g. a MIDI keyboard) are shown
   and are enabled.
   Since the Qt GUI pretty much hardwires the keystrokes used for various
   functions such as muting and queuing, we might not provided a keystroke
   mapping editor.  To be determined.
   In any event, many of the keystrokes configured in the "rc" file are
   supported, but one might have to edit the "rc" file directly, keeping
   in mind that it uses Gtkmm-2.4 conventions for keystrokes.
   \textsl{Linux} users can edit the keystrokes and copy the "rc" configuration
   into the Qt 5 "rc" file, \texttt{qpseq66.rc}.

\subsection{Seq66 Windows Setup}
\label{subsec:qt_portmidi_windows_setup}

   \textsl{Seq66 for Windows} (\textsl{qpseq66}) can be installed
   as a portable Zip package anywhere the user desires.  The package is
   self-contained, and contains all the the DLLs needed to run the program.
   \textsl{qpseq66} can also be installed via an NSIS-based installer,
   such as \texttt{seq66\_setup\_0.96.0-0.exe}.

\subsubsection{Seq66 Windows Issues}
\label{subsubsec:qt_portmidi_windows_setup_issues}

    When first starting \textsl{qpseq66} on \textsl{Windows}, one might
    experience some issues.  One issue is that the \textsl{Microsoft MIDI
    Mapper}, rumored to be removed in \textsl{Windows 8} and beyond, is still
    detected by the internal PortMidi library used in \textsl{qpseq66}.
    Another issue is that the built-in \textsl{Microsoft} wave-table
    synthesizer might not be accessible.

    We installed the
    \textsl{CoolSoft MIDIMapper} (\cite{midimapper}) and
    \textsl{VirtualMIDISYnth} (\cite{midisynth}) to try to get
    around these issues, and tried to turn off the
    \texttt{Windows System Sound} setup of
    \textsl{"Allow applications to restrict access to this device."}
    But we still had
    inaccessible devices, and the resulting errors would cause
    \textsl{qpseq66} to
    abort.  So we had to spend a lot of time adding support for
    the disabling of
    inaccessible ports, and saving and restoring the "rc" setup properly
    in the face of device-access errors.

    Here is some output logging on our Windows, generated using the
    \texttt{qpseq66} command-line option
    \texttt{-o log=virtualmidi.log}, which dumps the log file into

   \begin{verbatim}
       C:/Users/chris/AppData/Local/seq66/virtualmidi.log
   \end{verbatim}

\begin{verbatim}
   qpseq66 
   C:/Users/chris/AppData/Local/seq66/virtualmidi.log 
   2018-05-13 09:06:58 
   [MIDIMAPPER] 'mapper in : midiInGetDevCaps() error for device 'MIDIMAPPER':
      'The specified device identifier is out of range'
   pm_winmm_general_inputs(): no input devices
   PortMidi MMSystem 0: Microsoft MIDI Mapper output opened
   PortMidi MMSystem 1: CoolSoft MIDIMapper output closed
   PortMidi MMSystem 2: Microsoft GS Wavetable Synth output opened
   PortMidi MMSystem 3: VirtualMIDISynth #1 output closed
   [Opened MIDI file,
      'C:/Users/chris/Documents/Home/seq66/data/b4uacuse-gm-patchless.midi']
   [Writing rc configuration C:/Users/chris/AppData/Local/seq66/qpseq66.rc]
   PortMidi call failed: [-1] 'Bad pointer'
   PortMidi call failed: [-1] 'Bad pointer'
   Begin closing open devices...
   Warning: devices were left open. They have been closed.
\end{verbatim}

    By disabling the "closed" devices in the \texttt{qpseq66.rc} file,
    we are able to start.  Here are the settings:

\begin{verbatim}
      5    # number of MIDI clocks/busses
      # Output buss name: [0] 0:0 PortMidi:Microsoft MIDI Mapper
      0 0    # buss number, clock status
      # Output buss name: [1] 1:2 PortMidi:CoolSoft MIDIMapper
      1 -1    # buss number, clock status
      # Output buss name: [2] 2:3 PortMidi:Microsoft GS Wavetable Synth
      2 0    # buss number, clock status
      # Output buss name: [3] 3:4 PortMidi:VirtualMIDISynth #1
      3 -1    # buss number, clock status
\end{verbatim}

   These settings tell \textsl{Seq66} to ignore output busses 1 and 3,
   in order to allow it to start without error.

    We still have some minor issues at start up and at exit, but are now able
    to play a tune on the wavetable synthesizer using the
    \texttt{--bus 2} option, which forces output to PortMidi buss 2,
    "Microsoft GS Wavetable Synth".

    If an error occurs at start-up, a file called \texttt{erroneous.rc} is
    written to the configuration directory, and can be examined for issues with
    ports.

\subsubsection{Seq66 Windows Configuration Files}
\label{subsubsec:qt_portmidi_windows_setup_config}

    When you first run \textsl{qpseq66}
    on \textsl{Windows}, it will create a new configuration
    file, with inaccessible devices denoted in the
    \texttt{[midi-clock]} section of
    \texttt{C:/Users/username/AppData/Local/seq66/qpseq66.rc}
    by a \texttt{-1} value.

    On \textsl{Linux},
    the normal directory location of the \textsl{Seq66} configuration
    files is

\begin{verbatim}
    /home/username/.config/seq66
\end{verbatim}

   There are various configuration file names, depending on the name of the
   built application:

\begin{verbatim}
        seq66.rc      The RtMidi Native ALSA/JACK version.
        seq66.usr     The same, but for user-interface settings.
        seq66portmidi.rc    The PortMidi Gtkmm 2.4 version.
        seq66portmidi.usr   The same, but for user-interface settings.
        qseq66.rc           The RtMidi Qt 5 version for Linux.
        qseq66.usr          The same, but for user-interface settings.
        qpseq66.rc          The PortMidi Qt 5 version for Windows.
        qpseq66.usr         The same, but for user-interface settings.
\end{verbatim}

    On \textsl{Windows}, the conventional configuration file
    location is different:
    
\begin{verbatim}
    C:/Users/username/AppData/Local/seq66
\end{verbatim}

    The files are:

\begin{verbatim}
        qpseq66.rc          The PortMidi Qt 5 version for Windows.
        qpseq66.usr         The same, but for some user-interface settings.
\end{verbatim}

    Typical of \textsl{Microsoft}, access to the \texttt{AppData} directory
    is not obvious... "we cannot have users monkeying with the configuration".
    To access \texttt{AppData}, highlight the user-name directory
    (e.g. \texttt{C:/Users/chris}, then append
    "AppData" to the end of it.  Voila! It is now visible in
    \textsl{Windows Explorer}.
    It is a Windows \textsl{thang}.
    Then click on the \texttt{seq66} directory.
    Open \texttt{qpseq66.rc} and see what is in it (scroll down a bit):

\begin{verbatim}
    [midi-clock]
    2    # number of MIDI clocks/busses
    # Output buss name: [0] 0:0 PortMidi:Microsoft MIDI Mapper
    0 0  # buss number, clock status
    # Output buss name: [2] 1:1 PortMidi:Microsoft GS Wavetable Synth (virtual)
    1 0  # buss number, clock status
    # Output buss name: [3] 1:1 PortMidi:nanoKEY2
    2 0  # buss number, clock status
\end{verbatim}
    
\begin{verbatim}
    [midi-input]
    1    # number of input MIDI busses
    # The first number is the port number, and the second number
    # indicates whether it is disabled (0), or enabled (1).
    # [1] 0:1 PortMidi:nanoKEY2
    0 1
\end{verbatim}

   These settings can be accessed via
   \textbf{Edit / Preferences / MIDI Clock} and
   \textbf{Edit / Preferences / MIDI Input} to
	alter the ports accessible, in the \textsl{Windows}
   version of \textsl{Seq66}.
   The operating system may have some devices locked out, though.
   Generally, the device needs to have a "System Sound" setting that allows an
   application to grab exclusive access to the device.
   For now, do a web search for this error message, if you have issues:

\begin{verbatim}
   'The specified device is already in use.  Wait until it is free, and then
   try again.'
\end{verbatim}

%-------------------------------------------------------------------------------
% vim: ts=3 sw=3 et ft=tex
%-------------------------------------------------------------------------------


% Man page

%%% \input{manpage}

% Headless version

%-------------------------------------------------------------------------------
% seq66 headless
%-------------------------------------------------------------------------------
%
% \file        seq66 headless.tex
% \library     Documents
% \author      Chris Ahlstrom
% \date        2018-09-30
% \update      2018-10-01
% \version     $Revision$
% \license     $XPC_GPL_LICENSE$
%
%     Provides a discussion of the MIDI GUI headless that Seq66
%     supports.
%
%-------------------------------------------------------------------------------

\section{Seq66 Headless Version}
\label{sec:headless}

   \textsl{Seq66} can be built as a command-line application,
   as described in \sectionref{sec:build}.
   That is, it can be run from the command-line, but has no user interface.
   It can also be instantiated as a Linux daemon, for totally headless usage.
   Because there is not a lot of visibility into a headless process, the
   setup for \texttt{seq66cli} is a little complex, and the musician must get
   used to blind MIDI control.

\subsection{Seq66 Headless Setup}
\label{subsec:headless_setup}

   The first step in setting up a headless \texttt{seq66cli} session is
   to make sure that the GUI version (\texttt{seq66}) works as expected.
   The GUI and headless configurations need to do the following:
   
   \begin{enumerate}
      \item Access the correct inputs, especially a keyboard or pad controller
         that can be used for controlling the sequencer via MIDI, as well as
         inputing notes.
      \item The MIDI input must be configured with some \texttt{[midi-control]}
         values, so that the headless sequencer can do things like stop and
         start playback, select the next playlist or song, or change other
         sequencer controls.  Also note that an alternative is to provide a 
         \texttt{[midi-control-file]} specified in the "rc" file.
      \item Access the desired outputs, in order to play sounds.  This can
         sometimes be tricky, because \textsl{Seq66} can route all
         patterns to the same output, or can let the patterns decide the
         outputs for themselves.
      \item Use the desired play-list.  The headless sequencer can only select
         songs to play via a pre-configured play-list.
   \end{enumerate}

   Sometimes odd problems, such as the output synthesizer not working, not
   appearing in the list of outputs, can prove a real puzzle.
   Here are the steps used in this test; adapt them to your setup.  For
   simplicity, JACK is not running, and so ALSA is in force.

   \textbf{First}, after booting, plug in the MIDI keyboard or MIDI control
   pad.  Our example here will use the \textsl{Korg nanoKEY2} keyboard.  

   \textbf{Second}, start the desired (software) synthesizer.  We will use the
   synth \textsl{Yoshimi}, with a stock setup from our "Yoshimi Cookbook"
   project.  The order of starting the keyboard/pad and the synthesiser
   will alter the port numbers of these items.  Best to do things in the same
   order every time... be consistent.

   \textbf{Third}, to validate the setup, run a command from the command-line
   such as:

   \begin{verbatim}
      seq66 -b 2 -v -X data/sample.playlist
   \end{verbatim}

   The buss number ("2") may need to be different on your setup to get sound
   routed to the correct synthesizer.  Also, the path to the playlist might
   need to be an absolute path; normally playlists are stored in the
   \texttt{HOME/.config/sequencer66} directory and accessed from there.
   Verify that the main window shows the playlist name, and that the arrow keys
   modify the play-list or song selection.  If that works, verify that the MIDI
   keyboard or pad controller works to change the selection.
   Verify that the current song plays through the synthesizer that was started.
   If this setup works (MIDI controls have the proper effect and the tunes play
   through the synthesizer), proceed to the next step.

   \textbf{Fourth}, edit the \texttt{sequencer66.rc} and \texttt{seq66cli.rc}
   files as described below so that the former's settings of
   \texttt{[midi-clock]} or \texttt{[midi-clock-file]} are entered into the
   latter "rc" configuration.
   For our testing, we use a separate MIDI-control file 
   (\texttt{nanomap.rc}), which is set up
   in the main "rc" file to be read via a \texttt{[midi-clock-file]}
   setting.  The \texttt{nanomap.rc} file sets up our \textsl{nanoKEY2} as
   shown in this figure:

\begin{figure}[H]
   \centering 
%  \includegraphics[scale=1.6]{examples/nanokey-sample-rc.png}
   \includegraphics[scale=0.65]{roll.png}
   \caption{Sample nanoKEY2 Control Setup}
   \label{fig:headless_nanokey2_setup}
\end{figure}

   In this figure, the \textbf{OCT -} button on the nanoKEY2 is pressed until
   it is flashing (not seen in the figure).
   This means that the lowest note on the nanoKEY2 is MIDI note 0, the lowest
   note possible.  With these settings, the playlists and songs can be loaded
   and then played and paused.
   The \texttt{seq66cli.rc} file is edited so that the rather large
   \texttt{[midi-control]} section is replaced by the following:

   \begin{verbatim}
      [midi-control-file]
      nanomap.rc    # (/home/ahlstrom/.config/sequencer66/nanomap.rc)
   \end{verbatim}

   The \texttt{nanomap.rc} file is included in the "data" directory of the
   source-code package.

   \textbf{Fifth}, test the command-line \textsl{Seq66} by running the
   following command (your setup might vary) on the command line:

   \begin{verbatim}
      $ ./Seq66cli/seq66cli -b 2 -X data/sample.playlist} -v
   \end{verbatim}

   There is a play-list option to automatically unmute the sets when a new song
   is selected.  If set, then the first song should be ready to play.
   If it plays, and the play-list seems to work (as indicated by the console
   output and the proper playback), then run \texttt{seq66cli} as a daemon:

   \begin{verbatim}
      $ ./Seq66cli/seq66cli -b 2 -X data/sample.playlist} -v -o daemonize
   \end{verbatim}

   The keyboard controls and sound output should still work.

%-------------------------------------------------------------------------------
% vim: ts=3 sw=3 et ft=tex
%-------------------------------------------------------------------------------


% ---------------------------------------------------------------------
% The following sections are not so important.  Can read the
% sequencer64 manual for them.
% ---------------------------------------------------------------------
%
% MIDI implementation chart
%
% \pagebreak
% \input{midi_impl_chart}
% \pagebreak
%
% Important Concepts
%
% %-------------------------------------------------------------------------------
% concepts
%-------------------------------------------------------------------------------
%
% \file        concepts.tex
% \library     Documents
% \author      Chris Ahlstrom
% \date        2015-11-01
% \update      2021-05-18
% \version     $Revision$
% \license     $XPC_GPL_LICENSE$
%
%     Provides some concepts and terms needed to understand Seq66.
%
%-------------------------------------------------------------------------------

\section{Concepts}
\label{sec:concepts}

   The \textsl{Seq66} program is a loop-player machine with a 
   number of interfaces.  This section is useful
   to present some concepts and definitions of terms as
   they are used in \textsl{Seq66}.  Various terms have been used over
   the years to mean the same thing (e.g. "sequence", "pattern", "loop",
   "track", and "slot"), so it is good to clarify the terminology.

\subsection{Concepts / Terms}
\label{subsec:concepts_terms}

   This section doesn't provide comprehensive coverage of terms.  It
   covers terms that might be puzzling.

\subsubsection{Concepts / Terms / loop, pattern, track, sequence}
\label{subsubsec:concepts_terms_loop}

   \index{loop}
   \index{pattern}
   \index{sequence}
   \textsl{Loop} is a synonym for \textsl{pattern}, \textsl{track},
   or \textsl{sequence}; the terms are used interchangeably.
   Each loop is represented by a box (pattern slot) in the Pattern (main)
   Window.

   A loop is a unit of melody or rhythm
   extending for a small number of measures (in most cases).
   Each loop is represented by a box in the patterns panel.
   Each loo is editable.  All patterns can be layed out in
   a particular arrangement to generate a more complex song.

   \index{slot}
   A \textsl{slot} is a box in a pattern grid that holds a loop.

   Note that other sequencer applications use the term "sequence"
   to apply to the complete song, and not just to one track or pattern in the
   entire song.

\subsubsection{Concepts / Terms / armed, muted}
\label{subsubsec:concepts_terms_armed}

   \index{armed}
   An armed sequence is a MIDI pattern that will be heard.
   "Armed" is the opposite of "muted", and the same as "unmuted".
   Performing an \textsl{arm} operation in \textsl{Seq66}
   means clicking on an "unarmed" sequence in the patterns panel (the main
   window of \textsl{Seq66}).
   An unarmed sequence will not be heard, and it has a normal background.
   When the sequence is \textsl{armed}, it will be heard, and it has a
   more noticeable  background.
   A sequence can be armed or unarmed in many ways:

   \begin{itemize}
      \item Clicking on a sequence/pattern box.
      \item Pressing the hot-key for that sequence/pattern box.
      \item Opening up the Song Editor and starting playback; the
         sequences arm/unarm depending on the layout of the
         sequences and triggers in the piano roll of the Song Editor.
      \item Using a MIDI control, as configured in a 'ctrl' file, to
         toggle the armed status of a pattern.
   \end{itemize}

\subsubsection{Concepts / Terms / bank, screenset}
\label{subsubsec:concepts_terms_bank}

   \index{screen set}
   The \textsl{screen set}
   is a set of patterns that fit within the \textbf{4 x 8}
   grid of loops/patterns in the patterns panel.
   \textsl{Seq66} supports multiple screens sets, up to 32 of them,
   and a name can be given to each for clarity.
   Some other sizes, such as \textbf{8 x 8} and \textbf{12 x 8}, are
   partly supported.  For the most part, the column number is best left at 8.
   \index{bank}
   The term "bank" is \textsl{Kepler34}'s name for "screen set".

   \index{play screen}
   By default, only one set is active and playing at a time.  This set is
   informally termed the "play screen".

\subsubsection{Concepts / Terms / buss, bus, port}
\label{subsubsec:concepts_terms_buss}

   \index{bus}
   \index{buss}
   A \textsl{buss} (also spelled "bus" these days;
   \url{https://en.wikipedia.org/wiki/Busbar}) is an entity onto which
   MIDI events can be placed, in order to be heard or to affect the
   playback, or into which MIDI events can be received, for recording.
   A \textsl{buss} is just another name for port.
   \textsl{Seq66} can also perform some mapping of I/O ports
   for a more flexible studio setup.

\subsubsection{Concepts / Terms / performance, song, trigger}
\label{subsubsec:concepts_terms_performance}

   In the jargon of \textsl{Seq66}, a
   \index{performance}
   \index{song}
   \textsl{performance} or
   \textsl{song}is an organized collection of patterns that play a tune
   automatically.
   This layout of patterns is created using the song editor, sometimes
   called the "performance editor".
   This window controls the song playback in "Song Mode"
   (as opposed to "Live Mode").

   \index{trigger}
   The playback of each track is controlled by a set of triggers created for
   that track.
   A \textsl{trigger} is indicates when a sequence/pattern/loop
   should be played, and how much of the sequence (including repeats) should be
   played.  A song performance consists of a number of sequences, each
   triggered as the musician laid them out.

\subsubsection{Concepts / Terms / Auto-step, Step-Edit}
\label{subsubsec:concepts_terms_auto_step}

   \index{auto-step}
   \index{step-edit}
   Auto-step (step-edit) provide a way to add notes easily when a pattern is
   not playing.  It works in two ways.  In the first way, when drawing notes on
   the pattern-editor's piano roll, dragging the mouse automatically inserts
   notes of the configured length at intervals of the configured snap.
   In the second way, incoming MIDI notes (including chords)
   are automatically logged at the given snap interval, with a length a little
   less than the configured snap interval.

\subsubsection{Concepts / Terms / export}
\label{subsubsec:concepts_terms_export}

   \index{export}
   A \textsl{export} in \textsl{Seq66} is a way of writing a
   song-performance to a more standard MIDI file, so that it can be played
   exactly by other sequencers.
   An export collects all of the unmuted tracks that have
   performance information (triggers) associated with them, and creates one
   larger trigger for each track, repeating the events as indicated by the
   original performance.

\subsubsection{Concepts / Terms / group, mute-group}
\label{subsubsec:concepts_terms_group}

   \index{group}
   A \textsl{group} in \textsl{Seq66} is a
   set of patterns, that can arm (unmute) their playing state
   together.
   Every group contains all sequences in the active screen set. 
   This concept is similar to mute/unmute groups in hardware
   sequencers.
   \index{mute-group}
   Also known as a "mute-group".
   Mute-groups can be stored in the MIDI file or in a 'mutes' file.
   Each mute-group is associated with a keystroke or a MIDI control.
   When applied, the mute-group enables one or more patterns in the current
   screenset.

\subsubsection{Concepts / Terms / PPQN, pulses, ticks, clocks, divisions}
\label{subsubsec:concepts_terms_pulses}

   \index{pulses}
   The concept of "pulses per quarter note", or PPQN, is very important for
   MIDI timing.  To make it a bit more confusing, sometimes these pulses are
   referred to as "ticks", "clocks", and "divisions".
   To make it even more confusing, there are separate timing concepts to
   understand, such as "tempo", "beats per measure", "beats per minute", and
   "MIDI clocks".
   And, when JACK is involved, one must remember that JACK "ticks" come at 10
   times the rate of MIDI ticks.
   A full description of all these terms, and how they are calculated, is
   beyond the scope of this document.  Check out the source code.

\subsubsection{Concepts / Terms / queue, keep queue, snapshot, one-shot}
\label{subsubsec:concepts_terms_queue_mode}

   To "queue" a pattern means to ready it for playback on the next repeat of
   a pattern.  A pattern can be armed immediately, or it can be queued to
   play back the next time the pattern restarts.
   Pattern toggle occurs at the end of the pattern,
   rather than being set immediately.

   A set of queued patterns can be temporarily stored, so that a different
   set of playbacks can occur, before the original set of playbacks is
   restored.

   \index{keep queue}
   \index{queue!keep}
   The "keep queue" functionality allows the queue to be held without
   holding down a button the whole time.  Once this key is pressed,
   then the hot-keys for any pattern can be pressed, over and over,
   to queue each pattern.

   \index{snapshot}
   A \textsl{Seq66} \textsl{snapshot} is a briefly preserved
   state of the patterns.  One can press a snapshot key, change the state of
   the patterns for live playback, and then release the snapshot key to
   revert to the state when it was first pressed.  (One might call it a
   "revert" key.)

\subsection{Concepts / Sound Subsystems}
\label{subsec:concepts_sound_subsystems}

\subsubsection{Concepts / Sound Subsystems / ALSA}
\label{subsubsec:concepts_sound_alsa}

   \textsl{ALSA} is a audio/MIDI system for \textsl{Linux}, with components built
   into the \textsl{Linux} kernel. It is the main subsystem used by
   \textsl{Seq66}.
   It supports virtual port connections via the \texttt{aconnect} program.
   The name of the library used to build
   \textsl{ALSA} projects is \texttt{libasound} \cite{alsa}.

\subsubsection{Concepts / Sound Subsystems / PortMIDI}
\label{subsubsec:concepts_sound_portmidi}

   \textsl{PortMIDI} is a cross-platform API (applications programming
   interface) for MIDI refactored for \textsl{Seq66}.
   It is used in the "portmidi" C++ modules, and provides support for
   \textsl{Seq66} in \textsl{Microsoft Windows} (and potentially
   \textsl{Mac OSX}).
   See reference \cite{portmidi} for the PortMIDI home page; our version
   cuts out code that requires \textsl{Java}.

\subsubsection{Concepts / Sound Subsystems / JACK}
\label{subsubsec:concepts_sound_jack}

   \textsl{JACK} is a cross-platform
   API and infrastructure
   (with an emphasis on \textsl{Linux})
   to make it easier to connect and reroute MIDI
   and audio event between various applications and hardware ports.
   It should be preferred over \textsl{ALSA}, and is selected automatically if
   running.
   It supports virtual port connections via the \texttt{qjackctl} program or
   the \textsl{Non Session Manager}.
   See reference \cite{jack}.

%-------------------------------------------------------------------------------
% vim: ts=3 sw=3 et ft=tex
%-------------------------------------------------------------------------------

%
% Building and debugging Seq66
%
% %-------------------------------------------------------------------------------
% build
%-------------------------------------------------------------------------------
%
% \file        build.tex
% \library     Documents
% \author      Chris Ahlstrom
% \date        2015-11-06
% \update      2019-08-15
% \version     $Revision$
% \license     $XPC_GPL_LICENSE$
%
%     Provides the man page section of seq24-user-manual.tex.
%
%-------------------------------------------------------------------------------

\section{Building Seq66}
\label{sec:build}

   The project packaging for \textsl{Seq66}
   is (still) aimed at developers.
%  But note that we have Debian packages, conventional tarballs,
%  and \textsl{Windows} installers in the "Seq66
%  Packages" project (\cite{sequencer66packages}).
%  If one has packages built for other Linux distributions, let us know, and we
%  can stick them there for others to use.

   This section is a how-to on building
   \textsl{Seq66} from source code.  It's easy with
   these instructions. Note that there are many ways to build
   \textsl{Seq66}, each described in its own section.
   \textsl{Seq66} is available only in CLI (command-line) or Qt 5 forms.
   The latter can be built via \textsl{qtcreator} (easy-peasy)
   or via \textsl{GNU Automake}.

\subsection{Linux INSTALL Build}
\label{subsec:build_install}

   There are many build options.  Some are modifiable via the normal GNU
   \texttt{configure} script method.  Some are modifiable by
   editing the source code to \textbf{\#define} and \textbf{\#undefine} certain
   macros.  If you don't care about such options, start here.
   If you want to see what options are available, skip to
   \sectionref{subsubsec:build_configure}, which has many details one can
   adjust.
   The project includes the conventional GNU \texttt{configure} script, in the
   tarball and in the git-cloned project.  However, the
   the \texttt{bootstrap} script can do a lot of setup,
   as in the following instructions:

   \begin{enumber}
      \item Preload the dependencies, as listed in
         \sectionref{subsec:build_dependencies}.
          If some are missing, the
          \texttt{configure} script will tell you,
          or, at worst, a build error will.
      \item Check-out the desired branch, normally "master".
         Make a branch, if desired, to make changes.
         See the file \texttt{git.txt} in the
         \texttt{contrib/notes} project directory.
      \item From the top project directory, run the commands:
\begin{verbatim}
      $ ./bootstrap
      $ ./configure
\end{verbatim}
      \item For debugging without libtool getting in the way, just run
         one of the following commands, which run the
         \texttt{configure} script, adding the
         \texttt{--enable-debug} and
         \texttt{--disable-shared} options to it.
         The bootstrap script is our start-over
         way of setting up GNU autoconf.
\begin{verbatim}
      $ ./bootstrap --enable-debug
      $ ./bootstrap -ed
\end{verbatim}
      \item Run the make command:
\begin{verbatim}
      $ make
\end{verbatim}
      \item To install \textsl{Seq66}, become root and run:
\begin{verbatim}
      # make install
\end{verbatim}
   \end{enumber}

   Note that the Qt user-interface is standard, and the alternate MIDI
   engine, \textsl{portmidi}, can be configured and built, though best for
   \textsl{Windows}.
   See \sectionref{subsec:build_Qt}.

   Note that one has to build the documentation and Debian packaging
   separately, they are not part of the default build.

   The following command bootstraps and then configures
   for release mode, and greatly reduces the amount of compiler output:
 
\begin{verbatim}
   $ ./bootstrap --enable-release
   $ ./bootstrap -er
\end{verbatim}

   This script option runs:

\begin{verbatim}
   $ ./configure --enable-silent-rules
\end{verbatim}

   It results in abbreviated output, which makes it easier to see
   warnings that might pop up.
   This anti-verbosity option can be overridden at "make" time:

\begin{verbatim}
   $ make V=1
\end{verbatim}

   \texttt{V=0} is another way to quiet down the build.
   Note that the build can be sped up by telling \texttt{make}
   to use more cores.
   For example, if one has an 8-core system:

\begin{verbatim}
   $ make -j 8
\end{verbatim}

%  Of course, one can use fewer than the number of cores, if desired.

\subsection{Options for Seq66 Features}
\label{subsec:build_options}

   \textsl{Seq66} comes with options for the \texttt{configure} command
   and options represented by definable macros in the source code.

\subsubsection{"configure" Options}
\label{subsubsec:build_configure}

   The following \texttt{configure} options can be specified on the command
   line.  The default build uses rtmidi and Qt.

   \setcounter{ItemCounter}{0}      % Reset the ItemCounter for this list.

   \itempar{\texttt{--enable-rtmidi}}{build!enable rtmidi}
        \index{BUILD\_RTMIDI}
        This option can be bootstrapped directly using
        "\texttt{./bootstrap -er}.
        This is the default build.
        It creates a Qt executable, \texttt{qseq66},
        which can do JACK MIDI input/output and transport using the native API.
        \texttt{seq66} falls back to using ALSA if JACK is not running.  
        Like ALSA, the JACK support has an auto-connect feature
        that can be disabled using the "manual" ("virtual-port")
        option.

        We term this version "rtmidi" because we originally
        used the RtMidi (\cite{rtmidi}) project as the basis for the native
        JACK support, but it does not quite fit the usage model
        of \textsl{Seq66}, so we heavily refactored it.

   \itempar{\texttt{--enable-cli}}{build!enable cli}
        \index{BUILD\_RTCLI}
        Bootstrapping directly:
        "\texttt{./bootstrap -er -cli}".
        It sets up a command-line build of \texttt{seq66} that has the
        program name \texttt{seq66cli}.
        This application must be controlled via MIDI controls set up in the
        [midi-control] section of the "rc" file.
        See \sectionref{subsec:rc_file_midi_control}, for
        information on those controls, which include start, stop,
        pause, play-list navigation, and other commands.
        A single song can be loaded via the last command-line argument.
        A number of songs can be loaded via a play-list.
        See \sectionref{sec:playlist}.
        There is an option to make
        the application fork into the background as a daemon.
%
%  \itempar{\texttt{--enable-alsamidi}}{build!enable alsamidi}
%       \index{BUILD\_ALSAMIDI}
%       Bootstrapping directly:
%       "\texttt{./bootstrap -er -am}".
%       This executable is basically the original version of 
%       \textsl{Seq66}, with the original executable name of
%       \texttt{sequencer66}, which we're keeping around as a backup while we
%       work the remaining nits out of the "rtmidi" version of the application.

   \itempar{\texttt{--enable-portmidi}}{build!enable portmidi}
        \index{BUILD\_PORTMIDI}
        Bootstrapping directly:
        "\texttt{./bootstrap -er -pm}".
        This option builds the Linux/PortMIDI version of the application,
        \texttt{seq66portmidi},
        which is is meant as a way to pre-test the
        port to Windows.
        This version is best used for the \textsl{Windows} and
        \textsl{Mac} ports.

%  \itempar{\texttt{--disable-highlight}}{build!disable highlight}
%       \index{SEQ66\_HIGHLIGHT\_EMPTY\_SEQS}
%       Undefines the \texttt{SEQ66\_HIGHLIGHT\_EMPTY\_SEQS}
%       macro, which is otherwise defined by default.  If defined, the
%       application highlights empty patterns by coloring them yellow.
%       If not defined, empty sequences/patterns are shown in the normal
%       black-on-white coloring.  In either case, empty patterns will not be
%       played.
%       Applies only to the Gtkmm user-interface.
%
%   \itempar{\texttt{--disable-lash}}{build!disable lash}
%       \index{SEQ66\_LASH\_SUPPORT}
%       Undefines the \texttt{SEQ66\_LASH\_SUPPORT} macro, but it
%       is now undefined by default.
%       \textsl{Linux} only.
%       Even if this option is left defined,
%       however, \textsl{Seq66} will still not use LASH support unless
%       one specifies \texttt{--lash} on the \texttt{sequencer66} command-line or
%       turn on the new \texttt{[lash-session]} option in the "rc"
%       configuration file,
%       \texttt{\textasciitilde/.config/sequencer66/sequencer66.rc}.

    \itempar{\texttt{--disable-jack}}{build!disable jack}
        \index{SEQ66\_JACK\_SUPPORT}
        Undefines the \texttt{SEQ66\_JACK\_SUPPORT} macro, which is
        defined by default.  Even if defined,
        \textsl{Seq66} will still not use JACK support unless
        one specifies the various JACK options on the \texttt{sequencer66}
        command-line or turn them on in the "rc" configuration file,
        \texttt{\textasciitilde/.config/seq66/qseq66.rc}.
        This is an option for \textsl{Linux} only.

    \itempar{\texttt{--disable-jack-session}}{build!disable jack session}
        \index{SEQ66\_JACK\_SESSION}
        Undefines the \texttt{SEQ66\_JACK\_SESSION} macro, which is
        defined if JACK support is defined, and the
        \texttt{jack/session.h} file is found to be installed on the system.
        This option, if left defined, can be affected by
        command-line options and options in the "rc" configuration file.
        This is an option for \textsl{Linux} only.

%   \itempar{\texttt{--disable-pause}}{build!disable pause support}
%       \index{SEQ66\_PAUSE\_SUPPORT}
%       This option undefines the \texttt{SEQ66\_PAUSE\_SUPPORT} macro,
%       which is defined by default, and provides support for toggling between
%       a Play button and a Pause button, for actually pausing playback
%       in Live mode, and for a new Pause key, which defaults to a
%       period (".").
%
%   \itempar{\texttt{--disable-chords}}{build!disable chords support}
%       \index{SEQ66\_STAZED\_CHORD\_GENERATOR}
%       This option undefines the \texttt{SEQ66\_STAZED\_CHORD\_GENERATOR}
%       macro.  If this macro is defined,
%       then the application is built with support for a Chord button in the
%       pattern editor, which enables entering whole chords with a single
%       click.  This feature is grabbed from \textsl{Seq32} (\cite{seq32}).
%
%   \itempar{\texttt{--disable-transpose}}{build!disable transpose support}
%       \index{SEQ66\_STAZED\_TRANSPOSE}
%       This option undefines the \texttt{SEQ66\_STAZED\_TRANSPOSE}
%       macro.  If this macro is defined,
%       then the application is built with support for a Transpose button in the
%       pattern editor and the song editor.  The Transpose button in the
%       pattern editor allows a pattern to be exempt from transposition,
%       while the Transpose button in the song editor allows transposing the
%       entire song (except for exempt patterns).
%       This feature is grabbed from \textsl{Seq32} (\cite{seq32}).
%
%   \itempar{\texttt{--disable-multiwid}}{build!disable multi-wid support}
%       \index{SEQ66\_MULTI\_MAINWID}
%       Undefines the \texttt{SEQ66\_MULTI\_MAINWID} macro.
%       If this macro is defined (currently the default), then it is
%       possible to show multiple sets in the main window.
%       See \sectionref{subsec:patterns_panel_multiple},
%       which describes this mode.  Gtkmm only.

    To summarize, these option macros define/undefine the following build
    macros:

      \begin{itemize}
%       \item \texttt{SEQ66\_HIGHLIGHT\_EMPTY\_SEQS}
%       \item \texttt{SEQ66\_LASH\_SUPPORT}
        \item \texttt{SEQ66\_JACK\_SUPPORT}
        \item \texttt{SEQ66\_JACK\_SESSION}
%       \item \texttt{SEQ66\_PAUSE\_SUPPORT}
%       \item \texttt{SEQ66\_STAZED\_CHORD\_GENERATOR}
%       \item \texttt{SEQ66\_MULTI\_MAINWID}
      \end{itemize}

   A lot of options have become mandatory over the years, and are not discussed
   here.  And there may be more macros not discussed.  For the latest, see the
   \texttt{INSTALL} file in the source-code project.

\subsubsection{Manually-defined Macros in the Code}
\label{subsubsec:build_macros}

   As we have explored what \textsl{Seq24} does as we improve
   \textsl{Seq66}, we've found of things that might change the code
   for the worse in some minds.
   We mark those changes with macros.
   And sometimes we tried a change, but left it disabled.
   Look at those macros, modify them, and build
   the source code to one's preferences.  If one does not see a macro described
   below, it means we need to catch up with the documentation.

   The following items are not part of the configure script, but can
   be edited manually in the header file
   \texttt{libseq66/include/seq66-features.h}
   to achieve the desired settings:

   \setcounter{ItemCounter}{0}      % Reset the ItemCounter for this list.
   
%   \itempar{\texttt{SEQ66\_EDIT\_SEQUENCE\_HIGHLIGHT}}{build!seq highlight}
%       Defined in the \texttt{perform} module.
%       Provides the option to highlight the currently-editing sequence in the
%       main window view and in the song editor.  If the sequence is muted, it
%       is highlighted in black text on a cyan background.  If it is unmuted,
%       it is highlighted in cyan text on a black background.  The highlighting
%       follows whichever pattern editor or event editor has the focus.
%       Gtkmm only.
%
%   \itempar{\texttt{SEQ66\_USE\_NEW\_FONT}}{build!new font}
%       Already defined in the \texttt{font} module.
%       If defined, a new, anti-aliased,
%       bold font is used in the user-interface.  This new font is implemented
%       in new XPM files in \texttt{resources/pixmaps} directory:
%       \texttt{wen*.xpm}.  The font is slightly
%       larger, but changes the user-interface sizes only to an infinitesmal
%       degree.
%       Gtkmm only.
%
%       \textbf{Obsolete:}
%       \index{obsolete:compile-time font}
%       This option is no longer a compile-time option, but a run-time option.
%       It is now the default, but the usage of the old versus new font can be
%       set in the "user" configuration file.
%       Also, if the legacy mode is specified, the old font becomes the
%       default.
%
%   \itempar{\texttt{SEQ66\_USE\_EVENT\_MAP}}{build!event map}
%       Already defined in the \texttt{event\_list} module.
%       It enables the usage of an
%       \texttt{std::multimap}, instead of an \texttt{std::list},
%       to store MIDI events.  Because
%       the code does a lot of sorting of events, using the
%       \texttt{std::multimap} is actually a lot faster (especially under debug
%       mode, where it takes
%       many seconds for a medium-size MIDI file to load using the
%       \texttt{std::list} implementation.
%       But the \texttt{std::multimap} can be a limiting factor during playback.
%       We use the list implementation and sort the container only after
%       getting all the events loaded.
%
%   \itempar{\texttt{SEQ66\_USE\_MIDI\_VECTOR}}{build!midi vector}
%       Defined in the \texttt{seq66\_features.h} file.
%       It enables the usage of an
%       \texttt{std::vector} instead of \texttt{std::list},
%       to store MIDI data bytes.
%       It provides the preferred alternative to the list for storing and
%       counting the bytes of MIDI data.  It stops the reversing of
%       certain events due to the peculiarities of \texttt{std::list}.
%       This new implementation uses
%       \texttt{std::vector} and does not use \texttt{pop\_back()} to retrieve
%       the bytes for writing to a file.
%
%   \itempar{\texttt{SEQ66\_FOLLOW\_PROGRESS\_BAR}}{build!follow progress bar}
%       Already defined in the \texttt{app\_limits.h} module.
%       It enables the automatic scrolling (horizontal paging) of the pattern
%       editor and song editor piano rolls, to keep the progress bar in view at
%       all times.  This feature is useful for patterns that are longer than
%       the span of the editor windows.  Such scrolling is a common
%       feature of software MIDI sequencers.
%
%  \textbf{Obsolete}:  Still need to replicate the descriptions that follow
%     in the proper sections.
%
%   \itempar{\texttt{SEQ66\_USE\_GREY\_GRID}}{build!grey/normal grid}
%       \textbf{This item is no longer defined}.
%       Instead, the option is now part of the "rc" configuration file.  This
%       description will be moved to the correct section eventually.
%
%       This configuration item causes the pattern slots/boxes to be colored
%       grey (actually, they will be colored normally as per the current GTK
%       them).  Otherwise, they are colored black.  By default, this value is
%       defined (in the \texttt{mainwid} module).
%
%   \itempar{\texttt{SEQ66\_USE\_WHITE\_GRID}}{build!white grid}
%       \textbf{This item is no longer defined}.
%       Instead, the option is now part of the "user" configuration file.  This
%       description will be moved to the correct section eventually.
%
%       This configuration item causes the pattern slots/boxes to be colored
%       white.  Also definable(in the \texttt{mainwid} module).
%
%   \itempar{\texttt{SEQ66\_USE\_BRACKET\_GRID}}{build!normal grid}
%       \textbf{This item is no longer defined}.
%       Instead, the option is now part of the "user" configuration file.  This
%       description will be moved to the correct section eventually.
%       This configuration box that outlines the pattern slots/boxes is
%       painted over to convert the box to look like a pair of brackets.
%       By default, this value is defined (in the \texttt{mainwid} module).
%
%   \itempar{\texttt{SEQ66\_SEQNUMBER\_ON\_GRID}}{build!grid numbers}
%       \textbf{This item is no longer defined}.
%       Instead, the option is now part of the "rc" configuration file.  This
%       description will be moved to the correct section eventually.
%
%       If the "show sequence numbers" option is on, then each
%       of the blank pattern slots in the main window show the would-be
%       sequence number for that slot.  The background color of the numbers
%       will not match the background color of the grid (which matches the
%       chosen GTK theme).  But, no matter what the GTK background color, they
%       will at least be visible.  There is a little image of this style inside
%       the screenshot shown on the first page of this manual.
%
%       If \texttt{SEQ66\_USE\_WHITE\_GRID}
%       are defined, so that the grid cells are white, then the sequence
%       numbering looks a little nicer, as can be seen in the following
%       figure:
%
% \begin{figure}[H]
%  \centering 
%  \includegraphics[scale=0.75]{pattern-window-white-box-numbering.png}
%  \caption{Pattern Window Built for White Grid with Numbering}
%  \label{fig:build_white_box_numbering}
% \end{figure}
%
%       There is a little image of this style inside the screenshot shown on
%       the first page of this manual, as well.
%
%       If neither \texttt{SEQ66\_USE\_GREY\_GRID} nor
%       \texttt{SEQ66\_USE\_WHITE\_GRID} are defined, so that the grid slots
%       are black, then the numbering will be yellow on a black background, and
%       match perfectly.  This style is shown in the following figure:
%
% \begin{figure}[H]
%  \centering 
%  \includegraphics[scale=0.75]{pattern-window-black-box-numbering.png}
%  \caption{Pattern Window Built for Black Grid with Numbering}
%  \label{fig:build_black_box_numbering}
% \end{figure}
%
%     There is a little image of this style inside the screenshot shown on
%     the first page of this manual, as well.
%
%     Take your pick, modify the code accordingly before building it.
%     Perhaps these can eventually be options for the \texttt{configure}
%     script, or even run-time options!  Let us know!
%
%   \itempar{\texttt{SEQ66\_SOLID\_PIANOROLL\_GRID}}{build!solid piano-roll}
%       Enabling this macro makes the grid lines for the piano rolls
%       more solid, with about the same perception of lightness.
%       It also calls in some other tweaks, such as the positioning of
%       markers.  We currently like this look a little better, and so it is
%       the default.  See the \texttt{app\_limits.h}
%       header file for the definition of this variable.
%
%       Here is the pattern editor (sequence editor) with this alternate look.
%
% \begin{figure}[H]
%  \centering 
%  \includegraphics[scale=0.75]{pattern/pattern-editor-alternate-look.png}
%  \caption{Sequence Pattern Editor Alternate Look}
%  \label{fig:pattern_editor_alternate_look}
% \end{figure}
%
%       Note the smmoothness of the grid lines, the extra emphasis of the C
%       notes at each octave, the emphasis of the note-drawing snap lines that
%       mark the default length of a click-to-add note, the emphasis of the
%       beat and bars, and, finally, the new location of the
%       \textbf{END} marker.  Also note the dark-cyan background pattern,
%       discussed elsewhere in this document.
%
%       Here is the grid-styling for an 8/4 time signature in the song editor:
%
% \begin{figure}[H]
%  \centering 
%  \includegraphics[scale=0.75]{song-editor/song-editor-alternate-look.png}
%  \caption{Song Editor Alternate Look}
%  \label{fig:song_editor_alternate_look}
% \end{figure}
%
%     Also note the sequence numbers shown in the bottom left of each pattern
%     name box. This is a new feature, and, as noted elsewhere, is a new
%     option in the \textsl{File / Options / Keyboard} tab and in
%     the "rc" configuration file.
%
%   \itempar{\texttt{SEQ66\_USE\_VI\_SEQROLL\_MODE}}{build!vi seqroll}
%       Definable in the seqroll module, this macro allows the vi hjkl keys to
%       be used as arrow keys for moving notes.  Not yet tested.  We will not
%       make this a default, because it could drive non-vi users nuts.

   \itempar{\texttt{SEQ66\_USE\_DEBUG\_OUTPUT}}{build!debug output}
      Enable this macro in the
      \texttt{globals.h} header file, to see extra console
      output if the application is compiled for debugging.  This macro can be
      activated only if \texttt{SEQ66\_PLATFORM\_DEBUG} is defined, which is taken
      care of by the build process.  If set, this macro turns on extra
      console output for some modules.

%       \begin{itemize}
%          \item \texttt{globals}
%          \item \texttt{jack\_assistant}
%          \item \texttt{optionsfile}
%          \item \texttt{user\_settings}
%       \end{itemize}

      \index{SEQ66\_PLATFORM\_DEBUG}
      Again, note the macro \texttt{SEQ66\_PLATFORM\_DEBUG},
      defined in \texttt{platform\_macros.h} if the application is
      built in debug mode.

\subsection{Seq66 Build Dependencies}
\label{subsec:build_dependencies}

   With luck, the following dependencies will bring in their own
   dependencies when installed.  Build requirements:

     \begin{itemize}
%       \item libgtkmm-2.4-dev (dev is the header-file package)
%       \item libsigc++-2.0-dev
        \item libjack-jackd2-dev
        \item liblash-compat-dev (optional)
     \end{itemize}

   Runtime requirements:

     \begin{itemize}
%       \item libatk-adaptor (and its dependencies)
%       \item libgail-common (and its dependencies)
        \item valgrind (optional, very useful for debugging)
        \item gdb (optional, very useful for debugging)
        \item gprof and gcov (optional, useful for debugging)
     \end{itemize}

   Build tools requirements:

     \begin{itemize}
        \item automake and autoconf
        \item autoconf-archive
        \item g++
        \item make
        \item libtool
     \end{itemize}

   Documentation requirements (very optional):

     \begin{itemize}
        \item doxygen and doxygen-latex
        \item graphviz
        \item texlive
        \item latexmk
     \end{itemize}
      
   Debian packaging (optional):

     \begin{itemize}
        \item debhelper
        \item fakeroot
     \end{itemize}

\subsection{Linux Qt Builds}
\label{subsec:build_Qt}

   The \textsl{Linux} version of \textsl{Seq66} can be built with the Qt
   user-interface, using either the PortMIDI or RtMIDI (preferred) MIDI
   engines.
 
\begin{verbatim}
   $ ./bootstrap --enable-release -rm -qt
   $ ./bootstrap -er -rm -qt
   $ ./bootstrap --enable-release -pm -qt
   $ ./bootstrap -er -pm -qt
\end{verbatim}

   The Qt/RtMIDI combination are the official version
   of \textsl{Seq66} for \textsl{Linux}.
   Our PortMIDI engine, while modestly improved over legacy PortMIDI, is not
   quite as complete as the RtMIDI-derived implementation.

\subsection{Linux Qmake Build}
\label{subsec:build_qmake}

   We wanted to build \textsl{Seq66 for Windows} using GNU tools and the
   MSYS platform on either \textsl{Linux} (cross-compiling) or
   \textsl{Windows}.  But it proved easier to create the
   \texttt{*.pro} files necessary for \textsl{qmake} and be able to build
   on \textsl{Windows} and \textsl{Mac OSX}.
   \index{Qt Creator}
   \index{qtcreator}
   This setup and build can be done through \textsl{Qt Creator}.
   The command-line setup is straightforward.
   From the \textsl{Seq66} directory, run the following
   commands, using either the build directory specified by \textsl{Qt Creator},
   or making your own "shadow build" directory.

   \begin{verbatim}
        $ mkdir debug-build
        $ cd debug-build
        $ qmake -makefile -recursive "CONFIG += debug" ../sequencer66/qpseq66.pro
        $ make
   \end{verbatim}

    One can also use "CONFIG += release", or just leave that off entirely.

    The \texttt{qpseq66.pro} file can also be loaded in the
    nice IDE, \textsl{Qt Creator}, and be configured, built, and debugged
    there.  And one can tweak the GUI elements in that IDE.

\subsection{Windows Qmake Build}
\label{subsec:build_qmake_windows}

   The easiest option for a build on \textsl{Windows} is to install 
   \textsl{Qt Creator} in its open-source edition.
   The executable name is
   \texttt{qt-unified-windows-x86-3.0.5-online.exe} or somesuch.
   Rather than navigate through the corporate pages, just go to
   \url{https://download.qt.io/archive/online_installers/3.0/} and
   pick the latest version.
   
   When installing, be sure to select at least the the 32-bit Mingw tools,
   including \texttt{mingw32-make.exe}, and
   \texttt{qmake.exe}.  The \textsl{Windows}
   \textbf{PATH} must be modified to
   include the path to both executables, and the excutables
   \texttt{moc.exe}, \texttt{uic.exe}, \texttt{rcc.exe}, and
   \texttt{windeployqt.exe}.
   If the installation directory for \textsl{Qt Creator} is
   \texttt{ProgramFiles} (e.g. \texttt{C:/Program Files}), then add
   these directories to the user or system PATH:

   \begin{verbatim}
      %ProgramFiles%\Qt\5.11.1\mingw53_32\bin
      %ProgramFiles%\Qt\Tools\mingw530_32\bin
   \end{verbatim}

   Obviously, the version number might differ from "5.11.1".
   There is a build script in the \texttt{nsis} directory that
   automates the process of making the executable:
   \texttt{build\_release\_package.bat}.
   It also shows the steps needed to do a release build and create a
   \textsl{7-Zip} package, which include editing some macro variables with
   naming or version information.  If one wants to do it manually,
   follow these steps.  (We denote the DOS backslash path separator
   by "/", for our convenience.)

   \begin{enumerate}
      \item Using
         "git clone https://github.com/ahlstromcj/seq66.git"
         or the unpacking of a source-code tarball,
         create the \texttt{seq66} directory with all
         of the project files.
      \item Change to the directory above this directory.
      \item Create an empty "shadow" directory, e.g. \texttt{seq66-release}.
      \item Change to this "shadow" directory.  In that directory, run the
         following command:
         \texttt{qmake -makefile -recursive "CONFIG += release"
            ../seq66/seq66.pro}.
      \item Next, run the following command and wait for the build to
         complete:
         \texttt{mingw32-make > make.log 2>\&1}.
      \item Open \texttt{make.log} and make sure there are no errors.
         Note that the output directory is inside the "shadow" directory and
         is called \texttt{Seq66qt5/release}.
      \item Run the following command to copy necessary \textsl{Qt} DLLs to
         this directory: \linebreak
         \texttt{windeployqt Seq66qt5/release}.
      \item Create the \texttt{data} directory in
         \texttt{windeployqt Seq66qt5/release} and copy some data files:
         \begin{enumerate}
            \item \texttt{mkdir Seq66qt5/release/data}
            \item \texttt{copy ../seq66/data/*.rc Seq66qt5/release/data}
            \item \texttt{copy ../seq66/data/*.usr Seq66qt5/release/data}
            \item \texttt{copy ../seq66/data/*.midi Seq66qt5/release/data}
            \item \texttt{copy ../seq66/data/*.pdf Seq66qt5/release/data}
            \item \texttt{copy ../seq66/data/*.txt Seq66qt5/release/data}
            \item \texttt{copy ../seq66/data/*.playlist Seq66qt5/release/data}
         \end{enumerate}
      \item Change to the \texttt{Seq66qt5} directory and run:
         \texttt{7z a -r qpseq66-release-package-0.96.1.7z \linebreak release/*}
   \end{enumerate}

   \index{portable package}
   At this point, the \texttt{7z} file is useful as a "portable" package
   for the application.  It can also be used to build the installer, as
   shown in the next section.

   By the way, we have not tried using the Microsoft C++ compiler yet.
   If you try it and get the code to work, let us know!

\subsubsection{Windows Installer}
\label{subsec:build_installer_windows}

   Here, we unpack the \texttt{7z} release package and then use
   \textsl{NSIS} to build the installer.
   These steps equire \textsl{7-Zip}
   to be installed and accessible from the DOS
   command-line, as \texttt{7z.exe}.
   Requires \textsl{NSIS 3} to be installed, unless one wants to use
   NSIS on Linux to build the installer (our preferred method).
   We build the installer in \textsl{Linux} using the
   \textsl{nsis} package.  These instructions can be adopted to using the
   \textsl{Windows} GUI interface for \textsl{NSIS}.

   \begin{enumerate}
      \item Copy the file
         \texttt{7z a -r qpseq66-release-package-0.96.1.7z} to
         the top-level project directory, \texttt{seq66}.
      \item Run
         \texttt{7z x qpseq66-release-package-0.96.1.7z} to extract
         the contents to the \texttt{release} directory.
      \item Run the following commands:
         \begin{enumerate}
            \item \texttt{pushd nsis}
            \item \texttt{makensis Seq66Setup.nsi}
            \item \texttt{popd}
         \end{enumerate}
      \item Verify that the installer works.  It's name is like:
         \texttt{seq66\_setup\_0.90.0.exe}
   \end{enumerate}

   The \textsl{bash} script \texttt{packages} does all this, plus
   creates the source-tarball and some other actions for the developers.

%-------------------------------------------------------------------------------
% vim: ts=3 sw=3 et ft=tex
%-------------------------------------------------------------------------------

%
% Discussion of MIDI formats related to Seq24 and Seq66
%
% %-------------------------------------------------------------------------------
% midi_formats
%-------------------------------------------------------------------------------
%
% \file        midi_formats.tex
% \library     Documents
% \author      Chris Ahlstrom
% \date        2015-09-03
% \update      2018-10-31
% \version     $Revision$
% \license     $XPC_GPL_LICENSE$
%
%     Provides a discussion of the formats (legacy and new) of the last
%     track of an Seq24/Seq66 MIDI file.
%
%-------------------------------------------------------------------------------

\section{MIDI Format and Other MIDI Notes}
\label{sec:midi_format_and_midi_notes}

\subsection{Standard MIDI Format 0}
\label{subsec:midi_format_smf_0}

   \index{smf 0}
   \index{channel split}
   \textsl{Seq66} can read and import SMF 0 MIDI files, and performs
   channel splitting automatically.
   When an SMF 0 format is detected, \textsl{Seq66}
%  reads the file as if were an SMF 1 file, but
   puts all of the events into the
   same sequence/pattern.  While the file is being processed, a list of the
   channels present in the track is maintained.

   Tempo and Time Signature events are read, if present.
   When saving a \textsl{Seq66} MIDI file,
%  in non-legacy mode,
   the Tempo and Time Signature events are saved as MIDI events.
   \index{new!time/tempo saved}
   This allows other sequencers to read a \textsl{Seq66} MIDI file.
   This addition of Tempo can fix imported tracks that don't have a
   measure value.
%  (e.g. it has 0 instead of at least one measure) associated
%  with them; unfixed, these tracks have racing progress bars that don't
%  reflect the actually progresss through the track.

   Once the end-of-track is encountered for that pattern, one new empty
   pattern is created for each channel found in the original sequence.
   The events in the main patten are scanned, one by one, and added at the
   end of the appropriate patten.  If the event is a channel event,
   then the event is inserted into the patten that was created for that
   channel.  If the event is a non-channel event, then each patten gets a
   copy of that event.

   After processing, the MIDI buss information, track name, and other pieces of
   information are attached to each sequence.  The following figure shows in
   imported SMF 0 tune, split into tracks.

\begin{figure}[H]
   \centering 
%  \includegraphics[scale=0.65]{smf0/imported-smf-0-song.png}
   \includegraphics[scale=0.65]{roll.png}
   \caption{Imported SMF 0 MIDI Song}
   \label{fig:imported_smf_0_song}
\end{figure}

   The imported SMF 0 track is preserved, in
   pattern slot \#16.  It is highlighted in a dark cyan color to remind the
   user that it is a special \textsl{Seq66} pattern.
%  It has no channel number.  It is
%  assigned the non-existent MIDI channel of 0.
   If the original track had no
   title, this track is named "Untitled".
   One will either delete
   this track before saving the file, or keep it muted.

   Each single-channel track is given a title of either the form
   "N: Track-name" or, if the song was untitled, "Track N".
   The sequence number of each new track is the internal channel number
   (always the actual MIDI channel number minus one).
   The time-signature of each track is set to defaults, unless a
   time-signature event is encountered in the imported file.

   \index{tempo events}
   \index{time signature events}
   \textsl{Seq66} supports reading some other
   information a MIDI SMF 0 track might have, such as the Tempo and the
   Time Signature.  It saves this information in the first track
   of the MIDI file.
   The original SMF 0 track is also shown in the song editor, as in the
   following figure.

\begin{figure}[H]
   \centering 
%  \includegraphics[scale=0.65]{smf0/imported-smf-0-song-editor.png}
   \includegraphics[scale=0.65]{roll.png}
   \caption{SMF 0 MIDI Song in the Song Editor}
   \label{fig:imported_smf_0_song_editor}
\end{figure}

   One is free to edit the imported tune to heart's content.
   Here, we added one instance of each track, including the SMF 0 track,
   to show what the imported song looks like.

\subsection{Legacy Proprietary Track Format}
\label{subsec:legacy_midi_format}

   The authors of \textsl{Seq24} took trouble to ensure that the format
   of the MIDI files it writes are compatible with other MIDI applications.
   \textsl{Seq24} also stores its own information (triggers, MIDI control
   information, etc) in the file, but marked so that other sequencers can read
   the file and ignore its \textsl{Seq24}-specific information.

   \textsl{Seq66} continues that MIDI-compliant behavior, but has
   improved the compliance a bit. 
   We call that last chunk of sequencer-specific information the "proprietary
   track".
   Before we discuss that last, proprietary track, note that the normal MIDI
   tracks that
   precede it include the SeqSpec ("sequencer-specific")
   control tags.
%  shown in \tableref{table:seqspec_items_normal_tracks}.
%  These control tags are global constants in the \textsl{Seq24} source
%  code, ranging from 0x24240001 to 0x24240013.
%
%  \begin{table}[htb]
%     \centering
%     \caption{SeqSpec Items in Normal Tracks}
%     \label{table:seqspec_items_normal_tracks}
%     \begin{tabular}{l l}
%        \texttt{c\_midibus}        & \texttt{24 24 00 01 00 00 00 00} \\
%        \texttt{c\_midich}         & \texttt{24 24 00 02 00 00 00 00} \\
%        \texttt{c\_triggers}       & \texttt{24 24 00 04 00 00 00 00} \\
%        \texttt{c\_timesig}        & \texttt{24 24 00 06 00 00 00 00} \\
%        \texttt{c\_triggers\_new}  & \texttt{24 24 00 08 00 00 00 00} \\
%        \texttt{c\_musickey}       & \texttt{24 24 00 11 00} \\
%        \texttt{c\_musicscale}     & \texttt{24 24 00 12 00} \\
%        \texttt{c\_backsequence}   & \texttt{24 24 00 13 00 00 00 00} \\
%     \end{tabular}
%  \end{table}
%

   All of the SeqSpecs are shown in the next table.
   The \texttt{c\_triggers} tag is obsolete, but still present.

   \begin{table}[htb]
      \centering
      \caption{All SeqSpec Items}
      \label{table:seqspec_items_all}
      \begin{tabular}{l l}
         \texttt{c\_midibus}        & \texttt{24 24 00 01 00 00 00 00} \\
         \texttt{c\_midich}         & \texttt{24 24 00 02 00 00 00 00} \\
         \texttt{c\_midiclocks}     & \texttt{24 24 00 03 00 00 00 00} \\
         \texttt{c\_triggers}       & \texttt{24 24 00 04 00 00 00 00} \\
         \texttt{c\_notes}          & \texttt{24 24 00 05 00 00 00 00} \\
         \texttt{c\_timesig}        & \texttt{24 24 00 06 00 00 00 00} \\
         \texttt{c\_bpmtag}         & \texttt{24 24 00 07 00 00 00 00} \\
         \texttt{c\_triggers\_new}  & \texttt{24 24 00 08 00 00 00 00} \\
         \texttt{c\_mutegroups}     & \texttt{24 24 00 09 00 00 00 00} \\
         \texttt{c\_gap\_A}         & \texttt{24 24 00 0A 00 00 00 00} \\
         \texttt{c\_gap\_B}         & \texttt{24 24 00 0B 00 00 00 00} \\
         \texttt{c\_gap\_C}         & \texttt{24 24 00 0C 00 00 00 00} \\
         \texttt{c\_gap\_D}         & \texttt{24 24 00 0D 00 00 00 00} \\
         \texttt{c\_gap\_E}         & \texttt{24 24 00 0E 00 00 00 00} \\
         \texttt{c\_gap\_F}         & \texttt{24 24 00 0F 00 00 00 00} \\
         \texttt{c\_midictrl}       & \texttt{24 24 00 10 00} \\
         \texttt{c\_musickey}       & \texttt{24 24 00 11 00} \\
         \texttt{c\_musicscale}     & \texttt{24 24 00 12 00} \\
         \texttt{c\_backsequence}   & \texttt{24 24 00 13 00 00 00 00} \\
         \texttt{c\_transpose}      & \texttt{24 24 00 14 00 00 00 00} \\
         \texttt{c\_perf\_bp\_mess} & \texttt{24 24 00 15 00 00 00 00} \\
         \texttt{c\_perf\_bw}       & \texttt{24 24 00 16 00 00 00 00} \\
         \texttt{c\_tempo\_map}     & \texttt{24 24 00 17 00 00 00 00} \\
         \texttt{c\_reserver\_1}    & \texttt{24 24 00 18 00 00 00 00} \\
         \texttt{c\_reserver\_2}    & \texttt{24 24 00 19 00 00 00 00} \\
         \texttt{c\_tempo\_track}   & \texttt{24 24 00 1A 00 00 00 00} \\
         \texttt{c\_seq\_color}     & \texttt{24 24 00 1B 00 00 00 00} \\
         \texttt{c\_seq\_edit\_mode} & \texttt{24 24 00 1C 00 00 00 00} \\
      \end{tabular}
   \end{table}

   \index{saved control tags}
   The \texttt{c\_musickey},
   \texttt{c\_musicscale}, and
   \texttt{c\_backsequence}
   control tags are new with \textsl{Seq66}.
   They are saved as additional information in each sequence in which they
   have been specified in the sequence editor.
   For backward compatibility (and because it is probably the more
   common use case), these items can also be
   saved globally for the whole MIDI song, as an option.

   These tags
%  (created by the application, but not present in the
%  proprietary track, and perhaps also created by other MIDI applications)
   are preceded by the standard MIDI "FF 7F length" meta-event sequence.
   The following discussion applies to the final "proprietary" track as
   saved in the legacy \textsl{Seq24} format.

   After all the counted MIDI event
   tracks are read, \textsl{Seq24} checks for
   extra data after them.
   If there is extra data, \textsl{Seq24} reads a long value.
   The first one encountered is a MIDI "sequencer-specific"
   (\textsl{SeqSpec}) section.  It starts with

   \begin{verbatim}
      0x24240010
   \end{verbatim}

   which is a Seq24 "c\_midictrl" proprietary value flagged by the
   number "0x2424".
%  MIDI requires an "MTrk" marker to start a track, but requires
   MIDI requires this marker to be supported.  Some applications, like
   \textsl{timidity}, handle it.  Others % , like \textsl{midicvt},
   complain about an unexpected header marker.
   Next, MIDI wants to see this triad of bytes

   \begin{verbatim}
      status = FF, type= 7F (proprietary), length = whatever
   \end{verbatim}

   to precede proprietary data.
%  Now, as shown by \tableref{table:midi_file_support_table},
%  most applications accept the shortcut legacy format, but \textsl{midicvt}
%  does not.

   \textsl{Seq66} writes this information properly,
   starting with the \texttt{0x242400nn xFF 0x7F} marker.
   We also need to be able to read legacy Seq24 MIDI files, so that ability has
   been preserved in \textsl{Seq66}.

   At this point, we have the \textbf{c\_midictrl} information now.
   Next, we read a long value, seqs.  It is 0.

   \begin{verbatim}
      24 24 00 10 00 00 00 00
   \end{verbatim}

   Read the next long value, 0x24240003.  This is \textbf{c\_midiclocks}.
   We get a value of 0 for "TrackLength" (now a local variable called
   "busscount"):

   \begin{verbatim}
      24 24 00 03 00 00 00 00
   \end{verbatim}

   If the buss-count was greater than 0, then for each value, we would read a
   byte value represent the bus a clock was on, and setting the clock value
   of the master MIDI buss.
   Another check for more data is made.

   \begin{verbatim}
      24 24 00 05 00 20 00 00
   \end{verbatim}

   0x24240005 is \textbf{c\_notes}.  The value screen\_sets is read (two
   bytes) and
   here is 0x20 = 32.  For each screen-set:

   \begin{verbatim}
      len = read\_short()
   \end{verbatim}

   If non-zero, each of the \texttt{len} bytes is appended as a string.
   Here, len is 0 for all 32 screensets, so the screen-set notepad is set to
   an empty string.
   Another check for more data is made.

   \begin{verbatim}
      24 24 00 07 00 00 00 78
   \end{verbatim}

   0x24240007 is \textbf{c\_bpmtag}.  The long value is read and sets the
   perform object's bpm value.  Here, it is 120 bpm.
   Another check for more data is made.

   \begin{verbatim}
      24 24 00 09 00 00 04 00
   \end{verbatim}

   0x24240009 is \textbf{c\_mutegroups}.  The long value obtained here is
   1024.  If this value is not equal to the constant
   \textbf{c\_gmute\_tracks} (1024), a warning is emitted to the console,
   but processing continues anyway, 32 x 32 long values are read to select
   the given group-mute, and then set each of its 32 group-mute-states.

   In our sample file, 32 groups are specified, but all 32 group-mute-state
   values for each are 0.

   So, to summarize the legacy proprietary track's data, ignoring the data
   itself, which is mostly 0 values, as shown in
   \tableref{table:seqspec_items_legacy_track}

   \begin{table}[htb]
      \centering
      \caption{SeqSpec Items in Legacy Proprietary Track}
      \label{table:seqspec_items_legacy_track}
      \begin{tabular}{l l}
\texttt{c\_midictrl}    & \texttt{24 24 00 10 00 00 00 00} \\
\texttt{c\_midiclocks}  & \texttt{24 24 00 03 00 00 00 00} (buss count = 0) \\
\texttt{c\_notes}       & \texttt{24 24 00 05 00 20 00 00} (screen sets = 32) \\
\texttt{c\_bpmtag}      & \texttt{24 24 00 07 00 00 00 78} (bpm = 120) \\
\texttt{c\_mutegroups}  & \texttt{24 24 00 09 00 00 04 00} (mg = 1024) \\
      \end{tabular}
   \end{table}

   The new format (again, ignoring the data) takes up a few more bytes.
   It starts with the normal track marker and size data, followed by a
   made-up track name ("Seq66-S"),
   as shown in \tableref{table:seqspec_items_new_track}.

   \begin{table}[htb]
      \centering
      \caption{SeqSpec Items in New Proprietary Track}
      \label{table:seqspec_items_new_track}
      \begin{tabular}{l l}
\texttt{"MTrk" etc.}   & \texttt{4d 54 72 6b 00 00 11 0d 00 ...} \\
\texttt{Track name}    & \texttt{53 65 71 75 65 6e 63 65 72 32 34 2d 53} \\
\texttt{c\_midictrl}   & \texttt{ff 7f 04 24 24 00 10 00} (???) \\
\texttt{c\_midiclocks} & \texttt{ff 7f 04 24 24 00 03 00} (buss count = 0) \\
\texttt{c\_notes}      & \texttt{ff 7f 46 24 24 00 05 00 20 00...} (screen sets = 32) \\
\texttt{c\_bpmtag}     & \texttt{ff 7f 08 24 24 00 07 00 00 00 78} (bpm = 120) \\
\texttt{c\_mutegroups} & \texttt{ff 7f a1 08 24 24 00 09 00 00 04 00...} (mg = 1024) \\
      \end{tabular}
   \end{table}

   For the new format, the components of the final proprietary track size are
   as shown here:

   \begin{enumber}
      \item \textbf{Delta time}.  1 byte, always 0x00.
      \item \textbf{Sequence number}.  5 bytes.  OPTIONAL.
      \item \textbf{Track name}. 3 + 10 or 3 + 15
      \item \textbf{Series of proprietary specs}:
      \begin{itemize}
         \item \textbf{Prop header}:
         \begin{itemize}
            \item If legacy format, 4 bytes.
            \item Otherwise, 2 bytes + varinum\_size(length) + 4 bytes.
            \item Length of the prop data.
         \end{itemize}
      \end{itemize}
      \item \textbf{Track End}. 3 bytes.
   \end{enumber}

   Note that we still need to dig into all the new values that have accumulated
   over the last couple years!

\subsection{MIDI Information}
\label{subsec:midi_information}

   This section provides some useful, basic information about MIDI data.

\subsubsection{MIDI Variable-Length Value}
\label{subsubsec:midi_variable_length_value}

   \index{VLV}
   A variable-length value (VLV) is a quantity that uses additional bytes
   and continuation bits to encode large numbers without confusing a MIDI
   interpreter.
   See \url{https://en.wikipedia.org/wiki/Variable-length\_quantity}.

   The length of a variable length value obviously depends on the value it
   represents.  Here is a simple list of the numbers that can be represented
   by a VLV:

   \begin{verbatim}
      1 byte:  0x00 to 0x7F
      2 bytes: 0x80 to 0x3FFF
      3 bytes: 0x4000 to 0x001FFFFF
      4 bytes: 0x200000 to 0x0FFFFFFF
   \end{verbatim}

\subsubsection{MIDI Track Chunk}
\label{subsubsec:midi_track_chunk}

   \texttt{Track chunk == MTrk + length + track\_event [+ track\_event ...]}

   \begin{itemize}
      \item \textsl{MTrk} is 4 bytes representing the literal string "MTrk".
         This marks the beginning of a track.
      \item \textsl{length} is 4 bytes the number of bytes in the track
         chunk following this number.  That is, the marker and length are
         not counted in the length value.
      \item \textsl{track\_event} denotes a sequenced track event; usually
         there are many track events in a  track.  However, some of the
         events may simply be informational, and not modify the audio
         output.
   \end{itemize}

   A track event consists of a delta-time since the last event, and one of
   three types of events.
 
   \texttt{track\_event = v\_time + midi\_event | meta\_event | sysex\_event}
 
   \begin{itemize}
      \item \textsl{v\_time} is the variable length value for elapsed time
         (delta time) from the previous event to this event.
      \item \textsl{midi\_event} is any MIDI channel message such as note-on
         or note-off.
      \item \textsl{meta\_event} is an SMF meta event.
      \item \textsl{sysex\_event} is an SMF system exclusive event.
   \end{itemize}

\subsubsection{MIDI Meta Events}
\label{subsubsec:midi_meta_events}

   Meta events are non-MIDI data of various sorts consisting of a fixed prefix,
   type indicator, a length field, and actual event data.
 
   \texttt{meta\_event = 0xFF + meta\_type + v\_length + event\_data\_bytes}

   \begin{itemize}
      \item \textsl{meta\_type} is 1 byte, expressing one of the meta event
         types shown in the table that follows this list.
      \item \textsl{v\_length} is length of meta event data, a variable
         length value.
      \item \textsl{event\_data\_bytes} is the actual event data.
   \end{itemize}

   \begin{table}
      \centering
      \caption{MIDI Meta Event Types}
      \label{table:midi_meta_event_types}
      \begin{tabular}{l l}
         Type & Event \\
         0x00 & Sequence number \\
         0x01 & Text event \\
         0x02 & Copyright notice \\
         0x03 & Sequence or track name \\
         0x04 & Instrument name \\
         0x05 & Lyric text \\
         0x06 & Marker text \\
         0x07 & Cue point \\
         0x20 & MIDI channel prefix assignment \\
         0x2F & End of track \\
         0x51 & Tempo setting \\
         0x54 & SMPTE offset \\
         0x58 & Time Signature \\
         0x59 & Key Signature \\
         0x7F & Sequencer-Specific event \\
      \end{tabular}
   \end{table}

   \textsl{Timidity} reads the legacy and new formats and plays the tune.
   \textsl{Seq66}  saves the "b4uacuse" tune out, in both formats,
   with a "MIDI divisions" value of 192, versus its original value of 120.
   The song plays a little bit faster after this conversion.

   The \textsl{midicvt} application does not read the legacy \textsl{Seq24}
   file format.  It
   expects to see the MTrk marker.  Even if the \textsl{midicvt}
   \texttt{--ignore} option is provided,
   \textsl{midicvt} does not like the legacy \textsl{Seq24} format, and ends
   with an error message.
   However, as shown by \tableref{table:midi_file_support_table},
   most applications are more
   forgiving, and can read (or ignore) the legacy format.  The
   \textsl{gsequencer} application has some major issues in our
   installation, but it is probably our setup.  (No JACK running?)

   \begin{table}
      \centering
      \caption{Application Support for MIDI Files}
      \label{table:midi_file_support_table}
      \begin{tabular}{l l l l}
         \textbf{Application}  &
            \textbf{Legacy} &
            \textbf{New} & 
            \textbf{Original File} \\
         ardour       & TBD       & TBD       & TBD \\
         composite    & TBD       & TBD       & TBD \\
         gsequencer   & No        & No        & No \\
         lmms         & Yes       & Yes       & Yes \\
         midi2ly      & Yes       & Yes       & TBD \\
         midicvt      & No        & Yes       & Yes \\
         midish       & TBD       & TBD       & TBD \\
         muse         & TBD       & TBD       & TBD \\
         playmidi     & TBD       & TBD       & TBD \\
         pmidi        & TBD       & TBD       & TBD \\
         qtractor     & Yes       & Yes       & Yes \\
         rosegarden   & Yes       & Yes       & Yes \\
         superlooper  & TBD       & TBD       & TBD \\
         timidity     & Yes       & Yes       & Yes \\
      \end{tabular}
   \end{table}

\subsection{More MIDI Information}
\label{subsec:midi_information_more}

   This section goes into even more detail about the MIDI format, especially as
   it applies to the processing done by \textsl{Seq66}.
   The following sub-sections describe how \textsl{Seq66}
   parses a MIDI file.

\subsubsection{MIDI File Header, MThd}
\label{subsubsec:midi_format_header}

   The first thing in a MIDI file is The data of the header:

   \begin{verbatim}
      Header ID:     "MThd"         4 bytes
      MThd length:     6            4 bytes
      Format:        0, 1, 2        2 bytes
      No. of track:  1 or more      2 bytes
      PPQN:           192           2 bytes
   \end{verbatim}

   The header ID and it's length are always the same values.  The formats that
   Seq66 supports are 0 or 1.  SMF 0 has only one track, while SMF 1 can
   support an arbitary number of tracks.  The last value in the header is the
   PPQN value, which specifies the "pulses per quarter note", which is the
   basic time-resolution of events in the MIDI file.  Common values are 96 or
   192, but higher values are also common.  Seq66 and its precursor,
   Seq24, default to 192.

\subsubsection{MIDI Track, MTrk}
\label{subsubsec:midi_format_track}

   The next part of the MIDI file consists of the tracks specified in the file.
   In SMF 1 format, each track is assumed to cover a different MIDI channel,
   but always the same MIDI buss.  (The MIDI buss is not a data item in
   standard MIDI files, but it is a special data item in the sequencer-specific
   section of \textsl{Seq24/Seq66} MIDI files.)  Each track is tagged by
   a standard chunk marker, "MTrk".  Other markers are possible, and are to be
   ignored, if nothing else.  Here are the values read at the beginning of a
   track:

   \begin{verbatim}
      Track ID:      "MTrk"         4 bytes
      Track length:  varies         4 bytes
   \end{verbatim}

   The track length is the number of bytes that need to be read in order to get
   all of the data in the track.

   \textbf{Delta time}.
   The amount time that passes from one event to the next is the
   \textsl{delta time}.
   For some events, the time doesn't matter, and is set to 0.
   This values is a
   \textsl{variable length value}, also known as a "VLV" or a "varinum".   It
   provides a way of encoding arbitrarily large values, a byte at a time.

   \begin{verbatim}
      Delta time:    varies         1 or more bytes
   \end{verbatim}

   The running-time accumulator is incremented by the delta-time.
   The current time is adjusted as per the PPQN ratio, if needed, and passed
   along.

\subsubsection{Channel Events}
\label{subsubsec:midi_format_channel_events}

   \textbf{Status}.
   The byte after the delta time is examined by masking it against 0x80 to check
   the high bit.  If not set, it is a "running status", it is replaced with the
   "last status", which is 0 at first.

   \begin{verbatim}
      Status byte:   varies         1 byte
   \end{verbatim}

   If the high bit is set, it is a status byte.  What does the status mean?  To
   find out, the channel part of the status is masked out using the 0xF0 mask.
   If it is a 2-data-byte event (note on, note off, aftertouch, control-change,
   or pitch-wheel), then the two data bytes are read:

   \begin{verbatim}
      Data byte 0:   varies         1 byte
      Data byte 1:   varies         1 byte
   \end{verbatim}

   If the status is a Note On event, with velocity = data[1] = 0,
   then it is converted to a Note Off event, a fix for the output quirks of
   some MIDI devices.
   If it is a 1-data-btye event (Program Change or Channel Pressure), then only
   data byte 0 is read.
   The one or two data bytes are added to the event,
   the event is added to the current sequence,
   and the MIDI channel of the sequence is set.

\subsubsection{Meta Events Revisited}
\label{subsubsec:midi_format_meta_events_revisited}

   If the event status masks off to 0xF0 (0xF0 to 0xFF), then it is a Meta
   event.  If the Meta event byte is 0xFF, it is called a "Sequencer-specific",
   or "SeqSpec" event.  For this kind of event, then a type byte and the length
   of the event are read.

   \begin{verbatim}
      Meta type:     varies         1 byte
      Meta length:   varies         1 or more bytes
   \end{verbatim}

   If the type of the SeqSpec (0xFF) meta event is 0x7F, parsing checks to see
   if it is one of the Seq24 "proprietary" events.  These events are tagged
   with various values that mask off to 0x24240000.  The parser reads the
   tag:

   \begin{verbatim}
      Prop tag:     0x242400nn      4 bytes
   \end{verbatim}

   These tags provide a way to save and recover Seq24/Seq66 properties
   from the MIDI file: MIDI buss, MIDI channel, time signature, sequence
   triggers, and (new), the key, scale, and background sequence to use with the
   track/sequence.  Any leftover data for the tagged event is let go.  Unknown
   tags ate skipped.

   If the type of the SeqSpec (0xFF) meta event is 0x2F, then it is the
   End-of-Track marker.  The current time marks the length (in MIDI pulses) of
   the sequence.  Parsing is done for that track.

   If the type of the SeqSpec (0xFF) meta event is 0x03, then it is the
   sequence name.  The "length" number of bytes are read, and loaded as the
   sequence name.

   If the type of the SeqSpec (0xFF) meta event is 0x00, then it is the
   sequence number, which is read:

   \begin{verbatim}
      Seq number:    varies         2 bytes
   \end{verbatim}

   Note that the sequence number might be modified latter to account for the
   current \textsl{Seq24} screenset in force for a file import operation.

   Anything other SeqSpec type is simply skipped by reading the "length" number
   of bytes.

   The remaining sections simply describe MIDI meta events in more detail, for
   reference.

\subsection{Meta Events}
\label{subsubsec:midi_format_meta_events_summary}

Here, we summarize the MIDI meta events.

   \begin{enumber}
      \item \texttt{FF 00 02 ssss}: Sequence Number.
      \item \texttt{FF 01 len text}: Text Event.
      \item \texttt{FF 02 len text}: Copyright Notice.
      \item \texttt{FF 03 len text}: Sequence/Track Name.
      \item \texttt{FF 04 len text}: Instrument Name.
      \item \texttt{FF 05 len text}: Lyric.
      \item \texttt{FF 06 len text}: Marker.
      \item \texttt{FF 07 len text}: Cue Point.
      \item \texttt{FF 08 through 0F len text}: Other kinds of  text events.
      \item \texttt{FF 2F 00}: End of Track.
      \item \texttt{FF 51 03 tttttt}: Set Tempo, us/qn.
      \item \texttt{FF 54 05 hr mn se fr ff}: SMPTE Offset.
      \item \texttt{FF 58 04 nn dd cc bb}: Time Signature.
      \item \texttt{FF 59 02 sf mi}: Key Signature.
      \item \texttt{FF 7F len data}: Sequencer-Specific.
      \item \texttt{FF F0 len data F7}: System-Exclusive
   \end{enumber}

The next sections describe the events that \textsl{Sequencer} tries to handle.
These are:

   \begin{itemize}
      \item Sequence Number (0x00)
      \item Track Name (0x03)
      \item End-of-Track (0x2F)
      \item Set Tempo (0x51) (Seq66 only)
      \item Time Signature (0x58) (Seq66 only)
      \item Sequencer-Specific (0x7F) (Handled differently in Seq66)
      \item System Exclusive (0xF0) Sort of handled, functionality incomplete.
   \end{itemize}

\subsubsection{Sequence Number (0x00)}
\label{subsubsec:midi_format_meta_sequence_number}

   \begin{verbatim}
      FF 00 02 ss ss
   \end{verbatim}

   This optional event must occur at the beginning of a track,
   before any non-zero delta-times, and before any transmittable MIDI
   events.  It specifies the number of a sequence.

\subsubsection{Track/Sequence Name (0x03)}
\label{subsubsec:midi_format_meta_sequence_name}

   \begin{verbatim}
      FF 03 len text
   \end{verbatim}

   If in a format 0 track, or the first track in a format 1 file, the name
   of the sequence.  Otherwise, the name of the track.

\subsubsection{End of Track (0x2F)}
\label{subsubsec:midi_format_meta_end_of_track}

   \begin{verbatim}
      FF 2F 00
   \end{verbatim}

   This event is not optional.  It is included so that an exact ending
   point may be specified for the track, so that it has an exact length,
   which is necessary for tracks which are looped or concatenated.

\subsubsection{Set Tempo Event (0x51)}
\label{subsubsec:midi_format_meta_set_tempo}

   The MIDI Set Tempo meta event sets the tempo of a MIDI sequence in terms of
   the microseconds per quarter note.  This is a meta message, so this event is
   never sent over MIDI ports to a MIDI device.
   After the delta time, this event consists of six bytes of data:

   \begin{verbatim}
      FF 51 03 tt tt tt
   \end{verbatim}

   Example:

   \begin{verbatim}
      FF 51 03 07 A1 20
   \end{verbatim}

   \begin{enumber}
      \item 0xFF is the status byte that indicates this is a Meta event.
      \item 0x51 the meta event type that signifies this is a Set Tempo event.
      \item 0x03 is the length of the event, always 3 bytes.
      \item The remaining three bytes carry the number of microseconds per
         quarter note.  For example, the three bytes above form the hexadecimal
         value 0x07A120 (500000 decimal), which means that there are 500,000
         microseconds per quarter note.
   \end{enumber}

   Since there are 60,000,000 microseconds per minute, the event above
   translates to: set the tempo to 60,000,000 / 500,000 = 120 quarter notes per
   minute (120 beats per minute).

   This event normally appears in the first track. If not, the default tempo is
   120 beats per minute.  This event is important if the MIDI time division is
   specified in "pulses per quarter note", which does not itself define the
   length of the quarter note. The length of the quarter note is then
   determined by the Set Tempo meta event.

   Representing tempos as time per beat instead of beat per time allows
   absolutely exact DWORD-term synchronization with a time-based sync protocol
   such as SMPTE time code or MIDI time code.  This amount of accuracy
   in the tempo resolution allows a four-minute piece at 120 beats per minute
   to be accurate within 500 usec at the end of the piece.

\subsubsection{Time Signature Event (0x58)}
\label{subsubsec:midi_format_meta_time_sig}

   After the delta time, this event consists of seven bytes of data:

   \begin{verbatim}
      FF 58 04 nn dd cc bb
   \end{verbatim}

   The time signature is expressed as four numbers.
   \texttt{nn} and \texttt{dd} represent the numerator and denominator of the
   time signature as it would be notated.  The denominator is a negative power
   of two:  2 represents a quarter-note, 3 represents an eighth-note, etc.  The
   \texttt{cc} parameter expresses the number of MIDI clocks in a metronome
   click.  The \texttt{bb} parameter expresses the number of notated 32nd-notes
   in a MIDI quarter- note (24 MIDI Clocks).

   Example:

   \begin{verbatim}
      FF 58 04 04 02 18 08
   \end{verbatim}

   \begin{enumber}
      \item 0xFF is the status byte that indicates this is a Meta event.
      \item 0x58 the meta event type that signifies this is a Time Signature
         event.
      \item 0x04 is the length of the event, always 4 bytes.
      \item 0x04 is the numerator of the time signature, and ranges from 0x00
         to 0xFF.
      \item 0x02 is the log base 2 of the denominator, and is the power to
         which 2 must be raised to get the denominator.  Here, the denominator
         is 2 to 0x02, or 4, so the time signature is 4/4.
      \item 0x18 is the metronome pulse in terms of the number of
         MIDI clock ticks per click.  Assuming 24 MIDI clocks per quarter note,
         the value here (0x18 = 24) indidicates that the metronome will tick
         every 24/24 quarter note.  If the value of the sixth byte were 0x30 =
         48, the metronome clicks every two quarter notes, i.e. every
         half-note.
      \item 0x08 defines the number of 32nd notes per beat. This byte is
         usually 8 as there is usually one quarter note per beat, and one
         quarter note contains eight 32nd notes.
   \end{enumber}

   If a time signature event is not present in a MIDI sequence, a 4/4 signature
   is assumed.

   In \textsl{Seq66},
   the \texttt{c\_timesig} SeqSpec event is given priority.  The
   conventional time signature is used only if the \texttt{c\_timesig}
   SeqSpec is not present in the file.

\subsubsection{SysEx Event (0xF0)}
\label{subsubsec:midi_format_meta_sysex_event}

   If the meta event status value is 0xF0, it is called a "System-exclusive",
   or "SysEx" event.

   \textsl{Seq66} has some code in place to store these messages, but the
   data is currently not actually stored or used.  Although there is some
   infrastructure to support storing the SysEx event within a sequence, the
   SysEx information is simply skipped.  \textsl{Seq66} warns if the
   terminating 0xF7 SysEx terminator is not found at the expected length.
   Also, some malformed SysEx events have been encountered, and those are
   detected and skipped as well.

\subsubsection{Sequencer Specific (0x7F)}
\label{subsubsec:midi_format_meta_sequencer_specific}

   This data, also known as SeqSpec data, provides a way to encode information
   that a specific sequencer application needs, while marking it so that other
   sequences can safely ignore the information.

   \begin{verbatim}
      FF 7F len data
   \end{verbatim}

   In \textsl{Seq24 and Seq66},
   the data portion starts with four bytes
   that indicate the kind of data for a particular SeqSpec event:

   \begin{verbatim}
      c_midibus        ^  0x24240001  Track buss number
      c_midich         ^  0x24240002  Track channel number
      c_midiclocks     *  0x24240003  Track clocking (not fully implemented!)
      c_triggers       ^  0x24240004  See c_triggers_new; no offset here
      c_notes          *  0x24240005  Song data, notes (names of sets)
      c_timesig        ^  0x24240006  Track time signature (not global)
      c_bpmtag         *  0x24240007  Song beats/minute (global)
      c_triggers_new   ^  0x24240008  Track trigger data
      c_mutegroups     *  0x24240009  Song mute group data (global)
	   c_gap_A          s  0x2424000A  Gap A
	   c_gap_B          s  0x2424000B  Gap B
	   c_gap_C          s  0x2424000C  Gap C
	   c_gap_D          s  0x2424000D  Gap D
	   c_gap_E          s  0x2424000E  Gap E
	   c_gap_F          s  0x2424000F  Gap F
      c_midictrl       *  0x24240010  Song MIDI control (always empty)
      c_musickey       +  0x24240011  Track key (Seq66 only) (global too)
      c_musicscale     +  0x24240012  Track scale (Seq66 only) (global too)
      c_backsequence   +  0x24240013  Track background sequence (global too)
	   c_transpose      t  0x24240014  Track transpose value
	   c_perf_bp_mes    t  0x24240015  Perfedit beats/measure
	   c_perf_bw        t  0x24240016  Perfedit beat-width
	   c_tempo_map      t  0x24240017  Reserve seq32 tempo map
	   c_reserved_1     t  0x24240018  Reserved for expansion
	   c_reserved_2     t  0x24240019  Reserved for expansion
	   c_tempo_track    t  0x2424001A  Alternate tempo track no. (unimplemented)
	   c_seq_color      t  0x2424001B  Track color, from Kepler34
	   c_seq_edit_mode  t  0x2424001C  Unimplemented feature from Kepler34
	   c_seq_loopcount  t  0x2424001D  Future: Track N-play pattern

      * = global only; ^ = track only; + = both; s = reserved for seq32; t = TBD
   \end{verbatim}

	\begin{comment}
		\texttt{midi\_vector\_base::fill\_proprietary} adds the following tag values:
		\begin{itemize}
			\item c\_midibus; sequence::get\_midi\_bus().
			\item c\_midich; sequence::get\_midi\_channel().
			\item c\_timesig; sequence::get\_beats\_per\_bar() and get\_beat\_width().
			\item If non-global sequence features:
			\begin{itemize}
				\item c\_musical\_key; sequence::musical\_key(); if != key of C.
				\item c\_musical\_key; sequence::musical\_scale();
					 if != c\_scales\_off.
				\item c\_backsequence; sequence::background\_sequence();
					if valid
			\end{itemize}
			\item c\_transpose; sequence::transposable().
			\item c\_seq\_color; sequence::color(); if not c\_seq\_color\_none.
		\end{itemize}

		\texttt{midi\_vector\_base::song\_fill\_seq\_trigger} adds the
			 following tag values:
		\begin{itemize}
			\item c\_triggers\_new.
		\end{itemize}

		Items not dealt with in the code:
		\begin{itemize}
			\item c\_midiclocks
			\item c\_triggers (old)
			\item c\_notes
			\item c\_bpmtag
			\item c\_mutegroups
			\item c\_gap\_A to F
			\item c\_midictrl
	   	\item c\_perf\_bp\_mes
	   	\item c\_perf\_bw
	   	\item c\_tempo\_map
	   	\item c\_reserved\_1
	   	\item c\_reserved\_2
	   	\item c\_tempo\_track
	   	\item c\_seq\_edit\_mode
	   	\item c\_seq\_loopcount
		\end{itemize}

	\end{comment}

   In \textsl{Seq24}, these events are placed at the end of the song, but are
   not marked as SeqSpec data.  Most MIDI applications handle this situation
   fine, but some (e.g. midicvt) do not.  Therefore, \textsl{Seq66} makes
   sure to wrap each data item in the 0xFF 0x7F wrapper.

   Also, the last three items above (key, scale, and background sequence) can
   also be stored (by \textsl{Seq66}) with a particular sequence/track,
   as well as at the end of the song.  Not sure if this bit of extra
   flexibility is useful, but it is there.

\subsubsection{Non-Specific End of Sequence}
\label{subsubsec:midi_format_meta_sequence_ends}

   Any other statuses are deemed unsupportable in \textsl{Seq66}, and
   abort parsing with an error.

   If the --bus option is in force, it overrides the buss number (if any)
   stored with the sequence.  This option is useful for testing a setup.
   Note that it also applies to new sequences.

   At the end, \textsl{Seq66} adds the sequence to the encoded tune.

%-------------------------------------------------------------------------------
% vim: ts=3 sw=3 et ft=tex
%-------------------------------------------------------------------------------

%
% ---------------------------------------------------------------------
%
% Acknowledgments

%%% %-------------------------------------------------------------------------------
% seq66-user-manual
%-------------------------------------------------------------------------------
%
% \file        kudos.tex
% \library     Documents
% \author      Chris Ahlstrom
% \date        2016-08-29
% \update      2020-12-30
% \version     $Revision$
% \license     $XPC_GPL_LICENSE$
%
%     This document provides LaTeX documentation for Seq66.
%
%-------------------------------------------------------------------------------

\section{Kudos}
\label{sec:kudos}

   This section gives some credit where credit is due.
   We have contributors to acknowledge, and have not caught up with all the
   people who have helped this project:

   \begin{itemize}
      \item \textsl{Tim Deagan (tdeagan)}:
         Fixes to the mute-group support.
      \item \textsl{0rel}:
         An important fix to add and relink notes after a
         paste action in the pattern editor.
      \item \textsl{arnaud-jacquemin}:
         A bug report and fix for a regression in mute-groups support.
         Also suggestions for enhancing mute-group support.
      \item \textsl{Stan Preston (stazed)}:
         Ideas for many improvements based
         on his \textsl{seq32} project.  A lot of ideas.
         And a lot of code!
      \item \textsl{Animtim}:
         A number of bug reports and a new logo for \textsl{Sequencer64}.
      \item \textsl{jean-emmanuel}:
         Scrollable main-window support, other features and reports.
      \item \textsl{Olivier Humbert (trebmuh)}:
         French translation for the desktop files.
      \item \textsl{Oli Kester}:
         The creator of \textsl{Kepler34}, from which we got many
         clues on porting the user-interface to Qt 5 and Windows.
   \end{itemize}

   Also some bug-reporters and testers:

   \begin{itemize}
      \item \textsl{F0rth}:
         A request for scripting support, a possible future feature.
      \item \textsl{gimmeapill}:
         Testing, bug-reports, and, um, "marketing".
      \item \textsl{georgkrause}:
         A number of helpful bug reports.
      \item \textsl{goguetchapuisb}:
         Found that \textsl{Sequencer64} native JACK did not properly handle
         the copious Active Sensing messages emitted by Yamaha keyboards.
      \item \textsl{milkmiruku}:
         Mainwids issues and many ideas, suggestions, feature requests, and bug
         report.
      \item \textsl{muranyia}:
         Feature request for numbered piano keys and bug-reports.
      \item \textsl{simonvanderveldt}:
         Issues with window sizing and more.
      \item \textsl{ssj71}:
         A request for an LV2 plugin version, a possible future feature.
      \item \textsl{triss}:
         A request for OSC support, a possible future feature.  We added some
         OSC support in order to play well with the \textsl{Non Session
         Manager} (\textsl{NSM}).
      \item \textsl{layk}:
         Some bug reports, and, we are pretty sure, some nice videos that
         demonstrate \textsl{Seq66} on \textsl{YouTube}.  See
         \cite{layk}.
      \item \textsl{matt-bel}:
         Reported a regression from \textsl{Seq24}, which could use
         a MIDI control event to mute/unmute multiple patterns at once,
         a cool feature!
      \item \textsl{zigmhount}:
         A pending request for a control that would automatically set up a
         pattern for recording and playback with one "click".
      \item \textsl{grammoboy} and \textsl{J. Liles}:
         Bug reports and other help with \textsl{NSM} support.
      \item \textsl{Houston4444}:
         Similarly, help with \textsl{RaySession}, a work-alike of
         \textsl{NSM}, written in \textsl{Python}.
      \item \textsl{unfa}:
         Bug reports for coloring, and for inspiring the "*.palette" file
         feature, as well as making coloring more comprehensive.
   \end{itemize}

   ... and there are more to add to this list....

   There are a number of authors of \textsl{Seq24}.
   ideas from other \textsl{Seq24} fans),
   and some deep history,
   as one can see in \figureref{fig:menu_help_credits},
   and in \figureref{fig:menu_help_doc}.
   All of these authors, and more, have contributed to \textsl{Seq66},
   whether they know it or not.
   The original author is Rob C. Buse; where the word "I" occurs, that is
   probably him.  Without his work, we would never have started
   \textsl{Seq66}.

   From the original author:

   \begin{quotation}
      \textsl{Seq24} is a real-time MIDI sequencer. It was created to
      provide a very simple interface for editing and playing MIDI 'loops'.
      After searching for a software based sequencer that would provide the
      functionality needed for a live performance, there was little found in
      the software realm. I set out to create a very minimal sequencer that
      excludes the bloated features of the large software sequencers, and
      includes a small subset of features that I have found usable in
      performing. 

      Written by Rob C. Buse.  I wrote this program to fill a
      hole.  I figure it would be a waste if I was the only one
      using it.  So, I released it under the GPL.
   \end{quotation}

   Taking advantage of Rob's generosity,
   we've created a reboot, a refactoring, an improvement (we hope) of
   \textsl{Seq24}.  It preserves (we hope) the lean nature of \textsl{Seq24},
   while adding a few useful features.
   Without \textsl{Seq24} and its authors,
   \textsl{Seq66} would never have come into being.

%-------------------------------------------------------------------------------
% vim: ts=3 sw=3 et ft=tex
%-------------------------------------------------------------------------------


\section{Summary}
\label{sec:summary}

   Contact: If you have ideas about \textsl{Seq66} or a bug report,
   please email us (at \url{mailto:ahlstromcj@gmail.com}).
   If it's a bug report, please add \textbf{[BUG]} to the Subject, or use the
   GitHub bug-reporting interface.

% References

%%% %-------------------------------------------------------------------------------
% references
%-------------------------------------------------------------------------------
%
% \file        references.tex
% \library     Documents
% \author      Chris Ahlstrom
% \date        2015-08-31
% \update      2021-01-28
% \version     $Revision$
% \license     $XPC_GPL_LICENSE$
%
%     Provides the References section of the Seq66 manual. Rather
%     than use the bibtex package, our small set of references uses a
%     simpler method.
%
% Potential additional references:
%
%     cakewalk
%     WRK
%
%
%-------------------------------------------------------------------------------

\section{References}
\label{sec:references}

   The \textsl{Seq66} reference list.

{\RaggedRight
\begin{thebibliography}{99}

   \bibitem{alsa}
   ALSA team.
   \emph{Advanced Linux Sound Architecture (ALSA) project homepage.}
   \url{http://www.alsa-project.org/}.
   ALSA tools through version 1.0.29.
   2015.

   \bibitem{midimapper}
   Coolsoft.
   \emph{Coolsoft MIDIMapper (Windows).}
   \url{https://coolsoft.altervista.org/en/midimapper}.
   2018.

   \bibitem{midisynth}
   Coolsoft.
   \emph{Coolsoft VirtualMIDISynth (Windows).}
   \url{https://coolsoft.altervista.org/en/virtualmidisynth}.
   2018.

   \bibitem{combine}
   Jay Capela Music.
   \emph{"Combine": A Seq24 Demonstration.}
   \url{https://www.youtube.com/watch?v=fUiXbVT0bJQ}.
   2010.

   \bibitem{jack}
   JACK team.
   \emph{JACK Audio Connection Kit.}
   \url{http://jackaudio.org/}.
   2015.

   \bibitem{kepler34}
   Oli Kester.
   \emph{Kepler34: Seq24 for the 2010s.}
   \url{https://github.com/oli-kester/kepler34}.
   2010-2016.

   \bibitem{layk}
   Lassi Ylikojola.
   \emph{Many demo videos of Sequencer64}
   \url{https://www.youtube.com/watch?v=YStYVjFv1TM},
   \url{https://www.youtube.com/watch?v=GBlEP8Ffqss},
   \url{https://www.youtube.com/watch?v=4gG8SvJxJkA&t=28s},
   \url{https://www.youtube.com/watch?v=n4Z4WPK6FpA}.
   2010-2017.

   \bibitem{midicvt}
   Chris Ahlstrom.
   \emph{Extension of midicomp/midi2text to convert between MIDI and ASCII
      text format.}
   \url{https://github.com/ahlstromcj/midicvt}.
   2015-2016.

   \bibitem{midicontrol}
   linuxaudio.org.
   \emph{seq24: toggle sequences with a MIDI controller.}
   \url{http://wiki.linuxaudio.org/wiki/seq24togglemiditutorial}.
   2013.

   \bibitem{midicontroltable}
   midi.org.
   \emph{Summary of MIDI Messages.}
   \url{https://www.midi.org/specifications/item/table-1-summary-of-midi-message#2}.
   Year unknown.

   \bibitem{nanobasket}
   Roy Vegard.
   \emph{Configurator software for the Korg nanoSERIES of MIDI controllers.}
   \url{https://github.com/royvegard/Nano-Basket}.
   2015.

   \bibitem{portmidi}
   PortMedia team.
   \emph{Platform Independent Library for MIDI I/O.}
   \url{http://portmedia.sourceforge.net/}.
   2010.

   \bibitem{rtmidi}
   Gary P. Scavone.
   \emph{The RtMIDI Tutorial.}
   \url{https://www.music.mcgill.ca/~gary/rtmidi/}.
   2016.

   \bibitem{seq24}
   Seq24 Team.
   \emph{The home site for the Seq24 looping sequencer.}
   \url{http://www.filter24.org/seq24/download.html}.
   2010.

   \bibitem{seq24launchpad}
   Seq24 Team.
   \emph{The home site for the Seq66 looping sequencer.}
   \url{https://edge/launchpad.net/seq24}.
   2016.

   \bibitem{seq24launchpadmapper}
   Excds.
   \emph{A simple mapping for toggling the LEDs on the Novation launchpad
   together with seq24.}
   \url{https://github.com/Excds/seq24-launchpad-mapper}.
   2013.

   \bibitem{seq32}
   Stan Preston (stazed).
   \emph{The home site for the Seq32 looping sequencer.}
   \url{https://github.com/Stazed/seq32}.
   2016.

   \bibitem{seq66}
   Chris Ahlstrom.
   \emph{A reboot of the Seq24 project as "Seq66".}
   \url{https://github.com/ahlstromcj/seq66/}.
   2015-2017.

   \bibitem{timidity}
   Timidity++ Team.
   \emph{Download site for Timidity++ source code.}
   \url{http://sourceforge.net/projects/timidity/}.
   2015.

   \bibitem{vmpk}
   VMPK Team.
   \emph{Virtual MIDI Piano Keyboard.}
   \url{http://vmpk.sourceforge.net/}.
   2015.

   \bibitem{wootangent1}
   The Wootangent man.
   \emph{Seq66 Tutorial Videos, Part 1 and Part 2.}
   \url{http://wootangent.net/2010/10/linux-music-tutorial-seq24-part-1/}.
   \url{http://wootangent.net/2010/10/linux-music-tutorial-seq24-part-2/}.
   2010.

\end{thebibliography}
}

%-------------------------------------------------------------------------------
% vim: ts=3 sw=3 et ft=tex
%-------------------------------------------------------------------------------


\printindex

\end{document}

%-------------------------------------------------------------------------------
% vim: ts=3 sw=3 et ft=tex
%-------------------------------------------------------------------------------
