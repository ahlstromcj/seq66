%-------------------------------------------------------------------------------
% song_editor
%-------------------------------------------------------------------------------
%
% \file        song_editor.tex
% \library     Documents
% \author      Chris Ahlstrom
% \date        2015-08-31
% \update      2021-01-11
% \version     $Revision$
% \license     $XPC_GPL_LICENSE$
%
%     Provides the concepts.
%
%-------------------------------------------------------------------------------

\section{Song Editor}
\label{sec:song_editor}

   The \textsl{Seq66 Song Editor} combines all patterns
   into a complete tune with specified repetitions of each pattern.
   It shows one row per pattern/loop/sequence in numbered columns,
   with the placement of each pattern at various time locations in the song.
   \index{performance}
   In \textsl{Seq66} parlance, the song editor creates a
   \textsl{performance}, and the performance is implemented by a set of
   triggers.
   Triggers are internal timing items stored with each pattern when a
   \textsl{Seq66} MIDI tune is saved.
   \index{song mode}
   In \textbf{Song} mode, these triggers, not the user, control
   playback.

   \index{song editor!dual}
   Two song editor windows can be
   brought onscreen, as a convenience for arranging projects with a large
   number of sequences/patterns.
   The \textbf{Song} tab and a \textbf{Song} window can be shown at the
   same time.
   The \textbf{Song} editor activates
   the \textbf{Song} mode of \textsl{Seq66}.
   When the song editor has the focus of the application, it
   takes over control from the patterns panel, and controls playback.
   Once playback is started in the song editor, some actions
   in the patterns panel no longer have effect, effectively disabling live
   mode.  The song editor takes over the arming/unarming (unmuting/muting)
   shown in the patterns panel.  The highlighting of armed/unarmed patterns
   changes according to whether the pattern is playing in the song editor, or
   not.  If one tries to change the muting using a hot-key (or even a click) in
   the patterns panel, the song editor immediately returns the pattern to the
   state it has in the song editor.  The only way to manually change the muting
   then is to click the pattern's label in the song editor.
   Both the song editor and the patterns panel both reflect the change in
   muting in the user-interface, though with \textsl{opposite colors}.

\begin{figure}[H]
   \centering 
   \includegraphics[scale=0.75]{song-editor/song-editor-annotated.png}
   \caption{Song Editor Window, Annotated}
   \label{fig:song_editor_window_annotated}
\end{figure}

   Here are the features for the song editor, as layed out above.

   \begin{itemize}
      \item Enter "paint" mode.
         The \texttt{p} key enters paint mode, where additional triggers
         can be added by click-dragging on a pattern row.
         The \texttt{x} key leaves this mode.
      \item Start/Pause button functionality.
         When the song roll has keyboard focus,
         the \texttt{Space} key starts and stops playback, rewinding to the
         beginning when stopped.
         The \texttt{.} (period) key starts and pauses playback, without
         rewinding.
         This functionality is similar to that of the main window, but
         these keys are not reconfigurable in the song roll.
      \item Undo/Redo/Cut/Copy/Paste of a selected section.
         Provided by buttons and by these keystrokes:
         \texttt{Ctrl-Z}. Undo.
         \texttt{Shift-Ctrl-Z}. Redo.
         \texttt{Ctrl-X}. Cut.  Removes the selection.
            \index{keys!backspace}
            \index{keys!delete}
            \index{keys!ctrl-x}
            Can also be done with the \texttt{Delete} and
            \texttt{Backspace} keys.
            The deletion can be undone.
         \texttt{Ctrl-C}. Copy.
         \index{keys!ctrl-c}
         \index{keys!copy}
         Copies the trigger for later usage.
         \texttt{Ctrl-V}. Paste.
            \index{keys!ctrl-v}
            \index{keys!paste}
            Puts the roll into paste mode.
            When inserted, each insert goes immediately
            after the current item or the previous insertion.  The same can be
            done for whole patterns.
      \item Horizontal (Time) Zoom.
         Provided by buttons and by these keystrokes:
         \texttt{Z}. Zoom in.
         \texttt{z}. Zoom out.
         \texttt{0}. Reset zoom.
   \end{itemize}

   \begin{itemize}
      \index{shift left click}
      \item Toggling of the mute state of multiple patterns by holding the
         \texttt{Shift} key while left-clicking on the \textbf{M}
         or name field of a pattern.
      \index{pause}
      \index{progress bar}
      \item Optional pattern coloring (selected in the Patterns panel)
         and a configurable progress bar.
      \index{redo}
      \item \textbf{Undo} and \textbf{Redo} buttons.
      \index{transpose}
      \item A \textbf{Transpose} button and transposition drop-down selector.
      \index{untransposable color}
      \item Red coloring of events for patterns that are not transposable, such
         as drum tracks.
   \end{itemize}

   The song editor is not too complex, but for exposition, we break it into
   the top panel, the bottom panel, and the rest of the window.

\subsection{Song Editor / Top Panel}
\label{subsec:song_editor_top}

   The top panel shown earlier provides quick access to actions
   and configuration.

   \begin{enumber}
      \item \textbf{Undo}
      \item \textbf{Redo}
      \item \textbf{Follow Progress}
      \item \textbf{Toggle Section Painting}
      \item \textbf{Grid Snap}
      \item \textbf{Transpose}
      \item \textbf{L/R Loop}
      \item \textbf{L/R Collapse}
      \item \textbf{L/R Expand}
      \item \textbf{L/R Expand and Copy}
   \end{enumber}

   Details are described below.

   \setcounter{ItemCounter}{0}      % Reset the ItemCounter for this list.

   \itempar{Stop}{song editor!stop}
   Stops the playback of the song.
   \index{keys!esc (stop)}
   The keystroke for stopping playback is the \texttt{Escape} character.
   It can be configured to be another character (such as \texttt{Space}, which
   would make the space-bar toggle the playback status).

   \itempar{Play}{song editor!play}
   \index{L marker}
   Starts the playback of the song, starting at the \textbf{L marker}.
   The \textbf{L marker} serves as the start position for playback
   in the song editor.  One can change the start position only when the
   performance is not playing.
   \index{keys!space (play)}
   The default keystroke for starting playback is the \texttt{Space} character.
   \index{keys!esc (stop)}
   The default keystroke for stopping playback is the \texttt{Escape} character.
   \index{keys!period (pause)}
   The default keystroke for pausing playback is the \texttt{Period} character.

   \itempar{Undo}{song editor!undo}
   The \textbf{Undo} button rolls back the last change in the layout of a
   pattern.  Each time it is clicked, the most recent change is undone.
   It rolls back one change each time pressed.
   It is inactive if there is nothing to undo.
   Also implemented via \texttt{Ctrl-Z}.

   \itempar{Redo}{song editor!redo}
   The \textbf{Redo} button reapplies the last change undone by
   the \textbf{Undo} button.  It is inactive if there is nothing to redo.
   Also implemented via \texttt{Shift-Ctrl-Z}.

   \itempar{Follow Progress}{song editor!follow}
      \textbf{Follow Progress} toggles the mode of following progress
         for longer songs.

   \itempar{Section Painting}{song editor!paint}
      \textbf{Toggle Section Painting} toggles the ability
      to drag the mouse along the pattern's timeline to indicate when the
      pattern plays.  Short patterns will be duplicated one or more times as
      the mouse isdragged.

   \itempar{Grid Snap}{song editor!grid snap}
   Indicates the horizontal grid snap for movement actions and trigger drawing.
   Grid snap selects where the patterns are drawn.
   Unlike the \textbf{Grid Snap} of the pattern editor, the units
   of the song editor snap value are in fractions of a measure length.
   The following values are supported:
   1, 1/2, 1/4, 1/8, 1/16, and 1/32.
%  , and 1/3, 1/6, 1/12, and 1/24.

   \itempar{Transpose}{song editor!transpose}
   \textbf{Transpose} consists of two controls: one to apply
   transposition, and a drop-down menu to select the direction and amount
   of transposition.

   \itempar{L/R Loop}{song editor!play loop}
   \index{loop mode}
   Activates loop mode. When play is activated, plays the song and loop
   between the
   \index{L marker}
   \index{R marker}
   \textbf{L marker} and the \textbf{R marker}.
   This button is a state button, and its appearance indicates when it is
   depressed, and thus active.
   If this button is deactivated during playback, then playback
   continues past the \textbf{R marker}.
   Note that these markers can be placed using left
   and right mouse clicks, respectively.

   \itempar{L/R Collapse}{song editor!collapse}
   This button collapses the song between the \textbf{L marker} and the
   \textbf{R marker}.
   What this means is that, if there is song material (patterns) before the
   \textbf{L marker} and after the \textbf{R marker},
   and the \textbf{Collapse} button is
   pressed, any song material between the L and R markers is erased, and
   the song material after the \textbf{R marker} is moved leftward to
   the \textbf{L marker}.
   Collapsing occurs in all tracks present in the song editor.

   \itempar{L/R Expand}{song editor!expand}
   This button expands the song between the
   \textbf{L marker} and the \textbf{R marker}.
   It inserts blank space between these markers, moving the song material
   that is after the \textbf{R marker}
   to the right by the duration of the blank space.
   Expansion occurs in all tracks present in the song editor.

   \itempar{L/R Expand and Copy}{song editor!expand and copy}
   This button expands the song between the \textbf{L marker} and the
   \textbf{R marker} much like the \textbf{Expand} button.
   However, it also copies the original data that is present after the
   \textbf{R marker}, and pastes it into the newly-available space between
   the L and R markers.

\subsection{Song Editor / Patterns Panel}
\label{subsec:song_editor_patterns_panel}

   The patterns panel is at the left of the song roll.
   Here are the items to note in the patterns panel:

   \begin{enumber}
      \item \textbf{Number}.
         The number of the screen set.
      \item \textbf{Title}.
         \index{pattern!title}
         \index{pattern!name}
         The title is the name of the pattern, for easy reference.
      \item \textbf{Channel}.
         \index{pattern!channel}
         The channel number appears (redundantly)
         at the right of the title.
      \item \textbf{Buss-Channel}.
         \index{pattern!buss-channel}
         This pair of numbers shows the MIDI buss number used in the pattern
         and the channel used for the pattern.
      \item \textbf{Beat/Measure}.
         \index{pattern!beat}
         This pair of numbers is the standard time-signature of the pattern.
      \item \textbf{Mute Indicator}.
         \index{song editor!mute indicator}
         The letter M is in a black box if the track/pattern is muted, and a
         white box if it is unmuted.
         Left-clicking on the "M" (or the name of the pattern)
         mutes/unmutes the pattern.
         \index{shift left click}
         If the Shift key is held while left-clicking on the M or the pattern
         name, then
         the mute/unmute state of every other active pattern is toggled.
         This feature is useful for isolating a single track or pattern.
      \item \textbf{Empty Track}.
         Completely empty tracks (no track events or meta events)
         are indicated by a dark-gray filling in the pattern column.
         Tracks that have only meta information, but no playable event, are
         indicated by a yellow filling in the pattern column.
   \end{enumber}

   The patterns column shows a list of all of the patterns that have been
   created in the current song.  Each pattern in this list has a track of
   pattern layouts associated with it in the piano roll section.

   \index{patterns column!left-click}
   \index{patterns column!ctrl-left-click}
   \index{song editor!muting}
   Left-clicking on the pattern name or the "M" button toggles the muting
   (arming) status of the track.
   It does the same thing if the \texttt{Ctrl} key is held at the same time.

   \index{pattern!shift-left-click}
   \index{song editor!inverse muting}
   \index{song editor!solo}
   \index{shift-left-click solo}
   Shift-left-clicking on the pattern name or the "M" button toggles the muting
   (arming) status of \textsl{all other tracks} except the track that was
   selected.  This action is useful for quickly listening to a single sequence
   in isoloation.

   \index{patterns column!right-click}
   Right-clicking on the pattern name or the "M" button brings up the same
   pattern editing menu as discussed in
   \sectionref{subsubsec:patterns_pattern_filled}.
   Recall that this context menu has the following entries:
   \textbf{Edit...}, \textbf{Event Edit...}, \textbf{Cut}, \textbf{Copy},
   \textbf{Song}, \textbf{Disable Transpose}, and \textbf{MIDI Bus}.

\subsection{Song Editor / Song Roll}
\label{subsec:song_editor_arrangement_panel}

   Also known as the "arrangement panel".
   The "Piano Roll" section of the arrangement panel is where patterns or
   subsections are inserted, deleted, shrunk, lengthened, or moved.
   Here are features to note in the annotated piano roll area
   shown in \figureref{fig:song_editor_window_full_items}:

   \begin{enumber}
      \item \textbf{Selection}.
         Clicking inside a pattern or a pattern subsection selects it.
         Selection is denoted by an orange rectangle around the trigger
         section and a dark grey color in the selected trigger.
         A pattern subsection can be moved by the mouse and deleted by keystrokes
         shown later.
      \item \textbf{De-selection}.
         \index{song editor!section deselection}
         Left-clicking or right-clicking in an empty area of the song roll
         will deselect the selection.
      \item \textbf{Selecttion Movement}.
         \index{song editor!selection movement}
         If, instead of grabbing the section length handle, one grabs inside
         the pattern or pattern subsection, that item can be moved
         horizontally, as long as there is room.  A left-click
         inside the item shows it as selected.
         One can also highlight a pattern section (making it gray),
         then click the \texttt{p} key to enter "paint" mode, and move the
         pattern left or right with the arrow keys.
      \item \textbf{Selection Position}.
         A selection position is a marker that divides a pattern into two
         pieces, called \textsl{pattern subsections}.  This makes it easy to
         select smaller portions of a pattern for editing or deleting.  It
         is especially useful for making holes in a pattern.
      \item \textbf{Pattern Subsectioning}.
         \index{song editor!middle click}
         \index{pattern subsection}
         Middle-clicking (or ctrl-left-click)
         inside a pattern inserts a selection position
         marker in it, breaking the pattern into two equal pieces.
         We call each piece a \textsl{pattern subsection}.
         This division can be done over and over.
      \item \textbf{Section Length ("handle")}.
         \index{song editor!handle}
         \index{song editor!section length}
         Looking closely at the diagram where the arrows point, small
         squares in two corners of the patterns can be seen.
         Call them "handles".
         By grabbing
         a handle with a left-click, the handle can be moved horizontally
         to either lengthen or shorted the pattern or pattern subsection, if
         there is room to move in the desired direction.
         It doesn't matter if the item is selected or not.
      \item \textbf{Expansion}.
         \index{song editor!section expansion}
         Originally, all the long patterns of this sample song were continuous.
         But, by setting the L and R markers, and using the \textbf{Expand}
         button, we opened up some silent space in the song, just to be able
         to show it off.
   \end{enumber}

   \index{song editor!split pattern}
   A useful feature is to split a pattern section in the
   song editor, either in half, or at the nearest snap point to the mouse
   pointer.
%  The location of the split is determined by the
%  setting of the \textbf{File / Options / Mouse / Seq66 Options /
%  Middle-click splits song trigger at nearest snap (instead of halfway point)}
%  setting.
%  To split a pattern, left-click it to highlight it, move
   the mouse (if not splitting in half) to the desired pointer, and
   ctrl-left-click with the mouse.

   The \textsl{Seq24} help files refer to work in the song editor as the
   "Performance Editor" or "Performance Mode".  Adding a pattern in this
   window is a bit like adding a note in the pattern editor.
   One clicks, holds, and drags the mouse to insert a copy or copies of the
   pattern associated with the row in which one is dragging.
   The longer one drags, the more copies of the pattern that are inserted.

   \index{song editor!right-click-hold}
   \index{song editor!draw}
	Right-click on the arrangement panel (roll) to enter
   draw mode, and hold the button.
   \index{new!mod4 mode}
   \index{keys!mod4}
   \index{song editor!mod4}
   Just like the patterns panel, there is a feature to allow the Mod4 (the
   Super or Windows) key to keep the right-click in force even after it is
   released.  See \ref{new_mod4_mode}.  Pressing Mod4 before
   releasing the right-click that allows pattern-adding ("painting"), keeps
   pattern-adding in force after the right-click is released.  Now pattern
   can be entered at will with the left mouse button.  Right-click again to
   leave the pattern-adding mode.

   \index{new!paint mode}
   Another way to turn on painting:
   make sure that the performance editor piano roll has the
   keyboard focus by left-clicking in it, then press the
   \index{keys!p}
   \texttt{p} key.
%  This is just like pressing the right mouse button, but the draw/paint mode
%  sticks (as if the Mod4 mode were in force).
   While in the paint mode, one can add pattern clips with the left mouse
   button, via click or drag, and one can highlight a pattern clip and move it
   with the left and right arrow keys.

%  In the near future, movement of selected trigger segments will not
%  require the paint mode to be active just to move the segments left or
%  right.

   To get out of the paint mode, press the
   \index{keys!x}
   \texttt{x} key while in the sequence editor, to "x-scape"
   from the paint mode.
   These keys, however, do not work while the sequence is playing.

   \index{zoom keys}
   \index{keys!0}
   \index{keys!z}
   \index{keys!shift-z}
   The song editor supports zoom in the piano roll.
   This feature is not accessible via a button or a menu
   entry -- it is accessible only via keystrokes.
   After one has left-clicked in the piano roll, the \texttt{z}, \texttt{Z},
   and \texttt{0} can be used to zoom the piano-roll view.  The \texttt{z} key
   zooms out, the \texttt{Z} key zooms in, and the \texttt{0} key resets the
   zoom to the default value.  The zoom feature also modifies the time-line.

   \index{song editor!left-click-right-hold}
   \index{song editor!insert}
   A left-click with a simultaneous right-click-hold inserts one copy of the
   pattern.  The inserted pattern shows up as a box with a tiny
   representation of the notes visible inside.  Some patterns can
   be less than a measure in length, resulting in a tiny box.
   \index{song editor!right-left-hold-drag}
   \index{song editor!multiple insert}
   To keep adding more copies of the pattern, continue to hold both buttons
   and drag the mouse rightward.

   \index{song editor!middle-click}
   Middle-click on a pattern to drop a new selection position into the
   pattern,
   \index{song editor!pattern subsection}
   which breaks the pattern into two equal \textsl{pattern subsections}.
   Each middle-click on the pattern adds a new selection position,
   halving the size of the subsections as more pattern subsections are
   added.

   \index{song editor!left-click}
   \index{song editor!selection}
   When a pattern or a pattern subsection is left-clicked in the piano
   roll, it is marked with a dark gray filling.
   \index{song editor!right left hold drag}
   \index{song editor!deletion}
   When a right-left-hold-drag action is done in this gray area, the result
   is to \textsl{delete} that pattern section or subsection.
   \index{keys!delete}
   One can also hit the Delete key to \textsl{delete} that pattern section
   or subsection.

   MORE KEYS:

   One can page up and down vertically in the arrangement
   panel using the
   \index{keys!page-up} Page Up and 
   \index{keys!page-down} Page Down keys.
   One can go to the top using the 
   \index{keys!home} Home key,
   to the bottom using the
   \index{keys!end} End key.
   One can page left and right horizontally in the arrangement
   panel using the
   \index{keys!shift-page-up} Shift Page Up and 
   \index{keys!shift-page-down} Shift Page Down keys.
   One can go to the leftmost position using the 
   \index{keys!shift-home} Home key,
   and to the rightmost position using the
   \index{keys!shift-end} End key,

   If playback is started with the song editor as the
   active window, then the pattern boxes in the patterns panel
   show as armed/unarmed (unmuted/muted) depending upon whether or not the
   pattern is shown as playing (or not) at the current playback position in
   the song editor piano roll.
   song editor.

\subsection{Song Editor / Measures Ruler}
\label{subsec:song_editor_measures_ruler}

   \index{measures ruler}
   The measures ruler ("bar indicator")
   consists of a \textsl{timeline} at the top, a
   numbered patterns column at the left with a muting indicator, and the
   grid or roll section.  There are a lot of hidden details in the
   arrangement panel, as the figure shows.  Here are the main sections:

   The \textsl{measures ruler} is the ruled and numbered section at the top
   of the arrangement panel.  It provides a place to put the left and right
   markers.  In the \textsl{Seq24} documentation, it is called the "bar
   indicator".

   \index{measures ruler!left-click}
   Left-click in the measures ruler to move and drop an
   \index{L anchor}
   \index{L marker}
   \textbf{L marker} (\textbf{L anchor}) on the measures ruler.
   \index{measures ruler!right-click}
   Right-click in the measures ruler to drop an
   \index{R anchor}
   \index{R marker}
   \textbf{L marker} (\textbf{R anchor}) on the measures ruler.

   Once these anchors are in place, one can then use
	the \textsl{Collapse} and \textsl{Expand} buttons to modify the
   placement of the pattern events.

   Note that the \textbf{L marker} serves as the start position for playback
   in the song editor.  One can change the start position only when the
   performance is not playing.

   \index{new!marker mode}
   \index{new!movement mode}
   Another way to move the "L" and "R" markers has been added.
   To select which marker will move, first click the upper half of the time
   strip (otherwise, the "L" will move, prematurely) to give it keyboard focus.
   Then press the lower-case
   \index{keys!l}
   \texttt{l} key or the lower-case
   \index{keys!r}
   \texttt{r} key.
   \textsl{There is no visual feedback that one is in the movement mode.}
   Then press the left or right arrow key to move the selected ("L" or "R")
   key marker by one snap value at a time.

\begin{comment}

\subsection{Song Editor / Bottom Panel}
\label{subsec:song_editor_bottom}

   The bottom panel is simple, consisting of a stock horizontal scroll bar
   and a small button, called the \textbf{Grow} button, labelled with a
   "\textbf{$>$}".

   \index{grow button}
   \index{song editor!grow}
   The \textbf{Grow} button adds to the number of measures that exist
   in the song editor. The visual effect is very subtle, resulting only
   in a small change in the thumb of the horizontal scroll-bar, unless one
   is at the right end of the piano roll.  Then, one can see the added
   measures.  Usually about 128 at a time are added, but this depends on the
   value of PPQN in force.

\end{comment}

%-------------------------------------------------------------------------------
% vim: ts=3 sw=3 et ft=tex
%-------------------------------------------------------------------------------
