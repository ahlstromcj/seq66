%-------------------------------------------------------------------------------
% seq66 recording
%-------------------------------------------------------------------------------
%
% \file        seq66 recording.tex
% \library     Documents
% \author      Chris Ahlstrom
% \date        2023-11-25
% \update      2025-05-30
% \version     $Revision$
% \license     $XPC_GPL_LICENSE$
%
%     Provides a discussion of the MIDI GUI recording that Seq66
%     supports.
%
%-------------------------------------------------------------------------------

\section{Seq66 Recording In Depth}
\label{sec:recording}

   Recording in \textsl{Seq66} has been greatly enhanced.
   Multiple patterns can be recorded at once, with events
   routed to particular patterns based on the buss number the event came
   in on, or its channel number.
   Additional ways to toggle recording have been added.
   Additional recording alterations have been added, such as on-the-fly note
   mapping.
   It can get a bit complex to keep track of, hence this section,
   which walks the user through some scenarios.

   Before getting started, note that recording also
   can be done during count-in via the metronome feature.
   See \sectionref{subsection:edit_preferences_metronome}.

\subsection{Recording Scenarios}
\label{sec:recording_scenarios}

   The following recording scenarios are available:

   \begin{itemize}
      \item \textbf{Standard Recording}.
         This provides the normal method of recording. Open a pattern in a
         window and enable recording, or right-click on its live-grid
         slot and enable recording..
         The last slot selected for recording receives the events.
         Events are received from all enabled MIDI input ports.
         In this mode the record-button at the bottom of the main window
         is red.
         The slot popup-menu does not provide for setting an input buss
         in this mode.
      \item \textbf{Record-by-Bus}.
         In this mode, any pattern that specifies a specific input buss
         can be enabled for recording.
         Incoming events are routed to the first pattern that has an input
         buss matching the buss on which the event was recorded.
         In this mode the record-button at the bottom of the main window
         is green.
         To designate a recording buss, first go to
         \textbf{Edit / Preferences / MIDI Input} and enable
         \textbf{Record into patterns by bus}.
         Then right-click on the pattern and select an
         \textbf{Input bus}. (This popup menu entry only appears when
         record-by-bus is active.)
      \item \textbf{Record-by-Channel}.
         In this mode, one enables recording in patterns numbered from 0 to 15
         (channels 1 to 16) with output channels set from 0 to 15.
         Incoming events are analyze for their channel and are recording into
         the corresponding pattern number.
         In this mode the record-button at the bottom of the main window
         is yellow.
   \end{itemize}

   The "Record-by" options above are mutually exclusive.

\subsubsection{Standard Recording}
\label{subsubsec:recording_standard_recording}

   Standard recording allows the enabling of recording in a single pattern.
   It \textsl{requires} that the other two modes be turned off.
   Once a pattern is set to record, no other pattern can be set to record until
   the original pattern is set to not record.
   The single recording pattern gets all events from any MIDI
   input that is enabled.

   In standard recording there a many ways to enable recording into
   a pattern.

   \begin{itemize}
      \item Open the pattern editor in a window or tab or click on the
         pattern slot to record, then click the red record
         button at the bottom of the pattern editor.
         The pattern will also
         show a red circle to indicate that recording has been enabled.
      \item Right-click on the desired pattern in the grid and select
         \textbf{Record toggle} from the menu.
      \item Touch a pattern or click its hot key, then
         click the red record button at the bottom of the main window.
      \item Select the \textbf{RECORD} grid mode, then select a pattern.
         To change the selected slot, click on the red record button
         to disable recording, then click on the desired slot.
         Recording is active on that slot.
         (This option is more useful in the other recording scenarios.)
   \end{itemize}

   Again, remember that, with standard recording, only one pattern can
   accept events.
   Also note that this mode should be used if one
   expects to record SysEx events, which have no channel.

\subsubsection{Route-Input-By-Buss}
\label{subsubsec:recording_route_by_buss}

   \index{record!by buss}
   Route-input-by-buss (also known as record-by-buss)
   is enabled whenever a pattern in a song specifies an
   input buss \textsl{and}
   \textbf{Edit / Preferences / MIDI Input / Record into patterns by
   bus} is checked. It is an 'rc' file option.
   If it is enabled, it supercedes the
   \textbf{record-by-channel} option and
   disables that standard recording mode described above.

   When route-by-bus is enabled, an internal container is populated with
   all the patterns in the current play-set that specify an input buss.
   This container is rebuilt when a sequence is added or removed.
   It is stored in the \texttt{c\_midiinbus} SeqSpec in the song.

   If route-by-bus is enabled, a sequence with an input bus that matches the
   buss associated with the incoming event is looked up, and input is
   streamed to it.

   In this scenario, multiple patterns can be enabled for recording,
   and output from multiple input ports are routed appropriately.

   For the green record button to be enabled, at least one pattern
   must specify an input bus.
   Clicking on the green record button at the bottom of the main window
   will turn on recording for \textsl{all}
   patterns that specify an input buss.

\subsubsection{Record-By-Channel}
\label{subsubsec:recording_record_by_channel}

   Record-by-channel can be enabled in
   \textbf{Edit / Preferences / MIDI Input / Record into patterns by
   channel}, which is an 'rc' file option.

   It works by routing events to the first pattern that has specified an 
   output (not input) channel that matches the channel (if applicable) of
   the incoming MIDI event.
   The patterns applicable are entered into a list of patterns with specified
   output channels.

   \textbf{IMPORTANT}:
   In order for a channel to be recorded, it's corresponding pattern
   must exist. Hence,
   for convenience, there is a MIDI file installed, called
   \textbf{16-blank-patterns}, which specifies all the output channels.
   It can be copied and used for the first recording from a device
   playing on multiple channels, or from multiple devices, each playing
   on a different channel.

   Clicking on the yellow record button at the bottom of the main window
   will turn on recording for \textsl{all}
   patterns, except for patterns that specify "Free" (i.e. no forced channel).

   Note that the route-by-buss option, described above, supercedes this
   option.

\subsection{Recording Modes}
\label{sec:recording_modes}

   Recording can also transform (alter) the incoming events.

   \begin{itemize}
      \item \textbf{Tighten}.
         Partial quantization to the current snap value.
      \item \textbf{Quantize}.
         Full quantization to the current snap value.
      \item \textbf{Note-map}.
         If a 'drums' file is active, then notes are changed
         in note-value according to that file.
         This is most useful for drums, but other effects are possible.
   \end{itemize}

   These transformations can also be done after recording.

\subsection{Seq66-to-Seq66 Recording}
\label{sec:recording_seq_to_seq}

   This section describes a way to have one instance of \textsl{Seq66}
   (\texttt{qseq66}) interact with another.
   The first \textsl{Seq66} to start runs with a different client name
   (optional) and a different 'rc' file name (to not affect the normal
   configuration of \textsl{Seq66}).
   The first \textsl{Seq66} runs with virtual ports; the second
   \textsl{Seq66} runs normally and can auto-connect to these
   ports.

   The first instance:

   Run and exit to create the configuration files, then run with
   the "virtual" option; the first number is the number of output ports
   to make, and the second is the number of input ports to make.

   \begin{verbatim}
       $ qseq66 --client first66 --rc first66
       $ qseq66 --client first66 --rc first66 --option virtual=2,2
   \end{verbatim}

   In the \textbf{MIDI Clock} tab should be "first66:midi out 0" and
   "first66:midi out 1".
   In the \textbf{MIDI Input} tab should be "first66:midi in 0" and
   "first66:midi in 1".
   Among the ports shown, along with plugged in hardware, are:

   \begin{verbatim}
      $ aplaymidi -l
       Port    Client name                      Port name
       44:0    MPK mini Play mk3                MPK mini Play mk3 MIDI 1
      128:0    first66                          midi in 0
      128:1    first66                          midi in 1
      ahlstrom@mlstrycoo ~/Home/ca/mls/git/seq66
      $ arecordmidi -l
       Port    Client name                      Port name
       44:0    MPK mini Play mk3                MPK mini Play mk3 MIDI 1
      128:2    first66                          midi out 0
      128:3    first66                          midi out 1
   \end{verbatim}

   Leave this instance running, and run a plain \texttt{qseq66} command.
   It should show the "first66" output (clocks) and input ports.

   One should now be able to pass MIDI events between the two instances.

% VERIFY this at some point.

%-------------------------------------------------------------------------------
% vim: ts=3 sw=3 et ft=tex
%-------------------------------------------------------------------------------
